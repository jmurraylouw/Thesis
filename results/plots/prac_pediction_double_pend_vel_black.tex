\begin{figure}[htb]
    \centering
    \begin{tikzpicture}
        \begin{axis}[            
            xlabel = Time,
            ylabel = North velocity,
            x unit = \si{\second},
            y unit = \si{\metre/\second},
            xmin = 0,   xmax = 20,
            ymin = -2.5,  ymax = 2.5,
            grid = major,
            legend cell align = left,
            legend pos = north east,
            grid style = dashed,
            legend style = {font = \scriptsize},
            label style = {font = \scriptsize},
            tick label style = {font = \scriptsize},
            width = 0.95\columnwidth,
            height = 0.5\columnwidth,
            % initialize Dark2
            cycle list/Dark2,
            % combine it with 'mark list*':
            cycle multiindex* list = {
                Dark2\nextlist
            }
        ]
        
        \addplot+[mark = none, style = solid, ultra thick] 
        table[x = time, y = vel, col sep = comma] 
        {results/csv/step_predictions_Prac_2021-08-23_04_double_pend_m1_0.2_m2_0.1_l1-0.5_l2_0.62_wind-0.5.csv_dmd_201.csv};
        \addlegendentry{Measured}

        \addplot+[mark = none, style = dashed, ultra thick] 
        table[x = time, y = vel_hat, col sep = comma] 
        {results/csv/step_predictions_Prac_2021-08-23_04_double_pend_m1_0.2_m2_0.1_l1-0.5_l2_0.62_wind-0.5.csv_dmd_201.csv};
        \addlegendentry{DMDc prediction}

        % \addplot+[mark = none, style = solid, ultra thick] 
        % table[x = time, y = theta, col sep = comma] 
        % {results/csv/step_predictions_Prac_2021-08-23_04_double_pend_m1_0.2_m2_0.1_l1-0.5_l2_0.62_wind-0.5.csv_dmd_201.csv};
        % \addlegendentry{White-box model}

        \end{axis}
    \end{tikzpicture} 
    
    \caption{Practical flight data and model predictions with an elongated payload for a North velocity step input
        ($m_1 =$~\SI{0.2}{\kilo\gram}, $l_1 =$~\SI{0.5}{\meter}, $m_2 =$~\SI{0.1}{\kilo\gram}, $l_2 =$~\SI{0.6}{\meter})}
    \label{fig:prac_pediction_double_pend_vel_black}
\end{figure}
