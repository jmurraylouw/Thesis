\graphicspath{{results/fig/}}
{
\tikzset{external/figure name/.add={results/}{}}

\chapter{Practical implementation and results} \label{chap:results}

    \paragraph
    In Chapter~\ref{chap:system_id}, it was shown within simulations that both the white-box and black-box system identification models can
    accurately represent the dynamics of a multirotor with a suspended payload.
    In Chapter~\ref{chap:control}, it was also shown within simulations that the proposed controllers effectively achieve swing-damping control. 
    However, practical flights may differ significantly from simulations, which would affect the performance of these implementations.
    Wind is a common unmeasured disturbance that influences the flight dynamics of a multirotor, 
    but this disturbance was not considered in simulations.
    The practical dynamics and sensor noise may also differ from the simulation model, which further motivates the need for practical data.
    
    \paragraph
    In this chapter, the system identification techniques will be applied to practical flight data using the same methodology described in Chapter~\ref{chap:system_id}.
    The effect of different wind conditions and payloads on the performance of these techniques will be investigated.
    The performance of these techniques on a practical dynamic payload system will also be shown.
    Finally, \gls{HITL} simulations will be performed to determine whether the available hardware can handle the computational complexity of the \gls{MPC} implementation.

    \FloatBarrier\section{Parameter estimation with practical data}

        \paragraph
        In Section~\ref{sec:param_estimation}, a cable length parameter estimation technique was performed with simulation data and the resulting white-box models produced reasonably accurate representations of the simulated multirotor-payload dynamics.
        In this section, the cable length estimation technique will be applied to practical flight data with different payload masses and cable lengths.
        The effect of wind on the parameter estimation technique will also be investigated.
        Finally, the cable length estimator will be applied to data from a dynamic payload.
        
        % \subsection{Mass estimation} 
        
        %     As discussed in Chapter~\ref{chap:system_id}, for this method to work, the mass of the multirotor needs to be known.
        %     If a heavier battery or an extra accessory is added to the vehicle, 
        %     the new vehicle mass will first have to be estimated in a separate flight before the payload can be attached.
        %     This is one of the disadvantages of white-box modelling with parameter estimation based techniques.
        %     The method is designed with a specific system in mind 
        %     and needs to be adjusted and redesigned for every deviation from the pre-assumed configuration.
        %     In contrast, the data-driven technique is a general solution that works for a much wider range of system configurations.
        %     The data-driven technique is not readjusted for an added vehicle mass, 
        %     since the whole model is estimated instead of specific parameters.

        \FloatBarrier\subsection{Simple payload cable length estimation}
            
            \paragraph
            As discussed in Section~\ref{sec:length_estimation}, an \gls{FFT} of the payload angle data is used to estimate the natural frequency of the suspended payload, 
            which is used to estimate the cable length.
            \gls{PE} is used as the error metric to quantify the estimation accuracy.
            The \gls{PE} of the cable length estimation is calculated as,
            \begin{equation}
                PE = \frac{ l_{estimated} - l_{actual} }{ l_{actual} } \times 100 \% ,
            \end{equation}
            where $l_{actual}$ is the actual cable length and $l_{estimated}$ is the estimated cable length.
            The \gls{PE} can be interpreted as the percentage of the actual length by which the estimated length differs from the actual length.
            
            \begin{figure}[htb]
    \centering
    \begin{tikzpicture}
        \begin{axis}[            
            xlabel = Length of training data,
            ylabel = Percentage Error,
            x unit = \si{\second},
            y unit = \%,
            xmin = 0,   xmax = 12,
            ymin = -50,   ymax = 100,
            grid = major,
            legend cell align = left,
            legend pos = north east,
            grid style = dashed,
            legend style = {font = \scriptsize},
            label style = {font = \scriptsize},
            tick label style = {font = \scriptsize},
            width = 0.95\columnwidth,
            height = 0.5\columnwidth,
            % initialize Dark2
            cycle list/Dark2,
            % combine it with 'mark list*':
            cycle multiindex* list = {
                Dark2\nextlist
            }
        ]

        \addplot+[mark = none, style = solid, ultra thick] 
        table[x = train_time, y = percentage_error, col sep = comma] 
        {results/csv/cable_length_vs_train_time_Prac_2021-08-23_01_l-0.5_mp-0.2_wind-0.5.csv_10_0.5.csv};
        \addlegendentry{$m_p =$~\SI{0.2}{\kilo\gram}, $l =$~\SI{0.5}{\meter}}

        \addplot+[mark = none, style = solid, ultra thick] 
        table[x = train_time, y = percentage_error, col sep = comma] 
        {results/csv/cable_length_vs_train_time_Prac_2021-08-20_01_l-1_mp-0.1_wind-0.5.csv_10_1.csv};
        \addlegendentry{$m_p =$~\SI{0.1}{\kilo\gram}, $l =$~\SI{1}{\meter}}        

        \addplot+[mark = none, style = solid, ultra thick] 
        table[x = train_time, y = percentage_error, col sep = comma] 
        {results/csv/cable_length_vs_train_time_Prac_2021-08-20_02_l-2_mp-0.3-wind-0.5.csv_10_2.csv};
        \addlegendentry{$m_p =$~\SI{0.3}{\kilo\gram}, $l =$~\SI{2}{\meter}}

        \addplot+[mark = none, style = solid, ultra thick] 
        table[x = train_time, y = percentage_error, col sep = comma] 
        {results/csv/cable_length_vs_train_time_Prac_2021-08-20_03_l-1_mp-0.2_wind-0.5.csv_10_1.csv};
        \addlegendentry{$m_p =$~\SI{0.2}{\kilo\gram}, $l =$~\SI{1}{\meter}}

        \end{axis}
    \end{tikzpicture} 
    
    \caption{Plot of the error in cable length estimation as a function of length of training data 
    (wind speed $\approx~$\SI{0.5}{\metre/\second})}
    \label{fig:cable_length_vs_train_time}
\end{figure}


            \paragraph 
            Figure~\ref{fig:cable_length_vs_train_time} shows the \gls{PE} of cable length estimation for practical flight data with different payload masses and cable lengths using practical fligth data.
            Note that for each payload configuration, 
            the estimation converges to a constant error after a sufficient length of training data.
            For these payload configurations, the converged \gls{PE} ranges from 18.9~\% to 32.4~\%.
            These errors may be due to the large difference between the theoretical and the damped natural frequency.
            It appears that the \gls{PID} controllers damp the payload swing angles, 
            which affects the oscillation frequency of the payload.
            Hence, an inaccurate length is estimated from the frequency peak identified in the \gls{FFT}.
            % The estimation accuracy could be improved by including the effect of damping in the calculation of the cable length from the measured natural frequency.
            % However, it was shown by \cite{Slabber2020} that this is
            Unlike in the simulations, the actual cable attachment may be below the \gls{CoM} of the vehicle.
            This increases the effective suspended length and may also contribute to the error in parameters estimation.
            
            \begin{figure}[htb]
    \centering
    \begin{tikzpicture}
        \begin{axis}[            
            xlabel = Time,
            ylabel = Payload angle,
            x unit = \si{\second},
            y unit = \si{\radian},
            xmin = 0,   xmax = 20,
            ymin = -20,  ymax = 20,
            grid = major,
            legend cell align = left,
            legend pos = north east,
            grid style = dashed,
            legend style = {font = \scriptsize},
            label style = {font = \scriptsize},
            tick label style = {font = \scriptsize},
            width = 0.95\columnwidth,
            height = 0.5\columnwidth,
            % initialize Dark2
            cycle list/Dark2,
            % combine it with 'mark list*':
            cycle multiindex* list = {
                Dark2\nextlist
            }
        ]
        
        \addplot+[mark = none, style = solid, ultra thick] 
        table[x = time, y = theta, col sep = comma] 
        {results/csv/step_predictions_Prac_2021-08-20_02_l-2_mp-0.3-wind-0.5.csv_white_26.34.csv};
        \addlegendentry{Actual}

        \addplot+[mark = none, style = solid, ultra thick] 
        table[x = time, y = theta_hat, col sep = comma] 
        {results/csv/step_predictions_Prac_2021-08-20_02_l-2_mp-0.3-wind-0.5.csv_white_26.34.csv};
        \addlegendentry{Model prediction using the estimated length}

        \end{axis}
    \end{tikzpicture} 
    
    \caption{White-box model prediction for a North velocity step input
    ($l =$~\SI{0.5}{\metre}, $m_p =$~\SI{0.2}{\kilo\gram}.)}
    \label{fig:prediction_single_pend_white_prac}
\end{figure}


            \paragraph
            Figure~\ref{fig:prediction_single_pend_white_prac} compares the actual payload angle to the predicted angle of the white-box model for a velocity step in a practical flight. 
            The cable length estimated from this flight is \SI{2.64}{\metre} resulting in a \gls{PE} of 32.0~\%.
            The prediction matches the general shape of the practical data well.
            However, the transient response of the practical data, which can be seen in the first two oscillation peaks, is noticeably different from the model prediction.
            This is probably due to the inner loop controllers dynamics which affects the transient response of the practical data but are not accounted for in the white-box.
            
            \paragraph
            Furthermore, the practical swing angle peaks are attenuated by non-linear damping, which differs from the linear damping of the white-box model.
            This is a minor modelling error expected from a linearised model.

            % \input{results/plots/prediction_single_pend_white_prac_bad.tex}
            % \paragraph
            % Figure~\ref{fig:prediction_single_pend_white_prac_bad} shows the white-box prediction of practical data, when the step input is seen after the initial condition

            % \paragraph
            % The naive assumption regarding the speed of the inner loop \gls{PID} controller probably adds to the modelling error significantly.
            % As discussed in Chapter~\ref{chap:modelling}, the white-box model assumes that the time scale separation between the velocity and attitude controller is large enough for the acceleration setpoint to approximate the actual separation.
            % However, it appears that the attitude controller of the practical multirotor is slow enough to significantly affect the transient response of the swing angle.
            % Notice how the first two oscillation peaks differ noticeably from the expected linear damping shape.
            % Since the payload is attached below the \gls{CoM} of the multirotor, the oscillating pitch angle of the multirotor also affects the swing angle of the payload.

            % Recall from Chapter~\ref{chap:system_id} that the damping coefficient is not estimated, but a manually tuned value is used.

            \begin{figure}[htb]
    \centering
    \begin{tikzpicture}
        \begin{axis}[            
            xlabel = Length of training data,
            ylabel = Percentage Error,
            x unit = \si{\second},
            y unit = \%,
            xmin = 0,       xmax = 12,
            ymin = -50,     ymax = 100,
            grid = major,
            legend cell align = left,
            legend pos = north east,
            grid style = dashed,
            legend style = {font = \scriptsize},
            label style = {font = \scriptsize},
            tick label style = {font = \scriptsize},
            width = 0.95\columnwidth,
            height = 0.5\columnwidth,
            % initialize Dark2
            cycle list/Dark2,
            % combine it with 'mark list*':
            cycle multiindex* list = {
                Dark2\nextlist
            }
        ]
        
        \addplot+[mark = none, style = solid, ultra thick] 
        table[x = train_time, y = percentage_error, col sep = comma] 
        {results/csv/cable_length_vs_train_time_Prac_2021-08-20_03_l-1_mp-0.2_wind-0.5.csv_10_1.csv};
        \addlegendentry{wind speed $\approx~$\SI{0.5}{\metre/\second}}

        \addplot+[mark = none, style = solid, ultra thick] 
        table[x = train_time, y = percentage_error, col sep = comma] 
        {results/csv/cable_length_vs_train_time_Prac_2021-08-12_03_manual_x_vel_steps_2mps.csv_10_1.csv};
        \addlegendentry{wind speed $\approx~$\SI{2}{\metre/\second}}

        \addplot+[mark = none, style = solid, ultra thick] 
        table[x = train_time, y = percentage_error, col sep = comma] 
        {results/csv/cable_length_vs_train_time_Prac_2021-08-12_02_manual_x_vel_steps_4mps.csv_20_1.csv};
        \addlegendentry{wind speed $\approx~$\SI{4}{\metre/\second}}

        \addplot+[mark = none, style = solid, ultra thick] 
        table[x = train_time, y = percentage_error, col sep = comma] 
        {results/csv/cable_length_vs_train_time_Prac_2021-08-26_01_l-1_mp-0.2_wind-6.csv_10_1.csv};
        \addlegendentry{wind speed $\approx~$\SI{6}{\metre/\second}}

        \end{axis}
    \end{tikzpicture} 
    
    \caption{Cable length estimation error as a function of length of training data with wind disturbances
        ($m_p =$~\SI{0.2}{\kilo\gram}, $l =$~\SI{1}{\meter})}
    \label{fig:cable_length_vs_train_time_wind}
\end{figure}


            \paragraph
            Figure~\ref{fig:cable_length_vs_train_time_wind} shows the \gls{PE} of cable length estimation for flights with different wind conditions.
            These flights were all performed with the same payload.
            Wind conditions are referenced here by the wind speed recorded by the website, www.yr.no, for the hour of the day of the flight.
            It appears that the wind speed affects the parameter estimation result since the estimation error differs significantly for different wind speeds.
            This is probably due to the variable damping effect of the controllers at different wind speeds.
            From the considered flights, the largest \gls{PE} occurs at the highest wind speed, 
            and the lowest \gls{PE} at the lowest wind speed.
            However, only a few different wind speeds were tested and a trend cannot be identified conclusively from this small sample.

            \paragraph
            Note in Figure~\ref{fig:cable_length_vs_train_time_wind} that the estimation \gls{PE} converges for each considered flight,
            even with wind speeds up to \SI{6}{\metre/\second}.
            Therefore a dominant oscillation frequency emerges from each flight, 
            even when the multirotor is heavily affected by wind.
        
        \FloatBarrier\subsection{Dynamic payload cable length estimation} \label{sec:dynamic_payload_prac}

            \paragraph
            As discussed in Chapter~\ref{chap:system_id}, 
            the dynamical equations of the white-box model are fixed in the a~priori modelling phase.
            The model is then populated with values from parameter estimation techniques.
            However, when the dynamics of the observed system differ significantly from the pre-determined model,
            the parameter estimation algorithms still determine naive, best-fit values for the pre-determined model.

            \begin{figure}[!htb]
                \centering
                \includegraphics[width=0.6\linewidth]{practical_double_pendulum_2.jpg}
                \caption{Practical flight with a suspended elongated payload attached to Honeybee.}
                \label{fig:practical_double_pendulum}
            \end{figure}     

            \paragraph
            One of the a~priori modelling assumptions mentioned in Section~\ref{sec:param_estimation},
            is that the suspended payload is a point-mass.
            This reduced the considered suspended payload system to a single pendulum in the white-box model.
            Figure~\ref{fig:practical_double_pendulum} shows a photo of an elongated payload suspended from Honeybee during a practical flight
            This is a practical example of a dynamic payload that deviates significantly from the point-mass assumption.
            The mass distribution causes a rotation of the payload relative to the suspended cable,
            which significantly affects the flight dynamics.
            
            \begin{figure}
                \captionsetup[subfigure]{justification=centering}
                \centering  
                \begin{subfigure}[t]{0.45\columnwidth}
    \centering
    \begin{tikzpicture}
        \begin{axis}[            
            xlabel = Time,
            ylabel = Payload angle,
            x unit = \si{\second},
            y unit = \si{\degree},
            xmin = 0,   xmax = 7,
            ymin = -20,  ymax = 25,
            grid = major,
            legend cell align = left,
            legend pos = north east,
            grid style = dashed,
            legend style = {font = \scriptsize},
            label style = {font = \scriptsize},
            tick label style = {font = \scriptsize},
            width = 0.95\columnwidth,
            height = 0.95\columnwidth,
            % initialize Dark2
            cycle list/Dark2,
            % combine it with 'mark list*':
            cycle multiindex* list = {
                Dark2\nextlist
            }
        ]
        
        \pgfplotsset{cycle list shift=1}

        \addplot+[mark = none, style = solid, ultra thick] 
        table[x expr = \thisrow{time} - 4, y = theta, col sep = comma] 
        {results/csv/step_predictions_Prac_2021-08-23_04_double_pend_m1_0.2_m2_0.1_l1-0.5_l2_0.62_wind-0.5.csv_dmd_201.csv};

        \end{axis}
    \end{tikzpicture} 
    
    \caption{Measured payload angle data.}
    \label{fig:FFT_vel_step_double_pend}
\end{subfigure}
 % subfigure
                \begin{subfigure}[t]{0.45\columnwidth}
    \centering
    \begin{tikzpicture}
        \begin{axis}[            
            xlabel = Frequency,
            ylabel = Amplitude,
            x unit = \si{\hertz},
            % y unit = \si{\second},
            xmin = 0.3,  xmax = 1.7,
            ymin = 0,    ymax = 0.8,
            grid = major,
            legend cell align = left,
            legend pos = north east,
            grid style = dashed,
            legend style = {font = \scriptsize},
            label style = {font = \scriptsize},
            tick label style = {font = \scriptsize},
            width = 0.95\columnwidth,
            height = 0.95\columnwidth,
            % initialize Dark2
            cycle list/Dark2,
            % combine it with 'mark list*':
            cycle multiindex* list = {
                Dark2\nextlist
            }
        ]

        \addplot+[mark = none, style = solid, ultra thick] 
        table[x = f, y = P1, col sep = comma] 
        {results/csv/FFT_vel_step_Prac_2021-08-23_04_double_pend_m1_0.2_m2_0.1_l1-0.5_l2_0.62_wind-0.5.csv.csv};

        \end{axis}
    \end{tikzpicture} 
    
    \caption{FFT amplitude spectrum}
    \label{fig:FFT_double_pend_prac}
\end{subfigure}
 % subfigure
                \caption{White-box model prediction for a North velocity step input for a dynamic payload
                ($m_1 =$~\SI{0.2}{\kilo\gram}, $l_1 =$~\SI{0.5}{\meter}, $m_2 =$~\SI{0.1}{\kilo\gram}, $l_2 =$~\SI{0.6}{\meter}).}
                \label{fig:FFT_double_pend_prac_subfigs}  
            \end{figure}

            \paragraph
            Figure~\ref{fig:FFT_vel_step_double_pend} shows a snapshot of the payload angle data from a practical flight with a dynamic payload.
            Two superimposed frequencies are visible in the payload oscillations due to the double pendulum action of the elongated pendulum.
            The two peaks corresponding to these two frequencies can easily be identified from the \gls{FFT} amplitude spectrum in Figure~\ref{fig:FFT_double_pend_prac}.
            The cable length estimation method uses the frequency of the dominant peak and calculates the effective length corresponding to that frequency.
            This results in a single pendulum model that best matches the dynamic payload oscillations. 

            \begin{figure}[htb]
    \centering
    \begin{tikzpicture}
        \begin{axis}[            
            xlabel = Length of training data,
            ylabel = Estimated cable length,
            x unit = \si{\second},
            y unit = \si{\metre},
            xmin = 0,   xmax = 10,
            ymin = 0,   ymax = 4,
            grid = major,
            legend cell align = left,
            legend pos = north east,
            grid style = dashed,
            legend style = {font = \scriptsize},
            label style = {font = \scriptsize},
            tick label style = {font = \scriptsize},
            width = 0.95\columnwidth,
            height = 0.5\columnwidth,
            % initialize Dark2
            cycle list/Dark2,
            % combine it with 'mark list*':
            cycle multiindex* list = {
                Dark2\nextlist
            }
        ]

        \addplot+[mark = none, style = solid, ultra thick] 
        table[x = train_time, y = estimated_length, col sep = comma] 
        {results/csv/cable_length_vs_train_time_Prac_2021-08-23_04_double_pend_m1_0.2_m2_0.1_l1-0.5_l2_0.62_wind-0.5.csv_20_0.5.csv};
        
        \end{axis}
    \end{tikzpicture} 
    
    \caption{Estimated cable length as a function of length of training data for a dynamic payload
    ($m_1 =$~\SI{0.2}{\kilo\gram}, $l_1 =$~\SI{0.5}{\meter}, $m_2 =$~\SI{0.1}{\kilo\gram}, $l_2 =$~\SI{0.6}{\meter}) }
    \label{fig:double_pend_cable_length_vs_train_time}
\end{figure}

            
            \paragraph
            Figure~\ref{fig:double_pend_cable_length_vs_train_time} shows the estimated cable length 
            as a function of the length of training data for a practical dynamic payload.
            Note that the estimated length converges after a sufficient length of training data, showing that a dominant oscillation frequency can be identified.
            For this flight, the estimated cable length is \SI{1.03}{\metre}.
            
            \begin{figure}
                \captionsetup[subfigure]{justification=centering}
                \centering  
                \input{results/plots/prediction_double_pend_white_prac.tex} % subfigure
                \begin{subfigure}[t]{\columnwidth}
    \centering
    \begin{tikzpicture}
        \begin{axis}[            
            xlabel = Time,
            ylabel = Payload angle,
            x unit = \si{\second},
            y unit = \si{\degree},
            xmin = 0,   xmax = 19,
            ymin = -25,  ymax = 25,
            grid = major,
            legend cell align = left,
            legend pos = north east,
            grid style = dashed,
            legend style = {font = \scriptsize},
            label style = {font = \scriptsize},
            tick label style = {font = \scriptsize},
            width = 0.95\columnwidth,
            height = 0.5\columnwidth,
            % initialize Dark2
            cycle list/Dark2,
            % combine it with 'mark list*':
            cycle multiindex* list = {
                Dark2\nextlist
            }
        ]
        
        \addplot+[mark = none, style = solid, ultra thick] 
        table[x = time, y = theta, col sep = comma] 
        {results/csv/step_predictions_Prac_2021-08-23_04_double_pend_m1_0.2_m2_0.1_l1-0.5_l2_0.62_wind-0.5.csv_white_36.9.csv};
        % \addlegendentry{Actual}

        \addplot+[mark = none, style = solid, ultra thick] 
        table[x = time, y = theta_hat, col sep = comma] 
        {results/csv/step_predictions_Prac_2021-08-23_04_double_pend_m1_0.2_m2_0.1_l1-0.5_l2_0.62_wind-0.5.csv_white_36.9.csv};
        % \addlegendentry{Model prediction using the estimated length}

        \end{axis}
    \end{tikzpicture} 
    
    % \caption{Bad white-box model approximation.}
    \label{fig:prediction_double_pend_white_prac_bad}
\end{subfigure}
 % subfigure
                \caption{Data from a velocity step response with a dynamic payload 
                ($m_1 =$~\SI{0.2}{\kilo\gram}, $l_1 =$~\SI{0.5}{\meter}, $m_2 =$~\SI{0.1}{\kilo\gram}, $l_2 =$~\SI{0.6}{\meter}).}
                \label{fig:predictions_double_pend_prac_subfigs}  
            \end{figure}

            \paragraph
            Figure~\ref{fig:predictions_double_pend_prac_subfigs} shows two model predictions resulting from slightly different starting points in flight data.
            The two prediction runs in Figure~\ref{fig:predictions_double_pend_prac_subfigs} differ significantly from each other even though the starting points of the predictions are offset by only \SI{0.06}{\second}.
            Since the oscillations of the dynamic payload are irregular compared to the sinusoidal dynamics of the white-box model, the prediction accuracy is very sensitive to the initial condition.
            
            \begin{figure}[htb]
    \centering
    \begin{tikzpicture}
        \begin{axis}[            
            xlabel = Time,
            ylabel = Payload angle,
            x unit = \si{\second},
            y unit = \si{\degree},
            xmin = 0,   xmax = 10,
            ymin = -25,  ymax = 25,
            grid = major,
            legend cell align = left,
            legend pos = north east,
            grid style = dashed,
            legend style = {font = \scriptsize},
            label style = {font = \scriptsize},
            tick label style = {font = \scriptsize},
            width = 0.95\columnwidth,
            height = 0.5\columnwidth,
            % initialize Dark2
            cycle list/Dark2,
            % combine it with 'mark list*':
            cycle multiindex* list = {
                Dark2\nextlist
            }
        ]
        
        \addplot+[mark = none, style = solid, ultra thick] 
        table[x = time, y = theta_hat, col sep = comma] 
        {results/csv/step_predictions_Prac_2021-08-23_04_double_pend_m1_0.2_m2_0.1_l1-0.5_l2_0.62_wind-0.5.csv_white_36.9_diff_IC.csv};
        \addlegendentry{offset = \SI{0.00}{\second}}

        % \addplot+[mark = none, style = solid, ultra thick] 
        % table[x = time, y = theta_hat, col sep = comma] 
        % {results/csv/step_predictions_Prac_2021-08-23_04_double_pend_m1_0.2_m2_0.1_l1-0.5_l2_0.62_wind-0.5.csv_white_36.92_diff_IC.csv};
        % \addlegendentry{0.02}

        \addplot+[mark = none, style = solid, ultra thick] 
        table[x = time, y = theta_hat, col sep = comma] 
        {results/csv/step_predictions_Prac_2021-08-23_04_double_pend_m1_0.2_m2_0.1_l1-0.5_l2_0.62_wind-0.5.csv_white_36.95_diff_IC.csv};
        \addlegendentry{offset = \SI{0.05}{\second}}

        % \addplot+[mark = none, style = solid, ultra thick] 
        % table[x = time, y = theta_hat, col sep = comma] 
        % {results/csv/step_predictions_Prac_2021-08-23_04_double_pend_m1_0.2_m2_0.1_l1-0.5_l2_0.62_wind-0.5.csv_white_36.96_diff_IC.csv};
        % \addlegendentry{0.06}

        \addplot+[mark = none, style = solid, ultra thick] 
        table[x = time, y = theta_hat, col sep = comma] 
        {results/csv/step_predictions_Prac_2021-08-23_04_double_pend_m1_0.2_m2_0.1_l1-0.5_l2_0.62_wind-0.5.csv_white_37_diff_IC.csv};
        \addlegendentry{offset = \SI{0.10}{\second}}

        % \addplot+[mark = none, style = solid, ultra thick] 
        % table[x = time, y = theta_hat, col sep = comma] 
        % {results/csv/step_predictions_Prac_2021-08-23_04_double_pend_m1_0.2_m2_0.1_l1-0.5_l2_0.62_wind-0.5.csv_white_37_diff_IC.csv};
        % \addlegendentry{0.10}

        % \addplot+[mark = none, style = solid, ultra thick] 
        % table[x = time, y = theta_hat, col sep = comma] 
        % {results/csv/step_predictions_Prac_2021-08-23_04_double_pend_m1_0.2_m2_0.1_l1-0.5_l2_0.62_wind-0.5.csv_white_37.02_diff_IC.csv};
        % \addlegendentry{offset = \SI{0.12}{\second}}

        % \addplot+[mark = none, style = dashed, ultra thick] 
        % table[x = time, y = theta, col sep = comma] 
        % {results/csv/step_predictions_Prac_2021-08-23_04_double_pend_m1_0.2_m2_0.1_l1-0.5_l2_0.62_wind-0.5.csv_white_36.9_diff_IC.csv};
        % \addlegendentry{Actual}

        \end{axis}
    \end{tikzpicture} 
    
    \caption{White-box predictions from different initial conditions for a dynamic payload 
    ($m_1 =$~\SI{0.2}{\kilo\gram}, $l_1 =$~\SI{0.5}{\meter}, $m_2 =$~\SI{0.1}{\kilo\gram}, $l_2 =$~\SI{0.6}{\meter}) }
    \label{fig:prediction_double_pend_white_prac_diff_IC}
\end{figure}

            
            \paragraph
            Figure~\ref{fig:prediction_double_pend_white_prac_diff_IC} shows the white-box model predictions for the dynamic payload data with slightly different starting positions in the data.
            Note how much the predictions differ even though the starting points are so close together. 
            This shows how sensitive the white-box model is to the initial condition of the prediction.
            This is because the white-box model consists of ordinary differential equations which depend on the initial angular rate of the payload.
            Even though the oscillations of the dynamic payload have an approximate sinusoidal shape, the time derivative of the data differs significantly from the sinusoidal white-box dynamics.
            This is clear from numerous infliction points in the payload angle data shown in Figure~\ref{fig:predictions_double_pend_prac_subfigs}.
            
            \paragraph
            However, as the size of the swing angles attenuates the relative oscillation of the elongated payload also decreases.
            Therefore the effect of the superimposed higher frequency oscillations become less prominent and the system dynamics approximates a single pendulum more closely.
            For example, in Figure~\ref{fig:predictions_double_pend_prac_subfigs} it can be seen that the oscillations after \SI{12}{\second} are much less irregular than before. 
            Therefore the single pendulum model provides a decent representation of a practical dynamic payload for small swing angles.
            
            \paragraph
            Overall, the white-box model represents the general shape of the practical data, but does not capture the transient response of the system and is very sensitive to initial conditions. 
            
    \FloatBarrier\section{Data-driven system identification with practical data} \label{sec:data_driven_sys_id_prac}

        \paragraph
        In Chapter~\ref{chap:system_id} it was shown that the considered data-driven methods build accurate models of the system dynamics from simulation data.
        It was also shown in Chapter~\ref{chap:modelling} that the simulation environment is a realistic representation of the practical system.
        However, there are still differences between simulations and practical flights.
        Therefore the data-driven algorithms will be tested with real flight data to evaluate their suitability for practical implementations.

        \FloatBarrier\subsection{Wind disturbance} \label{sec:length_train_data_prac}

            \paragraph
            The wind conditions during practical flights have a large influence on the quality of the flight data gathered.
            Wind adds an unmeasured disturbance to the considered system which is detrimental to system identification.
            This disturbance consists of a randomly fluctuating force applied to the vehicle, cable and payload.
            It is very difficult to model these forces accurately and to determine accurate drag coefficients of the practical system for realistic simulations.

            \paragraph
            The mean force applied to the multirotor by the wind affects the mean offset in acceleration setpoint data because the velocity controller integrators compensate for the disturbance.
            The mean offset is subtracted from the acceleration setpoint data, 
            which results in a signal with a zero mean which is used for system identification.
            This accounts for the mean force applied by the wind.
            However, the wind speed also fluctuates from the mean randomly.
            This results in random process noise in the plant which cannot easily be removed from the measured data.

            \input{results/plots/wind_Ttrain.tex}

            \paragraph
            Figure~\ref{fig:wind_Ttrain} shows prediction error as a function of training data length for different wind conditions.
            This plot shows that the minimum prediction error decreases with decreasing wind speeds.
            This is expected since lower wind speeds correspond to less process noise which is beneficial for system identification.
            Note that the prediction error corresponding to \SI{6}{\metre/\second} winds does not vary much with the length of training data.
            % This may be because the unmeasured wind disturbance dominates the flight dynamics such the data-driven algorithms cannot identify a representative model with any length of training data. 
            The prediction error at this wind speed is quite large and a model generated from such data will probably not be useful for control.

            \begin{figure}[htb]
    \centering
    \begin{tikzpicture}
        \begin{axis}[            
            xlabel = Length of training data,
            ylabel = $\overline{NMAE}$ \phantom{~},
            x unit = \si{\second},
            y unit = \%,
            xmin = 5,     xmax = 120,
            ymin = 3.2,  ymax = 15,
            grid = major,
            legend cell align = left,
            legend pos = north east,
            grid style = dashed,
            legend style = {font = \scriptsize},
            label style = {font = \scriptsize},
            tick label style = {font = \scriptsize},
            width = 0.95\columnwidth,
            height = 0.5\columnwidth,
            % initialize Dark2
            cycle list/Dark2,
            % combine it with 'mark list*':
            cycle multiindex* list = {
                Dark2\nextlist
            }
        ]

        \addplot+[mark = none, style = solid, ultra thick] 
        table[x = T_train, y expr = {\thisrow{NMAE_mean}*100}, col sep = comma] 
        {results/csv/NMAE_vs_Ntrain_Prac_2021-08-12_03_manual_x_vel_steps_2mps.csv_dmd_angle.csv};
        \addlegendentry{DMD}

        \addplot+[mark = none, style = solid, ultra thick] 
        table[x = T_train, y expr = {\thisrow{NMAE_mean}*100}, col sep = comma] 
        {results/csv/NMAE_vs_Ntrain_Prac_2021-08-12_03_manual_x_vel_steps_2mps.csv_havok_angle.csv};
        \addlegendentry{HAVOK}

        \end{axis}
    \end{tikzpicture} 
    
    \caption{\gls{DMD} and \gls{HAVOK} prediction errors for different lengths of practical training data
    ($m =$~\SI{0.206}{\kilo\gram}, $l =$~\SI{1}{\meter}, $T_s =$~\SI{0.03}{\second}).}
    \label{fig:havok_vs_dmd_Ttrain_2mps}
\end{figure}


            \paragraph
            Figure~\ref{fig:havok_vs_dmd_Ttrain_2mps} compares the prediction errors of \gls{DMDc} and \gls{HAVOKc} models.
            This shows that the two techniques produce similar prediction errors for different wind conditions.
            The difference in prediction error is small and will probably not affect the performance of the controllers using these models.
            This affirms the observation in Chapter~\ref{chap:system_id} that the minor difference in the algorithm implementations has a negligible effect on prediction accuracy.
            Even for practical data, the \gls{DMDc} implementation is preferred over \gls{HAVOKc} for practical data due to lower computational complexity.

            
        \FloatBarrier\subsection{Hyperparameters}

            \paragraph
            As discussed in Section~\ref{sec:hyperparameters}, the prediction error generally improves for a higher number of delay-coordinates because the number of parameters in the model increases.
            However, the prediction error reaches a Pareto optimum, after which the error does not significantly decrease with an increasing number of terms.
            Figure~\ref{fig:havok_vs_dmd_q_2mps} shows the prediction error as a function of the number of delay-coordinates for practical flight data.
            Even though the Pareto elbow is not as smooth and clear as shown in Chapter~\ref{chap:system_id}, the elbow can still be identified.            
            % Note that the Pareto elbow is not as smooth and clear as shown in Section~\ref{sec:hyperparameters}.
            % This may be due to random 
            % The Pareto optimal models for practical data have significantly more delay-coordinates than

            \begin{figure}[htb]
    \centering
    \begin{tikzpicture}
        \begin{axis}[            
            xlabel = {Number of delay-coordinates, $q$},
            ylabel = $\overline{NMAE}_{mm}$ \phantom{~},
            % x unit = \si{\second},
            y unit = \%,
            xmin = 5,     xmax = 80,
            ymin = 3.2,  ymax = 7.5,
            grid = major,
            legend cell align = left,
            legend pos = north east,
            grid style = dashed,
            legend style = {font = \scriptsize},
            label style = {font = \scriptsize},
            tick label style = {font = \scriptsize},
            width = 0.95\columnwidth,
            height = 0.5\columnwidth,
            % initialize Dark2
            cycle list/Dark2,
            % combine it with 'mark list*':
            cycle multiindex* list = {
                Dark2\nextlist
            }
        ]

        \addplot+[mark = none, style = solid, ultra thick] 
        table[x = q, y expr = {\thisrow{NMAE_mean}*100}, col sep = comma] 
        {results/csv/NMAE_vs_q_Prac_2021-08-12_03_manual_x_vel_steps_2mps_q.csv_dmd_angle.csv};
        \addlegendentry{DMD}

        \addplot+[mark = none, style = solid, ultra thick] 
        table[x = q, y expr = {\thisrow{NMAE_mean}*100}, col sep = comma] 
        {results/csv/NMAE_vs_q_Prac_2021-08-12_03_manual_x_vel_steps_2mps_q.csv_havok_angle.csv};
        \addlegendentry{HAVOK}

        \end{axis}
    \end{tikzpicture} 
    
    \caption{DMD and HAVOK prediction errors for different number of delays included in the model
    ($m =$~\SI{0.206}{\kilo\gram}, $l =$~\SI{1}{\meter}, $T_s =$~\SI{0.03}{\second}).}
    \label{fig:havok_vs_dmd_q_2mps}
\end{figure}


            % \begin{figure}[htb]
    \centering
    \begin{tikzpicture}
        \begin{semilogyaxis}[            
            xlabel = Index of mode,
            ylabel = Singular value,
            % x unit = \si{\second},
            % y unit = \si{\second},
            xmin = 0,     xmax = 183,
            ymin = 1e-5,  ymax = 1e4,
            grid = major,
            legend cell align = left,
            legend pos = north east,
            grid style = dashed,
            legend style = {font = \scriptsize},
            label style = {font = \scriptsize},
            tick label style = {font = \scriptsize},
            width = 0.95\columnwidth,
            height = 0.5\columnwidth,
            % initialize Dark2
            cycle list/Dark2,
            % combine it with 'mark list*':
            cycle multiindex* list = {
                Dark2\nextlist
            }
        ]

        \addplot+[only marks, mark = *, ultra thin, mark options={scale=0.7}] 
        table[x = index, y = S, col sep = comma] 
        {results/csv/Singular_values_Prac_2021-08-12_03_manual_x_vel_steps_2mps_q.csv_havok_angle_Ttrain_50_q91_p52.csv};
        \addlegendentry{Significant modes}

        \addplot+[only marks, mark = *, ultra thin, mark options={scale=0.7}] 
        table[x = index, y = S, col sep = comma] 
        {results/csv/Singular_values_Prac_2021-08-12_03_manual_x_vel_steps_2mps_q.csv_havok_angle_Ttrain_50_q91_p52_trunc.csv};
        \addlegendentry{Truncated modes}

        \end{semilogyaxis}
    \end{tikzpicture} 
    
    \caption{Significant and truncated singular values of a \gls{HAVOKc} model produced from practical data
    ($m_p =$~\SI{0.2}{\kilo\gram}, $l =$~\SI{0.5}{\meter}, $T_s =$~\SI{0.03}{\second}, $T_{train} =$~\SI{60}{\second}.)}
    \label{fig:prac_singular_values}
\end{figure}

            % Difference between SITl and practical (same input steps)
            % This is due to wind disturbance
            % Plot wind vs less wind vs no wind
            
        \FloatBarrier\subsection{System parameters}

            \paragraph
            It was shown with multiple simulations in Section~\ref{sec:system_params} 
            that the system identification methods work for a range of different payload parameters.
            Figure~\ref{fig:prac_system_params} shows the prediction error for different payloads with practical data. 
            This shows that the proposed methods also work in practice with different payload configurations.
            The `double-descent' trend (discussed in Section~\ref{sec:sys_id_length_of_data}) is also seen in the practical data results where the prediction error increases slightly after a specific length of training data.            
            
            \begin{figure}[htb]
    \centering
    \begin{tikzpicture}
        \begin{axis}[            
            xlabel = Length of training data,
            ylabel = $\overline{NMAE}$ \phantom{~},
            x unit = \si{\second},
            y unit = \%,
            xmin = 5,     xmax = 120,
            ymin = 2.8,  ymax = 7.8,
            grid = major,
            legend cell align = left,
            legend pos = north east,
            grid style = dashed,
            legend style = {font = \scriptsize},
            label style = {font = \scriptsize},
            tick label style = {font = \scriptsize},
            width = 0.95\columnwidth,
            height = 0.5\columnwidth,
            % initialize Dark2
            cycle list/Dark2,
            % combine it with 'mark list*':
            cycle multiindex* list = {
                Dark2\nextlist
            }
        ]
         
        \addplot+[mark = none, style = solid, ultra thick] 
        table[x = T_train, y expr = {\thisrow{NMAE_mean}*100}, col sep = comma] 
        {results/csv/NMAE_vs_Ntrain_Prac_2021-08-20_01_l-1_mp-0.1_wind-0.5.csv_dmd_angle.csv};
        \addlegendentry{$m_p =$~\SI{0.1}{\kilo\gram}, $l = $~\SI{1}{\metre}}
        
        \addplot+[mark = none, style = solid, ultra thick] 
        table[x = T_train, y expr = {\thisrow{NMAE_mean}*100}, col sep = comma] 
        {results/csv/NMAE_vs_Ntrain_Prac_2021-08-20_02_l-2_mp-0.3-wind-0.5.csv_dmd_angle.csv};
        \addlegendentry{$m_p =$~\SI{0.3}{\kilo\gram}, $l = $~\SI{2}{\metre}}
        
        \addplot+[mark = none, style = solid, ultra thick] 
        table[x = T_train, y expr = {\thisrow{NMAE_mean}*100}, col sep = comma] 
        {results/csv/NMAE_vs_Ntrain_Prac_2021-08-23_01_l-0.5_mp-0.2_wind-0.5.csv_dmd_angle.csv};
        \addlegendentry{$m_p =$~\SI{0.2}{\kilo\gram}, $l = $~\SI{0.5}{\metre}}
        
        \addplot+[mark = none, style = solid, ultra thick] 
        table[x = T_train, y expr = {\thisrow{NMAE_mean}*100}, col sep = comma] 
        {results/csv/NMAE_vs_Ntrain_Prac_2021-08-20_03_l-1_mp-0.2_wind-0.5.csv_dmd_angle.csv};
        \addlegendentry{$m_p =$~\SI{0.2}{\kilo\gram}, $l = $~\SI{1}{\metre}}
        
        \end{axis}
    \end{tikzpicture} 

    \caption{\gls{DMDc} prediction error as a function of training data length for different payload parameters.}
    \label{fig:prac_system_params}
\end{figure}


            \paragraph
            Recall that the models producing these predictions do not use a~priori information about the plant. 
            Only input and output measurements are used in the model generation. 
            In contrast to the white-box technique, the effect of system parameters such as 
            multirotor mass, 
            payload mass, 
            cable length, 
            and damping coefficients 
            are inherently included in the estimated model.
            Therefore these parameters can all be varied 
            and the system identification algorithm should still be able to determine a prediction model of the system.

        \FloatBarrier\subsection{State predictions}

            \paragraph
            Figure~\ref{fig:prac_pediction_single_pend_theta_black} 
            shows the measured and predicted payload angle data of a suspended payload for a velocity step in a practical flight.
            Note that the model is generated from training data and is tested with a separate set of previously unseen, testing data.
            Figure~\ref{fig:prac_pediction_single_pend_theta_black} plots the state prediction against testing data and it is clear that the prediction data fits the measured data very well.
            
            \paragraph
            Recall from Section~\ref{sec:dmdc} that \gls{DMDc} produces a discrete, state-space model in the form:
            \begin{equation}
                \bm{x}_{k+1} = \bm{A}_{dmd} \bm{x}_k + \bm{A}_d \bm{d}_k + \bm{B}_{dmd} \bm{u}_k .
            \end{equation}
            The state prediction starts at the initial condition, $\bm{x}_0$ and $\bm{d}_0$,
            and predicts the state vector for each successive time-step, $\bm{x}_{k+1}$, 
            from the state vector, $\bm{x}_{k}$,
            delay vector, $\bm{d}_{k}$
            and input vector, $\bm{u}_{k}$
            at the previous time-step.
            Each of these time-step predictions results in a small error that accumulates with each successive time-step.
            Therefore the prediction error increases as the prediction horizon increases, 
            as shown in Figure~\ref{fig:prac_pediction_single_pend_theta_black}.

            \begin{figure}[htb]
    \centering
    \begin{tikzpicture}
        \begin{axis}[            
            xlabel = Time,
            ylabel = Payload angle,
            x unit = \si{\second},
            y unit = \si{\degree},
            xmin = 0,   xmax = 20,
            ymin = -20,  ymax = 20,
            grid = major,
            legend cell align = left,
            legend pos = north east,
            grid style = dashed,
            legend style = {font = \scriptsize},
            label style = {font = \scriptsize},
            tick label style = {font = \scriptsize},
            width = 0.95\columnwidth,
            height = 0.5\columnwidth,
            % initialize Dark2
            cycle list/Dark2,
            % combine it with 'mark list*':
            cycle multiindex* list = {
                Dark2\nextlist
            }
        ]
        
        \addplot+[mark = none, style = solid, ultra thick] 
        table[x = time, y = theta, col sep = comma] 
        {results/csv/step_predictions_Prac_2021-08-20_02_l-2_mp-0.3-wind-0.5.csv_dmd_737.csv};
        \addlegendentry{Measured}

        \addplot+[mark = none, style = dashed, ultra thick] 
        table[x = time, y = theta_hat, col sep = comma] 
        {results/csv/step_predictions_Prac_2021-08-20_02_l-2_mp-0.3-wind-0.5.csv_dmd_737.csv};
        \addlegendentry{DMDc prediction}

        \end{axis}
    \end{tikzpicture} 
    
    \caption{Model predictions of practical flight data with an suspended payload for a North velocity step input
        ($l =$~\SI{2}{\meter}, $m_p =$~\SI{0.3}{\kilo\gram})}
    \label{fig:prac_pediction_single_pend_theta_black}
\end{figure}


            \paragraph
            Figure~\ref{fig:prac_pediction_single_pend_vel_black} 
            shows the measured and predicted North velocity of the same flight.
            The oscillations in the velocity response due to the swinging payload are visible in this plot.
            The model predicts the frequency and size of these oscillations reasonably well.
            Note that predicting the payload angle is significantly easier than predicting the velocity response.
            The payload angle prediction inherently oscillates around a zero mean.
            However, the velocity response has a non-zero mean and depends on numerical integration of the acceleration setpoint data.
            A slight error in the correction of the setpoint offset 
            (discussed in Section~\ref{sec:noise} and Section~\ref{sec:length_train_data_prac})
            may result in a large error in the velocity prediction due to a build-up of integration error.
            However, despite this challenge, the model accurately predicts the velocity step size of the practical data.

            \begin{figure}[htb]
    \centering
    \begin{tikzpicture}
        \begin{axis}[            
            xlabel = Time,
            ylabel = North velocity,
            x unit = \si{\second},
            y unit = \si{\metre/\second},
            xmin = 0,   xmax = 20,
            ymin = -2,  ymax = 1.5,
            grid = major,
            legend cell align = left,
            legend pos = north east,
            grid style = dashed,
            legend style = {font = \scriptsize},
            label style = {font = \scriptsize},
            tick label style = {font = \scriptsize},
            width = 0.95\columnwidth,
            height = 0.5\columnwidth,
            % initialize Dark2
            cycle list/Dark2,
            % combine it with 'mark list*':
            cycle multiindex* list = {
                Dark2\nextlist
            }
        ]
        
        \addplot+[mark = none, style = solid, ultra thick] 
        table[x = time, y = vel, col sep = comma] 
        {results/csv/step_predictions_Prac_2021-08-20_02_l-2_mp-0.3-wind-0.5.csv_dmd_737.csv};
        \addlegendentry{Measured}

        \addplot+[mark = none, style = dashed, ultra thick] 
        table[x = time, y = vel_hat, col sep = comma] 
        {results/csv/step_predictions_Prac_2021-08-20_02_l-2_mp-0.3-wind-0.5.csv_dmd_737.csv};
        \addlegendentry{DMDc prediction}

        \end{axis}
    \end{tikzpicture} 
    
    \caption{Model predictions of practical flight data with a suspended payload for a North velocity step input
        ($l =$~\SI{2}{\meter}, $m_p =$~\SI{0.3}{\kilo\gram}, wind speed $\approx~$\SI{0.5}{\metre/\second})}
    \label{fig:prac_pediction_single_pend_vel_black}
\end{figure}


        \FloatBarrier\subsection{Extended dimensions}

            \paragraph
            Because the data-driven methods are only dependant on input and output data,
            the prediction model can easily be extended to include more dimensions by adding more state measurement variables.
            In this section, a prediction model is generated and discussed which includes both the North and East axes dynamics.
            Such a model could be used in a single \gls{MPC} velocity controller to damp the payload oscillations in both axes simultaneously.

            \paragraph
            For this model, the state vector is,
            \begin{equation}
                \bm{x} = \begin{bmatrix}
                    V_N & V_E & \theta_N & \theta_E 
                \end{bmatrix}^T ,
            \end{equation}
            and the corresponding input vector is,
            \begin{equation}
                \bm{u} = \begin{bmatrix}
                    A_{N_{sp}} & A_{E_{sp}}
                \end{bmatrix} .
            \end{equation}
            
            \begin{figure}
                \captionsetup[subfigure]{justification=centering}
                \centering
                \begin{subfigure}[t]{\columnwidth}
    \centering
    \begin{tikzpicture}
        \begin{axis}[            
            xlabel = Time,
            ylabel = Velocity,
            x unit = \si{\second},
            y unit = \si{\metre/\second},
            xmin = 0,   xmax = 75,
            ymin = -1.5,  ymax = 1.5,
            grid = major,
            legend cell align = left,
            legend pos = north east,
            grid style = dashed,
            legend style = {font = \scriptsize},
            label style = {font = \scriptsize},
            tick label style = {font = \scriptsize},
            width = 0.95\columnwidth,
            height = 0.3\columnwidth,
            % initialize Dark2
            cycle list/Dark2,
            % combine it with 'mark list*':
            cycle multiindex* list = {
                Dark2\nextlist
            }
        ]
        
        \addplot+[mark = none, style = solid, thick] 
        table[x = time, y = vel_sp.x, col sep = comma] 
        {results/csv/training_data_Prac_2021-08-23_02_l-0.5_mp-0.2_wind-0.5_XY_steps.csv.csv};
        \addlegendentry{$V_{N_{sp}}$}
        
        \addplot+[mark = none, style = solid, thick] 
        table[x = time, y = vel.x, col sep = comma] 
        {results/csv/training_data_Prac_2021-08-23_02_l-0.5_mp-0.2_wind-0.5_XY_steps.csv.csv};
        \addlegendentry{$V_N$}

        \addplot+[mark = none, style = solid, thick] 
        table[x = time, y = vel_sp.y, col sep = comma] 
        {results/csv/training_data_Prac_2021-08-23_02_l-0.5_mp-0.2_wind-0.5_XY_steps.csv.csv};
        \addlegendentry{$V_{E_{sp}}$}
        
        \addplot+[mark = none, style = solid, thick] 
        table[x = time, y = vel.y, col sep = comma] 
        {results/csv/training_data_Prac_2021-08-23_02_l-0.5_mp-0.2_wind-0.5_XY_steps.csv.csv};
        \addlegendentry{$V_E$}

        \end{axis}
    \end{tikzpicture} 
    
    \label{fig:training_data_vel_prac}
\end{subfigure}
 % subfigure
                \begin{subfigure}[t]{\columnwidth}
    \centering
    \begin{tikzpicture}
        \begin{axis}[            
            xlabel = Time,
            ylabel = Payload angle,
            x unit = \si{\second},
            y unit = \si{\radian},
            xmin = 0,   xmax = 75,
            ymin = -20,  ymax = 20,
            grid = major,
            legend cell align = left,
            legend pos = north east,
            grid style = dashed,
            legend style = {font = \scriptsize},
            label style = {font = \scriptsize},
            tick label style = {font = \scriptsize},
            width = 0.95\columnwidth,
            height = 0.3\columnwidth,
            % initialize Dark2
            cycle list/Dark2,
            % combine it with 'mark list*':
            cycle multiindex* list = {
                Dark2\nextlist
            }
        ]
        \pgfplotsset{cycle list shift=1}

        \addplot+[mark = none, style = solid, ultra thick] 
        table[x = time, y = theta.x, col sep = comma] 
        {results/csv/training_data_Prac_2021-08-23_02_l-0.5_mp-0.2_wind-0.5_XY_steps.csv.csv};
        \addlegendentry{$\theta_N$}
        
        \pgfplotsset{cycle list shift=1}

        \addplot+[mark = none, style = solid, ultra thick] 
        table[x = time, y = theta.y, col sep = comma] 
        {results/csv/training_data_Prac_2021-08-23_02_l-0.5_mp-0.2_wind-0.5_XY_steps.csv.csv};
        \addlegendentry{$\theta_E$}

        \end{axis}
    \end{tikzpicture} 
    
    \label{fig:training_data_theta_prac}
\end{subfigure}
 % subfigure
                \caption{Snapshot of training data with random velocity step inputs for the North and East axes
                    ($m_p =$~\SI{0.2}{\kilo\gram}, $l =$~\SI{0.5}{\meter}).}
                \label{fig:XY_train_data_subfigs}  
            \end{figure}

            \paragraph
            To generate training data, random steps are commanded in the North and East axes simultaneously to excite the dynamics in both axes. 
            Figure~\ref{fig:XY_train_data_subfigs} shows an example of the practical training data used for this extended dimension model.
            Clear oscillations in the payload angle and velocity response of both axes are visible. 

            \paragraph
            Figure~\ref{fig:xy_predictions} shows the state variable predictions of a \gls{DMDc} model 
            built from the data in Figure~\ref{fig:XY_train_data_subfigs}.
            It is clear that this model provides an accurate prediction of each state variable considered.
            This shows that the data-driven methods can be effectively extended to include both the North and East axes dynamics. 
            This is a great advantage of the proposed data-driven approach.
            The model is easily adapted for different use cases without redesigning estimation techniques or remodelling the plant manually.

            \begin{figure}
                \captionsetup[subfigure]{justification=centering}
                \centering
                \input{results/plots/prac_pediction_xy_theta_x.tex} % subfigure
                \begin{subfigure}{\columnwidth}
    \centering
    \begin{tikzpicture}
        \begin{axis}[            
            xlabel = Time,
            ylabel = East axis payload angle,
            x unit = \si{\second},
            y unit = \si{\degree},
            xmin = 0,   xmax = 20,
            ymin = -20,  ymax = 20,
            grid = major,
            legend cell align = left,
            legend pos = north east,
            grid style = dashed,
            legend style = {font = \scriptsize},
            label style = {font = \scriptsize},
            tick label style = {font = \scriptsize},
            width = 0.95\columnwidth,
            height = 0.3\columnwidth,
            % initialize Dark2
            cycle list/Dark2,
            % combine it with 'mark list*':
            cycle multiindex* list = {
                Dark2\nextlist
            }
        ]
        
        \addplot+[mark = none, style = solid, ultra thick] 
        table[x = time, y = theta_y, col sep = comma] 
        {results/csv/step_predictions_Prac_2021-08-23_02_l-0.5_mp-0.2_wind-0.5_XY_steps.csv_dmd_746.csv};
        
        \addplot+[mark = none, style = dashed, ultra thick] 
        table[x = time, y = theta_y_hat, col sep = comma] 
        {results/csv/step_predictions_Prac_2021-08-23_02_l-0.5_mp-0.2_wind-0.5_XY_steps.csv_dmd_746.csv};
        
        \end{axis}
    \end{tikzpicture} 
    
\end{subfigure}
 % subfigure
                \begin{subfigure}{\columnwidth}
    \centering
    \begin{tikzpicture}
        \begin{axis}[            
            xlabel = Time,
            ylabel = North velocity,
            x unit = \si{\second},
            y unit = \si{\metre/\second},
            xmin = 0,   xmax = 20,
            ymin = -2,  ymax = 2,
            grid = major,
            legend cell align = left,
            legend pos = north east,
            grid style = dashed,
            legend style = {font = \scriptsize},
            label style = {font = \scriptsize},
            tick label style = {font = \scriptsize},
            width = 0.95\columnwidth,
            height = 0.3\columnwidth,
            % initialize Dark2
            cycle list/Dark2,
            % combine it with 'mark list*':
            cycle multiindex* list = {
                Dark2\nextlist
            }
        ]
        
        \addplot+[mark = none, style = solid, ultra thick] 
        table[x = time, y = vel_x, col sep = comma] 
        {results/csv/step_predictions_Prac_2021-08-23_02_l-0.5_mp-0.2_wind-0.5_XY_steps.csv_dmd_746.csv};
        
        \addplot+[mark = none, style = dashed, ultra thick] 
        table[x = time, y = vel_x_hat, col sep = comma] 
        {results/csv/step_predictions_Prac_2021-08-23_02_l-0.5_mp-0.2_wind-0.5_XY_steps.csv_dmd_746.csv};
        
        \end{axis}
    \end{tikzpicture} 
    
\end{subfigure}
 % subfigure
                \begin{figure}[htb]
    \centering
    \begin{tikzpicture}
        \begin{axis}[            
            xlabel = Time,
            ylabel = North velocity,
            x unit = \si{\second},
            y unit = \si{\metre/\second},
            xmin = 0,   xmax = 20,
            ymin = -2.5,  ymax = 2.5,
            grid = major,
            legend cell align = left,
            legend pos = north east,
            grid style = dashed,
            legend style = {font = \scriptsize},
            label style = {font = \scriptsize},
            tick label style = {font = \scriptsize},
            width = 0.95\columnwidth,
            height = 0.5\columnwidth,
            % initialize Dark2
            cycle list/Dark2,
            % combine it with 'mark list*':
            cycle multiindex* list = {
                Dark2\nextlist
            }
        ]
        
        \addplot+[mark = none, style = solid, ultra thick] 
        table[x = time, y = vel_y, col sep = comma] 
        {results/csv/step_predictions_Prac_2021-08-23_02_l-0.5_mp-0.2_wind-0.5_XY_steps.csv_dmd_746.csv};
        \addlegendentry{Measured}

        \addplot+[mark = none, style = solid, ultra thick] 
        table[x = time, y = vel_y_hat, col sep = comma] 
        {results/csv/step_predictions_Prac_2021-08-23_02_l-0.5_mp-0.2_wind-0.5_XY_steps.csv_dmd_746.csv};
        \addlegendentry{DMD prediction}

        \end{axis}
    \end{tikzpicture} 
    
    \caption{Practical flight data and model predictions for the North and East directions
        ($m_p =$~\SI{0.2}{\kilo\gram}, $l =$~\SI{0.5}{\meter})}
    \label{fig:prac_pediction_xy_vel_y}
\end{figure}
 % subfigure
                \caption{Data-driven predictions of practical data for a model with both North and East axis dynamics.}
                \label{fig:xy_predictions}  
            \end{figure}

        \FloatBarrier\subsection{Dynamic payload} 

            \paragraph
            As mentioned in Section~\ref{sec:dynamic_payload_prac}, 
            one of the disadvantages of the white-box system identification approach
            is that it relies heavily on a~priori modelling assumptions.
            When a payload is attached to the multirotor that deviates from the white-box modelling assumptions, 
            the performance of the system identification method decreases.
            However, the data-driven methods can handle this deviation
            because it does not rely on these assumptions.
                        
            \input{results/plots/prac_pediction_double_pend_theta_black.tex}

            \paragraph
            Figure~\ref{fig:prac_pediction_double_pend_theta_black} shows the measured and predicted payload angle
            of the practical dynamic payload.
            The irregular oscillations due to the double pendulum action of an elongated payload are visible in the angle data.
            It is clear that the data-driven model represents the actual payload dynamics for unseen testing data well.
            It appears that the prediction differs slightly from the measurement data at the peaks, however, this does not appear to be a significant error.
            Overall, it is clear that the \gls{DMDc} model captures the multi-frequency oscillations of the dynamic payload well.

            \begin{figure}[htb]
    \centering
    \begin{tikzpicture}
        \begin{axis}[            
            xlabel = Time,
            ylabel = North velocity,
            x unit = \si{\second},
            y unit = \si{\metre/\second},
            xmin = 0,   xmax = 20,
            ymin = -2.5,  ymax = 2.5,
            grid = major,
            legend cell align = left,
            legend pos = north east,
            grid style = dashed,
            legend style = {font = \scriptsize},
            label style = {font = \scriptsize},
            tick label style = {font = \scriptsize},
            width = 0.95\columnwidth,
            height = 0.5\columnwidth,
            % initialize Dark2
            cycle list/Dark2,
            % combine it with 'mark list*':
            cycle multiindex* list = {
                Dark2\nextlist
            }
        ]
        
        \addplot+[mark = none, style = solid, ultra thick] 
        table[x = time, y = vel, col sep = comma] 
        {results/csv/step_predictions_Prac_2021-08-23_04_double_pend_m1_0.2_m2_0.1_l1-0.5_l2_0.62_wind-0.5.csv_dmd_201.csv};
        \addlegendentry{Measured}

        \addplot+[mark = none, style = dashed, ultra thick] 
        table[x = time, y = vel_hat, col sep = comma] 
        {results/csv/step_predictions_Prac_2021-08-23_04_double_pend_m1_0.2_m2_0.1_l1-0.5_l2_0.62_wind-0.5.csv_dmd_201.csv};
        \addlegendentry{DMDc prediction}

        % \addplot+[mark = none, style = solid, ultra thick] 
        % table[x = time, y = theta, col sep = comma] 
        % {results/csv/step_predictions_Prac_2021-08-23_04_double_pend_m1_0.2_m2_0.1_l1-0.5_l2_0.62_wind-0.5.csv_dmd_201.csv};
        % \addlegendentry{White-box model}

        \end{axis}
    \end{tikzpicture} 
    
    \caption{Practical flight data and model predictions with an elongated payload for a North velocity step input
        ($m_1 =$~\SI{0.2}{\kilo\gram}, $l_1 =$~\SI{0.5}{\meter}, $m_2 =$~\SI{0.1}{\kilo\gram}, $l_2 =$~\SI{0.6}{\meter}).}
    \label{fig:prac_pediction_double_pend_vel_black}
\end{figure}

            
            \paragraph
            Figure~\ref{fig:prac_pediction_double_pend_vel_black} shows the measured and predicted velocity of the same flight.
            The superimposed frequencies are not as visible in the velocity oscillations as they were in the payload angle data.
            However, the oscillations in the velocity response still appear irregular compared to the simple payload data shown in Figure~\ref{fig:prac_pediction_single_pend_vel_black}.
            In Figure~\ref{fig:prac_pediction_double_pend_vel_black}, 
            the size of the velocity prediction deviates from the measurement data at the velocity overshoot.
            However, this error does not appear significant enough to affect the corresponding \gls{MPC} controller.
            The shape of the velocity oscillations also appears to be captured well in the prediction.
            
            \paragraph
            Recall that the prediction is propagated from an initial condition using the given input data only.
            The model does not use state measurements to readjust after the initial condition is taken.
            Therefore an accumulation in prediction error is expected as the prediction horizon increases.
            Because the model prediction matches the shape of the practical testing data so closely,
            it is expected that this model can be used for a practical \gls{MPC} implementation on practical data. 

            % As discussed in Chapter~\ref{chap:control}, 
            % this prediction model can be used in an \gls{MPC} to optimise the control input to actively damp the payload swing angles.
            % It was also noted in Chapter~\ref{chap:control} that the accuracy of the model 
            % has a direct impact on the performance of the controller.
            % Because the model prediction matches the practical testing data so closely,
            % it is expected that this model will 
            
    \FloatBarrier\section{Hardware-in-the-Loop simulations}

        \paragraph
        In Section~\ref{sec:data_driven_sys_id_prac} it was shown that \gls{DMDc} can generate accurate state prediction models from practical flight data.
        It was also shown in Section~\ref{sec:mpc} that the \gls{DMDc} models can be used in an \gls{MPC} for effective swing-damping control.
        A \gls{HITL} simulation can now be used to test the complete control architecture with the final software running on the actual hardware.

        \paragraph
        As described in Section~\ref{sec:exp_design_hitl}, an \gls{OBC} runs a \gls{ROS} node which implements the \gls{MPC} algorithm.
        The \gls{MPC} is based on a \gls{DMDc} plant model generated from training data from a \gls{HITL} simulation.
        The \gls{OBC} receives state feedback and sends control signals to the \gls{FC}.
        The \gls{FC} runs the actual PX4 flight-stack firmware and executes the control signals received from the \gls{MPC} node.
        Sensor values are generated and sent to PX4 by the Gazebo simulator, which runs on a desktop computer connected to the \gls{FC} via \gls{USB}.
        In this way, we can safely determine whether the practical hardware is suitable for the computational complexity of the control algorithms.
        
        \paragraph
        In this section, the results of different \gls{HITL} simulations will be shown and discussed.
        The effect of the \gls{MPC} sample time on the \gls{CPU} consumption will also be investigated.
        Finally, it will be shown that an \gls{MPC} node, which is based on a data-driven system identification model, effectively implements swing-damping control in a \gls{HITL} simulation.

        \FloatBarrier\subsection{Effect of hyperparameters on computational requirements}    

            \begin{figure}[htb]
    \centering
    \begin{tikzpicture}
        \begin{axis}[            
            xlabel = {Number of delay-coordinates, $q$},
            ylabel = \%CPU,
            % x unit = \si{\second},
            % y unit = \%,
            xmin = 0,   xmax = 80,
            ymin = 0,   ymax = 100,
            grid = major,
            legend cell align = left,
            legend pos = north east,
            grid style = dashed,
            legend style = {font = \scriptsize},
            label style = {font = \scriptsize},
            tick label style = {font = \scriptsize},
            width = 0.95\columnwidth,
            height = 0.5\columnwidth,
            % initialize Dark2
            cycle list/Dark2,
            % combine it with 'mark list*':
            cycle multiindex* list = {
                Dark2\nextlist
            }
        ]

        \addplot+[mark = *, style = solid, ultra thick] 
        table {
            10  22.9
            30  51.1
            50  82.8
            70  98.7
        };

        \end{axis}
    \end{tikzpicture} 
    
    \caption{Maximum \%CPU used by the MPC node for different values of $q$ ($Ts =$~\SI{0.03}{\second})}
    \label{fig:CPU_vs_q}
\end{figure}


            \paragraph
            As discussed in Section~\ref{sec:mpc}, the computational complexity of the \gls{MPC} optimisation problem increases for larger state-space matrices, and the hyperparameter, $q$, determines the size of these matrices.
            Figure~\ref{fig:CPU_vs_q} shows the \scaleobj{0.8}{\%}CPU consumption of \gls{MPC} nodes based on models with different $q$ values.
            The \scaleobj{0.8}{\%}CPU value represents the average percentage of \gls{CPU} time used by the \gls{MPC} node running on the \gls{OBC}.
            From Figure~\ref{fig:CPU_vs_q} it is clear that \scaleobj{0.8}{\%}CPU increases with increasing values of $q$.
            This shows that the computational complexity increases for larger values of $q$.
            % This shows that the optimisation problem increases in complexity since more time is required to solve 

            \paragraph
            In Section~\ref{sec:hyperparameters}, it was shown that $q = 50$ is near the Pareto optimum for the practical data and the prediction accuracy does not increase significantly for $q > 50$.
            As shown in Figure~\ref{fig:CPU_vs_q}, an \gls{MPC} node with $q = 50$ and $T_s =$~\SI{0.03}{\second} run at \scaleobj{0.8}{\%}CPU~$= 82.8\%$ on the \gls{OBC}.
            The status of the \gls{QP} solver was also monitored, which showed that the optimisation problem was consistently solved within the given optimisation time for $q = 50$.
            The \gls{MPC} node achieves stable, swing-damping control with this model.
            However, for $q = 70$ the \gls{MPC} node uses \scaleobj{0.8}{\%}CPU~$= 98.7\%$ and the optimisation problem can not be solved fast enough for stable multirotor control.
            This results in an unstable controller and the multirotor-payload system crashes consistently with this \gls{MPC} node. 

            \begin{figure}[htb]
    \centering
    \begin{tikzpicture}
        \begin{axis}[            
            xlabel = {Number of delay-coordinates, $q$},
            ylabel = \%~RAM,
            % x unit = \si{\second},
            % y unit = \%,
            xmin = 0,   xmax = 80,
            ymin = 0,   ymax = 1.2,
            grid = major,
            legend cell align = left,
            legend pos = north east,
            grid style = dashed,
            legend style = {font = \scriptsize},
            label style = {font = \scriptsize},
            tick label style = {font = \scriptsize},
            width = 0.95\columnwidth,
            height = 0.5\columnwidth,
            % initialize Dark2
            cycle list/Dark2,
            % combine it with 'mark list*':
            cycle multiindex* list = {
                Dark2\nextlist
            }
        ]

        \pgfplotsset{cycle list shift=1} % So that RAM and CPU does not have same colour

        \addplot+[mark = *, style = solid, ultra thick] 
        table {
            10  0.4
            30  0.7
            50  0.8
            70  0.9
        };

        \end{axis}
    \end{tikzpicture} 
    
    \caption{Maximum \% RAM used by the MPC node for different values of $q$.}
    \label{fig:RAM_vs_q}
\end{figure}


            \paragraph
            Figure~\ref{fig:RAM_vs_q} shows that the \scaleobj{0.8}{\%}RAM also increases for larger values of $q$.
            The \scaleobj{0.8}{\%}RAM value represents the maximum percentage of \gls{RAM} space used by the \gls{MPC} node while running on the \gls{OBC}.
            It is clear that the \gls{OBC} has sufficient memory to handle the \gls{MPC}, since the \gls{MPC} node uses less than 1\% RAM, even for large models with $q = 70$.

        \FloatBarrier\subsection{Effect of sample time on computational requirements}    

            \begin{figure}[htb]
    \centering
    \begin{tikzpicture}
        \begin{axis}[            
            xlabel = {Sample time, $T_s$},
            ylabel = \%~CPU,
            x unit = \si{\milli\second},
            % y unit = \%,
            xmin = 15,   xmax = 45,
            ymin = 0,   ymax = 100,
            grid = major,
            legend cell align = left,
            legend pos = north east,
            grid style = dashed,
            legend style = {font = \scriptsize},
            label style = {font = \scriptsize},
            tick label style = {font = \scriptsize},
            width = 0.95\columnwidth,
            height = 0.5\columnwidth,
            % initialize Dark2
            cycle list/Dark2,
            % combine it with 'mark list*':
            cycle multiindex* list = {
                Dark2\nextlist
            }
        ]

        \addplot+[mark = *, style = solid, ultra thick] 
        table {
            20  98.6
            30  82.8
            40  40.2
        };

        \end{axis}
    \end{tikzpicture} 
    
    \caption{Maximum \% CPU used by the MPC node for different sample times ($q = 50$).}
    \label{fig:CPU_vs_Ts}
\end{figure}


            \paragraph
            The sample time of the controller also affects the \scaleobj{0.8}{\%}CPU consumption of the \gls{MPC} node.
            Figure~\ref{fig:CPU_vs_Ts} shows the measured \scaleobj{0.8}{\%}CPU of \gls{MPC} nodes with different sample times using a plant model with $q = 50$. 
            It is clear that \scaleobj{0.8}{\%}CPU decreases as $T_s$ increases.
            % In Section~\ref{sec:sample_time}, it was shown that the model prediction accuracy does not change much for different sample times in the range, $20 < q < 40$~ms.
            The default velocity controller in PX4 runs at \SI{50}{\hertz}, which corresponds to $T_s =$~\SI{20}{\milli\second}.
            However, an \gls{MPC} node with $T_s =$~\SI{20}{\milli\second} struggles to run fast enough on the given \gls{OBC}.
            This is clear from the high \scaleobj{0.8}{\%}CPU consumption (98.6\%) of the \gls{MPC} node running at $T_s =$~\SI{20}{\milli\second}.
            This node results in unstable control because the \gls{QP} problem cannot be solved within the given optimisation time.
            
            \paragraph
            An \gls{MPC} node with $T_s =$~\SI{40}{\milli\second} runs with a low \scaleobj{0.8}{\%}CPU, however it also results in unstable control.
            This is because the controller frequency is too low to provide adequate velocity control of the multirotor-payload dynamics.
            An \gls{MPC} node with $T_s =$~\SI{30}{\milli\second} provides stable control of the multirotor-payload system.
            The \gls{QP} problem is solved within the given optimisation time and, as shown in the section below, the resulting controller frequency of \SI{33.33}{\hertz} appears to be fast enough to control the multirotor-payload dynamics.
            
            \paragraph
            Note that the practical controller implementation experiences latency in various parts of the system.
            The latency can be attributed to various hardware communication channels and software execution times.
            \gls{HITL} simulations also determine whether the controller works despite the inherent latency of the system.

        \FloatBarrier\subsection{Velocity step response}    

            \begin{figure}[htb]
    \centering
    \begin{tikzpicture}
        \begin{axis}[            
            xlabel = Time,
            ylabel = North velocity,
            x unit = \si{\second},
            y unit = \si{\metre/\second},
            xmin = 0,   xmax = 15,
            ymin = 0.02,  ymax = 1.2,
            grid = major,
            legend cell align = left,
            legend pos = south east,
            grid style = dashed,
            legend style = {font = \scriptsize},
            label style = {font = \scriptsize},
            tick label style = {font = \scriptsize},
            width = 0.95\columnwidth,
            height = 0.5\columnwidth,
            % initialize Dark2
            cycle list/Dark2,
            % combine it with 'mark list*':
            cycle multiindex* list = {
                Dark2\nextlist
            }
        ]
        
        \addplot+[mark = none, style = solid, ultra thick] 
        table[x expr = \thisrow{time} - 7, y = vel_sp.x, col sep = comma] 
        {results/csv/HITL_step_PID_pid_single_step.csv};
        \addlegendentry{setpoint}

        \addplot+[mark = none, style = solid, ultra thick] 
        table[x expr = \thisrow{time} - 7, y = vel.x, col sep = comma] 
        {results/csv/HITL_step_PID_pid_single_step.csv};
        \addlegendentry{PID}

        \addplot+[mark = none, style = solid, ultra thick] 
        table[x expr = \thisrow{time} - 7, y = vel.x, col sep = comma] 
        {results/csv/HITL_step_MPC_mpc_great_responce.csv};
        \addlegendentry{MPC}

        \end{axis}
    \end{tikzpicture} 
    
    \caption{Velocity step responses of MPC and PID controllers for \gls{HITL} simulations
        ($q = 50$, $T_s =$\SI{0.03}{\second})}
    \label{fig:HITL_mpc_vel}
\end{figure}


            \paragraph
            Figure~\ref{fig:HITL_mpc_vel} plots the velocity responses of the \gls{PID} and \gls{MPC} controllers in \gls{HITL} simulations of the multirotor-payload system.
            The \gls{MPC} controller is based on a system identification model with $q = 50$ and a sample time of $T_s =$~\SI{30}{\milli\second}.
            It is clear from this plot that the \gls{MPC} node running on the \gls{OBC} provides stable control of the multirotor payload system and damps the velocity oscillations well.

            \begin{figure}[htb]
    \centering
    \begin{tikzpicture}
        \begin{axis}[            
            xlabel = Time,
            ylabel = Payload angle,
            x unit = \si{\second},
            y unit = \si{\degree},
            xmin = 0,   xmax = 15,
            ymin = -15,  ymax = 10,
            grid = major,
            legend cell align = left,
            legend pos = north east,
            grid style = dashed,
            legend style = {font = \scriptsize},
            label style = {font = \scriptsize},
            tick label style = {font = \scriptsize},
            width = 0.95\columnwidth,
            height = 0.5\columnwidth,
            % initialize Dark2
            cycle list/Dark2,
            % combine it with 'mark list*':
            cycle multiindex* list = {
                Dark2\nextlist
            }
        ]

        \pgfplotsset{cycle list shift=1} % So that MPC vs PID has consistent colour

        \addplot+[mark = none, style = solid, ultra thick] 
        table[x expr = \thisrow{time} - 7, y expr = \thisrow{angle.y} * 57.2958, col sep = comma] 
        {results/csv/HITL_step_PID_pid_single_step.csv};
        \addlegendentry{PID}

        \addplot+[mark = none, style = solid, ultra thick] 
        table[x expr = \thisrow{time} - 7, y expr = \thisrow{angle.y} * 57.2958, col sep = comma] 
        {results/csv/HITL_step_MPC_mpc_great_responce.csv};
        \addlegendentry{MPC}

        \end{axis}
    \end{tikzpicture} 
    
    \caption{Payload angle responses of MPC and PID controllers for \gls{HITL} simulations
        ($q = 50$, $T_s =$~\SI{0.03}{\second}).}
    \label{fig:HITL_mpc_theta}
\end{figure}


            \paragraph
            Figure~\ref{fig:HITL_mpc_theta} shows the payload angle data of the \gls{PID} and \gls{MPC} controllers during the velocity step response.
            From this plot, it can be seen that the \gls{MPC} controller damps the payload angle well.
            Overall, these plots show that the swing-damping control demonstrated in Chapter~\ref{chap:control}, is also achievable with the final controller software running on the actual hardware.
            This shows that the practical hardware is suitable for the computational complexity of the control algorithms and the controller architecture is practically feasible.

    \FloatBarrier\section{Summary}

        \paragraph
        In this chapter, the system identification techniques were applied to practical flight data.
        The cable length estimation technique described in Chapter~\ref{chap:system_id} was applied to data from flights with different payload masses and cable lengths.
        The results showed that the estimated length converged quickly for each payload configuration and that the estimation error was consistently quite large.
        The large error was attributed to the cable attachment being below the vehicle \gls{CoM}, and the damping effect of the controller, which were not considered in the parameter estimation algorithm. 
        
        \paragraph
        It was also shown that the estimated length converged for flight data with different wind conditions and that the estimation error was the largest for the data with the most wind.
        Furthermore, the cable length estimator was applied to flight data with a double pendulum payload.
        The estimated value converged to a length corresponding to the dominant frequency in the data, and the resulting white-box model prediction only represented the general shape of the data.
        The white-box model prediction was also shown to be very sensitive to initial conditions.

        \paragraph
        The data-driven system identification techniques were also successfully applied to practical flight data.
        As expected, the prediction error of the techniques decreased for flights with lower wind speeds.
        With a reasonable amount of wind, the techniques were able to generate accurate prediction models.
        However, at very high wind speeds, the resultant models were not usable.

        \paragraph
        The \gls{DMDc} and \gls{HAVOKc} state predictions performed equally well on data from various flights.
        \gls{DMDc} was chosen as the preferred method due to its lower complexity.
        It was shown that \gls{DMDc} could be applied to data from a wide range of different payload parameters and that it consistently generated accurate prediction models without prior knowledge of the payload dynamics.
        The Pareto front and `double-descent' trends discussed in Chapter~\ref{chap:system_id} were also identified in these results.
        
        \paragraph
        \gls{DMDc} was extended to capture the North and East velocity dynamics simultaneously, and the resulting prediction model accurately reconstructed the system dynamics for test data.
        Furthermore, \gls{DMDc} was applied to flight data with a dynamic payload representing a double pendulum system.
        \gls{DMDc} accurately reconstructed the dynamics of this system for unseen test data.
        Unlike the white-box model which only represented the dominant oscillation frequency, the \gls{DMDc} model also predicted the irregular oscillations well.

        \paragraph
        Finally, the full data-driven system identification with \gls{MPC} control architecture was tested in \gls{HITL} simulations.
        It was shown that the hardware successfully executes the \gls{MPC} algorithm at the desired frequency for the selected hyperparameters and sample time.
        The successful velocity step response proved that the various software systems work together seamlessly.
        It also showed that the control architecture can be implemented on the hardware of an actual multirotor for effective swing-damping control of the multirotor-payload system.
            
}