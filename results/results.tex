\graphicspath{{results/fig/}}

\chapter{Implementation and results}
\label{chap:results}

    \section{Parameter estimation with practical data}
        \subsection{Mass estimation}
        \subsection{Simple payload cable length estimation}
        \subsection{Complex payload cable length estimation}

    \section{Data-driven system identification with practical data}

        \FloatBarrier\subsection{Methodology}
            \begin{itemize}
                \item Method for generating data
                \item Plot example of training data
                \item picture of practical flight single and double
                \item Discuss difference between SITL and prac
                \begin{itemize}
                    \item Noise 
                    \item wind
                    \item CoM
                \end{itemize}
                \item plot hover of prac vs SITL to show noise and disturbance
            \end{itemize}

            \paragraph
            In Chapter~\ref{chap:system_id} it was noted that the optimal length of training data 
            often only included 2 velocity steps responses.
            This means the models were trained on a very small sample of step sizes 
            and need to extrapolate the dynamics of other step sizes.
        
            
        \FloatBarrier\subsection{Length of training data}
            plot NMAE vs Ttrain
            \begin{itemize}
                \item Plot strong wind vs less wind vs no wind
                \item HAVOK vs DMD
                \item Plot single pend vs double pend
            \end{itemize}

            \begin{figure}[htb]
    \centering
    \begin{tikzpicture}
        \begin{axis}[            
            xlabel = Length of training data,
            ylabel = $\overline{NMAE}$ \phantom{~},
            x unit = \si{\second},
            y unit = \%,
            xmin = 5,     xmax = 120,
            ymin = 3.2,  ymax = 15,
            grid = major,
            legend cell align = left,
            legend pos = north east,
            grid style = dashed,
            legend style = {font = \scriptsize},
            label style = {font = \scriptsize},
            tick label style = {font = \scriptsize},
            width = 0.95\columnwidth,
            height = 0.5\columnwidth,
            % initialize Dark2
            cycle list/Dark2,
            % combine it with 'mark list*':
            cycle multiindex* list = {
                Dark2\nextlist
            }
        ]

        \addplot+[mark = none, style = solid, ultra thick] 
        table[x = T_train, y expr = {\thisrow{NMAE_mean}*100}, col sep = comma] 
        {results/csv/NMAE_vs_Ntrain_Prac_2021-08-12_03_manual_x_vel_steps_2mps.csv_dmd_angle.csv};
        \addlegendentry{DMD}

        \addplot+[mark = none, style = solid, ultra thick] 
        table[x = T_train, y expr = {\thisrow{NMAE_mean}*100}, col sep = comma] 
        {results/csv/NMAE_vs_Ntrain_Prac_2021-08-12_03_manual_x_vel_steps_2mps.csv_havok_angle.csv};
        \addlegendentry{HAVOK}

        \end{axis}
    \end{tikzpicture} 
    
    \caption{\gls{DMD} and \gls{HAVOK} prediction errors for different lengths of practical training data
    ($m =$~\SI{0.206}{\kilo\gram}, $l =$~\SI{1}{\meter}, $T_s =$~\SI{0.03}{\second}).}
    \label{fig:havok_vs_dmd_Ttrain_2mps}
\end{figure}

            \begin{figure}[htb]
    \centering
    \begin{tikzpicture}
        \begin{axis}[            
            xlabel = Length of training data,
            ylabel = $\overline{NMAE}$ \phantom{~},
            x unit = \si{\second},
            y unit = \%,
            xmin = 5,     xmax = 120,
            ymin = 3.2,  ymax = 15,
            grid = major,
            legend cell align = left,
            legend pos = north east,
            grid style = dashed,
            legend style = {font = \scriptsize},
            label style = {font = \scriptsize},
            tick label style = {font = \scriptsize},
            width = 0.95\columnwidth,
            height = 0.5\columnwidth,
            % initialize Dark2
            cycle list/Dark2,
            % combine it with 'mark list*':
            cycle multiindex* list = {
                Dark2\nextlist
            }
        ]

        \addplot+[mark = none, style = solid, ultra thick] 
        table[x = T_train, y expr = {\thisrow{NMAE_mean}*100}, col sep = comma] 
        {results/csv/NMAE_vs_Ntrain_Prac_2021-08-20_03_l-1_mp-0.2_wind-0.5.csv_dmd_angle.csv};
        \addlegendentry{0.5 m/s winds}

        \addplot+[mark = none, style = solid, ultra thick] 
        table[x = T_train, y expr = {\thisrow{NMAE_mean}*100}, col sep = comma] 
        {results/csv/NMAE_vs_Ntrain_Prac_2021-08-12_03_manual_x_vel_steps_2mps.csv_dmd_angle.csv};
        \addlegendentry{2 m/s winds}

        \addplot+[mark = none, style = solid, ultra thick] 
        table[x = T_train, y expr = {\thisrow{NMAE_mean}*100}, col sep = comma] 
        {results/csv/NMAE_vs_Ntrain_Prac_2021-08-12_02_manual_x_vel_steps_4mps.csv_dmd_angle.csv};
        \addlegendentry{4 m/s winds}

        \addplot+[mark = none, style = solid, ultra thick] 
        table[x = T_train, y expr = {\thisrow{NMAE_mean}*100}, col sep = comma] 
        {results/csv/NMAE_vs_Ntrain_Prac_2021-08-26_01_l-1_mp-0.2_wind-6.csv_dmd_angle.csv};
        \addlegendentry{6 m/s winds}

        \end{axis}
    \end{tikzpicture} 
    
    \caption{Effect of wind on \gls{DMD} prediction errors for different lengths of practical training data
    ($m_p =$~\SI{0.2}{\kilo\gram}, $l =$~\SI{1}{\meter}, $T_s =$~\SI{0.03}{\second}).}
    \label{fig:wind_Ttrain}
\end{figure}


        \FloatBarrier\subsection{Hyperparameters}
            plot NMAE vs q

            \begin{figure}[htb]
    \centering
    \begin{tikzpicture}
        \begin{axis}[            
            xlabel = {Number of delay-coordinates, $q$},
            ylabel = $\overline{NMAE}$ \phantom{~},
            % x unit = \si{\second},
            y unit = \%,
            xmin = 5,     xmax = 90,
            ymin = 3.5,  ymax = 7.5,
            grid = major,
            legend cell align = left,
            legend pos = north east,
            grid style = dashed,
            legend style = {font = \scriptsize},
            label style = {font = \scriptsize},
            tick label style = {font = \scriptsize},
            width = 0.95\columnwidth,
            height = 0.5\columnwidth,
            % initialize Dark2
            cycle list/Dark2,
            % combine it with 'mark list*':
            cycle multiindex* list = {
                Dark2\nextlist
            }
        ]

        \addplot+[mark = none, style = solid, ultra thick] 
        table[x = q, y expr = {\thisrow{NMAE_mean}*100}, col sep = comma] 
        {results/csv/NMAE_vs_q_Prac_2021-08-12_03_manual_x_vel_steps_2mps_q.csv_dmd_angle.csv};
        \addlegendentry{DMD}

        \addplot+[mark = none, style = solid, ultra thick] 
        table[x = q, y expr = {\thisrow{NMAE_mean}*100}, col sep = comma] 
        {results/csv/NMAE_vs_q_Prac_2021-08-12_03_manual_x_vel_steps_2mps_q.csv_havok_angle.csv};
        \addlegendentry{HAVOK}

        % \addplot+[mark = none, style = solid, ultra thick] 
        % table[x = q, y expr = {\thisrow{NMAE_mean}*100}, col sep = comma] 
        % {results/csv/NMAE_vs_q_Prac_2021-08-20_03_l-1_mp-0.2_wind-0.5_q.csv_dmd_angle.csv};
        % \addlegendentry{DMD}

        % \addplot+[mark = none, style = solid, ultra thick] 
        % table[x = q, y expr = {\thisrow{NMAE_mean}*100}, col sep = comma] 
        % {results/csv/NMAE_vs_q_Prac_2021-08-20_03_l-1_mp-0.2_wind-0.5_q.csv_havok_angle.csv};
        % \addlegendentry{HAVOK}


        \end{axis}
    \end{tikzpicture} 
    
    \caption{\gls{DMD} and \gls{HAVOK} prediction errors for different number of delays included in the model
    ($m =$~\SI{0.206}{\kilo\gram}, $l =$~\SI{1}{\meter}, $T_s =$~\SI{0.03}{\second}, wind speed $\approx~$\SI{2}{\metre/\second}).}
    \label{fig:havok_vs_dmd_q_2mps}
\end{figure}

            \begin{figure}[htb]
    \centering
    \begin{tikzpicture}
        \begin{semilogyaxis}[            
            xlabel = Index of mode,
            ylabel = Singular value,
            % x unit = \si{\second},
            % y unit = \si{\second},
            xmin = 0,     xmax = 183,
            ymin = 1e-5,  ymax = 1e4,
            grid = major,
            legend cell align = left,
            legend pos = north east,
            grid style = dashed,
            legend style = {font = \scriptsize},
            label style = {font = \scriptsize},
            tick label style = {font = \scriptsize},
            width = 0.95\columnwidth,
            height = 0.5\columnwidth,
            % initialize Dark2
            cycle list/Dark2,
            % combine it with 'mark list*':
            cycle multiindex* list = {
                Dark2\nextlist
            }
        ]

        \addplot+[only marks, mark = *, ultra thin, mark options={scale=0.7}] 
        table[x = index, y = S, col sep = comma] 
        {results/csv/Singular_values_Prac_2021-08-12_03_manual_x_vel_steps_2mps_q.csv_havok_angle_Ttrain_50_q91_p52.csv};
        \addlegendentry{Significant modes}

        \addplot+[only marks, mark = *, ultra thin, mark options={scale=0.7}] 
        table[x = index, y = S, col sep = comma] 
        {results/csv/Singular_values_Prac_2021-08-12_03_manual_x_vel_steps_2mps_q.csv_havok_angle_Ttrain_50_q91_p52_trunc.csv};
        \addlegendentry{Truncated modes}

        \end{semilogyaxis}
    \end{tikzpicture} 
    
    \caption{Significant and truncated singular values of a \gls{HAVOK} model produced from practical data
    ($m_p =$~\SI{0.2}{\kilo\gram}, $l =$~\SI{0.5}{\meter}, $T_s =$~\SI{0.03}{\second}, $T_{train} =$~\SI{60}{\second}.)}
    \label{fig:prac_singular_values}
\end{figure}


            Difference between SITl and practical (same input steps)
            This is due to wind disturbance
            Plot wind vs less wind vs no wind
            
        \FloatBarrier\subsection{System parameters}

            \paragraph
            The proposed data-driven techniques work for a range of different
        Show that practical data works for different system parameters

        \paragraph
        Plot NMAE vs Ttrain - 1 graph
        \begin{itemize}
            \item m = 0.2 kg    l = 1 m  
            \item m = 0.2 kg    l = 2 m
            \item l = 1 m       m = 0.3 kg
            \item l = 1 m       m = 0.1 kg
        \end{itemize}

        \FloatBarrier\subsection{Predictions}
        \begin{itemize}
            \item Single pendulum: Plot 4 in grid showing good and bad
            \item Double pendulum: Plot 4 in grid showing good and bad
        \end{itemize}

        \FloatBarrier\subsection{Extended dimensions}
        \begin{itemize}
            \item plot error vs T-train for XY
            \item plot predictions
            \item plot error vs T-train for XYZ
        \end{itemize}
            
            


            
            

    \section{LQR}
        \FloatBarrier\subsection{Single pendulum}
        \begin{itemize}
            \item 
        \end{itemize}
        \FloatBarrier\subsection{Double pendulum}

    \section{MPC}
        \FloatBarrier\subsection{Single pendulum}
        \FloatBarrier\subsection{Double pendulum}

    \section{HIL}

    \section{Conclusion}


