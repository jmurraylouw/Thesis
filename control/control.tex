
{
\tikzset{external/figure name/.add={control/}{}}

\chapter{Control systems}
\label{chap:control}

    \FloatBarrier\section{Overview of controllers} \label{sec:controller_overview}

        \begin{table}[!h]
            \renewcommand{\arraystretch}{1.1}
            \centering
            \caption{Summary of the system identification techniques paired with the active damping controllers.}
            \begin{tabularx}{0.75\linewidth}{@{}lll@{}}
                \toprule
                \multicolumn{2}{c}{\textbf{System identification}}   & \textbf{Controller} \\
                \cmidrule(lr){1-2}
                Category    & Algorithm                     & \\
                \midrule
                White-box   & RLS mass estimator, and       & \gls{LQR} \\
                            & FFT cable length estimator    & \\
                Black-box   & DMDc, or                      & \gls{MPC} \\
                            & HAVOK                         & \\
                \bottomrule
            \end{tabularx}
            \label{tbl:controller_summary}
        \end{table}
    

    \FloatBarrier\section{Cascaded PID}

        Control system design
        Slower for less swing angles

    \FloatBarrier\section{LQR} \label{sec:lqr}

        Anton and Francois stuff

    \FloatBarrier\section{MPC} \label{sec:mpc}

        MATLAB
        QP solver
        C++ generation

    \FloatBarrier\section{Implentation and results}

        \paragraph
        After the system identification phase, active swing damping control can be applied
        to the multirotor and payload system.
        The control architectures are summarised in Table~\ref{tbl:controller_summary} 
        by pairing the system identification techniques along with the appropriate controllers.
        It was firstly shown in Chapter~\ref{chap:system_id} that the system identifcation techniques worked in simulation.
        % Emphasise that control is now applied in a full \gls{SITL} simulation.
        
        \FloatBarrier\subsection{Simple suspended payload} \label{sec:simple_payload_control}

            \paragraph
            The modelling assumptions of the white-box model discussed in Chapter~\ref{chap:modelling} 
            defines a point-mass suspended with a rigid cable which is attached to the \gls{CoM} of the multirotor.
            This is a simplisitic suspended payload model but represents the dynamics of many practical payloads well.
            In this section, the simulated payload model complies will all these assumptions.
            The simulation model used in this section was verified with practical data in Section~\ref{sec:model_verification}.
            This is also the payload model used for simulations with an \gls{LQR} controller by \cite{Erasmus2020} and \cite{Slabber2020}

            \begin{figure}[htb]
    \centering
    \begin{tikzpicture}
        \begin{axis}[            
            xlabel = Time,
            ylabel = Velocity,
            x unit = \si{\second},
            y unit = \si{\metre/\second},
            xmin = 0,   xmax = 16,
            ymin = -0.1,  ymax = 2.7,
            grid = major,
            legend cell align = left,
            legend pos = south east,
            grid style = dashed,
            legend style = {font = \scriptsize},
            label style = {font = \scriptsize},
            tick label style = {font = \scriptsize},
            width = 0.95\columnwidth,
            height = 0.5\columnwidth,
            % initialize Dark2
            cycle list/Dark2,
            % combine it with 'mark list*':
            cycle multiindex* list = {
                Dark2\nextlist
            }
        ]
        
        \addplot+[mark = none, style = solid, ultra thick] 
        table[x = time, y = vel_sp, col sep = comma] 
        {control/csv/compare_control_pid_Simulink_single_pend_mp0.3_l2.25_PID_vel_steps_tune_scale_0.7.mat.csv};
        \addlegendentry{Setpoint}

        \addplot+[mark = none, style = solid, ultra thick] 
        table[x = time, y = vel, col sep = comma] 
        {control/csv/compare_control_pid_Simulink_single_pend_mp0.3_l2.25_PID_vel_steps_tune_scale_0.7.mat.csv};
        \addlegendentry{PID}

        \addplot+[mark = none, style = solid, ultra thick] 
        table[x = time, y = vel, col sep = comma] 
        {control/csv/compare_control_mpc_Simulink_single_pend_mp0.3_l2.25_PID_vel_steps_tune_scale_0.7.mat.csv};
        \addlegendentry{MPC}

        \addplot+[mark = none, style = dashed, ultra thick] 
        table[x = time, y = vel, col sep = comma] 
        {control/csv/compare_control_lqr_Simulink_single_pend_mp0.3_l2.25_PID_vel_steps_tune_scale_0.7.mat.csv};
        \addlegendentry{LQR}


        \end{axis}
    \end{tikzpicture} 
    \caption{Velocity step response comparison for different controllers
    ($l =$~\SI{2}{\meter}, $m_p =$~\SI{0.3}{\kilo\gram})}
    \label{fig:compare_control_vel}
\end{figure}


            \paragraph
            From simulation results, it appears that both the \gls{MPC} and \gls{LQR} effectively damp the payload oscillations while controlling the velocity of the multirotor.
            Figure~\ref{fig:compare_control_vel} shows the velocity step responces of the MPC, \gls{LQR} and PID controllers for a multirotor with a suspended payload.
            From Figure~\ref{fig:compare_control_vel} it is clear that both the \gls{LQR} and \gls{MPC} controllers actively damp the velocity oscillation caused by the swinging payload.
            The \gls{PID} controller does not consider the payload angle, hence the oscillations are not damped as strongly.

            \paragraph
            For the \gls{MPC} and LQR, the respective models were first generated in the training phase of the simulation.
            Thereafter, the \gls{MPC} and \gls{LQR} were manually and iteratively tuned to produce a step response with a similar response time and overshoot.
            The PID response shown uses the same controller gains used in the training phase.

            % MPC:
            % vel weight = 2;
            % theta weight = 0;
            % dtheta weight = 10; 
            % mv weight = 1;
            % mvrate weight = 5;
            % Ty = 5;
            % Tu = 3;

            % LQR:
            % LQR.Q = diag([0.3 10 0 100]); 
            % LQR.R = 8

            \begin{figure}[htb]
    \centering
    \begin{tikzpicture}
        \begin{axis}[            
            xlabel = Time,
            ylabel = Payload angle,
            x unit = \si{\second},
            y unit = \si{\degree},
            xmin = 0,   xmax = 16,
            ymin = -20,  ymax = 15,
            grid = major,
            legend cell align = left,
            legend pos = south east,
            grid style = dashed,
            legend style = {font = \scriptsize},
            label style = {font = \scriptsize},
            tick label style = {font = \scriptsize},
            width = 0.95\columnwidth,
            height = 0.5\columnwidth,
            % initialize Dark2
            cycle list/Dark2,
            % combine it with 'mark list*':
            cycle multiindex* list = {
                Dark2\nextlist
            }
        ]
                
        \pgfplotsset{cycle list shift = 1}

        \addplot+[mark = none, style = solid, ultra thick] 
        table[x = time, y = theta, col sep = comma] 
        {control/csv/compare_control_pid_Simulink_single_pend_mp0.3_l2.25_PID_vel_steps_tune_scale_0.7.mat.csv};
        \addlegendentry{PID}

        \addplot+[mark = none, style = solid, ultra thick] 
        table[x = time, y = theta, col sep = comma] 
        {control/csv/compare_control_mpc_Simulink_single_pend_mp0.3_l2.25_MPC_vel_steps_tune_scale_0.7.csv};
        \addlegendentry{MPC}

        \addplot+[mark = none, style = dashed, ultra thick] 
        table[x = time, y = theta, col sep = comma] 
        {control/csv/compare_control_lqr_Simulink_single_pend_mp0.3_l2.25_PID_vel_steps_tune_scale_0.7.mat.csv};
        \addlegendentry{LQR}

        \end{axis}
    \end{tikzpicture} 
    \caption{Payload angle comparison for different controllers
    ($l =$~\SI{2}{\meter}, $m_p =$~\SI{0.3}{\kilo\gram})}
    \label{fig:compare_control_theta}
\end{figure}


            \paragraph
            Figure~\ref{fig:compare_control_theta} shows the payload angle data of the velocity step response.
            Both the \gls{MPC} and \gls{LQR} damp the payload angle well and the osciilations cease after only two or three swings.
            In this case, the \gls{MPC} response results in a smaller initial swing angle, however, this is dependant on the specific tuning of each controller.
            The \gls{LQR} can also be tuned to produce a similar swing angle.

            \begin{figure}[htb]
    \centering
    \begin{tikzpicture}
        \begin{axis}[            
            xlabel = Time,
            ylabel = North velocity,
            x unit = \si{\second},
            y unit = \si{\metre/\second},
            xmin = 0,   xmax = 16,
            ymin = -0.1,  ymax = 2,
            grid = major,
            legend cell align = left,
            legend pos = north east,
            grid style = dashed,
            legend style = {font = \scriptsize},
            label style = {font = \scriptsize},
            tick label style = {font = \scriptsize},
            width = 0.95\columnwidth,
            height = 0.5\columnwidth,
            % initialize Dark2
            cycle list/Dark2,
            % combine it with 'mark list*':
            cycle multiindex* list = {
                Dark2\nextlist
            }
        ]
        
        \pgfplotsset{cycle list shift = 2}

        \addplot+[mark = none, style = solid, ultra thick] 
        table[x = time, y = acc_sp, col sep = comma] 
        {control/csv/compare_control_mpc_Simulink_single_pend_mp0.3_l2.25_MPC_vel_steps_tune_scale_0.7.csv};
        \addlegendentry{MPC}

        \addplot+[mark = none, style = dashed, ultra thick] 
        table[x = time, y = acc_sp, col sep = comma] 
        {control/csv/compare_control_lqr_Simulink_single_pend_mp0.3_l2.25_LQR_vel_steps_tune_scale_0.7.csv};
        \addlegendentry{LQR}


        \end{axis}
    \end{tikzpicture} 
    \caption{Acceleration setpoint commanded by different controllers for a velocity step input
    ($l =$~\SI{2}{\meter}, $m_p =$~\SI{0.3}{\kilo\gram}).}
    \label{fig:compare_control_acc_sp}
\end{figure}


            \paragraph
            Figure~\ref{fig:compare_control_acc_sp} shows the acceleration setpoint commanded by the two controllers for this step response.
            This is probably due to the inherant similarity between the controller implementations as discussed in Section~\ref{sec:mpc}.
            The similarity in the acceleration setpoint responses also show that the energy expended in a velocity steps are roughly equal for these two controller implementations.
            However, this is also highly dependant on the weightings used in optimastion problem of both controllers.
            Both controllers also produce a non-zero steady-state setpoint as expected, which is required to counter areodynamic drag.

        \FloatBarrier\subsection{Different payload parameters}

            \paragraph
            The system identification and control implementations are required to perform well with different unknown payload parameters.
            Therefore, numerous flights with a range of different payload were simulated, the respective models were trained and the controllers were implemented.
            Figure~\ref{fig:compare_control_vel_l1} and Figure~\ref{fig:compare_control_vel_l05} show velocity step responces with \gls{LQR} and \gls{MPC} implentations with two payloads flights.
            Both the parameter estimation with \gls{LQR} implementation, and the DMDc with \gls{MPC} implementation, handle flights with different cable lengths and payload masses well.
            In each flight, \gls{LQR} and \gls{MPC} damp the payload oscillations and control the multirotor velocity well.
  
            \begin{figure}[hb]
    \centering
    \begin{tikzpicture}
        \begin{axis}[            
            xlabel = Time,
            ylabel = North velocity,
            x unit = \si{\second},
            y unit = \si{\metre/\second},
            xmin = 0,   xmax = 16,
            ymin = -0.1,  ymax = 2.5,
            grid = major,
            legend cell align = left,
            legend pos = south east,
            grid style = dashed,
            legend style = {font = \scriptsize},
            label style = {font = \scriptsize},
            tick label style = {font = \scriptsize},
            width = 0.95\columnwidth,
            height = 0.5\columnwidth,
            % initialize Dark2
            cycle list/Dark2,
            % combine it with 'mark list*':
            cycle multiindex* list = {
                Dark2\nextlist
            }
        ]
        
        \addplot+[mark = none, style = solid, ultra thick] 
        table[x = time, y = vel_sp, col sep = comma] 
        {control/csv/compare_control_mpc_Simulink_single_pend_mp0.2_l1_MPC_vel_steps_tune_scale_0.7.csv};
        \addlegendentry{Setpoint}

        \addplot+[mark = none, style = solid, ultra thick] 
        table[x = time, y = vel, col sep = comma] 
        {control/csv/compare_control_pid_Simulink_single_pend_mp0.2_l1_PID_vel_steps_tune_scale_0.7.csv};
        \addlegendentry{PID}

        \addplot+[mark = none, style = solid, ultra thick] 
        table[x = time, y = vel, col sep = comma] 
        {control/csv/compare_control_mpc_Simulink_single_pend_mp0.2_l1_MPC_vel_steps_tune_scale_0.7.csv};
        \addlegendentry{MPC}

        \addplot+[mark = none, style = dashed, ultra thick] 
        table[x = time, y = vel, col sep = comma] 
        {control/csv/compare_control_lqr_Simulink_single_pend_mp0.2_l1_LQR_vel_steps_tune_scale_0.7.csv};
        \addlegendentry{LQR}


        \end{axis}
    \end{tikzpicture} 
    \caption{Velocity step response comparison of different controllers
    ($l =$~\SI{1}{\meter}, $m_p =$~\SI{0.2}{\kilo\gram}).}
    \label{fig:compare_control_vel_l1}
    
\end{figure}
 % subfigure

            \begin{figure}[htb]
    \centering
    \begin{tikzpicture}
        \begin{axis}[            
            xlabel = Time,
            ylabel = North velocity,
            x unit = \si{\second},
            y unit = \si{\metre/\second},
            xmin = 0,   xmax = 16,
            ymin = -0.1,  ymax = 2.5,
            grid = major,
            legend cell align = left,
            legend pos = south east,
            grid style = dashed,
            legend style = {font = \scriptsize},
            label style = {font = \scriptsize},
            tick label style = {font = \scriptsize},
            width = 0.95\columnwidth,
            height = 0.5\columnwidth,
            % initialize Dark2
            cycle list/Dark2,
            % combine it with 'mark list*':
            cycle multiindex* list = {
                Dark2\nextlist
            }
        ]
        
        \addplot+[mark = none, style = solid, ultra thick] 
        table[x = time, y = vel_sp, col sep = comma] 
        {control/csv/compare_control_lqr_Simulink_single_pend_mp0.1_l0.5_LQR_vel_steps_tune_scale_0.7.csv};
        \addlegendentry{Setpoint}

        \addplot+[mark = none, style = solid, ultra thick] 
        table[x = time, y = vel, col sep = comma] 
        {control/csv/compare_control_pid_Simulink_single_pend_mp0.1_l0.5_PID_vel_steps_tune_scale_0.7.csv};
        \addlegendentry{PID}

        \addplot+[mark = none, style = solid, ultra thick] 
        table[x = time, y = vel, col sep = comma] 
        {control/csv/compare_control_mpc_Simulink_single_pend_mp0.1_l0.5_MPC_vel_steps_tune_scale_0.7.csv};
        \addlegendentry{MPC}

        \addplot+[mark = none, style = dashed, ultra thick] 
        table[x = time, y = vel, col sep = comma] 
        {control/csv/compare_control_lqr_Simulink_single_pend_mp0.1_l0.5_LQR_vel_steps_tune_scale_0.7.csv};
        \addlegendentry{LQR}


        \end{axis}
    \end{tikzpicture}
    \caption{Velocity step response comparison of different controllers
    ($l =$~\SI{0.5}{\meter}, $m_p =$~\SI{0.1}{\kilo\gram}).}
    \label{fig:compare_control_vel_l05}
\end{figure}
 % subfigure
                        
            \paragraph
            The controllers were not specifically tuned for each simulation.
            Instead, the same controller parameters were used for these simulations as for the simulations in Section~\ref{sec:simple_payload_control}.
            This shows that each control architecture is adaptibile to different payload parameters without manual intervention.

        \FloatBarrier\subsection{Variation of other system parameters}

            \paragraph
            The control architectures have been shown to be adaptable to variationd in the payload parameters.
            However, changing other system parameters may affect the performance of the different controllers. 
            As mentioned in Chapter~\ref{chap:system_id}, a disadvantage of the white-box system identification approach used by the LQR,
            is that parameter estimation techniques need to be manually designed for each unknown parameter.
            In the specific implementation, the LQR model assumes that the multirotor mass is known.
            Hence, changing the mass of the multirotor is detrimental to the accuracy of plant model and therefore affects the LQR performance.
            In contrast, the data-driven system identification method for the MPC plant model does not rely on such modelling assumptions.
            
            \begin{figure}
                \captionsetup[subfigure]{justification=centering}
                \centering  
                \begin{figure}[htb]
    \centering
    \begin{tikzpicture}
        \begin{axis}[            
            xlabel = Time,
            ylabel = Velocity,
            x unit = \si{\second},
            y unit = \si{\metre/\second},
            xmin = 0,   xmax = 16,
            ymin = -0.1,  ymax = 2.7,
            grid = major,
            legend cell align = left,
            legend pos = south east,
            grid style = dashed,
            legend style = {font = \scriptsize},
            label style = {font = \scriptsize},
            tick label style = {font = \scriptsize},
            width = 0.95\columnwidth,
            height = 0.5\columnwidth,
            % initialize Dark2
            cycle list/Dark2,
            % combine it with 'mark list*':
            cycle multiindex* list = {
                Dark2\nextlist
            }
        ]
        
        \addplot+[mark = none, style = solid, ultra thick] 
        table[x = time, y = vel_sp, col sep = comma] 
        {control/csv/compare_control_lqr_Simulink_single_pend_mp0.3_l1.5_LQR_vel_steps_tune_scale_0.7_1.5_x_mq.csv};
        \addlegendentry{Setpoint}

        \addplot+[mark = none, style = solid, ultra thick] 
        table[x = time, y = vel, col sep = comma] 
        {control/csv/compare_control_pid_Simulink_single_pend_mp0.3_l1.5_PID_vel_steps_tune_scale_0.7_1.5_x_mq.csv};
        \addlegendentry{PID}

        \addplot+[mark = none, style = solid, ultra thick] 
        table[x = time, y = vel, col sep = comma] 
        {control/csv/compare_control_mpc_Simulink_single_pend_mp0.3_l1.5_MPC_vel_steps_tune_scale_0.7_1.5_x_mq.csv};
        \addlegendentry{MPC}

        \addplot+[mark = none, style = dashed, ultra thick] 
        table[x = time, y = vel, col sep = comma] 
        {control/csv/compare_control_lqr_Simulink_single_pend_mp0.3_l1.5_LQR_vel_steps_tune_scale_0.7_1.5_x_mq.csv};
        \addlegendentry{LQR}


        \end{axis}
    \end{tikzpicture} 
    \caption{Velocity step responses with an altered multirotor mass
    ($m_q =$~\SI{0.3}{\kilo\gram}, $l =$~\SI{1.5}{\meter}, $m_p =$~\SI{0.3}{\kilo\gram})}
    \label{fig:compare_control_vel_mq_changed}
\end{figure}

                \begin{subfigure}[t]{\columnwidth}
    \centering
    \begin{tikzpicture}
        \begin{axis}[            
            xlabel = Time,
            ylabel = Payload angle,
            x unit = \si{\second},
            y unit = \si{\degree},
            xmin = 0,    xmax = 16,
            ymin = -18,  ymax = 6,
            grid = major,
            legend cell align = left,
            legend pos = south east,
            grid style = dashed,
            legend style = {font = \scriptsize},
            label style = {font = \scriptsize},
            tick label style = {font = \scriptsize},
            width = 0.95\columnwidth,
            height = 0.5\columnwidth,
            % initialize Dark2
            cycle list/Dark2,
            % combine it with 'mark list*':
            cycle multiindex* list = {
                Dark2\nextlist
            }
        ]
                
        \pgfplotsset{cycle list shift = 1}

        \addplot+[mark = none, style = solid, ultra thick] 
        table[x = time, y = theta, col sep = comma] 
        {control/csv/compare_control_pid_Simulink_single_pend_mp0.3_l0.5_PID_vel_steps_tune_scale_0.7_mq_0.546_mp_0.3.csv};
        \addlegendentry{PID}
        

        \addplot+[mark = none, style = solid, ultra thick] 
        table[x = time, y = theta, col sep = comma] 
        {control/csv/compare_control_mpc_Simulink_single_pend_mp0.3_l0.5_MPC_vel_steps_tune_scale_0.7_mq_0.546_mp_0.3.csv};
        \addlegendentry{MPC}

        \addplot+[mark = none, style = dashed, ultra thick] 
        table[x = time, y = theta, col sep = comma] 
        {control/csv/compare_control_lqr_Simulink_single_pend_mp0.3_l0.5_LQR_vel_steps_tune_scale_0.7_mq_0.546_mp_0.3_l_est_0.35696.csv};
        \addlegendentry{LQR}

        \end{axis}
    \end{tikzpicture} 

\end{subfigure}

                \caption{Velocity step responses with the multirotor mass decreased by \SI{0.25}{\kilo\gram}
                ($l =$~\SI{0.5}{\meter}, $m_p =$~\SI{0.3}{\kilo\gram})}
                \label{fig:mq_changed} 
            \end{figure}

            \paragraph
            Simulations were performed with an altered multirotor mass to demostrate how the control architectures handle changes in other system parameters.
            Figure~\ref{fig:mq_changed} shows the velocity step responses of the \gls{PID}, \gls{MPC} and \gls{LQR} implementations with an altered multirotor mass.
            For these simulations, the original multirotor mass, $m_Q =$~\SI{0.796}{\kilo\gram}, 
            was decreased by \SI{0.250}{\kilo\gram}, 
            resulting in a new multirotor mass of, $m_Q =$~\SI{0.546}{\kilo\gram}.
            The same system identification processes were naively followed as in the previous sections, without prior knowledge of the change in $m_Q$.
            The same tuned controller parameters were also used.

            \paragraph
            In Figure~\ref{fig:mq_changed} it appears that the LQR results in lower payload oscillations than the PID controller
            but enduces higher frequency oscillations.
            This results in a jittery velocity response with the LQR and is undesirable for a multirotor flight.
            The LQR control performance has degraded because the dynamics of the LQR plant model differs significantly from the actual dynamics. 
            In contrast, the MPC still results in a smooth velocity profile and damps the payload oscillations effectively, as in previous simulations.
            This is expected, since the system identification model used by the MPC included the effect of the changed mass by estimating the entire model without considering indivdual parameters.
            
            \paragraph
            It should be noted that another mass estimator can be implemented to estimate $m_Q$ in a flight stage before the payload is added.
            However, this involves manually redesigning the system identification procedure for each new system parameters that can be changed.
            In these simulations it was shown that changing non-estimated system parameters in the white-box approach can be detrimental to the control performance.
            Unlike the white-box approach, the black-box approach handles changes in different system parameters well without prior knowledge of these parameters.

        \FloatBarrier\subsection{Wind disturbance}

            \paragraph
            For zero steady-state error with a practical system, a controller needs to apply some form of disturbance rejection.
            Practical systems experience unmeasured disturbances and other deviations which are not accounted for by the plant model.
            For example, a mean force applied by wind could prevent zero steady-state tracking error of the multirotor velocity without disturbance rejection.
            As discuseed in Section~\ref{sec:lqr}, an integral state variable was added to the \gls{LQR} plant model for inegral action of the multirotor velocity tracking.
            As discuseed in Section~\ref{sec:mpc}, an unmeasured input was added to the \gls{MPC} plant model with a disturbance estimator to apply integral action to the multirotor velocity.

            \begin{figure}[htb]
    \centering
    \begin{tikzpicture}
        \begin{axis}[            
            xlabel = Time,
            ylabel = North velocity,
            x unit = \si{\second},
            y unit = \si{\metre/\second},
            xmin = 0,   xmax = 16,
            ymin = -0.1,  ymax = 2.5,
            grid = major,
            legend cell align = left,
            legend pos = south east,
            grid style = dashed,
            legend style = {font = \scriptsize},
            label style = {font = \scriptsize},
            tick label style = {font = \scriptsize},
            width = 0.95\columnwidth,
            height = 0.5\columnwidth,
            % initialize Dark2
            cycle list/Dark2,
            % combine it with 'mark list*':
            cycle multiindex* list = {
                Dark2\nextlist
            }
        ]
        
        \addplot+[mark = none, style = solid, ultra thick] 
        table[x = time, y = vel_sp, col sep = comma] 
        {control/csv/compare_control_mpc_Simulink_single_pend_mp0.3_l2.25_MPC_vel_steps_tune_scale_0.7_wind_step_disturb.csv};
        \addlegendentry{Setpoint}

        \pgfplotsset{cycle list shift = 1} % Shift to skip \gls{PID} colour used in other plots

        \addplot+[mark = none, style = solid, ultra thick] 
        table[x = time, y = vel, col sep = comma] 
        {control/csv/compare_control_mpc_Simulink_single_pend_mp0.3_l2.25_MPC_vel_steps_tune_scale_0.7_wind_step_disturb.csv};
        \addlegendentry{MPC}

        \addplot+[mark = none, style = dashed, ultra thick] 
        table[x = time, y = vel, col sep = comma] 
        {control/csv/compare_control_lqr_Simulink_single_pend_mp0.3_l2.25_LQR_vel_steps_tune_scale_0.7_wind_step_disturb.csv};
        \addlegendentry{LQR}


        \end{axis}
    \end{tikzpicture} 
    \caption{Effect of an unmeasured step input disturbance.
    ($l =$~\SI{2}{\meter}, $m_p =$~\SI{0.3}{\kilo\gram}).}
    \label{fig:compare_control_vel_wind_disturbance}
\end{figure}


            \paragraph
            Figure~\ref{fig:compare_control_vel_wind_disturbance} shows the responses of the controllers from Section~\ref{sec:simple_payload_control} with a constant wind disturbance starting at \SI{8}{\second}.
            At Time~=~\SI{8}{\second}, a wind speed of \SI{2}{\metre/\second} is applied as an unmeasured step input.
            This mostly affects the multirotor velocity because the wind causes a greater drag force on the multirotor, 
            hence a larger acceleration setpoint is required to maintain a constant velocity.

            \paragraph
            It appears that the MPC shows better disturbance rejection than the LQR when using the controller parameters which were tuned for good performance in Section~\ref{sec:simple_payload_control}.
            This is primarily because the weighting of the integral variable in the LQR optimisation was minimised to reduce overshoot.
            The integral weighting can be increased to improve integral action at the expense of increasing overshoot in the velocity response.

            \begin{figure}[htb]
    \centering
    \begin{tikzpicture}
        \begin{axis}[            
            xlabel = Time,
            ylabel = Velocity,
            x unit = \si{\second},
            y unit = \si{\metre/\second},
            xmin = 0,   xmax = 16,
            ymin = -0.1,  ymax = 3,
            grid = major,
            legend cell align = left,
            legend pos = south east,
            grid style = dashed,
            legend style = {font = \scriptsize},
            label style = {font = \scriptsize},
            tick label style = {font = \scriptsize},
            width = 0.95\columnwidth,
            height = 0.5\columnwidth,
            % initialize Dark2
            cycle list/Dark2,
            % combine it with 'mark list*':
            cycle multiindex* list = {
                Dark2\nextlist
            }
        ]
        
        \addplot+[mark = none, style = solid, ultra thick] 
        table[x = time, y = vel_sp, col sep = comma] 
        {control/csv/compare_control_mpc_Simulink_single_pend_mp0.3_l2.25_MPC_vel_steps_tune_scale_0.7_wind_step_disturb.csv};
        \addlegendentry{Setpoint}

        \pgfplotsset{cycle list shift = 2} % Shift to skip PID colour used in other plots

        % \addplot+[mark = none, style = solid, ultra thick] 
        % table[x = time, y = vel, col sep = comma] 
        % {control/csv/compare_control_mpc_Simulink_single_pend_mp0.3_l2.25_MPC_vel_steps_tune_scale_0.7_wind_step_disturb.csv};
        % \addlegendentry{MPC}

        \addplot+[mark = none, style = dashed, ultra thick] 
        table[x = time, y = vel, col sep = comma] 
        {control/csv/compare_control_lqr_Simulink_single_pend_mp0.3_l2.25_LQR_vel_steps_tune_scale_0.7_wind_step_disturb_intg_weight_0.1.csv};
        \addlegendentry{Integrator weighting: 0.1}

        \addplot+[mark = none, style = solid, ultra thick] 
        table[x = time, y = vel, col sep = comma] 
        {control/csv/compare_control_lqr_Simulink_single_pend_mp0.3_l2.25_LQR_vel_steps_tune_scale_0.7_wind_step_disturb_intg_weight_1.csv};
        \addlegendentry{Integrator weighting: 1}

        \addplot+[mark = none, style = solid, ultra thick] 
        table[x = time, y = vel, col sep = comma] 
        {control/csv/compare_control_lqr_Simulink_single_pend_mp0.3_l2.25_LQR_vel_steps_tune_scale_0.7_wind_step_disturb_intg_weight_10.csv};
        \addlegendentry{Integrator weighting: 10}


        \end{axis}
    \end{tikzpicture} 
    \caption{Different LQR responses for different integrator gains
    ($l =$~\SI{2}{\meter}, $m_p =$~\SI{0.3}{\kilo\gram}).}
    \label{fig:improve_lqr_integrator}
\end{figure}


            \paragraph
            Figure~\ref{fig:improve_lqr_integrator} shows the LQR responses with different integral state weightings.
            The other state variable weights are kept constant for each response.
            It is clear that the settling time and disturbance rejection of the LQR improves for larger integral state weighting.
            However, the overshoot increases significantly because of the integrator build up at the start of the response.
            % The overshoot could be decreased by adding a velocity derivative state to the LQR model, but this will involve ex

            \paragraph
            In contrast to the LQR, the \gls{MPC} shows good disturbance rejection while maintaining a low overshoot.
            This is because the disturbance estimator applies integral action which depends on deviation of the actual dynamics from the plant model.
            whereas the LQR applies integral action proportional to the integral of the tracking error.
            Therefore the MPC implementation produces less integrator build up which results in a lower overshoot.

            % Could add effect of bad models. ??

        \FloatBarrier\subsection{Dynamic payload}

            \paragraph
            Intro

            \begin{figure}[htb]
    \centering
    \begin{tikzpicture}
        \begin{axis}[            
            xlabel = Time,
            ylabel = Velocity,
            x unit = \si{\second},
            y unit = \si{\metre/\second},
            xmin = 0,   xmax = 16,
            ymin = -0.1,  ymax = 2.7,
            grid = major,
            legend cell align = left,
            legend pos = south east,
            grid style = dashed,
            legend style = {font = \scriptsize},
            label style = {font = \scriptsize},
            tick label style = {font = \scriptsize},
            width = 0.95\columnwidth,
            height = 0.5\columnwidth,
            % initialize Dark2
            cycle list/Dark2,
            % combine it with 'mark list*':
            cycle multiindex* list = {
                Dark2\nextlist
            }
        ]
        
        \addplot+[mark = none, style = solid, ultra thick] 
        table[x = time, y = vel_sp, col sep = comma] 
        {control/csv/compare_control_pid_Simulink_single_pend_mp0.3_l2.25_PID_vel_steps_tune_scale_0.7.mat.csv};
        \addlegendentry{Setpoint}

        \addplot+[mark = none, style = solid, ultra thick] 
        table[x = time, y = vel, col sep = comma] 
        {control/csv/compare_control_pid_Simulink_single_pend_mp0.3_l2.25_PID_vel_steps_tune_scale_0.7.mat.csv};
        \addlegendentry{PID}

        \addplot+[mark = none, style = solid, ultra thick] 
        table[x = time, y = vel, col sep = comma] 
        {control/csv/compare_control_mpc_Simulink_single_pend_mp0.3_l2.25_PID_vel_steps_tune_scale_0.7.mat.csv};
        \addlegendentry{MPC}

        \addplot+[mark = none, style = dashed, ultra thick] 
        table[x = time, y = vel, col sep = comma] 
        {control/csv/compare_control_lqr_Simulink_single_pend_mp0.3_l2.25_PID_vel_steps_tune_scale_0.7.mat.csv};
        \addlegendentry{LQR}


        \end{axis}
    \end{tikzpicture} 
    \caption{Velocity step response comparison for different controllers
    ($l =$~\SI{2}{\meter}, $m_p =$~\SI{0.3}{\kilo\gram})}
    \label{fig:compare_control_vel}
\end{figure}


            \paragraph
            vel control

            \begin{figure}[htb]
    \centering
    \begin{tikzpicture}
        \begin{axis}[            
            xlabel = Time,
            ylabel = Payload angle,
            x unit = \si{\second},
            y unit = \si{\degree},
            xmin = 0,   xmax = 16,
            ymin = -20,  ymax = 15,
            grid = major,
            legend cell align = left,
            legend pos = south east,
            grid style = dashed,
            legend style = {font = \scriptsize},
            label style = {font = \scriptsize},
            tick label style = {font = \scriptsize},
            width = 0.95\columnwidth,
            height = 0.5\columnwidth,
            % initialize Dark2
            cycle list/Dark2,
            % combine it with 'mark list*':
            cycle multiindex* list = {
                Dark2\nextlist
            }
        ]
                
        \pgfplotsset{cycle list shift = 1}

        \addplot+[mark = none, style = solid, ultra thick] 
        table[x = time, y = theta, col sep = comma] 
        {control/csv/compare_control_pid_Simulink_single_pend_mp0.3_l2.25_PID_vel_steps_tune_scale_0.7.mat.csv};
        \addlegendentry{PID}

        \addplot+[mark = none, style = solid, ultra thick] 
        table[x = time, y = theta, col sep = comma] 
        {control/csv/compare_control_mpc_Simulink_single_pend_mp0.3_l2.25_MPC_vel_steps_tune_scale_0.7.csv};
        \addlegendentry{MPC}

        \addplot+[mark = none, style = dashed, ultra thick] 
        table[x = time, y = theta, col sep = comma] 
        {control/csv/compare_control_lqr_Simulink_single_pend_mp0.3_l2.25_PID_vel_steps_tune_scale_0.7.mat.csv};
        \addlegendentry{LQR}

        \end{axis}
    \end{tikzpicture} 
    \caption{Payload angle comparison for different controllers
    ($l =$~\SI{2}{\meter}, $m_p =$~\SI{0.3}{\kilo\gram})}
    \label{fig:compare_control_theta}
\end{figure}

            
            \paragraph
            theta control

            \begin{figure}[htb]
    \centering
    \begin{tikzpicture}
        \begin{axis}[            
            xlabel = Time,
            ylabel = Velocity,
            x unit = \si{\second},
            y unit = \si{\metre/\second},
            xmin = 1.1,   xmax = 9.1,
            ymin = -0.1,  ymax = 2.4,
            grid = major,
            legend cell align = left,
            legend pos = south east,
            grid style = dashed,
            legend style = {font = \scriptsize},
            label style = {font = \scriptsize},
            tick label style = {font = \scriptsize},
            width = 0.95\columnwidth,
            height = 0.5\columnwidth,
            % initialize Dark2
            cycle list/Dark2,
            % combine it with 'mark list*':
            cycle multiindex* list = {
                Dark2\nextlist
            }
        ]
        
        \addplot+[mark = none, style = solid, ultra thick] 
        table[x = time, y = setpoint, col sep = comma] 
        {control/csv/mpc_prediction_vs_actual_Simulink_double_pend_m1-0.2_m2-0.1_l1-0.5_l2-0.6_MPC_vel_steps_tune_scale_0.7.csv};
        \addlegendentry{Setpoint}

        \addplot+[mark = none, style = solid, ultra thick] 
        table[x = time, y = prediction, col sep = comma] 
        {control/csv/mpc_prediction_vs_actual_Simulink_double_pend_m1-0.2_m2-0.1_l1-0.5_l2-0.6_MPC_vel_steps_tune_scale_0.7.csv};
        \addlegendentry{Optimised prediction}

        \addplot+[mark = none, style = dashed, ultra thick] 
        table[x = time, y = actual, col sep = comma] 
        {control/csv/mpc_prediction_vs_actual_Simulink_double_pend_m1-0.2_m2-0.1_l1-0.5_l2-0.6_MPC_vel_steps_tune_scale_0.7.csv};
        \addlegendentry{Actual response}

        \end{axis}
    \end{tikzpicture} 
    \caption{Velocity step response comparison of different controllers}
    \label{fig:mpc_prediction_vs_actual}
\end{figure}


            \paragraph
            Prediction vs actual


            \begin{itemize}
                \item subplot prediction of data driven model. subplot prediction of white-box model
                \item plot \gls{MPC} vs \gls{LQR} v \gls{PID} (no wind)
            \end{itemize}

    \FloatBarrier\section{Conclusion}

} % end of tikz set
    