\graphicspath{{control/fig/}}

\chapter{Control systems}
\label{chap:control}

    \FloatBarrier\section{Overview of controllers}

        \begin{table}[!h]
            \renewcommand{\arraystretch}{1.1}
            \centering
            \caption{Summary of the system identification techniques paired with the active damping controllers.}
            \begin{tabularx}{0.75\linewidth}{@{}lll@{}}
                \toprule
                \multicolumn{2}{c}{\textbf{System identification}}   & \textbf{Controller} \\
                \cmidrule(lr){1-2}
                Category    & Algorithm                     & \\
                \midrule
                White-box   & RLS mass estimator, and       & LQR \\
                            & FFT cable length estimator    & \\
                Black-box   & DMDc, or                      & MPC \\
                            & HAVOK                         & \\
                \bottomrule
            \end{tabularx}
            \label{tbl:controller_summary}
        \end{table}
    

    \FloatBarrier\section{Cascaded PID}

        Control system design
        Slower for less swing angles

    \FloatBarrier\section{LQR}

        Anton and Francois stuff

    \FloatBarrier\section{MPC}

        MATLAB
        QP solver
        C++ generation

    \FloatBarrier\section{Implentation and results}

        \paragraph
        After the system identification phase, active swing damping control can be applied
        to the multirotor and payload system.
        The control architectures are summarised in Table~\ref{tbl:controller_summary} 
        by pairing the system identification techniques along with the appropriate controllers.
        It was firstly shown in Chapter~\ref{chap:system_id} that the system identifcation techniques worked in simulation.
        % Emphasise that control is now applied in a full \gls{SITL} simulation.
        
        \FloatBarrier\subsection{Simple suspended payload}

            \paragraph
            The modelling assumptions of the white-box model discussed in Chapter~\ref{chap:modelling} 
            defines a point-mass suspended with a rigid cable which is attached to the \gls{CoM} of the multirotor.
            This is a simplisitic suspended payload model but represents the dynamics of many practical payloads well.
            In this section, the simulated payload model complies will all these assumptions.
            The simulation model used in this section was verified with practical data in Section~\ref{sec:model_verification}.
            This is also the payload model used for simulations with an \gls{LQR} controller by \cite{Erasmus2020} and \cite{Slabber2020}

            \input{control/plots/compare_control_vel.tex}

            \paragraph
            From simulation results, it appears that both the MPC and LQR effectively damp the payload oscillations while controlling the velocity of the multirotor.
            Figure~\ref{fig:compare_control_vel} shows the velocity step responces of the MPC, LQR and PID controllers for a multirotor with a suspended payload.
            For the MPC and LQR, the respective models were first generated in the training phase of the simulation.
            Thereafter, the MPC and LQR were manually and iteratively tuned to produce a step response with a similar response time and overshoot.
            The PID response shown is with the same controller gains used in the training phase.
            
            \paragraph
            From Figure~\ref{fig:compare_control_vel} it is clear that both the LQR and MPC controllers actively damp the velocity oscillation caused by the swinging payload.
            The oscillations clearly continue for a long

            MPC:
            vel weight = 2;
            theta weight = 0;
            dtheta weight = 10; 
            mv weight = 1;
            mvrate weight = 5;
            Ty = 5;
            Tu = 3;

            LQR:
            LQR.Q = diag([10 10 0 100]); 
            LQR.R = 8

            \begin{figure}[htb]
    \centering
    \begin{tikzpicture}
        \begin{axis}[            
            xlabel = Time,
            ylabel = Payload angle,
            x unit = \si{\second},
            y unit = \si{\degree},
            xmin = 0,   xmax = 16,
            ymin = -20,  ymax = 15,
            grid = major,
            legend cell align = left,
            legend pos = south east,
            grid style = dashed,
            legend style = {font = \scriptsize},
            label style = {font = \scriptsize},
            tick label style = {font = \scriptsize},
            width = 0.95\columnwidth,
            height = 0.5\columnwidth,
            % initialize Dark2
            cycle list/Dark2,
            % combine it with 'mark list*':
            cycle multiindex* list = {
                Dark2\nextlist
            }
        ]
                
        \pgfplotsset{cycle list shift = 1}

        \addplot+[mark = none, style = solid, ultra thick] 
        table[x = time, y = theta, col sep = comma] 
        {control/csv/compare_control_pid_Simulink_single_pend_mp0.3_l2.25_PID_vel_steps_tune_scale_0.7.mat.csv};
        \addlegendentry{PID}

        \addplot+[mark = none, style = solid, ultra thick] 
        table[x = time, y = theta, col sep = comma] 
        {control/csv/compare_control_mpc_Simulink_single_pend_mp0.3_l2.25_MPC_vel_steps_tune_scale_0.7.csv};
        \addlegendentry{MPC}

        \addplot+[mark = none, style = dashed, ultra thick] 
        table[x = time, y = theta, col sep = comma] 
        {control/csv/compare_control_lqr_Simulink_single_pend_mp0.3_l2.25_PID_vel_steps_tune_scale_0.7.mat.csv};
        \addlegendentry{LQR}

        \end{axis}
    \end{tikzpicture} 
    \caption{Payload angle comparison of different controllers
    ($l =$~\SI{2}{\meter}, $m_p =$~\SI{0.3}{\kilo\gram})}
    \label{fig:compare_control_theta}
\end{figure}


            \paragraph
            theta control

            \begin{figure}[htb]
    \centering
    \begin{tikzpicture}
        \begin{axis}[            
            xlabel = Time,
            ylabel = Velocity,
            x unit = \si{\second},
            y unit = \si{\metre/\second},
            xmin = 0,   xmax = 16,
            ymin = -0.1,  ymax = 2.2,
            grid = major,
            legend cell align = left,
            legend pos = south east,
            grid style = dashed,
            legend style = {font = \scriptsize},
            label style = {font = \scriptsize},
            tick label style = {font = \scriptsize},
            width = 0.95\columnwidth,
            height = 0.5\columnwidth,
            % initialize Dark2
            cycle list/Dark2,
            % combine it with 'mark list*':
            cycle multiindex* list = {
                Dark2\nextlist
            }
        ]
        
        \pgfplotsset{cycle list shift = 2}

        \addplot+[mark = none, style = solid, ultra thick] 
        table[x = time, y = acc_sp, col sep = comma] 
        {control/csv/compare_control_mpc_Simulink_single_pend_mp0.3_l2.25_MPC_vel_steps_tune_scale_0.7.csv};
        \addlegendentry{MPC}

        \addplot+[mark = none, style = dashed, ultra thick] 
        table[x = time, y = acc_sp, col sep = comma] 
        {control/csv/compare_control_lqr_Simulink_single_pend_mp0.3_l2.25_PID_vel_steps_tune_scale_0.7.mat.csv};
        \addlegendentry{LQR}


        \end{axis}
    \end{tikzpicture} 
    \caption{Acceleration setpoint commanded by different controllers for a velocity step input
    ($l =$~\SI{2}{\meter}, $m_p =$~\SI{0.3}{\kilo\gram})}
    \label{fig:compare_control_acc_sp}
\end{figure}


            \paragraph
            acc-sp plot

        \FloatBarrier\subsection{Different payloads}

            \begin{figure}
                \captionsetup[subfigure]{justification=centering}
                \centering  
                \input{control/plots/compare_control_vel_l-1.0_mp-0.2.tex} % subfigure
                \begin{figure}[htb]
    \centering
    \begin{tikzpicture}
        \begin{axis}[            
            xlabel = Time,
            ylabel = North velocity,
            x unit = \si{\second},
            y unit = \si{\metre/\second},
            xmin = 0,   xmax = 16,
            ymin = -0.1,  ymax = 2.5,
            grid = major,
            legend cell align = left,
            legend pos = south east,
            grid style = dashed,
            legend style = {font = \scriptsize},
            label style = {font = \scriptsize},
            tick label style = {font = \scriptsize},
            width = 0.95\columnwidth,
            height = 0.5\columnwidth,
            % initialize Dark2
            cycle list/Dark2,
            % combine it with 'mark list*':
            cycle multiindex* list = {
                Dark2\nextlist
            }
        ]
        
        \addplot+[mark = none, style = solid, ultra thick] 
        table[x = time, y = vel_sp, col sep = comma] 
        {control/csv/compare_control_lqr_Simulink_single_pend_mp0.1_l0.5_LQR_vel_steps_tune_scale_0.7.csv};
        \addlegendentry{Setpoint}

        \addplot+[mark = none, style = solid, ultra thick] 
        table[x = time, y = vel, col sep = comma] 
        {control/csv/compare_control_pid_Simulink_single_pend_mp0.1_l0.5_PID_vel_steps_tune_scale_0.7.csv};
        \addlegendentry{PID}

        \addplot+[mark = none, style = solid, ultra thick] 
        table[x = time, y = vel, col sep = comma] 
        {control/csv/compare_control_mpc_Simulink_single_pend_mp0.1_l0.5_MPC_vel_steps_tune_scale_0.7.csv};
        \addlegendentry{MPC}

        \addplot+[mark = none, style = dashed, ultra thick] 
        table[x = time, y = vel, col sep = comma] 
        {control/csv/compare_control_lqr_Simulink_single_pend_mp0.1_l0.5_LQR_vel_steps_tune_scale_0.7.csv};
        \addlegendentry{LQR}


        \end{axis}
    \end{tikzpicture}
    \caption{Velocity step response comparison of different controllers
    ($l =$~\SI{0.5}{\meter}, $m_p =$~\SI{0.1}{\kilo\gram}).}
    \label{fig:compare_control_vel_l05}
\end{figure}
 % subfigure
                \caption{}
                \label{fig:compare_control_different_payloads}  
            \end{figure}

            \paragraph
            Different payloads, compare tuning

        \FloatBarrier\subsection{Wind disturbance}

            \begin{figure}[htb]
    \centering
    \begin{tikzpicture}
        \begin{axis}[            
            xlabel = Time,
            ylabel = North velocity,
            x unit = \si{\second},
            y unit = \si{\metre/\second},
            xmin = 0,   xmax = 16,
            ymin = -0.1,  ymax = 2.5,
            grid = major,
            legend cell align = left,
            legend pos = south east,
            grid style = dashed,
            legend style = {font = \scriptsize},
            label style = {font = \scriptsize},
            tick label style = {font = \scriptsize},
            width = 0.95\columnwidth,
            height = 0.5\columnwidth,
            % initialize Dark2
            cycle list/Dark2,
            % combine it with 'mark list*':
            cycle multiindex* list = {
                Dark2\nextlist
            }
        ]
        
        \addplot+[mark = none, style = solid, ultra thick] 
        table[x = time, y = vel_sp, col sep = comma] 
        {control/csv/compare_control_mpc_Simulink_single_pend_mp0.3_l2.25_MPC_vel_steps_tune_scale_0.7_wind_step_disturb.csv};
        \addlegendentry{Setpoint}

        \pgfplotsset{cycle list shift = 1} % Shift to skip \gls{PID} colour used in other plots

        \addplot+[mark = none, style = solid, ultra thick] 
        table[x = time, y = vel, col sep = comma] 
        {control/csv/compare_control_mpc_Simulink_single_pend_mp0.3_l2.25_MPC_vel_steps_tune_scale_0.7_wind_step_disturb.csv};
        \addlegendentry{MPC}

        \addplot+[mark = none, style = dashed, ultra thick] 
        table[x = time, y = vel, col sep = comma] 
        {control/csv/compare_control_lqr_Simulink_single_pend_mp0.3_l2.25_LQR_vel_steps_tune_scale_0.7_wind_step_disturb.csv};
        \addlegendentry{LQR}


        \end{axis}
    \end{tikzpicture} 
    \caption{Effect of an unmeasured step input disturbance.
    ($l =$~\SI{2}{\meter}, $m_p =$~\SI{0.3}{\kilo\gram}).}
    \label{fig:compare_control_vel_wind_disturbance}
\end{figure}


            \paragraph
            Wind disturbance wind speed~=~\SI{2}{\metre/\second}

            \begin{figure}[htb]
    \centering
    \begin{tikzpicture}
        \begin{axis}[            
            xlabel = Time,
            ylabel = North velocity,
            x unit = \si{\second},
            y unit = \si{\metre/\second},
            xmin = 0,   xmax = 16,
            ymin = -0.1,  ymax = 3,
            grid = major,
            legend cell align = left,
            legend pos = south east,
            grid style = dashed,
            legend style = {font = \scriptsize},
            label style = {font = \scriptsize},
            tick label style = {font = \scriptsize},
            width = 0.95\columnwidth,
            height = 0.5\columnwidth,
            % initialize Dark2
            cycle list/Dark2,
            % combine it with 'mark list*':
            cycle multiindex* list = {
                Dark2\nextlist
            }
        ]
        
        \addplot+[mark = none, style = solid, ultra thick] 
        table[x = time, y = vel_sp, col sep = comma] 
        {control/csv/compare_control_mpc_Simulink_single_pend_mp0.3_l2.25_MPC_vel_steps_tune_scale_0.7_wind_step_disturb.csv};
        \addlegendentry{Setpoint}

        \pgfplotsset{cycle list shift = 2} % Shift to skip \gls{PID} colour used in other plots

        % \addplot+[mark = none, style = solid, ultra thick] 
        % table[x = time, y = vel, col sep = comma] 
        % {control/csv/compare_control_mpc_Simulink_single_pend_mp0.3_l2.25_MPC_vel_steps_tune_scale_0.7_wind_step_disturb.csv};
        % \addlegendentry{MPC}

        \addplot+[mark = none, style = dashed, ultra thick] 
        table[x = time, y = vel, col sep = comma] 
        {control/csv/compare_control_lqr_Simulink_single_pend_mp0.3_l2.25_LQR_vel_steps_tune_scale_0.7_wind_step_disturb_intg_weight_0.1.csv};
        \addlegendentry{Integrator weighting: 0.1}

        \addplot+[mark = none, style = solid, ultra thick] 
        table[x = time, y = vel, col sep = comma] 
        {control/csv/compare_control_lqr_Simulink_single_pend_mp0.3_l2.25_LQR_vel_steps_tune_scale_0.7_wind_step_disturb_intg_weight_1.csv};
        \addlegendentry{Integrator weighting: 1}

        \addplot+[mark = none, style = solid, ultra thick] 
        table[x = time, y = vel, col sep = comma] 
        {control/csv/compare_control_lqr_Simulink_single_pend_mp0.3_l2.25_LQR_vel_steps_tune_scale_0.7_wind_step_disturb_intg_weight_10.csv};
        \addlegendentry{Integrator weighting: 10}


        \end{axis}
    \end{tikzpicture} 
    \caption{Different \gls{LQR} responses for different integrator gains
    ($l =$~\SI{2}{\meter}, $m_p =$~\SI{0.3}{\kilo\gram}).}
    \label{fig:improve_lqr_integrator}
\end{figure}


            \paragraph
            Improve integrator. This is in contrast to the MPC response which shows good disturbance rejection and maintains a low overshoot.

            % Could add effect of bad models. ??

        \FloatBarrier\subsection{Dynamic payload}

            \paragraph
            Intro

            \begin{figure}[htb]
    \centering
    \begin{tikzpicture}
        \begin{axis}[            
            xlabel = Time,
            ylabel = Velocity,
            x unit = \si{\second},
            y unit = \si{\metre/\second},
            xmin = 0,   xmax = 16,
            ymin = -0.1,  ymax = 2.7,
            grid = major,
            legend cell align = left,
            legend pos = south east,
            grid style = dashed,
            legend style = {font = \scriptsize},
            label style = {font = \scriptsize},
            tick label style = {font = \scriptsize},
            width = 0.95\columnwidth,
            height = 0.5\columnwidth,
            % initialize Dark2
            cycle list/Dark2,
            % combine it with 'mark list*':
            cycle multiindex* list = {
                Dark2\nextlist
            }
        ]
        
        \addplot+[mark = none, style = solid, ultra thick] 
        table[x = time, y = vel_sp, col sep = comma] 
        {control/csv/compare_control_pid_Simulink_single_pend_mp0.3_l2.25_PID_vel_steps_tune_scale_0.7.mat.csv};
        \addlegendentry{Setpoint}

        \addplot+[mark = none, style = solid, ultra thick] 
        table[x = time, y = vel, col sep = comma] 
        {control/csv/compare_control_pid_Simulink_single_pend_mp0.3_l2.25_PID_vel_steps_tune_scale_0.7.mat.csv};
        \addlegendentry{PID}

        \addplot+[mark = none, style = solid, ultra thick] 
        table[x = time, y = vel, col sep = comma] 
        {control/csv/compare_control_mpc_Simulink_single_pend_mp0.3_l2.25_MPC_vel_steps_tune_scale_0.7.csv};
        \addlegendentry{MPC}

        \addplot+[mark = none, style = dashed, ultra thick] 
        table[x = time, y = vel, col sep = comma] 
        {control/csv/compare_control_lqr_Simulink_single_pend_mp0.3_l2.25_PID_vel_steps_tune_scale_0.7.mat.csv};
        \addlegendentry{LQR}


        \end{axis}
    \end{tikzpicture} 
    \caption{Velocity step response comparison of different controllers
    ($l =$~\SI{2}{\meter}, $m_p =$~\SI{0.3}{\kilo\gram})}
    \label{fig:compare_control_vel}
\end{figure}


            \paragraph
            vel control

            \begin{figure}[htb]
    \centering
    \begin{tikzpicture}
        \begin{axis}[            
            xlabel = Time,
            ylabel = Payload angle,
            x unit = \si{\second},
            y unit = \si{\degree},
            xmin = 0,   xmax = 16,
            ymin = -20,  ymax = 5,
            grid = major,
            legend cell align = left,
            legend pos = south east,
            grid style = dashed,
            legend style = {font = \scriptsize},
            label style = {font = \scriptsize},
            tick label style = {font = \scriptsize},
            width = 0.95\columnwidth,
            height = 0.5\columnwidth,
            % initialize Dark2
            cycle list/Dark2,
            % combine it with 'mark list*':
            cycle multiindex* list = {
                Dark2\nextlist
            }
        ]
                
        \pgfplotsset{cycle list shift = 1}

        \addplot+[mark = none, style = solid, ultra thick] 
        table[x = time, y = theta, col sep = comma] 
        {control/csv/compare_control_pid_Simulink_double_pend_m1-0.2_m2-0.1_l1-0.5_l2-0.6_PID_vel_steps_tune_scale_0.7.csv};
        \addlegendentry{PID}

        \addplot+[mark = none, style = solid, ultra thick] 
        table[x = time, y = theta, col sep = comma] 
        {control/csv/compare_control_mpc_Simulink_double_pend_m1-0.2_m2-0.1_l1-0.5_l2-0.6_MPC_vel_steps_tune_scale_0.7.csv};
        \addlegendentry{MPC}

        \addplot+[mark = none, style = dashed, ultra thick] 
        table[x = time, y = theta, col sep = comma] 
        {control/csv/compare_control_lqr_Simulink_double_pend_m1-0.2_m2-0.1_l1-0.5_l2-0.6_LQR_vel_steps_tune_scale_0.7_slow_minimal_oscillations.csv};
        \addlegendentry{LQR}

        \end{axis}
    \end{tikzpicture} 
    \caption{Payload angle comparison of different controllers
    ($l =$~\SI{2}{\meter}, $m_p =$~\SI{0.3}{\kilo\gram}).}
    \label{fig:compare_control_theta_dynamic}
\end{figure}


            \paragraph
            theta control

            \begin{itemize}
                \item subplot prediction of data driven model. subplot prediction of white-box model
                \item plot \gls{MPC} vs \gls{LQR} v \gls{PID} (no wind)
            \end{itemize}

    \FloatBarrier\section{Conclusion}
    