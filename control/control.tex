\graphicspath{{control/fig/}}

\chapter{Control systems}
\label{chap:control}

    \FloatBarrier\section{Overview of controllers}

        \begin{table}[!h]
            \renewcommand{\arraystretch}{1.1}
            \centering
            \caption{Summary of the system identification techniques paired with the active damping controllers.}
            \begin{tabularx}{0.75\linewidth}{@{}lll@{}}
                \toprule
                \multicolumn{2}{c}{\textbf{System identification}}   & \textbf{Controller} \\
                \cmidrule(lr){1-2}
                Category    & Algorithm                     & \\
                \midrule
                White-box   & RLS mass estimator, and       & LQR \\
                            & FFT cable length estimator    & \\
                Black-box   & DMDc, or                      & MPC \\
                            & HAVOK                         & \\
                \bottomrule
            \end{tabularx}
            \label{tbl:controller_summary}
        \end{table}
    

    \FloatBarrier\section{Cascaded PID}

        Control system design
        Slower for less swing angles

    \FloatBarrier\section{LQR}

        Anton and Francois stuff

    \FloatBarrier\section{MPC}

        MATLAB
        QP solver
        C+ generation

    \FloatBarrier\section{Implentation and results}

        \paragraph
        After the system identification phase, active swing damping control can be applied
        to the multirotor and payload system.
        The control architectures are summarised in Table~\ref{tbl:controller_summary} 
        by pairing the system identification techniques along with the appropriate controllers.
        It was firstly shown in Chapter~\ref{chap:system_id} that the system identifcation techniques worked in simulation.
        % Emphasise that control is now applied in a full \gls{SITL} simulation.
        
        \FloatBarrier\subsection{Simple suspended payload}

            \paragraph
            intro

            \begin{figure}[htb]
    \centering
    \begin{tikzpicture}
        \begin{axis}[            
            xlabel = Time,
            ylabel = Velocity,
            x unit = \si{\second},
            y unit = \si{\metre/\second},
            xmin = 0,   xmax = 16,
            ymin = -0.1,  ymax = 2.7,
            grid = major,
            legend cell align = left,
            legend pos = south east,
            grid style = dashed,
            legend style = {font = \scriptsize},
            label style = {font = \scriptsize},
            tick label style = {font = \scriptsize},
            width = 0.95\columnwidth,
            height = 0.5\columnwidth,
            % initialize Dark2
            cycle list/Dark2,
            % combine it with 'mark list*':
            cycle multiindex* list = {
                Dark2\nextlist
            }
        ]
        
        \addplot+[mark = none, style = solid, ultra thick] 
        table[x = time, y = vel_sp, col sep = comma] 
        {control/csv/compare_control_pid_Simulink_single_pend_mp0.3_l2.25_PID_vel_steps_tune_scale_0.7.mat.csv};
        \addlegendentry{Setpoint}

        \addplot+[mark = none, style = solid, ultra thick] 
        table[x = time, y = vel, col sep = comma] 
        {control/csv/compare_control_pid_Simulink_single_pend_mp0.3_l2.25_PID_vel_steps_tune_scale_0.7.mat.csv};
        \addlegendentry{PID}

        \addplot+[mark = none, style = solid, ultra thick] 
        table[x = time, y = vel, col sep = comma] 
        {control/csv/compare_control_mpc_Simulink_single_pend_mp0.3_l2.25_PID_vel_steps_tune_scale_0.7.mat.csv};
        \addlegendentry{MPC}

        \addplot+[mark = none, style = dashed, ultra thick] 
        table[x = time, y = vel, col sep = comma] 
        {control/csv/compare_control_lqr_Simulink_single_pend_mp0.3_l2.25_PID_vel_steps_tune_scale_0.7.mat.csv};
        \addlegendentry{LQR}


        \end{axis}
    \end{tikzpicture} 
    \caption{Velocity step response comparison for different controllers
    ($l =$~\SI{2}{\meter}, $m_p =$~\SI{0.3}{\kilo\gram})}
    \label{fig:compare_control_vel}
\end{figure}


            \paragraph
            vel control

            MPC:
            vel weight = 2;
            theta weight = 0;
            dtheta weight = 10; 
            mv weight = 1;
            mvrate weight = 5;
            Ty = 5;
            Tu = 3;

            LQR:
            LQR.Q = diag([10 10 0 100]); 
            LQR.R = 8

            \begin{figure}[htb]
    \centering
    \begin{tikzpicture}
        \begin{axis}[            
            xlabel = Time,
            ylabel = Payload angle,
            x unit = \si{\second},
            y unit = \si{\degree},
            xmin = 0,   xmax = 16,
            ymin = -20,  ymax = 15,
            grid = major,
            legend cell align = left,
            legend pos = south east,
            grid style = dashed,
            legend style = {font = \scriptsize},
            label style = {font = \scriptsize},
            tick label style = {font = \scriptsize},
            width = 0.95\columnwidth,
            height = 0.5\columnwidth,
            % initialize Dark2
            cycle list/Dark2,
            % combine it with 'mark list*':
            cycle multiindex* list = {
                Dark2\nextlist
            }
        ]
                
        \pgfplotsset{cycle list shift = 1}

        \addplot+[mark = none, style = solid, ultra thick] 
        table[x = time, y = theta, col sep = comma] 
        {control/csv/compare_control_pid_Simulink_single_pend_mp0.3_l2.25_PID_vel_steps_tune_scale_0.7.mat.csv};
        \addlegendentry{PID}

        \addplot+[mark = none, style = solid, ultra thick] 
        table[x = time, y = theta, col sep = comma] 
        {control/csv/compare_control_mpc_Simulink_single_pend_mp0.3_l2.25_MPC_vel_steps_tune_scale_0.7.csv};
        \addlegendentry{MPC}

        \addplot+[mark = none, style = dashed, ultra thick] 
        table[x = time, y = theta, col sep = comma] 
        {control/csv/compare_control_lqr_Simulink_single_pend_mp0.3_l2.25_PID_vel_steps_tune_scale_0.7.mat.csv};
        \addlegendentry{LQR}

        \end{axis}
    \end{tikzpicture} 
    \caption{Payload angle comparison for different controllers
    ($l =$~\SI{2}{\meter}, $m_p =$~\SI{0.3}{\kilo\gram})}
    \label{fig:compare_control_theta}
\end{figure}


            \paragraph
            theta control

            \begin{figure}[htb]
    \centering
    \begin{tikzpicture}
        \begin{axis}[            
            xlabel = Time,
            ylabel = North velocity,
            x unit = \si{\second},
            y unit = \si{\metre/\second},
            xmin = 0,   xmax = 16,
            ymin = -0.1,  ymax = 2,
            grid = major,
            legend cell align = left,
            legend pos = north east,
            grid style = dashed,
            legend style = {font = \scriptsize},
            label style = {font = \scriptsize},
            tick label style = {font = \scriptsize},
            width = 0.95\columnwidth,
            height = 0.5\columnwidth,
            % initialize Dark2
            cycle list/Dark2,
            % combine it with 'mark list*':
            cycle multiindex* list = {
                Dark2\nextlist
            }
        ]
        
        \pgfplotsset{cycle list shift = 2}

        \addplot+[mark = none, style = solid, ultra thick] 
        table[x = time, y = acc_sp, col sep = comma] 
        {control/csv/compare_control_mpc_Simulink_single_pend_mp0.3_l2.25_MPC_vel_steps_tune_scale_0.7.csv};
        \addlegendentry{MPC}

        \addplot+[mark = none, style = dashed, ultra thick] 
        table[x = time, y = acc_sp, col sep = comma] 
        {control/csv/compare_control_lqr_Simulink_single_pend_mp0.3_l2.25_LQR_vel_steps_tune_scale_0.7.csv};
        \addlegendentry{LQR}


        \end{axis}
    \end{tikzpicture} 
    \caption{Acceleration setpoint commanded by different controllers for a velocity step input
    ($l =$~\SI{2}{\meter}, $m_p =$~\SI{0.3}{\kilo\gram}).}
    \label{fig:compare_control_acc_sp}
\end{figure}


            \paragraph
            acc-sp plot

            \begin{figure}[htb]
    \centering
    \begin{tikzpicture}
        \begin{axis}[            
            xlabel = Time,
            ylabel = North velocity,
            x unit = \si{\second},
            y unit = \si{\metre/\second},
            xmin = 0,   xmax = 16,
            ymin = -0.1,  ymax = 2.5,
            grid = major,
            legend cell align = left,
            legend pos = south east,
            grid style = dashed,
            legend style = {font = \scriptsize},
            label style = {font = \scriptsize},
            tick label style = {font = \scriptsize},
            width = 0.95\columnwidth,
            height = 0.5\columnwidth,
            % initialize Dark2
            cycle list/Dark2,
            % combine it with 'mark list*':
            cycle multiindex* list = {
                Dark2\nextlist
            }
        ]
        
        \addplot+[mark = none, style = solid, ultra thick] 
        table[x = time, y = vel_sp, col sep = comma] 
        {control/csv/compare_control_mpc_Simulink_single_pend_mp0.3_l2.25_MPC_vel_steps_tune_scale_0.7_wind_step_disturb.csv};
        \addlegendentry{Setpoint}

        \pgfplotsset{cycle list shift = 1} % Shift to skip \gls{PID} colour used in other plots

        \addplot+[mark = none, style = solid, ultra thick] 
        table[x = time, y = vel, col sep = comma] 
        {control/csv/compare_control_mpc_Simulink_single_pend_mp0.3_l2.25_MPC_vel_steps_tune_scale_0.7_wind_step_disturb.csv};
        \addlegendentry{MPC}

        \addplot+[mark = none, style = dashed, ultra thick] 
        table[x = time, y = vel, col sep = comma] 
        {control/csv/compare_control_lqr_Simulink_single_pend_mp0.3_l2.25_LQR_vel_steps_tune_scale_0.7_wind_step_disturb.csv};
        \addlegendentry{LQR}


        \end{axis}
    \end{tikzpicture} 
    \caption{Effect of an unmeasured step input disturbance.
    ($l =$~\SI{2}{\meter}, $m_p =$~\SI{0.3}{\kilo\gram}).}
    \label{fig:compare_control_vel_wind_disturbance}
\end{figure}


            \paragraph
            Wind disturbance wind speed~=~\SI{2}{\metre/\second}

            \begin{figure}[htb]
    \centering
    \begin{tikzpicture}
        \begin{axis}[            
            xlabel = Time,
            ylabel = Velocity,
            x unit = \si{\second},
            y unit = \si{\metre/\second},
            xmin = 0,   xmax = 16,
            ymin = -0.1,  ymax = 3,
            grid = major,
            legend cell align = left,
            legend pos = south east,
            grid style = dashed,
            legend style = {font = \scriptsize},
            label style = {font = \scriptsize},
            tick label style = {font = \scriptsize},
            width = 0.95\columnwidth,
            height = 0.5\columnwidth,
            % initialize Dark2
            cycle list/Dark2,
            % combine it with 'mark list*':
            cycle multiindex* list = {
                Dark2\nextlist
            }
        ]
        
        \addplot+[mark = none, style = solid, ultra thick] 
        table[x = time, y = vel_sp, col sep = comma] 
        {control/csv/compare_control_mpc_Simulink_single_pend_mp0.3_l2.25_MPC_vel_steps_tune_scale_0.7_wind_step_disturb.csv};
        \addlegendentry{Setpoint}

        \pgfplotsset{cycle list shift = 2} % Shift to skip PID colour used in other plots

        % \addplot+[mark = none, style = solid, ultra thick] 
        % table[x = time, y = vel, col sep = comma] 
        % {control/csv/compare_control_mpc_Simulink_single_pend_mp0.3_l2.25_MPC_vel_steps_tune_scale_0.7_wind_step_disturb.csv};
        % \addlegendentry{MPC}

        \addplot+[mark = none, style = dashed, ultra thick] 
        table[x = time, y = vel, col sep = comma] 
        {control/csv/compare_control_lqr_Simulink_single_pend_mp0.3_l2.25_LQR_vel_steps_tune_scale_0.7_wind_step_disturb_intg_weight_0.1.csv};
        \addlegendentry{Integrator weighting: 0.1}

        \addplot+[mark = none, style = solid, ultra thick] 
        table[x = time, y = vel, col sep = comma] 
        {control/csv/compare_control_lqr_Simulink_single_pend_mp0.3_l2.25_LQR_vel_steps_tune_scale_0.7_wind_step_disturb_intg_weight_1.csv};
        \addlegendentry{Integrator weighting: 1}

        \addplot+[mark = none, style = solid, ultra thick] 
        table[x = time, y = vel, col sep = comma] 
        {control/csv/compare_control_lqr_Simulink_single_pend_mp0.3_l2.25_LQR_vel_steps_tune_scale_0.7_wind_step_disturb_intg_weight_10.csv};
        \addlegendentry{Integrator weighting: 10}


        \end{axis}
    \end{tikzpicture} 
    \caption{Different LQR responses for different integrator gains
    ($l =$~\SI{2}{\meter}, $m_p =$~\SI{0.3}{\kilo\gram}).}
    \label{fig:improve_lqr_integrator}
\end{figure}


            \paragraph
            Improve integrator

            \begin{figure}
                \captionsetup[subfigure]{justification=centering}
                \centering  
                \begin{figure}[hb]
    \centering
    \begin{tikzpicture}
        \begin{axis}[            
            xlabel = Time,
            ylabel = North velocity,
            x unit = \si{\second},
            y unit = \si{\metre/\second},
            xmin = 0,   xmax = 16,
            ymin = -0.1,  ymax = 2.5,
            grid = major,
            legend cell align = left,
            legend pos = south east,
            grid style = dashed,
            legend style = {font = \scriptsize},
            label style = {font = \scriptsize},
            tick label style = {font = \scriptsize},
            width = 0.95\columnwidth,
            height = 0.5\columnwidth,
            % initialize Dark2
            cycle list/Dark2,
            % combine it with 'mark list*':
            cycle multiindex* list = {
                Dark2\nextlist
            }
        ]
        
        \addplot+[mark = none, style = solid, ultra thick] 
        table[x = time, y = vel_sp, col sep = comma] 
        {control/csv/compare_control_mpc_Simulink_single_pend_mp0.2_l1_MPC_vel_steps_tune_scale_0.7.csv};
        \addlegendentry{Setpoint}

        \addplot+[mark = none, style = solid, ultra thick] 
        table[x = time, y = vel, col sep = comma] 
        {control/csv/compare_control_pid_Simulink_single_pend_mp0.2_l1_PID_vel_steps_tune_scale_0.7.csv};
        \addlegendentry{PID}

        \addplot+[mark = none, style = solid, ultra thick] 
        table[x = time, y = vel, col sep = comma] 
        {control/csv/compare_control_mpc_Simulink_single_pend_mp0.2_l1_MPC_vel_steps_tune_scale_0.7.csv};
        \addlegendentry{MPC}

        \addplot+[mark = none, style = dashed, ultra thick] 
        table[x = time, y = vel, col sep = comma] 
        {control/csv/compare_control_lqr_Simulink_single_pend_mp0.2_l1_LQR_vel_steps_tune_scale_0.7.csv};
        \addlegendentry{LQR}


        \end{axis}
    \end{tikzpicture} 
    \caption{Velocity step response comparison of different controllers
    ($l =$~\SI{1}{\meter}, $m_p =$~\SI{0.2}{\kilo\gram}).}
    \label{fig:compare_control_vel_l1}
    
\end{figure}
 % subfigure
                \begin{figure}[hb]
    \centering
    \begin{tikzpicture}
        \begin{axis}[            
            xlabel = Time,
            ylabel = North velocity,
            x unit = \si{\second},
            y unit = \si{\metre/\second},
            xmin = 0,   xmax = 16,
            ymin = -0.1,  ymax = 2.5,
            grid = major,
            legend cell align = left,
            legend pos = south east,
            grid style = dashed,
            legend style = {font = \scriptsize},
            label style = {font = \scriptsize},
            tick label style = {font = \scriptsize},
            width = 0.95\columnwidth,
            height = 0.5\columnwidth,
            % initialize Dark2
            cycle list/Dark2,
            % combine it with 'mark list*':
            cycle multiindex* list = {
                Dark2\nextlist
            }
        ]
        
        \addplot+[mark = none, style = solid, ultra thick] 
        table[x = time, y = vel_sp, col sep = comma] 
        {control/csv/compare_control_mpc_Simulink_single_pend_mp0.2_l1_MPC_vel_steps_tune_scale_0.7.csv};
        \addlegendentry{Setpoint}

        \addplot+[mark = none, style = solid, ultra thick] 
        table[x = time, y = vel, col sep = comma] 
        {control/csv/compare_control_pid_Simulink_single_pend_mp0.2_l1_PID_vel_steps_tune_scale_0.7.csv};
        \addlegendentry{PID}

        \addplot+[mark = none, style = solid, ultra thick] 
        table[x = time, y = vel, col sep = comma] 
        {control/csv/compare_control_mpc_Simulink_single_pend_mp0.2_l1_MPC_vel_steps_tune_scale_0.7.csv};
        \addlegendentry{MPC}

        \addplot+[mark = none, style = dashed, ultra thick] 
        table[x = time, y = vel, col sep = comma] 
        {control/csv/compare_control_lqr_Simulink_single_pend_mp0.2_l1_LQR_vel_steps_tune_scale_0.7.csv};
        \addlegendentry{LQR}


        \end{axis}
    \end{tikzpicture} 
    \caption{Velocity step response comparison of different controllers
    ($l =$~\SI{1}{\meter}, $m_p =$~\SI{0.2}{\kilo\gram}).}
    \label{fig:compare_control_vel_l1}
    
\end{figure}
 % subfigure
                \begin{figure}[hb]
    \centering
    \begin{tikzpicture}
        \begin{axis}[            
            xlabel = Time,
            ylabel = North velocity,
            x unit = \si{\second},
            y unit = \si{\metre/\second},
            xmin = 0,   xmax = 16,
            ymin = -0.1,  ymax = 2.5,
            grid = major,
            legend cell align = left,
            legend pos = south east,
            grid style = dashed,
            legend style = {font = \scriptsize},
            label style = {font = \scriptsize},
            tick label style = {font = \scriptsize},
            width = 0.95\columnwidth,
            height = 0.5\columnwidth,
            % initialize Dark2
            cycle list/Dark2,
            % combine it with 'mark list*':
            cycle multiindex* list = {
                Dark2\nextlist
            }
        ]
        
        \addplot+[mark = none, style = solid, ultra thick] 
        table[x = time, y = vel_sp, col sep = comma] 
        {control/csv/compare_control_mpc_Simulink_single_pend_mp0.2_l1_MPC_vel_steps_tune_scale_0.7.csv};
        \addlegendentry{Setpoint}

        \addplot+[mark = none, style = solid, ultra thick] 
        table[x = time, y = vel, col sep = comma] 
        {control/csv/compare_control_pid_Simulink_single_pend_mp0.2_l1_PID_vel_steps_tune_scale_0.7.csv};
        \addlegendentry{PID}

        \addplot+[mark = none, style = solid, ultra thick] 
        table[x = time, y = vel, col sep = comma] 
        {control/csv/compare_control_mpc_Simulink_single_pend_mp0.2_l1_MPC_vel_steps_tune_scale_0.7.csv};
        \addlegendentry{MPC}

        \addplot+[mark = none, style = dashed, ultra thick] 
        table[x = time, y = vel, col sep = comma] 
        {control/csv/compare_control_lqr_Simulink_single_pend_mp0.2_l1_LQR_vel_steps_tune_scale_0.7.csv};
        \addlegendentry{LQR}


        \end{axis}
    \end{tikzpicture} 
    \caption{Velocity step response comparison of different controllers
    ($l =$~\SI{1}{\meter}, $m_p =$~\SI{0.2}{\kilo\gram}).}
    \label{fig:compare_control_vel_l1}
    
\end{figure}
 % subfigure
                \caption{}
                \label{fig:compare_control_different_payloads}  
            \end{figure}

            \paragraph
            Different payloads, compare tuning

            \begin{itemize}
                \item plot acc-sp
                \item plot \gls{MPC} vs \gls{LQR} v \gls{PID} (no wind) step = 2 m/s
                \item plot different system paramaters
                \item plot with wind disturbacne control
            \end{itemize}

        \FloatBarrier\subsection{Dynamic payload}

            \paragraph
            Intro

            \begin{figure}[htb]
    \centering
    \begin{tikzpicture}
        \begin{axis}[            
            xlabel = Time,
            ylabel = Velocity,
            x unit = \si{\second},
            y unit = \si{\metre/\second},
            xmin = 0,   xmax = 16,
            ymin = -0.1,  ymax = 2.7,
            grid = major,
            legend cell align = left,
            legend pos = south east,
            grid style = dashed,
            legend style = {font = \scriptsize},
            label style = {font = \scriptsize},
            tick label style = {font = \scriptsize},
            width = 0.95\columnwidth,
            height = 0.5\columnwidth,
            % initialize Dark2
            cycle list/Dark2,
            % combine it with 'mark list*':
            cycle multiindex* list = {
                Dark2\nextlist
            }
        ]
        
        \addplot+[mark = none, style = solid, ultra thick] 
        table[x = time, y = vel_sp, col sep = comma] 
        {control/csv/compare_control_pid_Simulink_single_pend_mp0.3_l2.25_PID_vel_steps_tune_scale_0.7.mat.csv};
        \addlegendentry{Setpoint}

        \addplot+[mark = none, style = solid, ultra thick] 
        table[x = time, y = vel, col sep = comma] 
        {control/csv/compare_control_pid_Simulink_single_pend_mp0.3_l2.25_PID_vel_steps_tune_scale_0.7.mat.csv};
        \addlegendentry{PID}

        \addplot+[mark = none, style = solid, ultra thick] 
        table[x = time, y = vel, col sep = comma] 
        {control/csv/compare_control_mpc_Simulink_single_pend_mp0.3_l2.25_PID_vel_steps_tune_scale_0.7.mat.csv};
        \addlegendentry{MPC}

        \addplot+[mark = none, style = dashed, ultra thick] 
        table[x = time, y = vel, col sep = comma] 
        {control/csv/compare_control_lqr_Simulink_single_pend_mp0.3_l2.25_PID_vel_steps_tune_scale_0.7.mat.csv};
        \addlegendentry{LQR}


        \end{axis}
    \end{tikzpicture} 
    \caption{Velocity step response comparison for different controllers
    ($l =$~\SI{2}{\meter}, $m_p =$~\SI{0.3}{\kilo\gram})}
    \label{fig:compare_control_vel}
\end{figure}


            \paragraph
            vel control

            \begin{figure}[htb]
    \centering
    \begin{tikzpicture}
        \begin{axis}[            
            xlabel = Time,
            ylabel = Payload angle,
            x unit = \si{\second},
            y unit = \si{\degree},
            xmin = 0,   xmax = 16,
            ymin = -20,  ymax = 15,
            grid = major,
            legend cell align = left,
            legend pos = south east,
            grid style = dashed,
            legend style = {font = \scriptsize},
            label style = {font = \scriptsize},
            tick label style = {font = \scriptsize},
            width = 0.95\columnwidth,
            height = 0.5\columnwidth,
            % initialize Dark2
            cycle list/Dark2,
            % combine it with 'mark list*':
            cycle multiindex* list = {
                Dark2\nextlist
            }
        ]
                
        \pgfplotsset{cycle list shift = 1}

        \addplot+[mark = none, style = solid, ultra thick] 
        table[x = time, y = theta, col sep = comma] 
        {control/csv/compare_control_pid_Simulink_single_pend_mp0.3_l2.25_PID_vel_steps_tune_scale_0.7.mat.csv};
        \addlegendentry{PID}

        \addplot+[mark = none, style = solid, ultra thick] 
        table[x = time, y = theta, col sep = comma] 
        {control/csv/compare_control_mpc_Simulink_single_pend_mp0.3_l2.25_MPC_vel_steps_tune_scale_0.7.csv};
        \addlegendentry{MPC}

        \addplot+[mark = none, style = dashed, ultra thick] 
        table[x = time, y = theta, col sep = comma] 
        {control/csv/compare_control_lqr_Simulink_single_pend_mp0.3_l2.25_PID_vel_steps_tune_scale_0.7.mat.csv};
        \addlegendentry{LQR}

        \end{axis}
    \end{tikzpicture} 
    \caption{Payload angle comparison for different controllers
    ($l =$~\SI{2}{\meter}, $m_p =$~\SI{0.3}{\kilo\gram})}
    \label{fig:compare_control_theta}
\end{figure}


            \paragraph
            theta control

            \begin{itemize}
                \item subplot prediction of data driven model. subplot prediction of white-box model
                \item plot \gls{MPC} vs \gls{LQR} v \gls{PID} (no wind)
            \end{itemize}

    \FloatBarrier\section{Conclusion}
    