\FloatBarrier\section{MPC} \label{sec:mpc}

    \paragraph
    \gls{MPC} refers to a family of controllers which apply control actions based on predictions with a separately identifiable model \cite{Garcia1989}.
    \gls{MPC} does not refer to a specific algorithm implementation, but rather to the general control approach.
    A specific \gls{MPC} implementation is dependant on the plant model representation \cite{Garcia1989}.
    In this section, an overview will be given of the specific MPC implementation used in this work.
    The MPC algorithm will be explained and the design process to tune this implementation will be discussed.
    Finally, the control response of the tuned MPC will be shown and discussed for a simulated system.
    
    % ?? Lit study: include different types of MPC, e.g. DMC, MAC
    % ?? different types of models \cite{}, 

    \FloatBarrier\subsection{Model}

        \paragraph
        In the proposed control architecture for a multirotor with an unknown suspended payload,
        the MPC is used with a estimated model from the data-driven techniques discussed in Chapter~\ref{chap:system_id}.
        DMDc and HAVOKc produce discrete, linear state-space models of the system dynamics.

        MIMO
        State space
        Discrete

        MV, MO, UO, UD,
        diagram of model



    \FloatBarrier\subsection{Receding horizon}

        \paragraph
        Diagram of prediction horizon and control horizon

        \paragraph
        controller desicion, ${z_k}^T = [ u( k | k )^T ~~ u( k+1 | k)^T ~~ ... ~~ u( k+p-1 | k )^T ] $

    \FloatBarrier\subsection{Algoriithm}

        \paragraph
        Different implementations. e.g. list
        Based on model
        Other MPC optimiser options.
        MATLAB chosen. Why MATLAB?
        
        \paragraph
        The \gls{MPC} implementation from the Model Predictive Control Toolbox\texttrademark ~solves the control optimisattion problem as a \gls{QP} at each time interval \cite{MPCtoolbox2019}.
        This QP usually consists of three features, namely,
        \begin{itemize}
            \item the objective function,
            \item the constraints, and
            \item the controller desion
        \end{itemize}
        The objective function provides a scalar value that quatifies the controller performance.
        The controller desicion is the \gls{MV} adjustments determined by the QP solver that minimise the objective function.
        The constraints are conditions that the controller decision need to satisfy, such as bounds on the MO, MV, and \gls{MV} increment values.

        \paragraph
        Constraints cause non-linear ..., 
        not applied

        \paragraph
        The objective function used by the Model Predictive Control Toolbox\texttrademark is documented well by the corresponding user manual \cite{MPCtoolbox2019}.
        An overview of this implementation, which is used in this work, will be given here. 
        The objective function consists of the sum of three terms that each quantify a specific aspect of the control performance,
        and is calculated as,
        \begin{equation}
            J(z_k) = J_y(z_k) + J_u(z_k) + J_{\Delta u}(z_k),
        \end{equation}
        where $z_k$ is the controller decison at time-step $k$.
        The three scalar performance measures are denoted as
        $J_y(z_k)$ for output setpoint tracking,
        $J_u(z_k)$ for \gls{MV} tracking, and
        $J_{\delta u}(z_k)$ for \gls{MV} move suppression
        Each performance measure includes weigths that balanse the competing objectives of the different terms.
        These weights need to be manually tuned for a desired controller performance.

        \paragraph
        \textbf{Output setpoint tracking}  \newline
        The performance measure of the output setpoint tracking is calulated as,
        \begin{equation}
            J_y(z_k) = \sum_{j = 1}^{n_y} ~ \sum_{i = 1}^{N_p} 
                \left\{ 
                    {w^y}_{j} 
                        \left[ 
                            r_j( k+i | k) - y_j( k+i | k)
                        \right]
                \right\}^2 ,
        \end{equation}
        where the symbols are denoted as, \newline
        \begin{tabular}{ l l } 
            $k$             & Current control interval time-step \\
            $n_y$           & Number of output variables \\ 
            $N_p$           & Prediction horizon \\ 
            $y_j( k+i | k)$ & Predicted value of $j$\textsuperscript{th} output variable at $i$\textsuperscript{th} time-step from $k$ \\
            $r_j( k+i | k)$ & Reference value of $j$\textsuperscript{th} output variable at $i$\textsuperscript{th} time-step from $k$ \\
            ${w^y}_{j}$     & Tuning weight for $j$\textsuperscript{th} output variable \\
        \end{tabular}

        \paragraph
        The controller receives the reference values, $r_j( k+i | k)$, for the prediction horizon starting at time-step $k$.
        Using the inernal plant model, the predicted output values, $y_j( k+i | k)$, is determined based on the controller decision, $z_k$.
        The values of $N_p$ and ${w^y}_{j}$ are design choises and are constant controller specifications.
        The value of $n_y$ is also constant and is determined from the plant model.

        \paragraph
        \textbf{Manipulated variable tracking}  \newline
        In some control applications, it is desirable to keep specific \gls{MV} variables close to a target value.
        In the multirotor use case, lower \gls{MV} values are prefered because this corresponds to lower energy use.
        The \gls{MV} target values are therefore set to zero for our use case. 
        The performance measure of the manipulated variable tracking is calculated as,
        \begin{equation}
            J_u(z_k) = \sum_{j = 1}^{n_u} ~ \sum_{i = 0}^{N_p - 1} 
                \left\{ 
                    {w^u}_{j} 
                        \left[ 
                            u_j( k+i | k) - u_{j,sp}( k+i | k)
                        \right]
                \right\}^2 ,
        \end{equation}
        where the symbols are denoted as, \newline
        \begin{tabular}{ l l } 
            $n_u$           & Number of manipulated variables \\ 
            $u_j( k+i | k)$ & Control decision of $j$\textsuperscript{th} \gls{MV} at $i$\textsuperscript{th} time-step from $k$ \\
            $u_{j,target}( k+i | k)$ & Target value of $j$\textsuperscript{th} \gls{MV} at $i$\textsuperscript{th} time-step from $k$ \\
            ${w^u}_{j}$   & Tuning weight for $j$\textsuperscript{th} \gls{MV} \\
        \end{tabular}

        \paragraph
        The desired $u_{j,target}( k+i | k)$ values can be received for the prediction horizon starting at time-step $k$.
        However, for our use case, all $u_{j,target}( k+i | k)$ values are constant and zero.
        The value of $n_u$ is fixed by the plant model.
        The ${w^u}_{j}$ values are also constant and are determined as a design choise.
        
        \paragraph
        \textbf{Manipulated variable increment suppression}  \newline
        Large increments or moves in the \gls{MV} values is often undesirable for good control performance.
        In the multirotor use case, large increments of the acceleration \gls{MV}s result in large jerks which may cause the system to go beyond the accurate vector subspace of the plant model.
        High frequency moves in the acceleration \gls{MV}s may also cause jittery flight dynamics because acceleration setpoint changes correspond to attidue changes.
        The performance measure of \gls{MV} tracking is calulated as,
        \begin{equation}
            J_\Delta u (z_k) = \sum_{j = 1}^{n_u} ~ \sum_{i = 0}^{N_p - 1} 
                \left\{ 
                    {w^{\Delta u}}_{j} 
                        \left[ 
                            u_j( k+i | k) - u_j( k+i-1 | k)
                        \right]
                \right\}^2 ,
        \end{equation}
        where the symbols are denoted as, \newline
        \begin{tabular}{ l l } 
            ${w^u}_{j}$   & Tuning weight for movement in the $j$\textsuperscript{th} \gls{MV} \\
        \end{tabular}

        \paragraph
        Explain


% ?? convert acronyms to \gls{}

    \FloatBarrier\subsection{Integral action}
        
        \paragraph
        

    \FloatBarrier\subsection{Tuning}

        \paragraph
        In the controller tuning process, a large value of ${w^y}_{j}$ will correspond to more agrresive maneuvers of the $j$\textsuperscript{th} OV, because of the tracking error of that OV will be heavily penalised.
        For the tuning process, small values of ${w^u}_{j}$, correspond to aggresive maneuvers, because the values of the \gls{MV}s are not heavily penalised in the objective function. 

        \paragraph
        The value of $N_p$ determines the number of decision values in $z_k$.
        Therefore the computational complexity of the QP problem increases with larger values of $N_p$.
        However, if the 
        Hence, the value of $N_p$ needs
        It is often state as a 

    MATLAB
    QP solver
    C++ generation

    \begin{figure}[htb]
    \centering
    \begin{tikzpicture}
        \begin{axis}[            
            xlabel = Time,
            ylabel = North velocity,
            x unit = \si{\second},
            y unit = \si{\metre/\second},
            xmin = 0,   xmax = 16,
            ymin = -0.1,  ymax = 2.4,
            grid = major,
            legend cell align = left,
            legend pos = south east,
            grid style = dashed,
            legend style = {font = \scriptsize},
            label style = {font = \scriptsize},
            tick label style = {font = \scriptsize},
            width = 0.95\columnwidth,
            height = 0.5\columnwidth,
            % initialize Dark2
            cycle list/Dark2,
            % combine it with 'mark list*':
            cycle multiindex* list = {
                Dark2\nextlist
            }
        ]
        
        \addplot+[mark = none, style = solid, ultra thick] 
        table[x = time, y = vel_sp, col sep = comma] 
        {control/csv/compare_control_mpc_Simulink_single_pend_mp0.3_l2.25_MPC_vel_steps_tune_scale_0.7.csv};
        \addlegendentry{Setpoint}

        \addplot+[mark = none, style = solid, ultra thick] 
        table[x = time, y = vel, col sep = comma] 
        {control/csv/compare_control_mpc_Simulink_single_pend_mp0.1_l0.5_MPC_vel_steps_tune_scale_0.7.csv};
        \addlegendentry{($l =$~\SI{0.5}{\meter}, $m_p =$~\SI{0.1}{\kilo\gram})}

        \addplot+[mark = none, style = solid, ultra thick] 
        table[x = time, y = vel, col sep = comma] 
        {control/csv/compare_control_mpc_Simulink_single_pend_mp0.2_l1_MPC_vel_steps_tune_scale_0.7.csv};
        \addlegendentry{($l =$~\SI{1}{\meter}, $m_p =$~\SI{0.2}{\kilo\gram})}

        \addplot+[mark = none, style = solid, ultra thick] 
        table[x = time, y = vel, col sep = comma] 
        {control/csv/compare_control_mpc_Simulink_single_pend_mp0.3_l2.25_MPC_vel_steps_tune_scale_0.7.csv};
        \addlegendentry{($l =$~\SI{2}{\meter}, $m_p =$~\SI{0.3}{\kilo\gram})}

        \end{axis}
    \end{tikzpicture} 
    \caption{\gls{MPC} velocity step responses with different payloads}
    \label{fig:mpc_tuning_plot}
\end{figure}
