\FloatBarrier\section{MPC} \label{sec:mpc}

    \paragraph
    \gls{MPC} refers to a family of controllers which apply control actions based on predictions with a separately identifiable model \cite{Garcia1989}.
    \gls{MPC} does not refer to a specific algorithm implementation, but rather to the general control approach.
    A specific \gls{MPC} implementation is dependant on the plant model representation \cite{Garcia1989}.
    In this section, an overview will be given of the specific MPC implementation used in this work.
    The MPC algorithm will be explained and the design process to tune this implementation will be discussed.
    Finally, the control response of the tuned MPC will be shown and discussed for a simulated system.
    
    % ?? Lit study: include different types of MPC, e.g. DMC, MAC
    % ?? different types of models \cite{}, 

    \FloatBarrier\section{Model}

        \paragraph
        In the proposed control architecture for a multirotor with an unknown suspended payload,
        the MPC is used with a estimated model from the data-driven techniques discussed in Chapter~\ref{chap:system_id}.
        DMDc and HAVOKc produce discrete, linear state-space models of the system dynamics.

        MIMO
        State space
        Discrete

    \FloatBarrier\section{Algoriithm}

        \paragraph
        Different implementations. e.g. list
        Based on model
        Other MPC optimiser options.
        MATLAB chosen.
        Why MATLAB?
        QP
        Cost function


    \FloatBarrier\section{}
    \FloatBarrier\section{Tuning}

    MATLAB
    QP solver
    C++ generation

    \input{control/plots/mpc_tuning_plot.tex}