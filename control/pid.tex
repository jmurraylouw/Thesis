\FloatBarrier\section{Cascaded PID control} \label{sec:cascaded_pid}

    \paragraph
    % This section gives a brief overview of the \gls{PID} control architecture design process used in this work.
    % The \gls{PID} controllers will be designed based on a linearized model of the multirotor without a payload.
    % The design process used to determine the controller gains will also be discussed in this section.
    % Finally the controllers will be implemented with MATLAB\textsuperscript{\textregistered} and Simulink\texttrademark~and are verified with practical flight data.
    \gls{PID} control is a popular linear control technique that applies a control signal proportional to the error signal, 
    the integral of the error signal, 
    and the derivative of the error signal.
    This is the default control architecture used in the PX4 flight-stack \cite{Meier2015}.
    PX4 was chosen as the multirotor flight-stack because it is open-source and widely used in industry and research.
    The \gls{PID} implementation of PX4 does not provide active swing-damping of the payload, however, it is used in the system identification flight stage discussed in Chapter~\ref{chap:system_id}.
    
    \paragraph
    The default PX4 control architecture consists of multiple cascaded \gls{PID} controller loops.
    This is divided into two main sections, the inner-loop attitude controllers and the outer-loop translational controllers.
    Figure~\ref{fig:pid_architecture} shows a high-level overview of the PX4 control architecture without showing state feedback.
    
    \begin{figure}[htb]
      \centering
    
      % Define distances
      \def\blockheight{7em}
      \def\largeblockheight{1.45*\blockheight}
      \def\blockwidth{4.1em}
      \def\nodesep{1.25*\blockwidth}
    
      \footnotesize
      \begin{tikzpicture}[>={Stealth[inset=0pt,length=3pt,angle'=60,round]},
          simple_block/.style = {draw, fill=light_blue, text centered, inner sep=0pt, text width=\blockwidth, minimum height=\blockheight},
          larger_block/.style = {draw, fill=light_blue, text centered, inner sep=0pt, text width=\blockwidth, minimum height=\largeblockheight, anchor=north},
          no_border_block/.style = {text centered, inner ysep=0.5em},
          frame_block/.style = {draw, text centered, inner ysep=2.25em, inner xsep=0.5em, dashed}
      ]
        
        %===================================================================================
        
        % Reference
        \node (ref) [] {};
        
        % Position controller
        \path (ref.east)+(0.85*\nodesep,0) node (pos_ctrl) [simple_block] {Position Control};
        \path (pos_ctrl.north)+(0,0) node (pos_ctrl_text) [no_border_block, anchor=south] {\textbf{P}};
        \path (pos_ctrl.south)+(0.65*\nodesep, -0.075*\blockheight) node (pos_ctrl_freq_text) [no_border_block, anchor=north] {50 Hz};
        
        % Velocity controller
        \path (pos_ctrl.east)+(0.85*\nodesep,0) node (vel_ctrl) [simple_block] {Velocity Control};
        \path (vel_ctrl.north)+(0,0) node (vel_ctrl_text) [no_border_block, anchor=south] {\textbf{PID}};
        
        % Acceleration to Attitude
        \path (vel_ctrl.north east)+(1.1*\nodesep,0) node (force_2_att) [larger_block, text width=5.6em] {Acceleration and Yaw to Attitude};
      
        % Inertial Frame
        \node[fit=(pos_ctrl) (vel_ctrl)] (inertial) [frame_block] {};
        \path (inertial.south)+(0,0) node (inertial_text) [no_border_block, anchor=north] {Inertial Frame};
        
        % Angle controller
        \path (force_2_att.north east)+(0.95*\nodesep,0) node (angle_ctrl) [simple_block, anchor=north] {Angle Control};
        \path (angle_ctrl.north)+(0,0) node (angle_ctrl_text) [no_border_block, anchor=south] {\textbf{P}};
        \path (angle_ctrl.south)+(0, -0.075*\blockheight) node (angle_ctrl_freq_text) [no_border_block, anchor=north] {250 Hz};
        
        % Angular Rate controller
        \path (angle_ctrl.east)+(0.9*\nodesep,0) node (ang_vel_ctrl) [simple_block] {Angular Rate Control};
        \path (ang_vel_ctrl.north)+(0,0) node (ang_vel_ctrl_text) [no_border_block, anchor=south] {\textbf{PID}};
        \path (ang_vel_ctrl.south)+(0, -0.075*\blockheight) node (ang_vel_ctrl_freq_text) [no_border_block, anchor=north] {1 kHz};
        
        % Mixer
        \path (ang_vel_ctrl.north east)+(\nodesep,0) node (mixer) [larger_block, text width=3em] {Mixer};
        
        % Body Frame
        \node[fit=(angle_ctrl) (ang_vel_ctrl)] (body) [frame_block] {};
        \path (body.south)+(0,0) node (body_text) [no_border_block, anchor=north] {Body Frame};
        
        \path (mixer.east)+(0.5*\nodesep,0) node (cmd) [] {};
        
        %===================================================================================
        
        \path[draw,->] (ref.east) -- node[anchor=south, pos=0.05, align=left] {$\bm{X}_\text{sp}$} (pos_ctrl.west);
        
        \path[draw,->] (pos_ctrl.east) -- node[anchor=south] {$\bm{V}_\text{sp}$} (vel_ctrl.west);
        
        \path[draw,->] (vel_ctrl.east) -- node[anchor=south, pos=0.65] {$\bm{A}_\text{sp}$} (vel_ctrl.east -| force_2_att.west);
        \path[draw,->] ([yshift=-0.25*\largeblockheight]force_2_att.west -| ref.east) -- node[anchor=south, pos=0] {$\psi_\text{sp}$} ([yshift=-0.25*\largeblockheight]force_2_att.west);
        
        \path[draw,->] (force_2_att.east |- angle_ctrl.west) -- node[anchor=south, pos=0.4] {$\bm{q}_\text{sp}$} (angle_ctrl.west);
        \path[draw,->] ([yshift=-0.25*\largeblockheight]force_2_att.east) -- node[text centered, inner sep=2pt, anchor=south, pos=0.915] {$\delta_{T_\text{sp}}$} ([yshift=-0.25*\largeblockheight]mixer.west);
        
        \path[draw,->] (angle_ctrl.east) -- node[anchor=south] {$\bm{\Omega}_\text{sp}$} (ang_vel_ctrl.west);
        
        \path[draw,->] ([yshift=0.25*\blockheight]ang_vel_ctrl.east) -- node[text centered, inner sep=2pt, anchor=south, pos=0.6] {$\delta_{A_\text{sp}}$} ([yshift=0.25*\blockheight]ang_vel_ctrl.east -| mixer.west);
        \path[draw,->] (ang_vel_ctrl.east) -- node[text centered, inner sep=2pt, anchor=south, pos=0.6] {$\delta_{E_\text{sp}}$} (ang_vel_ctrl.east -| mixer.west);
        \path[draw,->] ([yshift=-0.25*\blockheight]ang_vel_ctrl.east) -- node[text centered, inner sep=2pt, anchor=south, pos=0.6] {$\delta_{R_\text{sp}}$} ([yshift=-0.25*\blockheight]ang_vel_ctrl.east -| mixer.west);
        
        \path[draw,->] (mixer.east) -- node[text centered, inner sep=5pt, anchor=south] {$\bm{T}_\text{sp}$} (cmd.west);
      
      \end{tikzpicture}
      
      \caption{Cascaded PID control architecture of PX4 \cite{PX4_controller_diagrams}}
      \label{fig:pid_architecture}

    \end{figure}
    

    \paragraph
    The setpoint vectors in Figure~\ref{fig:pid_architecture} are denoted by 
    $\bm{X}_{sp}$ for position, 
    $\bm{V}_{sp}$ for velocity, 
    $\bm{A}_{sp}$ for acceleration, 
    $\bm{q}_{sp}$ for the attitude quaternion, 
    $\bm{\Omega}_{sp}$ for angular rate, and
    $\bm{T}_{sp}$ for motor thrust.
    The virtual actuator commands are denoted by 
    $\delta_{A_{sp}}$,
    $\delta_{E_{sp}}$,
    $\delta_{R_{sp}}$, and
    $\delta_{T_{sp}}$,
    for the virtual 
    aileron,
    elevator,
    rudder, and 
    thrust commands.

    \paragraph
    The inner-most control element is the mixer, which converts virtual actuator commands to actuator thrust commands.
    % This is achieved with simple matrix multiplication.
    The attitude controller includes the angle and angular rate controllers, which send commands to the mixer.
    The translational controller consists of the position and velocity controllers, which send commands to the attitude controller.
    The rate of each controller is also shown in Figure~\ref{fig:pid_architecture}.
    Note that the outer-loop controllers deal with slow dynamics and therefore run at slower rates than the inner-loop controllers.
    % The inner-loop controllers deal with the fastest dynamics and are required to run at the highest rate.
    % The translation controllers control slower dynamics. hence they run at a lower rate.

    \FloatBarrier\subsection{Angular rate controller}

        \paragraph
        Three separate linear \gls{PID} controllers are used to control angular rates in the pitch, roll, and yaw axes of the multirotor body frame.
        The angular rate controller receives angular rate estimates from the PX4 state estimator and outputs virtual actuator commands to the mixer.
        Figure~\ref{fig:angular_rate_controller} shows a diagram of the pitch angular rate controller.
        The yaw and roll controllers are not shown here, but they replicate the same structure.
        
        \input{control/diagrams/angular_rate_controller.tex}

        \paragraph
        In Figure~\ref{fig:angular_rate_controller},
        $P_{\Omega_X}$,
        $I_{\Omega_X}$, and
        $D_{\Omega_X}$ are the \gls{PID} gains for the pitch angular rate controller.
        As defined in Chapter~\ref{chap:modelling}, 
        $\Omega_{B_X}$ is the pitch angular rate in the body axis,  
        $\Omega_{B_{X,{sp}}}$ is the pitch angular rate setpoint, and
        $\delta_{E_{sp}}$ is the virtual elevator setpoint.
        
        \paragraph
        Some elements improve the practical implementation of the controller, but have a negligible effect on the design of the gains, namely:
        \begin{itemize}
            \item A \gls{LPF} is added in the derivative path to reduce the effect of sensor noise.
            \item The derivative path is implemented on the plant output signal, instead of the error signal to eliminate derivative kick.
            \item Saturation is applied to the integral path to eliminate integral wind-up.
            \item The control signal is saturated to avoid dangerously large setpoint commands.
        \end{itemize}
        These effects are also described in detail by \cite{Erasmus2020}.
        
        \paragraph
        Classical control theory was used by \cite{Grobler2020} to design the controller gains of the practical multirotor, Honeybee.
        This same process is also explained in detail by \cite{Erasmus2020}.
        The controller gains were designed for a transient response that is as fast as possible while retaining fast disturbance rejection, and minimal overshoot.            
        It was determined by \cite{Grobler2020} that the default PX4 angular rate controller gains of the ZMR250 airframe provide excellent control for Honeybee and satisfy these design requirements.
        ZMR250 refers to a popular racing drone design.
        The practical system will be discussed in Chapter~\ref{chap:exp_design}.

        \paragraph
        For a \SI{1}{\radian/\second} step response, the pitch rate controllers result in a 
        3.6~\% overshoot, 
        \SI{0.024}{\second} rise time, 
        \SI{0.8}{\second} 2~\% settling time, 
        and \SI{138}{\radian/\second} bandwidth \cite{Grobler2020}.
        These gains are well suited for Honeybee and will also be implemented in this work.
        The default roll-rate and yaw-rate controller gains are also implemented for the same reason.
        These gains are documented in Appendix~\ref{appen:pid_gains}.

    \FloatBarrier\subsection{Angle controller}

        \paragraph
        The pitch, roll, and yaw angles are controlled by the angle controller in the body frame.
        A quaternion based controller is implemented by PX4 for angle control based on work by \cite{Brescianini2013}.
        The structure and design of this controller is well explained by \cite{Slabber2020}, \cite{Erasmus2020} and \cite{Grobler2020} but is only briefly discussed here.
        
        \paragraph
        For the control law proposed by \cite{Brescianini2013}, the error quaternion is calculated as:
        \begin{equation}
            \bm{q}_e = \bm{q}^{-1} \cdot \bm{q}_{sp},
        \end{equation}
        where $\bm{q}_e$ is the quaternion error, 
        $\bm{q}$ is the current attitude quaternion of the multirotor,  
        and $\bm{q}_{sp}$ is the attitude quaternion setpoint.
        The resulting control law is given as:
        \begin{equation} \label{eq:angle_law}
            \bm{\Omega}_{sp} = 2 \bm{P}_{q} \phantom{.} \mbox{sgn}(q_{0_e}) \bm{q}_{v_e} , 
            \qquad \qquad \mbox{sgn}(q_{0_e}) = 
            \begin{cases}
                1,  & \quad q_{0_e}  \geq   0 \\
                -1, & \quad q_{0_e}  <      0
            \end{cases} ,
        \end{equation}
        where 
        $\bm{\Omega}_{sp}$ is a vector of the roll, pitch, and yaw angular rate setpoints,
        $\bm{P}_{q}$ is a vector of the corresponding proportional gains,
        $q_{0_e}$ is the error of the quaternion magnitude component, and
        $\bm{q}_{v_e}$ is the error of the vector component of the quaternion.
        The implementation of this control law is illustrated in Figure~\ref{fig:angle_controller} for a single element of $\bm{\Omega}_{sp}$.

        \begin{figure}[htb]
    \centering
    \footnotesize
  
    % Define distances
    \def\blockheight{3em}
    \def\blockwidth{3em}
    \def\nodesep{2*\blockwidth}
  
    \begin{tikzpicture}[>={Stealth[inset=0pt,length=3pt,angle'=60,round]},
      simple_block/.style = {draw, fill=light_blue, text centered, inner sep=0pt, minimum width=\blockwidth, minimum height=\blockheight},
      text_block/.style = {draw, fill=light_blue, text centered, inner sep=0pt, text width=2*\blockwidth, minimum height=\blockheight},
      frame_block/.style = {draw, text centered, inner ysep=1.25em, inner xsep=1.25em, dashed},
      sat_block/.style = {draw, minimum width=\blockwidth, minimum height=\blockheight, path picture = 
          {
          % Get the width and height of the path picture node
          \pgfpointdiff{\pgfpointanchor{path picture bounding box}{north east}}%
              {\pgfpointanchor{path picture bounding box}{south west}}
          \pgfgetlastxy\x\y
          % Scale the x and y vectors so that the range
          % -1 to 1 is slightly shorter than the size of the node
          \tikzset{x=\x*.4, y=\y*.4}
          %
          % Draw annotation
          \draw (-1,0) -- (1,0) (0,-1) -- (0,1); 
          \draw (-1,-.6) -- (-.6,-.6) -- (.6,.6) -- (1,.6);
          }
      },
      gain/.style = {draw, fill=light_blue, inner sep=0pt, isosceles triangle, minimum height=\blockheight, isosceles triangle apex angle=60},
    ]
  
      %===============================================================================
      \node[text centered, yshift=0.25*\blockheight] (ref) {};
  
      % Controller
      \path(ref)+(0.75*\nodesep,-0.25*\blockheight) node (err_quat) [text_block] {Error quaternion};
      \path(err_quat)+(1.5*\nodesep,0) node (select_axis) [text_block] {Extract component};
      \path(select_axis)+(0, 1.5*\blockheight) node (select_mag) [text_block] {Extract magnitude};
      \path(select_mag)+(1.1*\nodesep,0) node (sgn) [simple_block] {sgn};
      \path(select_axis)+(1.9*\nodesep, 0.25*\blockheight) node (mult) [simple_block] {$\times$};
      \path(mult)+(0.65*\nodesep, 0) node (p_gain) [gain] {\scriptsize $2 P$};
      \path(p_gain)+(0.75*\nodesep, 0) node (sat_ctrl) [sat_block] {};
      \path(sat_ctrl)+(0.75*\nodesep, 0) node (output) [] {};
  
      %===============================================================================
      \path[draw,->] (ref.east) -- node[text centered, anchor=south, pos=0.1] {$\bm{q}_\text{sp}$} ([yshift=0.25*\blockheight]err_quat.west);
      \path[draw,->] ([yshift=-0.25*\blockheight]err_quat.west -| ref.east) -- node[text centered, anchor=north, pos=0.1] {$\bm{q}$} ([yshift=-0.25*\blockheight]err_quat.west);
      \path[draw,->] (err_quat.east) -- node[text centered, anchor=south,pos=0.4] {$\bm{q}_\text{e}$} (select_axis.west);
      \path[draw,->]([xshift=0.3*\nodesep]err_quat.east) |- (select_mag.west);
      \path[draw,->] (select_mag.east) -- node[text centered, anchor=south] {$q_{0_\text{e}}$} (sgn.west);
      \path[draw,->] (sgn.east) -- ([xshift=0.15*\nodesep]sgn.east) |- ([yshift=0.25*\blockheight]mult.west);
      \path[draw,->] (select_axis.east) -- node[text centered, anchor=south, pos=0.45]{$q_{j_\text{e}}$} ([yshift=-0.25*\blockheight]mult.west);
      \path[draw,->] (mult.east) -- (p_gain.west);
      \path[draw,->] (p_gain.east) -- (sat_ctrl.west);
      \path[draw,->] (sat_ctrl.east) -- node[text centered, anchor=south] {$\Omega_\text{sp}$} (output.west);
    \end{tikzpicture}

    \caption{Quaternion based angle controller \cite{PX4_controller_diagrams}}
    \label{fig:angle_controller}
  
  \end{figure}
        
        \paragraph
        In Figure~\ref{fig:angle_controller}, the $j$ subscript denotes a specific element in the vectors $\bm{q}_e$, $\bm{P}_{q}$, and $\bm{\Omega}_{sp}$, where $j = \{ 1, 2, 3 \}$.
        The attitude quaternion setpoint is taken as input, 
        the current attitude is received from the PX4 estimator, and the resulting angular rate setpoint is produced as the control signal.

        \paragraph
        % An integrator is not required in this controller, 
        % because it is placed between the velocity and angular rate controllers, which both include integrators to reject steady-state disturbances \cite{Erasmus2020}.
        Only a proportional term is applied in this control law.
        The proportional gains were designed for Honeybee by \cite{Grobler2020} for a transient response that is faster than the default ZMR250 response.
        These gains are documented in Appendix~\ref{appen:pid_gains}.
        This results in a \SI{1}{\radian} step response with 
        little to no overshoot, 
        \SI{0.3}{\second} rise time, 
        \SI{0.47}{\second} for a 2~\% settling time, 
        and \SI{11}{\radian/\second} bandwidth \cite{Grobler2020}.
        This has a large time-scale separation from the angular rate controller, which has a bandwidth of \SI{138}{\radian/\second}.
        A fast angle response is desired in this work, hence the same gains designed by \cite{Grobler2020} will be used.

    \FloatBarrier\subsection{Translational controller}

        \paragraph
        The translational controller consists of the position and velocity controllers.
        Position control is not considered in this work, therefore it will not be discussed in this section.
        A diagram of the North velocity controller is shown in Figure~\ref{fig:velocity_controller}.
        The West and Down velocity controllers duplicate the same configuration.
        
        % From: https://github.com/PX4/PX4-user_guide/blob/master/assets/diagrams/mc_control_arch_tikz.tex
\begin{figure}[ht]
  \centering

  % Define distances
  \def\blockheight{2.2em}
  \def\blockwidth{2.2em}
  \def\nodesep{1.5*\blockwidth}

  \begin{tikzpicture}[>={Stealth[inset=0pt,length=3pt,angle'=60,round]},
    simple_block/.style = {draw, fill=light_blue, text centered, inner sep=0pt, minimum width=\blockwidth, minimum height=\blockheight},
    frame_block/.style = {draw, text centered, inner ysep=1.25em, inner xsep=1.25em, dashed},
    port/.style = {inner sep=0pt, font=\tiny},
    sum/.style n args = {4}{draw, circle, minimum size=1.5em, alias=sum,
      append after command = {
        node at (sum.north) [port, below=1pt] {$#1$}
        node at (sum.west) [port, right=1pt] {$#2$}
        node at (sum.south) [port, above=1pt] {$#3$}
        node at (sum.east) [port, left=1pt] {$#4$}
      }
    },
    sat_block/.style = {draw, minimum width=\blockwidth, minimum height=\blockheight, path picture = 
      {
      % Get the width and height of the path picture node
      \pgfpointdiff{\pgfpointanchor{path picture bounding box}{north east}}%
          {\pgfpointanchor{path picture bounding box}{south west}}
      \pgfgetlastxy\x\y
      % Scale the x and y vectors so that the range
      % -1 to 1 is slightly shorter than the size of the node
      \tikzset{x=\x*.4, y=\y*.4}
      %
      % Draw annotation
      \draw (-1,0) -- (1,0) (0,-1) -- (0,1); 
      \draw (-1,-.6) -- (-.6,-.6) -- (.6,.6) -- (1,.6);
      }
    },
    sat_block_int/.style = {draw, minimum width=\blockwidth, minimum height=\blockheight, path picture = 
      {
      % Get the width and height of the path picture node
      \pgfpointdiff{\pgfpointanchor{path picture bounding box}{north east}}%
          {\pgfpointanchor{path picture bounding box}{south west}}
      \pgfgetlastxy\x\y
      % Scale the x and y vectors so that the range
      % -1 to 1 is slightly shorter than the size of the node
      \tikzset{x=\x*.4, y=\y*.4}
      %
      % Draw annotation
      \draw (-.8,-.9) -- (-.6,-.9) -- (.1,.5) -- (.3,.5);
      }
    }
  ]
    %===================================================================================
    \node[text centered] (ref) {};

    % Controller
    \path(ref)+(\nodesep,0) node (ref_sum) [sum={}{+}{-}{}] {};
  %   \path(ref_sum)+(\nodesep,0) node (ctrl_k) [simple_block] {$K$};
    \path(ref_sum)+(2*\nodesep,0) node (ctrl_i) [sat_block_int, fill=light_blue, text centered, inner sep=0pt] {$~~\int$};
    \path(ctrl_i)+(\nodesep,0) node (ctrl_i_gain) [simple_block] {$I_{V_N}$};
  %   \path(ref_sum)+(\nodesep,-1.5*\blockheight) node (ctrl_kd) [simple_block] {$K$};
    \path(ref_sum)+(\nodesep,-1.5*\blockheight) node (ctrl_d_filt) [simple_block] {LPF};
    \path(ref_sum)+(2*\nodesep,-1.5*\blockheight) node (ctrl_d) [simple_block] {$\frac{\partial}{\partial t}$};
    \path(ctrl_d)+(\nodesep,0) node (ctrl_d_gain) [simple_block] {$D_{V_N}$};
    \path(ref_sum)+(3*\nodesep,1.5*\blockheight) node (ctrl_p_gain) [simple_block] {$P_{V_N}$};
    \path(ctrl_i_gain)+(\nodesep,0) node (ctrl_sum) [sum={+}{+}{-}{}] {};
    \path(ctrl_sum)+(\nodesep,0) node (sat_ctrl) [sat_block] {};
    \path(sat_ctrl.east)+(1*\nodesep,0) node (ctrl) [] {};

    \path(ref_sum.south |- ctrl_d_filt.west)+(-\nodesep,0) node(output) [] {};

    %===================================================================================
    \path[draw,->] (ref.east) -- node[anchor=south, pos=0] {$V_{N_{sp}}$} (ref_sum.west);

    \path[draw,->] (ref_sum.east) -- (ctrl_i.west);
  %   \path[draw,->] (ctrl_k.east) -- (ctrl_i.west);
    \path[draw,->] (output.east) |- node[anchor=south, pos=0] {$V_N$} (ctrl_d_filt.west);
  %   \path[draw,->] (ctrl_kd.east) -- (ctrl_d_filt.west);
    \path[draw,->] (output.east -| ref_sum.south) -- (ref_sum.south);
    \path[draw,->] ([xshift=0.65*\nodesep]ref_sum.east) |- (ctrl_p_gain.west);
    \path[draw,->] (ctrl_i.east) -- (ctrl_i_gain.west);
    \path[draw,->] (ctrl_d_filt.east) -- (ctrl_d.west);
    \path[draw,->] (ctrl_d.east) -- (ctrl_d_gain.west);
    \path[draw,->] (ctrl_d_gain.east) -| (ctrl_sum.south);
    \path[draw,->] (ctrl_i_gain.east) -- (ctrl_sum.west);
    \path[draw,->] (ctrl_p_gain.east) -| (ctrl_sum.north);
    \path[draw,->] (ctrl_sum.east) -- (sat_ctrl.west);
    \path[draw,->] (sat_ctrl.east) -- node[anchor=south] {$A_{N_{sp}}$} (ctrl.west);
  
  \end{tikzpicture}

  \caption{Velocity controller diagram \cite{PX4_controller_diagrams}}
  \label{fig:velocity_controller}

\end{figure}

        \paragraph
        To drive the multirotor to a given velocity setpoint, the velocity controller commands an acceleration setpoint in the inertial frame.
        This acceleration setpoint is transformed to an attitude setpoint which is used by the angle controller.
        This transformation is based on work done by \cite{Mellinger2011}. 

        \paragraph
        Recall from Chapter~\ref{chap:system_id}, that the cascaded \gls{PID} controller is used for velocity step inputs in the training data flight stage for system identification.
        The system identification methods produce linear models, which are used in swing damping controllers to minimise the payload angles during flight.
        The velocity controller gains used for Honeybee by \cite{Grobler2020} result in aggressive velocity responses, which produce large payload swing angles.
        Such large payload angles are undesirable in safe flights.
        Therefore the velocity controller gains are redesigned for a slower transient response and smaller payload swing angles.

        \paragraph
        The derivation of the transfer function, $G_{V_N}(s)$, of the North velocity dynamics 
        of a multirotor with a suspended payload was derived by \cite{Slabber2020} and is described here briefly.
        Firstly, the non-linear dynamics were derived with Lagrangian mechanics as described in Chapter~\ref{chap:system_id}.
        The equations were then linearised around hover with the small-angle approximation.
        The linearised equations for the longitudinal or North velocity dynamics can be presented in the state-space form as,
        \begin{align}
            \dot{\textbf{X}}_{long} &= \textbf{A}_{long} \textbf{X}_{long} + \textbf{B}_{long} \mbox{U}_{long}~\mbox{and}\\
            \mbox{Y}_{long} &= \textbf{C}_{long} \textbf{X}_{long},
        \end{align}
        where 
        \begin{align}
            \renewcommand\arraystretch{3}
            \textbf{X}_{long} & = 
            \begin{bmatrix}
                V_N~~\dot{\theta}~~\theta
            \end{bmatrix}^T, \\
            % 
            U_{long} & = A_N, \\
            % 
            \textbf{A}_{long} &= 
            \begin{bmatrix}
                0 & \frac{ c }{ l m_Q } & \frac{ g m_p }{ m_Q }\\[0.5em]
                0 & -\frac{ c (m_Q + m_p) }{ l^2 m_Q m_p } & -\frac{ g (m_Q + m_p) }{ l m_Q } \\[0.5em]
                0 & 1 & 0
            \end{bmatrix}^T, \\
            % 
            \textbf{B}_{long}& = 
            \begin{bmatrix}
                1 & -\frac{1}{l}
            \end{bmatrix}^T, \\
            % 
            \textbf{C}_{long} &= 
            \begin{bmatrix}
                1 & 0 & 0
            \end{bmatrix}.
        \end{align}
        For the definition of these symbols, refer to Chapter~\ref{chap:system_id}.
        The input and output of the plant are $\mbox{U}_{long} = A_N  \mbox{ and } \mbox{Y}_{long} = V_N$, respectively.
        Note that the actual input of the plant is $A_{N_{sp}}$, however, it is assumed that $A_{N_{sp}} \approx A_N$ due to the large time-scale separation between the velocity controller and attitude controller.
        The transfer function can therefore be calculated as, 
        \begin{align}\label{eq:lin_vel_quad_pl}
            G_{V_N}(s) &= \frac{V_N(s)}{A_N(s)} = \textbf{C}_{long} \left(s \textbf{I} - \textbf{A}_{long}\right)^{-1} \textbf{B}_{long} \\
            G_{V_N}(s) &= \dfrac{s^2 + \frac{c}{m_p l^2}s + \dfrac{g}{l}}{s \left[s^2 + \dfrac{c(m_Q + m_p)}{m_Q m_p l^2}s + \dfrac{g(m_Q + m_p)}{m_Q l}\right]} 
        \end{align}

        % This file was created by matlab2tikz.
%
%The latest updates can be retrieved from
%  http://www.mathworks.com/matlabcentral/fileexchange/22022-matlab2tikz-matlab2tikz
%where you can also make suggestions and rate matlab2tikz.
%
\definecolor{mycolor1}{rgb}{0.00000,0.44700,0.74100}%
\definecolor{mycolor2}{rgb}{0.75000,0.75000,0.00000}%
\definecolor{mycolor3}{rgb}{0.75000,0.00000,0.75000}%
\definecolor{mycolor4}{rgb}{0.00000,0.75000,0.75000}%
%
\begin{figure}[htb]
    \centering
    \begin{tikzpicture}

        \begin{axis}[            
          xlabel = Imaginary,
          ylabel = Real,
          % x unit = \si{\hertz},
          % y unit = \si{\second},
          xmin=-5, xmax=0.2,
          ymin=-4, ymax=4,
          grid = major,
          % legend cell align = left,
          % legend pos = north east,
          grid style = dashed,
          % legend style = {font = \scriptsize},
          label style = {font = \scriptsize},
          tick label style = {font = \scriptsize},
          width = 0.55\columnwidth,
          height = 0.55\columnwidth,
          % initialize Dark2
          cycle list/Dark2,
          % combine it with 'mark list*':
          cycle multiindex* list = {
              Dark2\nextlist
          }
        ]


        \addplot [forget plot]
          table[row sep=crcr]{%
        0	0\\
        };

        \addplot [only marks, mark=o, mark options={solid, mycolor1}, style = solid, ultra thick,  forget plot]
          table[row sep=crcr]{%
        6.77962264341521e-16	3.13209195267317\\
        6.77962264341521e-16	-3.13209195267317\\
        -4.43649167310231	0\\
        -3.72687109858774	0\\
        -4.01084873037111	0\\
        -0.563508326896292	0\\
        };

        \addplot [color=mycolor1, only marks, mark=x, mark options={solid, mycolor1}, style = solid, ultra thick, forget plot]
          table[row sep=crcr]{%
        0	0\\
        0	0\\
        -15.1021256953078	5.68694734536982\\
        -15.1021256953078	-5.68694734536982\\
        2.63677968348475e-16	3.32301249776487\\
        2.63677968348475e-16	-3.32301249776487\\
        -4.01084873036827	0\\
        -3.59932833650225	0\\
        };
        
        \addplot [color=blue, style = solid, ultra thick, forget plot]
          table[row sep=crcr]{%
        0	0\\
        -9.32246040892286e-05	0.00990567878862648\\
        -0.000127453628897554	0.011582118625967\\
        -0.000174249921288412	0.0135421789181326\\
        -0.000238227139628144	0.0158337823582395\\
        -0.000325692309374581	0.0185129129317074\\
        -0.000445267010383679	0.0216449503043758\\
        -0.000608736146613293	0.0253062120655832\\
        -0.000832207423781231	0.0295857282376454\\
        -0.00113769480251062	0.0345872701584621\\
        -0.00155528007460376	0.0404316493088447\\
        -0.00212606198621601	0.0472592878906528\\
        -0.0029061767592857	0.0552330371085411\\
        -0.00397227348739993	0.0645411733502692\\
        -0.00542896012633655	0.0754004242853774\\
        -0.00741890925404723	0.0880587463804895\\
        -0.0101365363302348	0.102797360886462\\
        -0.0138464437938615	0.119931207255272\\
        -0.0189081621154179	0.139806413397939\\
        -0.0258090973290887	0.162792490484991\\
        -0.035207961866339	0.189265550470218\\
        -0.0479912015894711	0.219576633985741\\
        -0.0653447922894401	0.253995793713132\\
        -0.0888428061551879	0.292617237507333\\
        -0.12055154208073	0.335202505300727\\
        -0.163142439375009	0.38092530477761\\
        -0.219996234253677	0.427958714193607\\
        -0.295260718032634	0.472800778075537\\
        -0.393785261908563	0.509130579596385\\
        -0.520772373218682	0.525688024954305\\
        -0.680820322471942	0.501495443620636\\
        -0.875857182456015	0.388435516092755\\
        -1.0450301319122	0.0281285534880206\\
        -1.0457759677391	0\\
        -1.01838222580291	0\\
        -0.851210510744232	0\\
        -0.704606608529748	0\\
        -0.6509024435522	0\\
        -0.625727647206798	0\\
        -0.625649140037765	0\\
        -0.625570833656501	0\\
        -0.621554169813395	0\\
        -0.617650226726051	0\\
        -0.617583975233301	0\\
        -0.617517887478138	0\\
        -0.603421561309573	0\\
        -0.596420039077046	0\\
        -0.591514446091983	0\\
        -0.586777397062126	0\\
        -0.583413011005967	0\\
        -0.580602282288155	0\\
        -0.578487711715258	0\\
        -0.578487210507054	0\\
        -0.578487210507053	0\\
        -0.578471880082543	0\\
        -0.578471379433451	0\\
        -0.578471379433452	0\\
        -0.578458484845096	0\\
        -0.578456081906861	0\\
        -0.578455581815639	0\\
        -0.578455581815642	0\\
        -0.578442685812611	0\\
        -0.578426920164549	0\\
        -0.578342325857482	0\\
        -0.57825868627497	0\\
        -0.57809420730504	0\\
        -0.577933358833733	0\\
        -0.577776021881057	0\\
        -0.576517267875102	0\\
        -0.575462720029927	0\\
        -0.573795152162137	0\\
        -0.571828946843418	0\\
        -0.571820366805587	0\\
        -0.571811804449108	0\\
        -0.571290135619241	0\\
        -0.571290135619236	0\\
        -0.57128212714951	0\\
        -0.571282127149511	0\\
        -0.571274135149959	0\\
        -0.571274135149963	0\\
        -0.571191867131511	0\\
        -0.571111324241162	0\\
        -0.570955201207737	0\\
        -0.568914817345882	0\\
        -0.567441550915185	0\\
        -0.566373976606202	0\\
        -0.565598396226468	0\\
        -0.56503390346761	0\\
        -0.564622496434981	0\\
        -0.564322367621217	0\\
        -0.564103263129987	0\\
        -0.563943226627834	0\\
        -0.563509415514412	0\\
        -0.563508326896292	0\\
        };
        
        \addplot [color=mycolor2, style = solid, ultra thick, forget plot]
          table[row sep=crcr]{%
        -3.59932833650225	0\\
        -3.59932688489678	0\\
        -3.59932635188927	0\\
        -3.59932562316476	0\\
        -3.59932462685228	0\\
        -3.5993232646867	0\\
        -3.59932140230842	0\\
        -3.59931885600319	0\\
        -3.59931537455879	0\\
        -3.59931061444526	0\\
        -3.59930410585158	0\\
        -3.59929520619257	0\\
        -3.59928303643079	0\\
        -3.59926639381958	0\\
        -3.59924363221572	0\\
        -3.59921249773114	0\\
        -3.59916990268487	0\\
        -3.59911161402649	0\\
        -3.5990318225971	0\\
        -3.59892254526449	0\\
        -3.59877279062006	0\\
        -3.59856738626253	0\\
        -3.59828531439937	0\\
        -3.59789731914549	0\\
        -3.59736240840963	0\\
        -3.5966226264009	0\\
        -3.59559501860956	0\\
        -3.59415883093191	0\\
        -3.59213417359219	0\\
        -3.58924437890119	0\\
        -3.58504462506678	0\\
        -3.57877329201916	0\\
        -3.57180246813451	0\\
        -3.5717674866943	0\\
        -3.57173248568008	0\\
        -3.56900019081697	0\\
        -3.55262059541039	0\\
        -3.52086545803969	0\\
        -3.45422717487675	0\\
        -3.45374878635684	0\\
        -3.45326725222871	0\\
        -3.42011397449408	0\\
        -3.31873003890076	0\\
        -3.29918094619669	-1.27447953107161e-06\\
        -3.29983047655695	-0.0201624805068181\\
        -3.45238341109556	-0.269028450536322\\
        -3.53971057108474	-0.304134643288724\\
        -3.60607721366432	-0.312289771145505\\
        -3.67453400682908	-0.305648826842205\\
        -3.72591030021547	-0.290077333445448\\
        -3.7706648465994	-0.267831589850459\\
        -3.80546877167648	-0.243505782134715\\
        -3.80547713855985	-0.243499040013184\\
        -3.80547713855624	-0.243499040015789\\
        -3.80573308299711	-0.243292567108971\\
        -3.80574144231611	-0.243285816038911\\
        -3.80574144231726	-0.243285816037974\\
        -3.80595676204465	-0.243111757335416\\
        -3.80599689151444	-0.243079282744436\\
        -3.80600524328057	-0.243072522732111\\
        -3.80600524327608	-0.243072522735138\\
        -3.80622063211827	-0.242898019764033\\
        -3.80648400016679	-0.24268421317178\\
        -3.80789811175794	-0.241528034610376\\
        -3.80929783620179	-0.240369896333226\\
        -3.81205498877172	-0.238047908173089\\
        -3.81475714541383	-0.235718552407091\\
        -3.81740592764484	-0.233382084454409\\
        -3.83879818974857	-0.212362052058492\\
        -3.85699728244912	-0.190791533975926\\
        -3.88629793532574	-0.144912154598395\\
        -3.92167902005018	-0.00989859016352513\\
        -3.92183644023442	0\\
        -3.9318829120426	0\\
        -4.01009938249036	0\\
        -4.01009939062695	0\\
        -4.01084770851601	0\\
        -4.01084724250877	0\\
        -4.01084872805684	0\\
        -4.01084873312454	0\\
        -4.01084873115323	0\\
        -4.01084873045138	0\\
        -4.01084873038073	0\\
        -4.0108487303821	0\\
        -4.01084873038482	0\\
        -4.0108487303676	0\\
        -4.01084873036945	0\\
        -4.01084873037227	0\\
        -4.01084873037004	0\\
        -4.01084873036846	0\\
        -4.01084873036995	0\\
        -4.01084873036881	0\\
        -4.01084873037792	0\\
        -4.01084873037111	0\\
        };
        
        \addplot [color=mycolor3, style = solid, ultra thick, forget plot]
          table[row sep=crcr]{%
        -4.01084873036827	0\\
        -4.01084873036822	0\\
        -4.01084873036821	0\\
        -4.0108487303679	0\\
        -4.01084873036825	0\\
        -4.01084873036814	0\\
        -4.01084873036843	0\\
        -4.01084873036844	0\\
        -4.01084873036803	0\\
        -4.01084873036796	0\\
        -4.0108487303684	0\\
        -4.0108487303681	0\\
        -4.01084873036828	0\\
        -4.01084873036826	0\\
        -4.01084873036871	0\\
        -4.01084873036851	0\\
        -4.01084873036812	0\\
        -4.01084873036839	0\\
        -4.01084873036821	0\\
        -4.0108487303683	0\\
        -4.01084873036819	0\\
        -4.01084873036823	0\\
        -4.01084873036825	0\\
        -4.01084873036858	0\\
        -4.01084873036858	0\\
        -4.01084873036809	0\\
        -4.01084873036825	0\\
        -4.01084873036837	0\\
        -4.01084873036811	0\\
        -4.01084873036824	0\\
        -4.01084873036871	0\\
        -4.01084873036794	0\\
        -4.01084873036869	0\\
        -4.01084873036847	0\\
        -4.01084873036802	0\\
        -4.01084873036965	0\\
        -4.0108487303689	0\\
        -4.01084873036842	0\\
        -4.01084873036742	0\\
        -4.01084873036839	0\\
        -4.01084873036949	0\\
        -4.01084873037119	0\\
        -4.01084873036774	0\\
        -4.01084873037134	0\\
        -4.0108487303678	0\\
        -4.01084873036616	0\\
        -4.0108487303674	0\\
        -4.01084873037073	0\\
        -4.01084873037053	0\\
        -4.01084873036799	0\\
        -4.01084873037199	0\\
        -4.010848730372	0\\
        -4.01084873037223	0\\
        -4.01084873037943	0\\
        -4.0108487303669	0\\
        -4.01084873036673	0\\
        -4.01084873036444	0\\
        -4.01084873037507	0\\
        -4.01084873036395	0\\
        -4.01084873036721	0\\
        -4.01084873037622	0\\
        -4.01084873036674	0\\
        -4.01084873036679	0\\
        -4.01084873036939	0\\
        -4.01084873037214	0\\
        -4.01084873037376	0\\
        -4.01084873036736	0\\
        -4.01084873037162	0\\
        -4.01084873036306	0\\
        -4.0108487303731	0\\
        -4.01084873037341	0\\
        -4.01084873033022	0\\
        -4.01084873031481	0\\
        -4.01084873041662	0\\
        -4.01084873355721	0\\
        -4.01084872545341	0\\
        -4.01084975221632	0\\
        -4.01085021821957	0\\
        -4.01159235033179	0\\
        -4.01159234528768	0\\
        -4.01902450429601	0\\
        -4.02596037031718	0\\
        -4.0386492114254	0\\
        -4.1675163668624	0\\
        -4.24528501024616	0\\
        -4.29869259980289	0\\
        -4.33658035119409	0\\
        -4.36381884762432	0\\
        -4.383532272634	0\\
        -4.39785294610613	0\\
        -4.40827959724512	0\\
        -4.41588198396425	0\\
        -4.43644016603877	0\\
        -4.43649167310231	0\\
        };
        
        \addplot [color=mycolor4, style = solid, ultra thick,  forget plot]
          table[row sep=crcr]{%
        2.63677968348475e-16	-3.32301249776487\\
        -1.31214832611581e-05	-3.32301038947714\\
        -1.79392575183623e-05	-3.3230096149401\\
        -2.45258844943641e-05	-3.32300855564513\\
        -3.35307362275272e-05	-3.32300710671829\\
        -4.58415224439679e-05	-3.32300512450265\\
        -6.26717180182701e-05	-3.32300241207874\\
        -8.56800178853923e-05	-3.32299869926033\\
        -0.000117133514316636	-3.32299361485674\\
        -0.000160130529007207	-3.32298664802499\\
        -0.000218904775377837	-3.32297709407664\\
        -0.000299240282573054	-3.32296397785774\\
        -0.000409036935734297	-3.32294594427886\\
        -0.000559080388750577	-3.32292109985816\\
        -0.000764088439226984	-3.3228867797106\\
        -0.00104412970736539	-3.32283919853868\\
        -0.00142654033729761	-3.32277291698664\\
        -0.00194850007823433	-3.32268000753874\\
        -0.00266046720821312	-3.32254872159981\\
        -0.00363070182774483	-3.32236131458573\\
        -0.00495110168111612	-3.32209043283889\\
        -0.00674446794276384	-3.3216930301684\\
        -0.00917295841433663	-3.32110005216779\\
        -0.0124465339355543	-3.32019897506649\\
        -0.0168279392280582	-3.31880468585784\\
        -0.0226257599876148	-3.31661265387104\\
        -0.0301570045914138	-3.31312922323879\\
        -0.0396432842670908	-3.30758450495411\\
        -0.050985147502277	-3.29887128556869\\
        -0.0633740114740033	-3.28564410228744\\
        -0.0748591756324617	-3.26683477499833\\
        -0.0824022383140798	-3.24274307693549\\
        -0.0837037835347287	-3.22249906266598\\
        -0.0837000258510392	-3.22241303186436\\
        -0.083696193678354	-3.22232709379058\\
        -0.0832012673970071	-3.21603471286962\\
        -0.0768159743799015	-3.19100038910971\\
        -0.0656462439911673	-3.1709964729403\\
        -0.0552246664240886	-3.15897515219499\\
        -0.0551842401984368	-3.15893572040191\\
        -0.0551438612847535	-3.15889638106883\\
        -0.0529958903028031	-3.15686836587157\\
        -0.0507578679116363	-3.15488253529138\\
        -0.0507185573401022	-3.15484874952483\\
        -0.0506792988088516	-3.15481504514254\\
        -0.0412046907721941	-3.14764701507814\\
        -0.035606901438469	-3.14419436490932\\
        -0.0312965757086386	-3.14187299104054\\
        -0.0268152297966944	-3.13974212927357\\
        -0.0234359941577662	-3.1383130380198\\
        -0.0204851981737404	-3.13718349658071\\
        -0.0181877863837409	-3.136377576666\\
        -0.018187233913734	-3.13637739043626\\
        -0.0181872339137343	-3.13637739043626\\
        -0.0181703337304087	-3.1363716953695\\
        -0.0181697817580199	-3.13637150942138\\
        -0.0181697817580186	-3.13637150942138\\
        -0.0181555640067375	-3.13636672099528\\
        -0.0181529142186111	-3.13636582883176\\
        -0.0181523627429811	-3.13636564316454\\
        -0.0181523627429902	-3.13636564316454\\
        -0.0181381403810455	-3.13636085611411\\
        -0.0181207498466382	-3.1363550059489\\
        -0.0180273732560299	-3.13632365520333\\
        -0.0179349451298657	-3.13629272424294\\
        -0.0177528776188839	-3.13623209007463\\
        -0.0175744367714533	-3.13617304212881\\
        -0.0173995163167068	-3.1361155219942\\
        -0.0159867184047959	-3.13566402649744\\
        -0.0147848036949842	-3.13529807105546\\
        -0.012850106263255	-3.13474349812057\\
        -0.0105153100242299	-3.13412992324476\\
        -0.0105049945890333	-3.13412734578256\\
        -0.0104946993130616	-3.13412477451955\\
        -0.00986538199684112	-3.13396980191144\\
        -0.00986538199684506	-3.13396980191144\\
        -0.00985568923368085	-3.13396744881797\\
        -0.00985568923367008	-3.13396744881797\\
        -0.00984601544904212	-3.13396510135241\\
        -0.00984601544903613	-3.13396510135242\\
        -0.00974637999429442	-3.13394098286229\\
        -0.00964873589116239	-3.13391745125067\\
        -0.00945918819534919	-3.13387206740735\\
        -0.00694858912970409	-3.13330753507127\\
        -0.00509743499329995	-3.13293437346775\\
        -0.00373608474790688	-3.13268298037936\\
        -0.00273664132365675	-3.13251073500534\\
        -0.00200372239108815	-3.13239101029018\\
        -0.00146666404379714	-3.13230680380456\\
        -0.00107333295446521	-3.13224701765147\\
        -0.000785370808661368	-3.13220425597092\\
        -0.000574605169718495	-3.13217349727977\\
        -1.44168944404832e-06	-3.13209215304114\\
        6.77962264341521e-16	-3.13209195267317\\
        };
        
        \addplot [color=red, style = solid, ultra thick,  forget plot]
          table[row sep=crcr]{%
        2.63677968348475e-16	3.32301249776487\\
        -1.31214832611581e-05	3.32301038947714\\
        -1.79392575183623e-05	3.3230096149401\\
        -2.45258844943641e-05	3.32300855564513\\
        -3.35307362275272e-05	3.32300710671829\\
        -4.58415224439679e-05	3.32300512450265\\
        -6.26717180182701e-05	3.32300241207874\\
        -8.56800178853923e-05	3.32299869926033\\
        -0.000117133514316636	3.32299361485674\\
        -0.000160130529007207	3.32298664802499\\
        -0.000218904775377837	3.32297709407664\\
        -0.000299240282573054	3.32296397785774\\
        -0.000409036935734297	3.32294594427886\\
        -0.000559080388750577	3.32292109985816\\
        -0.000764088439226984	3.3228867797106\\
        -0.00104412970736539	3.32283919853868\\
        -0.00142654033729761	3.32277291698664\\
        -0.00194850007823433	3.32268000753874\\
        -0.00266046720821312	3.32254872159981\\
        -0.00363070182774483	3.32236131458573\\
        -0.00495110168111612	3.32209043283889\\
        -0.00674446794276384	3.3216930301684\\
        -0.00917295841433663	3.32110005216779\\
        -0.0124465339355543	3.32019897506649\\
        -0.0168279392280582	3.31880468585784\\
        -0.0226257599876148	3.31661265387104\\
        -0.0301570045914138	3.31312922323879\\
        -0.0396432842670908	3.30758450495411\\
        -0.050985147502277	3.29887128556869\\
        -0.0633740114740033	3.28564410228744\\
        -0.0748591756324617	3.26683477499833\\
        -0.0824022383140798	3.24274307693549\\
        -0.0837037835347287	3.22249906266598\\
        -0.0837000258510392	3.22241303186436\\
        -0.083696193678354	3.22232709379058\\
        -0.0832012673970071	3.21603471286962\\
        -0.0768159743799015	3.19100038910971\\
        -0.0656462439911673	3.1709964729403\\
        -0.0552246664240886	3.15897515219499\\
        -0.0551842401984368	3.15893572040191\\
        -0.0551438612847535	3.15889638106883\\
        -0.0529958903028031	3.15686836587157\\
        -0.0507578679116363	3.15488253529138\\
        -0.0507185573401022	3.15484874952483\\
        -0.0506792988088516	3.15481504514254\\
        -0.0412046907721941	3.14764701507814\\
        -0.035606901438469	3.14419436490932\\
        -0.0312965757086386	3.14187299104054\\
        -0.0268152297966944	3.13974212927357\\
        -0.0234359941577662	3.1383130380198\\
        -0.0204851981737404	3.13718349658071\\
        -0.0181877863837409	3.136377576666\\
        -0.018187233913734	3.13637739043626\\
        -0.0181872339137343	3.13637739043626\\
        -0.0181703337304087	3.1363716953695\\
        -0.0181697817580199	3.13637150942138\\
        -0.0181697817580186	3.13637150942138\\
        -0.0181555640067375	3.13636672099528\\
        -0.0181529142186111	3.13636582883176\\
        -0.0181523627429811	3.13636564316454\\
        -0.0181523627429902	3.13636564316454\\
        -0.0181381403810455	3.13636085611411\\
        -0.0181207498466382	3.1363550059489\\
        -0.0180273732560299	3.13632365520333\\
        -0.0179349451298657	3.13629272424294\\
        -0.0177528776188839	3.13623209007463\\
        -0.0175744367714533	3.13617304212881\\
        -0.0173995163167068	3.1361155219942\\
        -0.0159867184047959	3.13566402649744\\
        -0.0147848036949842	3.13529807105546\\
        -0.012850106263255	3.13474349812057\\
        -0.0105153100242299	3.13412992324476\\
        -0.0105049945890333	3.13412734578256\\
        -0.0104946993130616	3.13412477451955\\
        -0.00986538199684112	3.13396980191144\\
        -0.00986538199684506	3.13396980191144\\
        -0.00985568923368085	3.13396744881797\\
        -0.00985568923367008	3.13396744881797\\
        -0.00984601544904212	3.13396510135241\\
        -0.00984601544903613	3.13396510135242\\
        -0.00974637999429442	3.13394098286229\\
        -0.00964873589116239	3.13391745125067\\
        -0.00945918819534919	3.13387206740735\\
        -0.00694858912970409	3.13330753507127\\
        -0.00509743499329995	3.13293437346775\\
        -0.00373608474790688	3.13268298037936\\
        -0.00273664132365675	3.13251073500534\\
        -0.00200372239108815	3.13239101029018\\
        -0.00146666404379714	3.13230680380456\\
        -0.00107333295446521	3.13224701765147\\
        -0.000785370808661368	3.13220425597092\\
        -0.000574605169718495	3.13217349727977\\
        -1.44168944404832e-06	3.13209215304114\\
        6.77962264341521e-16	3.13209195267317\\
        };

        \addplot [color=black!50!green, style = solid, ultra thick,  forget plot]
          table[row sep=crcr]{%
        0	0\\
        -9.32246040892286e-05	-0.00990567878862648\\
        -0.000127453628897554	-0.011582118625967\\
        -0.000174249921288412	-0.0135421789181326\\
        -0.000238227139628144	-0.0158337823582395\\
        -0.000325692309374581	-0.0185129129317074\\
        -0.000445267010383679	-0.0216449503043758\\
        -0.000608736146613293	-0.0253062120655832\\
        -0.000832207423781231	-0.0295857282376454\\
        -0.00113769480251062	-0.0345872701584621\\
        -0.00155528007460376	-0.0404316493088447\\
        -0.00212606198621601	-0.0472592878906528\\
        -0.0029061767592857	-0.0552330371085411\\
        -0.00397227348739993	-0.0645411733502692\\
        -0.00542896012633655	-0.0754004242853774\\
        -0.00741890925404723	-0.0880587463804895\\
        -0.0101365363302348	-0.102797360886462\\
        -0.0138464437938615	-0.119931207255272\\
        -0.0189081621154179	-0.139806413397939\\
        -0.0258090973290887	-0.162792490484991\\
        -0.035207961866339	-0.189265550470218\\
        -0.0479912015894711	-0.219576633985741\\
        -0.0653447922894401	-0.253995793713132\\
        -0.0888428061551879	-0.292617237507333\\
        -0.12055154208073	-0.335202505300727\\
        -0.163142439375009	-0.38092530477761\\
        -0.219996234253677	-0.427958714193607\\
        -0.295260718032634	-0.472800778075537\\
        -0.393785261908563	-0.509130579596385\\
        -0.520772373218682	-0.525688024954305\\
        -0.680820322471942	-0.501495443620636\\
        -0.875857182456015	-0.388435516092755\\
        -1.0450301319122	-0.0281285534880206\\
        -1.04577608825783	0\\
        -1.07466060925661	0\\
        -1.35273584619381	0\\
        -1.99050642263175	0\\
        -2.54484646683592	0\\
        -2.99240008386602	0\\
        -2.99426556195052	0\\
        -2.99613204398591	0\\
        -3.10225084245539	0\\
        -3.27833071542214	0\\
        -3.29918094619669	1.27447953107161e-06\\
        -3.29983047655695	0.0201624805068181\\
        -3.45238341109556	0.269028450536322\\
        -3.53971057108474	0.304134643288724\\
        -3.60607721366432	0.312289771145505\\
        -3.67453400682908	0.305648826842205\\
        -3.72591030021547	0.290077333445448\\
        -3.7706648465994	0.267831589850459\\
        -3.80546877167648	0.243505782134715\\
        -3.80547713855985	0.243499040013184\\
        -3.80547713855624	0.243499040015789\\
        -3.80573308299711	0.243292567108971\\
        -3.80574144231611	0.243285816038911\\
        -3.80574144231726	0.243285816037974\\
        -3.80595676204465	0.243111757335416\\
        -3.80599689151444	0.243079282744436\\
        -3.80600524328057	0.243072522732111\\
        -3.80600524327608	0.243072522735138\\
        -3.80622063211827	0.242898019764033\\
        -3.80648400016679	0.24268421317178\\
        -3.80789811175794	0.241528034610376\\
        -3.80929783620179	0.240369896333226\\
        -3.81205498877172	0.238047908173089\\
        -3.81475714541383	0.235718552407091\\
        -3.81740592764484	0.233382084454409\\
        -3.83879818974857	0.212362052058492\\
        -3.85699728244912	0.190791533975926\\
        -3.88629793532574	0.144912154598395\\
        -3.92167902005018	0.00989859016352513\\
        -3.92183437934347	0\\
        -3.91210007708306	0\\
        -3.8529683777011	0\\
        -3.85296837766809	0\\
        -3.85251302247743	0\\
        -3.85251302248161	0\\
        -3.85206282006141	0\\
        -3.85206282003782	0\\
        -3.84765278531155	0\\
        -3.84367878868839	0\\
        -3.836739981431	0\\
        -3.78409305974124	0\\
        -3.76260856764159	0\\
        -3.75064689584588	0\\
        -3.74321944160966	0\\
        -3.73833707919077	0\\
        -3.73501605415243	0\\
        -3.73270672198126	0\\
        -3.73107692942619	0\\
        -3.72991490533506	0\\
        -3.72687851842757	0\\
        -3.72687109858774	0\\
        };
        
        \addplot [color=red, only marks, mark=square, mark options={solid, red}, style = solid, ultra thick, forget plot]
          table[row sep=crcr]{%
        -15.5127260871559	7.75374247372119\\
        -15.5127260871559	-7.75374247372119\\
        -0.057275337250322	3.29276215270708\\
        -0.057275337250322	-3.29276215270708\\
        -4.01084873036824	0\\
        -3.59078886811371	0\\
        -0.454779031876892	0.521060574738263\\
        -0.454779031876892	-0.521060574738263\\
        };

        \end{axis}
    \end{tikzpicture}
    \caption{Root locus plot of the North velocity dynamics including PID controller}
    \label{fig:rootlocus_cl}
\end{figure}

        \paragraph
        A payload with $m_p =$~\SI{0.2}{\kilo\gram} and $l =$~\SI{1}{\metre} is considered for the controller design.
        Figure~\ref{fig:rootlocus_cl} shows a root locus plot of the closed loop system which includes the \gls{PID} controller.
        % The suspended payload adds two lightly damped poles at $s = \pm 3.293 + -0.0573$.
        The gains were tuned in an iterative process to produce a slow step response that stimulates the payload dynamics enough for system identification.
        These gains were also tested iteratively with different payload parameters in the ranges, $0.1 \leq m_p \leq 0.3$~kg and $0.5 \leq l \leq 2$~m, 
        to establish safe flights with different payloads.

        \begin{figure}
            \captionsetup[subfigure]{justification=centering}
            \centering  
            \begin{subfigure}[t]{0.49\columnwidth}
    \centering
    \begin{tikzpicture}
        \begin{axis}[            
            xlabel = Time,
            ylabel = North velocity,
            x unit = \si{\second},
            y unit = \si{\metre/\second},
            xmin = 0,   xmax = 16,
            ymin = -0.1,  ymax = 2.5,
            grid = major,
            legend cell align = left,
            legend pos = south east,
            grid style = dashed,
            legend style = {font = \scriptsize},
            label style = {font = \scriptsize},
            tick label style = {font = \scriptsize},
            width = 0.99\columnwidth,
            height = 0.95\columnwidth,
            % initialize Dark2
            cycle list/Dark2,
            % combine it with 'mark list*':
            cycle multiindex* list = {
                Dark2\nextlist
            }
        ]
        
        \addplot+[mark = none, style = solid, ultra thick] 
        table[x = time, y = vel_sp, col sep = comma] 
        {control/csv/compare_control_pid_Simulink_single_pend_mp0.1_l0.5_PID_vel_steps_tune_scale_0.7.csv};
        \addlegendentry{Setpoint}

        % \addplot+[mark = none, style = solid, ultra thick] 
        % table[x = time, y = vel, col sep = comma] 
        % {control/csv/compare_control_pid_Simulink_single_pend_mp0.1_l0.5_PID_vel_steps_tune_scale_0.7.csv};
        % \addlegendentry{($l =$~\SI{0.5}{\meter}, $m_p =$~\SI{0.1}{\kilo\gram})}

        % \addplot+[mark = none, style = solid, ultra thick] 
        % table[x = time, y = vel, col sep = comma] 
        % {control/csv/compare_control_pid_Simulink_single_pend_mp0.2_l1_PID_vel_steps_tune_scale_0.7.csv};
        % \addlegendentry{($l =$~\SI{1}{\meter}, $m_p =$~\SI{0.2}{\kilo\gram})}

        \addplot+[mark = none, style = solid, ultra thick] 
        table[x = time, y = vel, col sep = comma] 
        {control/csv/compare_control_pid_Simulink_single_pend_mp0.2_l1_PID_vel_steps_tune_scale_0.7.csv};
        \addlegendentry{Velocity response}

        \end{axis}
    \end{tikzpicture} 
    \caption{North velocity.}
    \label{fig:pid_tuning_plot_vel}
\end{subfigure}
 % subfigure
            \begin{subfigure}[t]{0.45\columnwidth}
    \centering
    \begin{tikzpicture}
        \begin{axis}[            
            xlabel = Time,
            ylabel = Payload angle,
            x unit = \si{\second},
            y unit = \si{\degree},
            xmin = 0,   xmax = 16,
            ymin = -20,  ymax = 8,
            grid = major,
            legend cell align = left,
            legend pos = south east,
            grid style = dashed,
            legend style = {font = \scriptsize},
            label style = {font = \scriptsize},
            tick label style = {font = \scriptsize},
            width = 0.95\columnwidth,
            height = 0.95\columnwidth,
            % initialize Dark2
            cycle list/Dark2,
            % combine it with 'mark list*':
            cycle multiindex* list = {
                Dark2\nextlist
            }
        ]

        \pgfplotsset{cycle list shift = 2}
        
        \addplot+[mark = none, style = solid, ultra thick] 
        table[x = time, y = theta, col sep = comma] 
        {control/csv/compare_control_pid_Simulink_single_pend_mp0.2_l1_PID_vel_steps_tune_scale_0.7.csv};

        \end{axis}
    \end{tikzpicture} 
    \caption{Payload angle}
    \label{fig:pid_tuning_plot_theta}
\end{subfigure}
 % subfigure
            \caption{\gls{PID} velocity step response 
            ($l =$~\SI{1}{\meter}, $m_p =$~\SI{0.2}{\kilo\gram})}
            \label{fig:pid_tuning_subfigs}  
        \end{figure}

        \paragraph
        Figure~\ref{fig:pid_tuning_subfigs} shows a velocity step response for the resulting cascaded \gls{PID} controller with a specific payload.
        The gains are documented in Appendix~\ref{appen:pid_gains}.
        Note that the response results in a large overshoot, however, this aids in keeping the payload angles low.
        The velocity oscillations are clearly visible in Figure~\ref{fig:pid_tuning_plot_vel}.
        The current \gls{PID} controller does not provide an adequate way of actively damping the payload oscillations,
        hence an active swing damping controller is required.
