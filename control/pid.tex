\FloatBarrier\section{Cascaded PID}

    \paragraph
    This section gives a brief overview of the PID control architecture.
    The controllers are designed based on a linearised model of the multirotor without a payload.
    The design process used to determine the controller gains are also discussed in this section.
    Finally the controllers are implemented with MATLAB and Simulink and are verified with practical flight data.
    % ?? remember to add practical verification

    \paragraph
    The PX4-Autopilot was chosen as the multirotor flight-stack because it is open-source and widely used in industry and research.
    The default PX4 control architecture for a multirotor consists of multiple cascaded PID controller loops.
    This is devided into two main categories, the inner loop attitude controllers and the outher loop translational controllers.

    % From: https://github.com/PX4/PX4-user_guide/blob/master/assets/diagrams/mc_control_arch_tikz.tex
\begin{figure}[htb]
      \centering
    
      % Define distances
      \def\blockheight{7em}
      \def\largeblockheight{1.45*\blockheight}
      \def\blockwidth{4.1em}
      \def\nodesep{1.25*\blockwidth}
    
      \footnotesize
      \begin{tikzpicture}[>={Stealth[inset=0pt,length=3pt,angle'=60,round]},
          simple_block/.style = {draw, fill=light_blue, text centered, inner sep=0pt, text width=\blockwidth, minimum height=\blockheight},
          larger_block/.style = {draw, fill=light_blue, text centered, inner sep=0pt, text width=\blockwidth, minimum height=\largeblockheight, anchor=north},
          no_border_block/.style = {text centered, inner ysep=0.5em},
          frame_block/.style = {draw, text centered, inner ysep=2.25em, inner xsep=0.5em, dashed}
      ]
        
        %===================================================================================
        
        % Reference
        \node (ref) [] {};
        
        % Position controller
        \path (ref.east)+(0.85*\nodesep,0) node (pos_ctrl) [simple_block] {Position Control};
        \path (pos_ctrl.north)+(0,0) node (pos_ctrl_text) [no_border_block, anchor=south] {\textbf{P}};
        \path (pos_ctrl.south)+(0.65*\nodesep, -0.075*\blockheight) node (pos_ctrl_freq_text) [no_border_block, anchor=north] {50 Hz};
        
        % Velocity controller
        \path (pos_ctrl.east)+(0.85*\nodesep,0) node (vel_ctrl) [simple_block] {Velocity Control};
        \path (vel_ctrl.north)+(0,0) node (vel_ctrl_text) [no_border_block, anchor=south] {\textbf{PID}};
        
        % Acceleration to Attitude
        \path (vel_ctrl.north east)+(1.1*\nodesep,0) node (force_2_att) [larger_block, text width=5.6em] {Acceleration and Yaw to Attitude};
      
        % Inertial Frame
        \node[fit=(pos_ctrl) (vel_ctrl)] (inertial) [frame_block] {};
        \path (inertial.south)+(0,0) node (inertial_text) [no_border_block, anchor=north] {Inertial Frame};
        
        % Angle controller
        \path (force_2_att.north east)+(0.95*\nodesep,0) node (angle_ctrl) [simple_block, anchor=north] {Angle Control};
        \path (angle_ctrl.north)+(0,0) node (angle_ctrl_text) [no_border_block, anchor=south] {\textbf{P}};
        \path (angle_ctrl.south)+(0, -0.075*\blockheight) node (angle_ctrl_freq_text) [no_border_block, anchor=north] {250 Hz};
        
        % Angular Rate controller
        \path (angle_ctrl.east)+(0.9*\nodesep,0) node (ang_vel_ctrl) [simple_block] {Angular Rate Control};
        \path (ang_vel_ctrl.north)+(0,0) node (ang_vel_ctrl_text) [no_border_block, anchor=south] {\textbf{PID}};
        \path (ang_vel_ctrl.south)+(0, -0.075*\blockheight) node (ang_vel_ctrl_freq_text) [no_border_block, anchor=north] {1 kHz};
        
        % Mixer
        \path (ang_vel_ctrl.north east)+(\nodesep,0) node (mixer) [larger_block, text width=3em] {Mixer};
        
        % Body Frame
        \node[fit=(angle_ctrl) (ang_vel_ctrl)] (body) [frame_block] {};
        \path (body.south)+(0,0) node (body_text) [no_border_block, anchor=north] {Body Frame};
        
        \path (mixer.east)+(0.5*\nodesep,0) node (cmd) [] {};
        
        %===================================================================================
        
        \path[draw,->] (ref.east) -- node[anchor=south, pos=0.05, align=left] {$\bm{X}_\text{sp}$} (pos_ctrl.west);
        
        \path[draw,->] (pos_ctrl.east) -- node[anchor=south] {$\bm{V}_\text{sp}$} (vel_ctrl.west);
        
        \path[draw,->] (vel_ctrl.east) -- node[anchor=south, pos=0.65] {$\bm{A}_\text{sp}$} (vel_ctrl.east -| force_2_att.west);
        \path[draw,->] ([yshift=-0.25*\largeblockheight]force_2_att.west -| ref.east) -- node[anchor=south, pos=0] {$\psi_\text{sp}$} ([yshift=-0.25*\largeblockheight]force_2_att.west);
        
        \path[draw,->] (force_2_att.east |- angle_ctrl.west) -- node[anchor=south, pos=0.4] {$\bm{q}_\text{sp}$} (angle_ctrl.west);
        \path[draw,->] ([yshift=-0.25*\largeblockheight]force_2_att.east) -- node[text centered, inner sep=2pt, anchor=south, pos=0.915] {$\delta_{T_\text{sp}}$} ([yshift=-0.25*\largeblockheight]mixer.west);
        
        \path[draw,->] (angle_ctrl.east) -- node[anchor=south] {$\bm{\Omega}_\text{sp}$} (ang_vel_ctrl.west);
        
        \path[draw,->] ([yshift=0.25*\blockheight]ang_vel_ctrl.east) -- node[text centered, inner sep=2pt, anchor=south, pos=0.6] {$\delta_{A_\text{sp}}$} ([yshift=0.25*\blockheight]ang_vel_ctrl.east -| mixer.west);
        \path[draw,->] (ang_vel_ctrl.east) -- node[text centered, inner sep=2pt, anchor=south, pos=0.6] {$\delta_{E_\text{sp}}$} (ang_vel_ctrl.east -| mixer.west);
        \path[draw,->] ([yshift=-0.25*\blockheight]ang_vel_ctrl.east) -- node[text centered, inner sep=2pt, anchor=south, pos=0.6] {$\delta_{R_\text{sp}}$} ([yshift=-0.25*\blockheight]ang_vel_ctrl.east -| mixer.west);
        
        \path[draw,->] (mixer.east) -- node[text centered, inner sep=5pt, anchor=south] {$\bm{T}_\text{sp}$} (cmd.west);
      
      \end{tikzpicture}
      
      \caption{Cascaded \gls{PID} control architecture of PX4 \cite{PX4userguide}}
      \label{fig:pid_architecture}

    \end{figure}
    

    % ?? Fix notation of symbols and define them

    \paragraph
    Figure~\ref{fig:pid_architecture} shows a high level overview of the PX4 control architecture, but does not show state feedback. 
    The inner-most element is the mixer, which converts virtual actuator commands to actuator thrust commands.
    % This is achieved with simple matrix multiplication.
    The attitude controller includes the angle and angular rate controllers, which send sommands to the mixer.
    The translational controller consists of the position and velocity controllers, which send commands to the attitude controller.
    The rate of each controller is also shown in Figure~\ref{fig:pid_architecture}.
    % The inner loop controllers deal with the fastest dynamics and are required to run at the highest rate.
    % The translation controllers control slower dynamics. hence they run a lower rate.

    \FloatBarrier\subsection{Angular rate controller}

        \paragraph
        Three seperate linear PID controllers are used to the control angular rate of the multirotor in the pitch, roll, and yaw axis of the body frame.
        The angular rate controller receives angular rate estimates from the PX4 state estimator, and outputs virtual actuator commands to the mixer.
        Figure~\ref{fig:angular_rate_controller} shows a diagram of the PX4 angular rate controller.
        
        % From: https://github.com/PX4/PX4-user_guide/blob/master/assets/diagrams/mc_control_arch_tikz.tex
\begin{figure}[ht]
    \centering
  
    % Define distances
    \def\blockheight{2.2em}
    \def\blockwidth{2.2em}
    \def\nodesep{1.5*\blockwidth}
  
    \begin{tikzpicture}[>={Stealth[inset=0pt,length=3pt,angle'=60,round]},
      simple_block/.style = {draw, fill=light_blue, text centered, inner sep=0pt, minimum width=\blockwidth, minimum height=\blockheight},
      frame_block/.style = {draw, text centered, inner ysep=1.25em, inner xsep=1.25em, dashed},
      port/.style = {inner sep=0pt, font=\tiny},
      sum/.style n args = {4}{draw, circle, minimum size=1.5em, alias=sum,
        append after command = {
          node at (sum.north) [port, below=1pt] {$#1$}
          node at (sum.west) [port, right=1pt] {$#2$}
          node at (sum.south) [port, above=1pt] {$#3$}
          node at (sum.east) [port, left=1pt] {$#4$}
        }
      },
      sat_block/.style = {draw, minimum width=\blockwidth, minimum height=\blockheight, path picture = 
        {
        % Get the width and height of the path picture node
        \pgfpointdiff{\pgfpointanchor{path picture bounding box}{north east}}%
            {\pgfpointanchor{path picture bounding box}{south west}}
        \pgfgetlastxy\x\y
        % Scale the x and y vectors so that the range
        % -1 to 1 is slightly shorter than the size of the node
        \tikzset{x=\x*.4, y=\y*.4}
        %
        % Draw annotation
        \draw (-1,0) -- (1,0) (0,-1) -- (0,1); 
        \draw (-1,-.6) -- (-.6,-.6) -- (.6,.6) -- (1,.6);
        }
      },
      sat_block_int/.style = {draw, minimum width=\blockwidth, minimum height=\blockheight, path picture = 
        {
        % Get the width and height of the path picture node
        \pgfpointdiff{\pgfpointanchor{path picture bounding box}{north east}}%
            {\pgfpointanchor{path picture bounding box}{south west}}
        \pgfgetlastxy\x\y
        % Scale the x and y vectors so that the range
        % -1 to 1 is slightly shorter than the size of the node
        \tikzset{x=\x*.4, y=\y*.4}
        %
        % Draw annotation
        \draw (-.8,-.9) -- (-.6,-.9) -- (.1,.5) -- (.3,.5);
        }
      }
    ]
      %===================================================================================
      \node[text centered] (ref) {};
  
      % Controller
      \path(ref)+(\nodesep,0) node (ref_sum) [sum={}{+}{-}{}] {};
      \path(ref_sum)+(\nodesep,0) node (ctrl_k) [simple_block] {$K$};
      \path(ctrl_k)+(2*\nodesep,0) node (ctrl_i) [sat_block_int, fill=light_blue, text centered, inner sep=0pt] {$~~\int$};
      \path(ctrl_i)+(\nodesep,0) node (ctrl_i_gain) [simple_block] {$I$};
      \path(ref_sum)+(\nodesep,-1.5*\blockheight) node (ctrl_kd) [simple_block] {$K$};
      \path(ctrl_kd)+(\nodesep,0) node (ctrl_d_filt) [simple_block] {LPF};
      \path(ref_sum)+(3*\nodesep,-1.5*\blockheight) node (ctrl_d) [simple_block] {$\frac{\partial}{\partial t}$};
      \path(ctrl_d)+(\nodesep,0) node (ctrl_d_gain) [simple_block] {$D$};
      \path(ref_sum)+(4*\nodesep,1.5*\blockheight) node (ctrl_p_gain) [simple_block] {$P$};
      \path(ctrl_i_gain)+(\nodesep,0) node (ctrl_sum) [sum={+}{+}{-}{}] {};
      \path(ctrl_sum)+(\nodesep,0) node (sat_ctrl) [sat_block] {};
      \path(sat_ctrl.east)+(0.675*\nodesep,0) node (ctrl) [] {};
  
      \path(ref_sum.south |- ctrl_d_filt.west)+(-\nodesep,0) node(output) [] {};
  
      %===================================================================================
      \path[draw,->] (ref.east) -- node[anchor=south, pos=0] {$\Omega_\text{sp}$} (ref_sum.west);
  
      \path[draw,->] (ref_sum.east) -- (ctrl_k.west);
      \path[draw,->] (ctrl_k.east) -- (ctrl_i.west);
      \path[draw,->] (output.east) |- node[anchor=south, pos=0] {$\Omega$} (ctrl_kd.west);
      \path[draw,->] (ctrl_kd.east) -- (ctrl_d_filt.west);
      \path[draw,->] (output.east -| ref_sum.south) -- (ref_sum.south);
      \path[draw,->] ([xshift=0.65*\nodesep]ctrl_k.east) |- (ctrl_p_gain.west);
      \path[draw,->] (ctrl_i.east) -- (ctrl_i_gain.west);
      \path[draw,->] (ctrl_d_filt.east) -- (ctrl_d.west);
      \path[draw,->] (ctrl_d.east) -- (ctrl_d_gain.west);
      \path[draw,->] (ctrl_d_gain.east) -| (ctrl_sum.south);
      \path[draw,->] (ctrl_i_gain.east) -- (ctrl_sum.west);
      \path[draw,->] (ctrl_p_gain.east) -| (ctrl_sum.north);
      \path[draw,->] (ctrl_sum.east) -- (sat_ctrl.west);
      \path[draw,->] (sat_ctrl.east) -- node[anchor=south] {$\delta_\text{sp}$} (ctrl.west);
    
    \end{tikzpicture}

    \caption{Angular rate controller diagram in the Laplace domain \cite{PX4_controller_diagrams}}
    \label{fig:angular_rate_controller}
  
  \end{figure}

        \paragraph
        Derivative kick.
        LPF.
        Saturation.
        
        \paragraph
        Classical control theory was used by \cite{Grobler2020} to design the controller gains of the practical multirotor, Honeybee.
        This same process is also explained in detail by \cite{Erasmus2020}.
        The controller gains were designed for a transient response that is as fast as possible, while retaining fast disturbance rejection, and minimal overshoot.            
        It was determined by \cite{Grobler2020} that the default PX4 angular rate controller gains of the ZMR250 airframe provide excellent control for Honeybee and satisfy these design requirements.
        
        \paragraph
        For a \SI{1}{\radian/\second} step responce, the pitch rate controllers result in a 
        3.6~\% overshoot, 
        \SI{0.024}{\second} rise time, 
        \SI{0.8}{\second} for a 2~\% settling time, 
        and \SI{138}{\radian/\second} bandwidth \cite{Grobler2020}.
        These gains are well suited for Honeybee and will also be implemented in this work.
        The default roll-rate and yaw-rate controller gains are also implemented.

    \FloatBarrier\subsection{Angle controller}

        \paragraph
        The pitch, roll, and yaw angles are controlled by the angle controller in the body frame.
        A quaternion based controller is implemented by PX4 for angle control based on work by \cite{Brescianini2013}.
        The structure and design of this controller is well explained by \cite{Grobler2020}, \cite{Erasmus2020}, and \cite{Slabber2020} and is only briefly discussed here.
        
        \paragraph
        For the control law proposed by \cite{Brescianini2013}, the error quaternion is calculated as:
        \begin{equation}
            \bar{\bm{q}}_e = \bar{\bm{q}}^{-1} \cdot \bar{\bm{q}}_{sp},
        \end{equation}
        where $\bar{\bm{q}}_e$ is the quaternion error, 
        $\bar{\bm{q}}$ is the attitude quaternion of the multirotor,  
        and $\bar{\bm{q}}_{sp}$ is the attitude quaternion setpoint.
        The resulting control law is given as:
        \begin{equation} \label{eq:angle_law}
            \bm{\varOmega}_{sp} = 2 \bm{P}_{q} \mbox{sgn}(q_{0_e}) \bm{q}_{v_e} , 
            \qquad \qquad \mbox{sgn}(q_{0_e}) = 
            \begin{cases}
                1,  & \quad q_{0_e}  \geq   0 \\
                -1, & \quad q_{0_e}  <      0
            \end{cases} 
        \end{equation}

        %  ?? Fix symbols be matrix and correspond with diagram and explain matrix
        
        \paragraph
        The implementation of this control law is illustrated in Figure~\ref{fig:angle_controller}.
        The attitude setpoint is taken as input, the current attitude is received from the PX4 estimator, and the resulting anglular rate setpoint is produced as the output.

        \begin{figure}[htb]
    \centering
    \footnotesize
  
    % Define distances
    \def\blockheight{3em}
    \def\blockwidth{3em}
    \def\nodesep{2*\blockwidth}
  
    \begin{tikzpicture}[>={Stealth[inset=0pt,length=3pt,angle'=60,round]},
      simple_block/.style = {draw, fill=light_blue, text centered, inner sep=0pt, minimum width=\blockwidth, minimum height=\blockheight},
      text_block/.style = {draw, fill=light_blue, text centered, inner sep=0pt, text width=2*\blockwidth, minimum height=\blockheight},
      frame_block/.style = {draw, text centered, inner ysep=1.25em, inner xsep=1.25em, dashed},
      sat_block/.style = {draw, minimum width=\blockwidth, minimum height=\blockheight, path picture = 
          {
          % Get the width and height of the path picture node
          \pgfpointdiff{\pgfpointanchor{path picture bounding box}{north east}}%
              {\pgfpointanchor{path picture bounding box}{south west}}
          \pgfgetlastxy\x\y
          % Scale the x and y vectors so that the range
          % -1 to 1 is slightly shorter than the size of the node
          \tikzset{x=\x*.4, y=\y*.4}
          %
          % Draw annotation
          \draw (-1,0) -- (1,0) (0,-1) -- (0,1); 
          \draw (-1,-.6) -- (-.6,-.6) -- (.6,.6) -- (1,.6);
          }
      },
      gain/.style = {draw, fill=light_blue, inner sep=0pt, isosceles triangle, minimum height=\blockheight, isosceles triangle apex angle=60},
    ]
  
      %===============================================================================
      \node[text centered, yshift=0.25*\blockheight] (ref) {};
  
      % Controller
      \path(ref)+(0.75*\nodesep,-0.25*\blockheight) node (err_quat) [text_block] {Error quaternion};
      \path(err_quat)+(1.5*\nodesep,0) node (select_axis) [text_block] {Extract component};
      \path(select_axis)+(0, 1.5*\blockheight) node (select_mag) [text_block] {Extract magnitude};
      \path(select_mag)+(1.1*\nodesep,0) node (sgn) [simple_block] {sgn};
      \path(select_axis)+(1.9*\nodesep, 0.25*\blockheight) node (mult) [simple_block] {$\times$};
      \path(mult)+(0.65*\nodesep, 0) node (p_gain) [gain] {\scriptsize $2 P_{\Omega_j}$};
      \path(p_gain)+(0.75*\nodesep, 0) node (sat_ctrl) [sat_block] {};
      \path(sat_ctrl)+(0.75*\nodesep, 0) node (output) [] {};
  
      %===============================================================================
      \path[draw,->] (ref.east) -- node[text centered, anchor=south, pos=0.1] {$\bm{q}_\text{sp}$} ([yshift=0.25*\blockheight]err_quat.west);
      \path[draw,->] ([yshift=-0.25*\blockheight]err_quat.west -| ref.east) -- node[text centered, anchor=north, pos=0.1] {$\bm{q}$} ([yshift=-0.25*\blockheight]err_quat.west);
      \path[draw,->] (err_quat.east) -- node[text centered, anchor=south,pos=0.4] {$\bm{q}_\text{e}$} (select_axis.west);
      \path[draw,->]([xshift=0.3*\nodesep]err_quat.east) |- (select_mag.west);
      \path[draw,->] (select_mag.east) -- node[text centered, anchor=south] {$q_{0_\text{e}}$} (sgn.west);
      \path[draw,->] (sgn.east) -- ([xshift=0.15*\nodesep]sgn.east) |- ([yshift=0.25*\blockheight]mult.west);
      \path[draw,->] (select_axis.east) -- node[text centered, anchor=south, pos=0.45]{$q_{j_\text{e}}$} ([yshift=-0.25*\blockheight]mult.west);
      \path[draw,->] (mult.east) -- (p_gain.west);
      \path[draw,->] (p_gain.east) -- (sat_ctrl.west);
      \path[draw,->] (sat_ctrl.east) -- node[text centered, anchor=south] {$\Omega_{j_{sp}}$} (output.west);
    \end{tikzpicture}

    \caption{Quaternion based angle controller diagram \cite{PX4userguide}}
    \label{fig:angle_controller}
  
  \end{figure}

        \paragraph
        An integrator is not required in this controller, because it is placed between the velocity and angular rate controllers which both include integrators to reject steady-state disturbacnes \cite{Erasmus2020}.
        Therefore only a proportional term is applied in this control law.
        The proportional gain was designed by \cite{Grobler2020} for a faster transient response and bandwidth than the default ZMR250 gain.
        This results in a \SI{1}{\radian} step response with a
        0~\% overshoot, 
        \SI{0.3}{\second} rise time, 
        \SI{0.47}{\second} for a 2~\% settling time, 
        and \SI{11}{\radian/\second} bandwidth \cite{Grobler2020}.
        This still has a good time-scale seperation from the angular rate controller, which has a bandwidth of \SI{138}{\radian/\second}.
        The same gains designed by \cite{Grobler2020} will be used in this work.

    \FloatBarrier\subsection{Translational controller}

        Mention acceleration to attitude conversion.

        Explain velocity controller

        
        The translational controller consists of the position and velocity conyrollers.
        Position control is not required for training data in this work, therefore it will not be discussed in this section.
        For more details about the PX4 position control implementation and design, readers are refered to the work by \cite{Grobler2020}.
        A diagram of the North velocity controller is shown in Figure~\ref{fig:velocity_controller}.
        The West and Down velocity controllers duplicate the same configuration.

        % From: https://github.com/PX4/PX4-user_guide/blob/master/assets/diagrams/mc_control_arch_tikz.tex
\begin{figure}[ht]
  \centering

  % Define distances
  \def\blockheight{2.2em}
  \def\blockwidth{2.2em}
  \def\nodesep{1.5*\blockwidth}

  \begin{tikzpicture}[>={Stealth[inset=0pt,length=3pt,angle'=60,round]},
    simple_block/.style = {draw, fill=light_blue, text centered, inner sep=0pt, minimum width=\blockwidth, minimum height=\blockheight},
    frame_block/.style = {draw, text centered, inner ysep=1.25em, inner xsep=1.25em, dashed},
    port/.style = {inner sep=0pt, font=\tiny},
    sum/.style n args = {4}{draw, circle, minimum size=1.5em, alias=sum,
      append after command = {
        node at (sum.north) [port, below=1pt] {$#1$}
        node at (sum.west) [port, right=1pt] {$#2$}
        node at (sum.south) [port, above=1pt] {$#3$}
        node at (sum.east) [port, left=1pt] {$#4$}
      }
    },
    sat_block/.style = {draw, minimum width=\blockwidth, minimum height=\blockheight, path picture = 
      {
      % Get the width and height of the path picture node
      \pgfpointdiff{\pgfpointanchor{path picture bounding box}{north east}}%
          {\pgfpointanchor{path picture bounding box}{south west}}
      \pgfgetlastxy\x\y
      % Scale the x and y vectors so that the range
      % -1 to 1 is slightly shorter than the size of the node
      \tikzset{x=\x*.4, y=\y*.4}
      %
      % Draw annotation
      \draw (-1,0) -- (1,0) (0,-1) -- (0,1); 
      \draw (-1,-.6) -- (-.6,-.6) -- (.6,.6) -- (1,.6);
      }
    },
    sat_block_int/.style = {draw, minimum width=\blockwidth, minimum height=\blockheight, path picture = 
      {
      % Get the width and height of the path picture node
      \pgfpointdiff{\pgfpointanchor{path picture bounding box}{north east}}%
          {\pgfpointanchor{path picture bounding box}{south west}}
      \pgfgetlastxy\x\y
      % Scale the x and y vectors so that the range
      % -1 to 1 is slightly shorter than the size of the node
      \tikzset{x=\x*.4, y=\y*.4}
      %
      % Draw annotation
      \draw (-.8,-.9) -- (-.6,-.9) -- (.1,.5) -- (.3,.5);
      }
    }
  ]
    %===================================================================================
    \node[text centered] (ref) {};

    % Controller
    \path(ref)+(\nodesep,0) node (ref_sum) [sum={}{+}{-}{}] {};
  %   \path(ref_sum)+(\nodesep,0) node (ctrl_k) [simple_block] {$K$};
    \path(ref_sum)+(2*\nodesep,0) node (ctrl_i) [sat_block_int, fill=light_blue, text centered, inner sep=0pt] {$~~\int$};
    \path(ctrl_i)+(\nodesep,0) node (ctrl_i_gain) [simple_block] {$I_{V_N}$};
  %   \path(ref_sum)+(\nodesep,-1.5*\blockheight) node (ctrl_kd) [simple_block] {$K$};
    \path(ref_sum)+(\nodesep,-1.5*\blockheight) node (ctrl_d_filt) [simple_block] {LPF};
    \path(ref_sum)+(2*\nodesep,-1.5*\blockheight) node (ctrl_d) [simple_block] {$\frac{d}{dt}$};
    \path(ctrl_d)+(\nodesep,0) node (ctrl_d_gain) [simple_block] {$D_{V_N}$};
    \path(ref_sum)+(3*\nodesep,1.5*\blockheight) node (ctrl_p_gain) [simple_block] {$P_{V_N}$};
    \path(ctrl_i_gain)+(\nodesep,0) node (ctrl_sum) [sum={+}{+}{-}{}] {};
    \path(ctrl_sum)+(\nodesep,0) node (sat_ctrl) [sat_block] {};
    \path(sat_ctrl.east)+(1*\nodesep,0) node (ctrl) [] {};

    \path(ref_sum.south |- ctrl_d_filt.west)+(-\nodesep,0) node(output) [] {};

    %===================================================================================
    \path[draw,->] (ref.east) -- node[anchor=south, pos=0] {$V_{N_{sp}}$} (ref_sum.west);

    \path[draw,->] (ref_sum.east) -- (ctrl_i.west);
  %   \path[draw,->] (ctrl_k.east) -- (ctrl_i.west);
    \path[draw,->] (output.east) |- node[anchor=south, pos=0] {$V_N$} (ctrl_d_filt.west);
  %   \path[draw,->] (ctrl_kd.east) -- (ctrl_d_filt.west);
    \path[draw,->] (output.east -| ref_sum.south) -- (ref_sum.south);
    \path[draw,->] ([xshift=0.65*\nodesep]ref_sum.east) |- (ctrl_p_gain.west);
    \path[draw,->] (ctrl_i.east) -- (ctrl_i_gain.west);
    \path[draw,->] (ctrl_d_filt.east) -- (ctrl_d.west);
    \path[draw,->] (ctrl_d.east) -- (ctrl_d_gain.west);
    \path[draw,->] (ctrl_d_gain.east) -| (ctrl_sum.south);
    \path[draw,->] (ctrl_i_gain.east) -- (ctrl_sum.west);
    \path[draw,->] (ctrl_p_gain.east) -| (ctrl_sum.north);
    \path[draw,->] (ctrl_sum.east) -- (sat_ctrl.west);
    \path[draw,->] (sat_ctrl.east) -- node[anchor=south] {$A_{N_{sp}}$} (ctrl.west);
  
  \end{tikzpicture}

  \caption{Velocity controller diagram \cite{PX4userguide}}
  \label{fig:velocity_controller}

\end{figure}

        \paragraph
        Recall from Chapter~\ref{chap:system_id}, that the cascaded PID controller is used for velocity step inputs in the training data phase for system identifcation.
        The system identification methods produce linear models, which are used in swing damping controllers to minimise the payload angles during flight.
        Therefore training data with large payload angles is undesirable, because it is very different from the vector subspace that the swing damping controllers will operate in.
        The velocity controller gains used for Honeybee by \cite{Grobler2020} result in aggressive veloicty responses, which produces large payload swing angles.
        Therefore the velocity controller gains are redesigned for a slower transient reponse which results in smaller payload swing angles.

        \paragraph
        The trasfer function, $G_{V_N}(s)$, of the plant considered for North velocity control was derived by \cite{Erasmus2020}.
        As a brief summary, the derivation starts by linearising the multirotor pitch rate dynamics at hover with the small angle approximation.
        The closed-loop transfer function including the pitch rate controller is then determined.
        This function is then used in the derivation of the closed-loop transfer function of the pitch dynamics, which includes the pitch angle controller.
        Finally, this is used to derive the open-loop, North velocity transfer function, $G_{V_N}(s)$, with $A_{N,sp}(s)$ as input and $V_N$(s) as output 
        For more detail regarding this derivation, readers are refered to the work by \cite{Erasmus2020}.  

        \begin{figure}
            \captionsetup[subfigure]{justification=centering}
            \centering  
            \begin{subfigure}[t]{0.45\columnwidth}
    \centering
    \begin{tikzpicture}
        \begin{axis}[            
            xlabel = Imaginary,
            ylabel = Real,
            % x unit = \si{\hertz},
            % y unit = \si{\second},
            xmin = 0.3,  xmax = 1.7,
            ymin = 0,    ymax = 0.8,
            grid = major,
            legend cell align = left,
            legend pos = north east,
            grid style = dashed,
            legend style = {font = \scriptsize},
            label style = {font = \scriptsize},
            tick label style = {font = \scriptsize},
            width = 0.95\columnwidth,
            height = 0.95\columnwidth,
            % initialize Dark2
            cycle list/Dark2,
            % combine it with 'mark list*':
            cycle multiindex* list = {
                Dark2\nextlist
            }
        ]

        % \addplot+[mark = none, style = solid, ultra thick] 
        % table[x = f, y = P1, col sep = comma] 
        % {};

        \end{axis}
    \end{tikzpicture} 
    
    \caption{Open loop transfer function, $G_{V_N}$.}
    \label{fig:rootlocus_open_loop}
\end{subfigure}
 % subfigure
            \begin{subfigure}[t]{0.45\columnwidth}
    \centering
    \begin{tikzpicture}
        \begin{axis}[            
            xlabel = Frequency,
            ylabel = Amplitude,
            x unit = \si{\hertz},
            % y unit = \si{\second},
            xmin = 0.3,  xmax = 1.7,
            ymin = 0,    ymax = 0.8,
            grid = major,
            legend cell align = left,
            legend pos = north east,
            grid style = dashed,
            legend style = {font = \scriptsize},
            label style = {font = \scriptsize},
            tick label style = {font = \scriptsize},
            width = 0.95\columnwidth,
            height = 0.95\columnwidth,
            % initialize Dark2
            cycle list/Dark2,
            % combine it with 'mark list*':
            cycle multiindex* list = {
                Dark2\nextlist
            }
        ]

        % \addplot+[mark = none, style = solid, ultra thick] 
        % table[x = f, y = P1, col sep = comma] 
        % {};

        \end{axis}
    \end{tikzpicture} 
    
    \caption{Closed loop transfer function including the North velocity controller and $G_{V_N}$}
    \label{fig:rootlocus_closed_loop}
\end{subfigure}
 % subfigure
            \caption{Root locus plots of the North velocity controller}
            \label{fig:rootlocus_subfigs}  
        \end{figure}

        \begin{figure}[htb]
    \centering
    \begin{tikzpicture}
        \begin{axis}[            
            xlabel = Time,
            ylabel = Velocity,
            x unit = \si{\second},
            y unit = \si{\metre/\second},
            xmin = 0,   xmax = 16,
            ymin = -0.1,  ymax = 3,
            grid = major,
            legend cell align = left,
            legend pos = south east,
            grid style = dashed,
            legend style = {font = \scriptsize},
            label style = {font = \scriptsize},
            tick label style = {font = \scriptsize},
            width = 0.95\columnwidth,
            height = 0.5\columnwidth,
            % initialize Dark2
            cycle list/Dark2,
            % combine it with 'mark list*':
            cycle multiindex* list = {
                Dark2\nextlist
            }
        ]
        
        \addplot+[mark = none, style = solid, ultra thick] 
        table[x = time, y = vel_sp, col sep = comma] 
        {control/csv/compare_control_pid_Simulink_single_pend_mp0.1_l0.5_PID_vel_steps_tune_scale_0.7.csv};
        \addlegendentry{Setpoint}

        \addplot+[mark = none, style = solid, ultra thick] 
        table[x = time, y = vel, col sep = comma] 
        {control/csv/compare_control_pid_Simulink_single_pend_mp0.1_l0.5_PID_vel_steps_tune_scale_0.7.csv};
        \addlegendentry{($l =$~\SI{0.5}{\meter}, $m_p =$~\SI{0.1}{\kilo\gram})}

        \addplot+[mark = none, style = solid, ultra thick] 
        table[x = time, y = vel, col sep = comma] 
        {control/csv/compare_control_pid_Simulink_single_pend_mp0.2_l1_PID_vel_steps_tune_scale_0.7.csv};
        \addlegendentry{($l =$~\SI{1}{\meter}, $m_p =$~\SI{0.2}{\kilo\gram})}

        \addplot+[mark = none, style = dashed, ultra thick] 
        table[x = time, y = vel, col sep = comma] 
        {control/csv/compare_control_pid_Simulink_single_pend_mp0.3_l2.25_PID_vel_steps_tune_scale_0.7.mat.csv};
        \addlegendentry{($l =$~\SI{2}{\meter}, $m_p =$~\SI{0.3}{\kilo\gram})}

        \end{axis}
    \end{tikzpicture} 
    \caption{\gls{PID} velocity step responses with different payloads}
    \label{fig:pid_tuning_plot}
\end{figure}


        Figure~\ref{fig:pid_tuning_plot} shows velocity step response for a cascaded PID controller with a payload producing large oscillations.
        % ?? Talk about pid response. What do you want to show?
        % ?? Check connection with system id chapter for training data controller
