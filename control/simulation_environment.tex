\FloatBarrier\section{Simulation Environment} \label{sec:simulation_environment}

    \paragraph
    The controllers in this chapter are tested within a Simulink\texttrademark~simulation environment.
    This is used rather than the \gls{SITL}/Gazebo simulation environment because it allows us to iterate designs faster.
    Simulink\texttrademark~provides a graphical interface for control system design which is helpful for rapid development.
    In contrast, the \gls{SITL}/Gazebo simulation environment requires text-based control laws in C\texttt{++} and requires \gls{ROS} nodes as an interface an between the \gls{MPC} and PX4.
    This requires a lot more development time for control system design than the graphical tools in Simulink\texttrademark.
    The \gls{SITL}/Gazebo simulation also has a longer runtime per simulation, which further adds to the development time of the iterative design process.
    
    \paragraph
    The multirotor and suspended payload system is modelled in Simulink\texttrademark~with the differential equations derived in Chapter~\ref{chap:modelling}.
    The resulting controllers are also implemented in Simulink\texttrademark.
    Using the cascaded \gls{PID} control architecture (discussed in the sections below), this simulation environment was verified against practical data with and without a payload.
    The plots in 
    Figure~\ref{fig:prac_vs_sim_vel_step_no_payload}, \ref{fig:prac_vs_sim_vel_step_with_payload}, and \ref{fig:prac_vs_sim_theta_with_payload} 
    from Chapter~\ref{chap:modelling}
    show how well the simulations match the actual system dynamics.
    The controllers will therefore be designed, tested and compared using this simulation environment.

    