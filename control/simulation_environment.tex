\FloatBarrier\section{Simulation Environment} \label{sec:simulation_environment}

    \paragraph
    The controllers in this chapter are tested with a MATLAB/Simulink simulation environment.
    This is used rather than the SITL and Gazebo simulation environment because it allows us to iterate designs faster.
    Simulink provides a graphical interface for control system design which enables us to change the control system design with ease.
    In contrast, the SITL and Gazebo simulation environment requires text-based control laws with C\texttt{++} and requires ROS nodes to interface an MPC with PX4.
    This requires a lot more development time for control system design than the graphical tools in Simulink.
    The SITL and Gazebo simulation also has a longer runtime per simulation, which further adds to the development time of the iterative control system design process.
    
    \paragraph
    The quadrotor and suspended payload system is modelled in Simulink with the differential equations derived in Chapter~\ref{chap:modelling}.
    The resulting controllers are also applied in Simulink.
    Using the cascade PID control architecture (discussed in the sections below), this simulation environment was verified against practical data with and without a payload.
    The plots in 
    Figure~\ref{fig:prac_vs_sim_vel_step_no_payload}, 
    Figure~\ref{fig:prac_vs_sim_vel_step_with_payload}, and
    Figure~\ref{fig:prac_vs_sim_theta_with_payload} 
    from Chapter~\ref{chap:modelling}
    show how well the simulations match the actual system dynamics.
    The controllers will therefore be designed, tested and compared using this simulation environment.

    