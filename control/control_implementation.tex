{
\tikzset{external/figure name/.add={control_implementation/}{}}

\FloatBarrier\section{Implentation and results}
\label{sec:control_implementation}

    \paragraph
    After the system identification phase, active swing damping control can be applied
    to the multirotor and payload system.
    The control architectures are summarised in Table~\ref{tbl:controller_summary} 
    by pairing the system identification techniques along with the appropriate controllers.
    In this work, the \emph{\gls{MPC} architecture} refers to the entire control implementation, 
    which includes the data-driven system identification process and the resulting \gls{MPC} based on the estimated model.
    Likewise, the \emph{\gls{LQR} architecture} includes the white-box modelling, the parameter estimation process, and the resulting \gls{LQR} determined from the system identification model.
    
    \paragraph
    It was firstly shown in Chapter~\ref{chap:system_id} that the system identifcation techniques worked in simulation.
    
    \FloatBarrier\subsection{Simple suspended payload} \label{sec:simple_payload_control}

        \paragraph
        The modelling assumptions of the white-box model discussed in Chapter~\ref{chap:modelling} 
        defines a point-mass suspended with a rigid cable which is attached to the \gls{CoM} of the multirotor.
        This is a simplisitic suspended payload model but represents the dynamics of many practical payloads well.
        In this section, the simulated payload model complies will all these assumptions.
        The simulation model used in this section was verified with practical data in Section~\ref{sec:model_verification}.
        This is also the payload model used for simulations with an \gls{LQR} controller by \cite{Erasmus2020} and \cite{Slabber2020}

        \begin{figure}[htb]
    \centering
    \begin{tikzpicture}
        \begin{axis}[            
            xlabel = Time,
            ylabel = Velocity,
            x unit = \si{\second},
            y unit = \si{\metre/\second},
            xmin = 0,   xmax = 16,
            ymin = -0.1,  ymax = 2.7,
            grid = major,
            legend cell align = left,
            legend pos = south east,
            grid style = dashed,
            legend style = {font = \scriptsize},
            label style = {font = \scriptsize},
            tick label style = {font = \scriptsize},
            width = 0.95\columnwidth,
            height = 0.5\columnwidth,
            % initialize Dark2
            cycle list/Dark2,
            % combine it with 'mark list*':
            cycle multiindex* list = {
                Dark2\nextlist
            }
        ]
        
        \addplot+[mark = none, style = solid, ultra thick] 
        table[x = time, y = vel_sp, col sep = comma] 
        {control/csv/compare_control_pid_Simulink_single_pend_mp0.3_l2.25_PID_vel_steps_tune_scale_0.7.mat.csv};
        \addlegendentry{Setpoint}

        \addplot+[mark = none, style = solid, ultra thick] 
        table[x = time, y = vel, col sep = comma] 
        {control/csv/compare_control_pid_Simulink_single_pend_mp0.3_l2.25_PID_vel_steps_tune_scale_0.7.mat.csv};
        \addlegendentry{PID}

        \addplot+[mark = none, style = solid, ultra thick] 
        table[x = time, y = vel, col sep = comma] 
        {control/csv/compare_control_mpc_Simulink_single_pend_mp0.3_l2.25_PID_vel_steps_tune_scale_0.7.mat.csv};
        \addlegendentry{MPC}

        \addplot+[mark = none, style = dashed, ultra thick] 
        table[x = time, y = vel, col sep = comma] 
        {control/csv/compare_control_lqr_Simulink_single_pend_mp0.3_l2.25_PID_vel_steps_tune_scale_0.7.mat.csv};
        \addlegendentry{LQR}


        \end{axis}
    \end{tikzpicture} 
    \caption{Velocity step response comparison for different controllers
    ($l =$~\SI{2}{\meter}, $m_p =$~\SI{0.3}{\kilo\gram})}
    \label{fig:compare_control_vel}
\end{figure}


        \paragraph
        From simulation results, it appears that both the \gls{MPC} and \gls{LQR} effectively damp the payload oscillations while controlling the velocity of the multirotor.
        Figure~\ref{fig:compare_control_vel} shows the velocity step responses of the \gls{MPC}, \gls{LQR} and \gls{PID} controllers for a multirotor with a suspended payload.
        From Figure~\ref{fig:compare_control_vel} it is clear that both the \gls{LQR} and \gls{MPC} controllers actively damp the velocity oscillation caused by the swinging payload.
        The \gls{PID} controller does not consider the payload angle, hence the oscillations are not damped as strongly.

        \paragraph
        For the \gls{MPC} and LQR, the respective models were first generated in the training phase of the simulation.
        Thereafter, the \gls{MPC} and \gls{LQR} were manually and iteratively tuned to produce a step response with a similar response time and overshoot.
        The \gls{PID} response shown uses the same controller gains used in the training phase.

        % MPC:
        % vel weight = 2;
        % theta weight = 0;
        % dtheta weight = 10; 
        % mv weight = 1;
        % mvrate weight = 5;
        % Ty = 5;
        % Tu = 3;

        % LQR:
        % LQR.Q = diag([0.3 10 0 100]); 
        % LQR.R = 8

        \begin{figure}[htb]
    \centering
    \begin{tikzpicture}
        \begin{axis}[            
            xlabel = Time,
            ylabel = Payload angle,
            x unit = \si{\second},
            y unit = \si{\degree},
            xmin = 0,   xmax = 16,
            ymin = -20,  ymax = 15,
            grid = major,
            legend cell align = left,
            legend pos = south east,
            grid style = dashed,
            legend style = {font = \scriptsize},
            label style = {font = \scriptsize},
            tick label style = {font = \scriptsize},
            width = 0.95\columnwidth,
            height = 0.5\columnwidth,
            % initialize Dark2
            cycle list/Dark2,
            % combine it with 'mark list*':
            cycle multiindex* list = {
                Dark2\nextlist
            }
        ]
                
        \pgfplotsset{cycle list shift = 1}

        \addplot+[mark = none, style = solid, ultra thick] 
        table[x = time, y = theta, col sep = comma] 
        {control/csv/compare_control_pid_Simulink_single_pend_mp0.3_l2.25_PID_vel_steps_tune_scale_0.7.mat.csv};
        \addlegendentry{PID}

        \addplot+[mark = none, style = solid, ultra thick] 
        table[x = time, y = theta, col sep = comma] 
        {control/csv/compare_control_mpc_Simulink_single_pend_mp0.3_l2.25_MPC_vel_steps_tune_scale_0.7.csv};
        \addlegendentry{MPC}

        \addplot+[mark = none, style = dashed, ultra thick] 
        table[x = time, y = theta, col sep = comma] 
        {control/csv/compare_control_lqr_Simulink_single_pend_mp0.3_l2.25_PID_vel_steps_tune_scale_0.7.mat.csv};
        \addlegendentry{LQR}

        \end{axis}
    \end{tikzpicture} 
    \caption{Payload angle comparison for different controllers
    ($l =$~\SI{2}{\meter}, $m_p =$~\SI{0.3}{\kilo\gram})}
    \label{fig:compare_control_theta}
\end{figure}


        \paragraph
        Figure~\ref{fig:compare_control_theta} shows the payload angle data of the velocity step response.
        Both the \gls{MPC} and \gls{LQR} damp the payload angle well and the osciilations cease after only two or three swings.
        In this case, the \gls{MPC} response results in a smaller initial swing angle, however, this is dependant on the specific tuning of each controller.
        The \gls{LQR} can also be tuned to produce a similar swing angle.

        \begin{figure}[htb]
    \centering
    \begin{tikzpicture}
        \begin{axis}[            
            xlabel = Time,
            ylabel = North velocity,
            x unit = \si{\second},
            y unit = \si{\metre/\second},
            xmin = 0,   xmax = 16,
            ymin = -0.1,  ymax = 2,
            grid = major,
            legend cell align = left,
            legend pos = north east,
            grid style = dashed,
            legend style = {font = \scriptsize},
            label style = {font = \scriptsize},
            tick label style = {font = \scriptsize},
            width = 0.95\columnwidth,
            height = 0.5\columnwidth,
            % initialize Dark2
            cycle list/Dark2,
            % combine it with 'mark list*':
            cycle multiindex* list = {
                Dark2\nextlist
            }
        ]
        
        \pgfplotsset{cycle list shift = 2}

        \addplot+[mark = none, style = solid, ultra thick] 
        table[x = time, y = acc_sp, col sep = comma] 
        {control/csv/compare_control_mpc_Simulink_single_pend_mp0.3_l2.25_MPC_vel_steps_tune_scale_0.7.csv};
        \addlegendentry{MPC}

        \addplot+[mark = none, style = dashed, ultra thick] 
        table[x = time, y = acc_sp, col sep = comma] 
        {control/csv/compare_control_lqr_Simulink_single_pend_mp0.3_l2.25_LQR_vel_steps_tune_scale_0.7.csv};
        \addlegendentry{LQR}


        \end{axis}
    \end{tikzpicture} 
    \caption{Acceleration setpoint commanded by different controllers for a velocity step input
    ($l =$~\SI{2}{\meter}, $m_p =$~\SI{0.3}{\kilo\gram}).}
    \label{fig:compare_control_acc_sp}
\end{figure}


        \paragraph
        Figure~\ref{fig:compare_control_acc_sp} shows the acceleration setpoint commanded by the two controllers for this step response.
        This is probably due to the inherant similarity between the controller implementations as discussed in Section~\ref{sec:mpc}.
        The similarity in the acceleration setpoint responses also show that the energy expended in a velocity steps are roughly equal for these two controller implementations.
        However, this is also highly dependant on the weightings used in optimastion problem of both controllers.
        Both controllers also produce a non-zero steady-state setpoint as expected, which is required to counter areodynamic drag.

    \FloatBarrier\subsection{Different payload parameters}

        \paragraph
        The system identification and control implementations are required to perform well with different unknown payload parameters.
        Therefore, numerous flights with a range of different payload were simulated, the respective models were trained and the controllers were implemented.
        Figure~\ref{fig:compare_control_vel_l1} and Figure~\ref{fig:compare_control_vel_l05} show velocity step responses with \gls{LQR} and \gls{MPC} implentations with two payloads flights.
        Both the parameter estimation with \gls{LQR} implementation, and the \gls{DMDc} with \gls{MPC} implementation, handle flights with different cable lengths and payload masses well.
        In each flight, \gls{LQR} and \gls{MPC} damp the payload oscillations and control the multirotor velocity well.

        \begin{figure}[hb]
    \centering
    \begin{tikzpicture}
        \begin{axis}[            
            xlabel = Time,
            ylabel = North velocity,
            x unit = \si{\second},
            y unit = \si{\metre/\second},
            xmin = 0,   xmax = 16,
            ymin = -0.1,  ymax = 2.5,
            grid = major,
            legend cell align = left,
            legend pos = south east,
            grid style = dashed,
            legend style = {font = \scriptsize},
            label style = {font = \scriptsize},
            tick label style = {font = \scriptsize},
            width = 0.95\columnwidth,
            height = 0.5\columnwidth,
            % initialize Dark2
            cycle list/Dark2,
            % combine it with 'mark list*':
            cycle multiindex* list = {
                Dark2\nextlist
            }
        ]
        
        \addplot+[mark = none, style = solid, ultra thick] 
        table[x = time, y = vel_sp, col sep = comma] 
        {control/csv/compare_control_mpc_Simulink_single_pend_mp0.2_l1_MPC_vel_steps_tune_scale_0.7.csv};
        \addlegendentry{Setpoint}

        \addplot+[mark = none, style = solid, ultra thick] 
        table[x = time, y = vel, col sep = comma] 
        {control/csv/compare_control_pid_Simulink_single_pend_mp0.2_l1_PID_vel_steps_tune_scale_0.7.csv};
        \addlegendentry{PID}

        \addplot+[mark = none, style = solid, ultra thick] 
        table[x = time, y = vel, col sep = comma] 
        {control/csv/compare_control_mpc_Simulink_single_pend_mp0.2_l1_MPC_vel_steps_tune_scale_0.7.csv};
        \addlegendentry{MPC}

        \addplot+[mark = none, style = dashed, ultra thick] 
        table[x = time, y = vel, col sep = comma] 
        {control/csv/compare_control_lqr_Simulink_single_pend_mp0.2_l1_LQR_vel_steps_tune_scale_0.7.csv};
        \addlegendentry{LQR}


        \end{axis}
    \end{tikzpicture} 
    \caption{Velocity step response comparison of different controllers
    ($l =$~\SI{1}{\meter}, $m_p =$~\SI{0.2}{\kilo\gram}).}
    \label{fig:compare_control_vel_l1}
    
\end{figure}
 % subfigure

        \begin{figure}[htb]
    \centering
    \begin{tikzpicture}
        \begin{axis}[            
            xlabel = Time,
            ylabel = North velocity,
            x unit = \si{\second},
            y unit = \si{\metre/\second},
            xmin = 0,   xmax = 16,
            ymin = -0.1,  ymax = 2.5,
            grid = major,
            legend cell align = left,
            legend pos = south east,
            grid style = dashed,
            legend style = {font = \scriptsize},
            label style = {font = \scriptsize},
            tick label style = {font = \scriptsize},
            width = 0.95\columnwidth,
            height = 0.5\columnwidth,
            % initialize Dark2
            cycle list/Dark2,
            % combine it with 'mark list*':
            cycle multiindex* list = {
                Dark2\nextlist
            }
        ]
        
        \addplot+[mark = none, style = solid, ultra thick] 
        table[x = time, y = vel_sp, col sep = comma] 
        {control/csv/compare_control_lqr_Simulink_single_pend_mp0.1_l0.5_LQR_vel_steps_tune_scale_0.7.csv};
        \addlegendentry{Setpoint}

        \addplot+[mark = none, style = solid, ultra thick] 
        table[x = time, y = vel, col sep = comma] 
        {control/csv/compare_control_pid_Simulink_single_pend_mp0.1_l0.5_PID_vel_steps_tune_scale_0.7.csv};
        \addlegendentry{PID}

        \addplot+[mark = none, style = solid, ultra thick] 
        table[x = time, y = vel, col sep = comma] 
        {control/csv/compare_control_mpc_Simulink_single_pend_mp0.1_l0.5_MPC_vel_steps_tune_scale_0.7.csv};
        \addlegendentry{MPC}

        \addplot+[mark = none, style = dashed, ultra thick] 
        table[x = time, y = vel, col sep = comma] 
        {control/csv/compare_control_lqr_Simulink_single_pend_mp0.1_l0.5_LQR_vel_steps_tune_scale_0.7.csv};
        \addlegendentry{LQR}


        \end{axis}
    \end{tikzpicture}
    \caption{Velocity step response comparison of different controllers
    ($l =$~\SI{0.5}{\meter}, $m_p =$~\SI{0.1}{\kilo\gram}).}
    \label{fig:compare_control_vel_l05}
\end{figure}
 % subfigure
                    
        \paragraph
        The controllers were not specifically tuned for each simulation.
        Instead, the same controller parameters were used for these simulations as for the simulations in Section~\ref{sec:simple_payload_control}.
        This shows that each control architecture is adaptibile to different payload parameters without manual intervention.

    \FloatBarrier\subsection{Variation of other system parameters}

        \paragraph
        The control architectures have been shown to be adaptable to variationd in the payload parameters.
        However, changing other system parameters may affect the performance of the different controllers. 
        As mentioned in Chapter~\ref{chap:system_id}, a disadvantage of the white-box system identification approach used by the LQR,
        is that parameter estimation techniques need to be manually designed for each unknown parameter.
        In the specific implementation, the \gls{LQR} model assumes that the multirotor mass is known.
        Hence, changing the mass of the multirotor is detrimental to the accuracy of plant model and therefore affects the \gls{LQR} performance.
        In contrast, the data-driven system identification method for the \gls{MPC} plant model does not rely on such modelling assumptions.
        
        \begin{figure}
            \captionsetup[subfigure]{justification=centering}
            \centering  
            \begin{figure}[htb]
    \centering
    \begin{tikzpicture}
        \begin{axis}[            
            xlabel = Time,
            ylabel = Velocity,
            x unit = \si{\second},
            y unit = \si{\metre/\second},
            xmin = 0,   xmax = 16,
            ymin = -0.1,  ymax = 2.7,
            grid = major,
            legend cell align = left,
            legend pos = south east,
            grid style = dashed,
            legend style = {font = \scriptsize},
            label style = {font = \scriptsize},
            tick label style = {font = \scriptsize},
            width = 0.95\columnwidth,
            height = 0.5\columnwidth,
            % initialize Dark2
            cycle list/Dark2,
            % combine it with 'mark list*':
            cycle multiindex* list = {
                Dark2\nextlist
            }
        ]
        
        \addplot+[mark = none, style = solid, ultra thick] 
        table[x = time, y = vel_sp, col sep = comma] 
        {control/csv/compare_control_lqr_Simulink_single_pend_mp0.3_l1.5_LQR_vel_steps_tune_scale_0.7_1.5_x_mq.csv};
        \addlegendentry{Setpoint}

        \addplot+[mark = none, style = solid, ultra thick] 
        table[x = time, y = vel, col sep = comma] 
        {control/csv/compare_control_pid_Simulink_single_pend_mp0.3_l1.5_PID_vel_steps_tune_scale_0.7_1.5_x_mq.csv};
        \addlegendentry{PID}

        \addplot+[mark = none, style = solid, ultra thick] 
        table[x = time, y = vel, col sep = comma] 
        {control/csv/compare_control_mpc_Simulink_single_pend_mp0.3_l1.5_MPC_vel_steps_tune_scale_0.7_1.5_x_mq.csv};
        \addlegendentry{MPC}

        \addplot+[mark = none, style = dashed, ultra thick] 
        table[x = time, y = vel, col sep = comma] 
        {control/csv/compare_control_lqr_Simulink_single_pend_mp0.3_l1.5_LQR_vel_steps_tune_scale_0.7_1.5_x_mq.csv};
        \addlegendentry{LQR}


        \end{axis}
    \end{tikzpicture} 
    \caption{Velocity step responses with an altered multirotor mass
    ($m_q =$~\SI{0.3}{\kilo\gram}, $l =$~\SI{1.5}{\meter}, $m_p =$~\SI{0.3}{\kilo\gram})}
    \label{fig:compare_control_vel_mq_changed}
\end{figure}

            \begin{subfigure}[t]{\columnwidth}
    \centering
    \begin{tikzpicture}
        \begin{axis}[            
            xlabel = Time,
            ylabel = Payload angle,
            x unit = \si{\second},
            y unit = \si{\degree},
            xmin = 0,    xmax = 16,
            ymin = -18,  ymax = 6,
            grid = major,
            legend cell align = left,
            legend pos = south east,
            grid style = dashed,
            legend style = {font = \scriptsize},
            label style = {font = \scriptsize},
            tick label style = {font = \scriptsize},
            width = 0.95\columnwidth,
            height = 0.5\columnwidth,
            % initialize Dark2
            cycle list/Dark2,
            % combine it with 'mark list*':
            cycle multiindex* list = {
                Dark2\nextlist
            }
        ]
                
        \pgfplotsset{cycle list shift = 1}

        \addplot+[mark = none, style = solid, ultra thick] 
        table[x = time, y = theta, col sep = comma] 
        {control/csv/compare_control_pid_Simulink_single_pend_mp0.3_l0.5_PID_vel_steps_tune_scale_0.7_mq_0.546_mp_0.3.csv};
        \addlegendentry{PID}
        

        \addplot+[mark = none, style = solid, ultra thick] 
        table[x = time, y = theta, col sep = comma] 
        {control/csv/compare_control_mpc_Simulink_single_pend_mp0.3_l0.5_MPC_vel_steps_tune_scale_0.7_mq_0.546_mp_0.3.csv};
        \addlegendentry{MPC}

        \addplot+[mark = none, style = dashed, ultra thick] 
        table[x = time, y = theta, col sep = comma] 
        {control/csv/compare_control_lqr_Simulink_single_pend_mp0.3_l0.5_LQR_vel_steps_tune_scale_0.7_mq_0.546_mp_0.3_l_est_0.35696.csv};
        \addlegendentry{LQR}

        \end{axis}
    \end{tikzpicture} 

\end{subfigure}

            \caption{Velocity step responses with the multirotor mass decreased by \SI{0.25}{\kilo\gram}
            ($l =$~\SI{0.5}{\meter}, $m_p =$~\SI{0.3}{\kilo\gram})}
            \label{fig:mq_changed} 
        \end{figure}

        \paragraph
        Simulations were performed with an altered multirotor mass to demonstrate how the control architectures handle changes in other system parameters.
        Figure~\ref{fig:mq_changed} shows the velocity step responses of the \gls{PID}, \gls{MPC} and \gls{LQR} implementations with an altered multirotor mass.
        For these simulations, the original multirotor mass, $m_Q =$~\SI{0.796}{\kilo\gram}, 
        was decreased by \SI{0.250}{\kilo\gram}, 
        resulting in a new multirotor mass of, $m_Q =$~\SI{0.546}{\kilo\gram}.
        The same system identification processes were naively followed as in the previous sections, without prior knowledge of the change in $m_Q$.
        The same tuned controller parameters were also used.

        \paragraph
        In Figure~\ref{fig:mq_changed} it appears that the \gls{LQR} results in lower payload oscillations than the \gls{PID} controller
        but induces higher frequency oscillations.
        This results in a jittery velocity response with the \gls{LQR} and is undesirable for a multirotor flight.
        The \gls{LQR} control performance has degraded because the dynamics of the \gls{LQR} plant model differs significantly from the actual dynamics. 
        In contrast, the \gls{MPC} still results in a smooth velocity profile and damps the payload oscillations effectively, as in previous simulations.
        This is expected since the system identification model used by the \gls{MPC} included the effect of the changed mass by estimating the entire model without considering indivdual parameters.
        
        \paragraph
        It should be noted that another mass estimator can be implemented to estimate $m_Q$ in a flight stage before the payload is added.
        However, this involves manually redesigning the system identification procedure for each new unknown system parameter.
        In these simulations, it was shown that changing non-estimated system parameters in the white-box approach can be detrimental to the control performance.
        Unlike the white-box approach, the black-box approach handles changes in different system parameters well without prior knowledge of these parameters.

    \FloatBarrier\subsection{Wind disturbance}

        \paragraph
        For zero steady-state error with a practical system, a controller needs to apply some form of disturbance rejection.
        Practical systems experience unmeasured disturbances and other deviations which are not accounted for by the plant model.
        For example, a mean force applied by wind could prevent zero steady-state tracking error of the multirotor velocity without disturbance rejection.
        As discussed in Section~\ref{sec:lqr}, an integral state variable was added to the \gls{LQR} plant model for integral action of the multirotor velocity tracking.
        As discussed in Section~\ref{sec:mpc}, an unmeasured input was added to the \gls{MPC} plant model with a disturbance estimator to apply integral action to the multirotor velocity.

        \begin{figure}[htb]
    \centering
    \begin{tikzpicture}
        \begin{axis}[            
            xlabel = Time,
            ylabel = North velocity,
            x unit = \si{\second},
            y unit = \si{\metre/\second},
            xmin = 0,   xmax = 16,
            ymin = -0.1,  ymax = 2.5,
            grid = major,
            legend cell align = left,
            legend pos = south east,
            grid style = dashed,
            legend style = {font = \scriptsize},
            label style = {font = \scriptsize},
            tick label style = {font = \scriptsize},
            width = 0.95\columnwidth,
            height = 0.5\columnwidth,
            % initialize Dark2
            cycle list/Dark2,
            % combine it with 'mark list*':
            cycle multiindex* list = {
                Dark2\nextlist
            }
        ]
        
        \addplot+[mark = none, style = solid, ultra thick] 
        table[x = time, y = vel_sp, col sep = comma] 
        {control/csv/compare_control_mpc_Simulink_single_pend_mp0.3_l2.25_MPC_vel_steps_tune_scale_0.7_wind_step_disturb.csv};
        \addlegendentry{Setpoint}

        \pgfplotsset{cycle list shift = 1} % Shift to skip \gls{PID} colour used in other plots

        \addplot+[mark = none, style = solid, ultra thick] 
        table[x = time, y = vel, col sep = comma] 
        {control/csv/compare_control_mpc_Simulink_single_pend_mp0.3_l2.25_MPC_vel_steps_tune_scale_0.7_wind_step_disturb.csv};
        \addlegendentry{MPC}

        \addplot+[mark = none, style = dashed, ultra thick] 
        table[x = time, y = vel, col sep = comma] 
        {control/csv/compare_control_lqr_Simulink_single_pend_mp0.3_l2.25_LQR_vel_steps_tune_scale_0.7_wind_step_disturb.csv};
        \addlegendentry{LQR}


        \end{axis}
    \end{tikzpicture} 
    \caption{Effect of an unmeasured step input disturbance.
    ($l =$~\SI{2}{\meter}, $m_p =$~\SI{0.3}{\kilo\gram}).}
    \label{fig:compare_control_vel_wind_disturbance}
\end{figure}


        \paragraph
        Figure~\ref{fig:compare_control_vel_wind_disturbance} shows the responses of the controllers from Section~\ref{sec:simple_payload_control} with a constant wind disturbance starting at \SI{8}{\second}.
        At Time~=~\SI{8}{\second}, a wind speed of \SI{2}{\metre/\second} is applied to the simulation model as an unmeasured step input.
        This mostly affects the multirotor velocity because the wind causes a greater drag force on the multirotor, 
        hence a larger acceleration setpoint is required to maintain a constant velocity.
        For both system identification approaches, the models were trained without wind.

        \paragraph
        It appears that the \gls{MPC} shows better disturbance rejection than the \gls{LQR} when using the controller parameters which were tuned for good performance in Section~\ref{sec:simple_payload_control}.
        This is primarily because the weighting of the integral variable in the \gls{LQR} optimisation was minimised to reduce overshoot.
        The integral weighting can be increased to improve integral action at the expense of increasing overshoot in the velocity response.

        \begin{figure}[htb]
    \centering
    \begin{tikzpicture}
        \begin{axis}[            
            xlabel = Time,
            ylabel = Velocity,
            x unit = \si{\second},
            y unit = \si{\metre/\second},
            xmin = 0,   xmax = 16,
            ymin = -0.1,  ymax = 3,
            grid = major,
            legend cell align = left,
            legend pos = south east,
            grid style = dashed,
            legend style = {font = \scriptsize},
            label style = {font = \scriptsize},
            tick label style = {font = \scriptsize},
            width = 0.95\columnwidth,
            height = 0.5\columnwidth,
            % initialize Dark2
            cycle list/Dark2,
            % combine it with 'mark list*':
            cycle multiindex* list = {
                Dark2\nextlist
            }
        ]
        
        \addplot+[mark = none, style = solid, ultra thick] 
        table[x = time, y = vel_sp, col sep = comma] 
        {control/csv/compare_control_mpc_Simulink_single_pend_mp0.3_l2.25_MPC_vel_steps_tune_scale_0.7_wind_step_disturb.csv};
        \addlegendentry{Setpoint}

        \pgfplotsset{cycle list shift = 2} % Shift to skip PID colour used in other plots

        % \addplot+[mark = none, style = solid, ultra thick] 
        % table[x = time, y = vel, col sep = comma] 
        % {control/csv/compare_control_mpc_Simulink_single_pend_mp0.3_l2.25_MPC_vel_steps_tune_scale_0.7_wind_step_disturb.csv};
        % \addlegendentry{MPC}

        \addplot+[mark = none, style = dashed, ultra thick] 
        table[x = time, y = vel, col sep = comma] 
        {control/csv/compare_control_lqr_Simulink_single_pend_mp0.3_l2.25_LQR_vel_steps_tune_scale_0.7_wind_step_disturb_intg_weight_0.1.csv};
        \addlegendentry{Integrator weighting: 0.1}

        \addplot+[mark = none, style = solid, ultra thick] 
        table[x = time, y = vel, col sep = comma] 
        {control/csv/compare_control_lqr_Simulink_single_pend_mp0.3_l2.25_LQR_vel_steps_tune_scale_0.7_wind_step_disturb_intg_weight_1.csv};
        \addlegendentry{Integrator weighting: 1}

        \addplot+[mark = none, style = solid, ultra thick] 
        table[x = time, y = vel, col sep = comma] 
        {control/csv/compare_control_lqr_Simulink_single_pend_mp0.3_l2.25_LQR_vel_steps_tune_scale_0.7_wind_step_disturb_intg_weight_10.csv};
        \addlegendentry{Integrator weighting: 10}


        \end{axis}
    \end{tikzpicture} 
    \caption{Different LQR responses for different integrator gains
    ($l =$~\SI{2}{\meter}, $m_p =$~\SI{0.3}{\kilo\gram}).}
    \label{fig:improve_lqr_integrator}
\end{figure}


        \paragraph
        Figure~\ref{fig:improve_lqr_integrator} shows the \gls{LQR} responses with different integral state weightings.
        The other state variable weights are kept constant for each response.
        It is clear that the settling time and disturbance rejection of the \gls{LQR} improves for larger integral state weighting.
        However, the overshoot increases significantly because of the integrator build-up at the start of the response.
        % The overshoot could be decreased by adding a velocity derivative state to the \gls{LQR} model, but this will involve ex

        \paragraph
        In contrast to the LQR, the \gls{MPC} shows good disturbance rejection while maintaining a low overshoot.
        This is because the disturbance estimator applies integral action which depends on the deviation of the actual dynamics from the plant model.
        whereas the \gls{LQR} applies integral action proportional to the integral of the tracking error.
        Therefore the \gls{MPC} implementation produces less integrator build up which results in a lower overshoot.

        % Could add effect of bad models. ??

    \FloatBarrier\subsection{Dynamic payload}

        \paragraph
        As discussed in Section~\ref{sec:dynamic_payload}, some payloads have dynamics that differ significantly from a suspended rigid mass.
        In this work, these payloads are referred to as dynamic payloads.
        An example of such a payload is an elongated payload, which can be represented by a double pendulum model. 
        
        \paragraph
        In Section~\ref{sec:dynamic_payload}, the proposed system identification techniques were tested on simulated flight data with such a payload.
        For the white-box system identification approach, it was shown that the white-box model captures the dynamics of the dominant frequency of the oscillating payload but ignores the higher frequency dynamics.
        For the black-box approach, a prediction model was generated from a set of training data.
        This model accurately predicted the multirotor and payload dynamics, including the low and high-frequency dynamics, of a set of testing data.

        \begin{figure}[htb]
    \centering
    \begin{tikzpicture}
        \begin{axis}[            
            xlabel = Time,
            ylabel = Velocity,
            x unit = \si{\second},
            y unit = \si{\metre/\second},
            xmin = 0,   xmax = 16,
            ymin = -0.1,  ymax = 2.7,
            grid = major,
            legend cell align = left,
            legend pos = south east,
            grid style = dashed,
            legend style = {font = \scriptsize},
            label style = {font = \scriptsize},
            tick label style = {font = \scriptsize},
            width = 0.95\columnwidth,
            height = 0.5\columnwidth,
            % initialize Dark2
            cycle list/Dark2,
            % combine it with 'mark list*':
            cycle multiindex* list = {
                Dark2\nextlist
            }
        ]
        
        \addplot+[mark = none, style = solid, ultra thick] 
        table[x = time, y = vel_sp, col sep = comma] 
        {control/csv/compare_control_pid_Simulink_single_pend_mp0.3_l2.25_PID_vel_steps_tune_scale_0.7.mat.csv};
        \addlegendentry{Setpoint}

        \addplot+[mark = none, style = solid, ultra thick] 
        table[x = time, y = vel, col sep = comma] 
        {control/csv/compare_control_pid_Simulink_single_pend_mp0.3_l2.25_PID_vel_steps_tune_scale_0.7.mat.csv};
        \addlegendentry{PID}

        \addplot+[mark = none, style = solid, ultra thick] 
        table[x = time, y = vel, col sep = comma] 
        {control/csv/compare_control_mpc_Simulink_single_pend_mp0.3_l2.25_PID_vel_steps_tune_scale_0.7.mat.csv};
        \addlegendentry{MPC}

        \addplot+[mark = none, style = dashed, ultra thick] 
        table[x = time, y = vel, col sep = comma] 
        {control/csv/compare_control_lqr_Simulink_single_pend_mp0.3_l2.25_PID_vel_steps_tune_scale_0.7.mat.csv};
        \addlegendentry{LQR}


        \end{axis}
    \end{tikzpicture} 
    \caption{Velocity step response comparison for different controllers
    ($l =$~\SI{2}{\meter}, $m_p =$~\SI{0.3}{\kilo\gram})}
    \label{fig:compare_control_vel}
\end{figure}


        \paragraph
        For the simulations in this section, accurate system identification models were generated as described in Section~\ref{sec:dynamic_payload} and used in the \gls{MPC} and \gls{LQR} controllers.
        Figure~\ref{fig:compare_control_vel_dynamic} shows the reslting controller responses with a dynamic payload.
        It appears that the velocity responses of both the \gls{LQR} and the \gls{MPC} are less smooth than with a simple suspended payload and show small velocity oscillations.
        
        \begin{figure}[htb]
    \centering
    \begin{tikzpicture}
        \begin{axis}[            
            xlabel = Time,
            ylabel = Payload angle,
            x unit = \si{\second},
            y unit = \si{\degree},
            xmin = 0,   xmax = 16,
            ymin = -20,  ymax = 15,
            grid = major,
            legend cell align = left,
            legend pos = south east,
            grid style = dashed,
            legend style = {font = \scriptsize},
            label style = {font = \scriptsize},
            tick label style = {font = \scriptsize},
            width = 0.95\columnwidth,
            height = 0.5\columnwidth,
            % initialize Dark2
            cycle list/Dark2,
            % combine it with 'mark list*':
            cycle multiindex* list = {
                Dark2\nextlist
            }
        ]
                
        \pgfplotsset{cycle list shift = 1}

        \addplot+[mark = none, style = solid, ultra thick] 
        table[x = time, y = theta, col sep = comma] 
        {control/csv/compare_control_pid_Simulink_single_pend_mp0.3_l2.25_PID_vel_steps_tune_scale_0.7.mat.csv};
        \addlegendentry{PID}

        \addplot+[mark = none, style = solid, ultra thick] 
        table[x = time, y = theta, col sep = comma] 
        {control/csv/compare_control_mpc_Simulink_single_pend_mp0.3_l2.25_MPC_vel_steps_tune_scale_0.7.csv};
        \addlegendentry{MPC}

        \addplot+[mark = none, style = dashed, ultra thick] 
        table[x = time, y = theta, col sep = comma] 
        {control/csv/compare_control_lqr_Simulink_single_pend_mp0.3_l2.25_PID_vel_steps_tune_scale_0.7.mat.csv};
        \addlegendentry{LQR}

        \end{axis}
    \end{tikzpicture} 
    \caption{Payload angle comparison for different controllers
    ($l =$~\SI{2}{\meter}, $m_p =$~\SI{0.3}{\kilo\gram})}
    \label{fig:compare_control_theta}
\end{figure}

        
        \paragraph
        Figure~\ref{fig:compare_control_theta_dynamic} shows the suspension cable angle for a velocity step response.
        It appears that the \gls{LQR} and \gls{MPC} damp the payload oscillations with a similar response time.
        However, the superimposed, high-frequency oscillations are smaller in the \gls{LQR} response than in the \gls{MPC} response.
        The \gls{LQR} naively damps the payload oscillations because its controller gain is determined from the same dynamical model as for a simple suspended payload.
        Because the angle of the payload relative to the cable is not measured or considered in the \gls{LQR} plant model, it does not directly damp these high-frequency oscillations.
        The \gls{MPC} should account for the superimposed frequency and provide a smoother response than the \gls{LQR}, since the black-box model includes the double pendulum dynamics in its prediction model.
        Therefore it is expected that the \gls{MPC} optimiser determines a smooth trajectory that damps both the low and high-frequency oscillations. 

        \begin{figure}[htb]
    \centering
    \begin{tikzpicture}
        \begin{axis}[            
            xlabel = Time,
            ylabel = Velocity,
            x unit = \si{\second},
            y unit = \si{\metre/\second},
            xmin = 0,   xmax = 9.1,
            ymin = -0.1,  ymax = 2.4,
            grid = major,
            legend cell align = left,
            legend pos = south east,
            grid style = dashed,
            legend style = {font = \scriptsize},
            label style = {font = \scriptsize},
            tick label style = {font = \scriptsize},
            width = 0.95\columnwidth,
            height = 0.5\columnwidth,
            % initialize Dark2
            cycle list/Dark2,
            % combine it with 'mark list*':
            cycle multiindex* list = {
                Dark2\nextlist
            }
        ]
        
        \addplot+[mark = none, style = solid, ultra thick] 
        table[x = time, y = vel_sp, col sep = comma] 
        {control/csv/compare_control_mpc_Simulink_double_pend_m1-0.2_m2-0.1_l1-0.5_l2-0.6_MPC_vel_steps_tune_scale_0.7_slow_minimal_oscillations.csv};
        \addlegendentry{Setpoint}

        \addplot+[mark = none, style = solid, ultra thick] 
        table[x = time, y = prediction.vel, col sep = comma] 
        {control/csv/mpc_prediction_vs_actual_Simulink_double_pend_m1-0.2_m2-0.1_l1-0.5_l2-0.6_MPC_vel_steps_tune_scale_0.7.csv};
        \addlegendentry{Optimised prediction}

        \addplot+[mark = none, style = solid, ultra thick] 
        table[x = time, y = vel, col sep = comma] 
        {control/csv/compare_control_mpc_Simulink_double_pend_m1-0.2_m2-0.1_l1-0.5_l2-0.6_MPC_vel_steps_tune_scale_0.7.csv};
        \addlegendentry{Actual response}

        \end{axis}
    \end{tikzpicture} 
    \caption{Optimised prediction and actual velocity response of the MCP with a dynamic payload}
    \label{fig:mpc_vel_prediction_vs_actual}
\end{figure}


        \paragraph
        However, the \gls{MPC} does not provide a smooth velocity or payload angle response.
        Even though the \gls{MPC} generates a smooth optimised trajectory with the plant model, the actual response of the simulated system differs from this prediction.
        Figure~\ref{fig:mpc_vel_prediction_vs_actual} shows the predicted velocity of the \gls{MPC} optimiser, given the velocity setpoint and initial condition at Time~=~\SI{1.1}{\second}.
        The actual simulated response, resulting from replanning at every time-step with the \gls{MPC}, is also shown in Figure~\ref{fig:mpc_vel_prediction_vs_actual}.
        For the first part of the velocity response, the the actual response matches the optimised trajectory well.
        However, after Time~=~\SI{3.2}{\second}, the predicted dynamics is noticeably different from the actual response to the optimised input sequence.

        \begin{figure}[htb]
    \centering
    \begin{tikzpicture}
        \begin{axis}[            
            xlabel = Time,
            ylabel = Payload angle,
            x unit = \si{\second},
            y unit = \si{\degree},
            xmin = 0,   xmax = 9,
            ymin = -12,  ymax = 5,
            grid = major,
            legend cell align = left,
            legend pos = south east,
            grid style = dashed,
            legend style = {font = \scriptsize},
            label style = {font = \scriptsize},
            tick label style = {font = \scriptsize},
            width = 0.95\columnwidth,
            height = 0.5\columnwidth,
            % initialize Dark2
            cycle list/Dark2,
            % combine it with 'mark list*':
            cycle multiindex* list = {
                Dark2\nextlist
            }
        ]
                
        \pgfplotsset{cycle list shift = 1}

        \addplot+[mark = none, style = solid, ultra thick] 
        table[x = time, y = prediction.theta, col sep = comma] 
        {control/csv/mpc_prediction_vs_actual_Simulink_double_pend_m1-0.2_m2-0.1_l1-0.5_l2-0.6_MPC_vel_steps_tune_scale_0.7.csv};
        \addlegendentry{Optimised prediction}

        \addplot+[mark = none, style = solid, ultra thick] 
        table[x = time, y = theta, col sep = comma] 
        {control/csv/compare_control_mpc_Simulink_double_pend_m1-0.2_m2-0.1_l1-0.5_l2-0.6_MPC_vel_steps_tune_scale_0.7.csv};
        \addlegendentry{Actual response}

        \end{axis}
    \end{tikzpicture} 
    \caption{Optimised prediction and actual payload angle response of the MCP with a dynamic payload}
    \label{fig:mpc_theta_prediction_vs_actual}
\end{figure}


        \paragraph
        Figure~\ref{fig:mpc_theta_prediction_vs_actual} shows the predicted and actual payload angle response for this simulation, starting at the same time-step.
        The \gls{MPC} optimiser also determined a smooth trajectory for the payload angle, but the actual response differs significantly from this trajectory.
        Even though the black-box model predictions accurately matched the testing data, Figure~\ref{fig:mpc_vel_prediction_vs_actual} and Figure~\ref{fig:mpc_theta_prediction_vs_actual}  
        show that the model is not an accurate approximation of the simulated system for all values of the state and input vectors. 

        \paragraph
        The estimated model provides accurate predictions in the domain of state and input vectors considered in the training data.
        However, the \gls{MPC} generates trajectories that are beyond this domain of training data.
        The multirotor with a simple suspended payload represents a mildly non-linear system, hence the linear approximation was effective for control with an \gls{MPC}.
        However, a double pendulum system reveals highly non-linear dynamics with multiple fixed points.
        This system also includes an unmeasured state variable which adds complexity to dynamics.
        
        \paragraph
        From the results in Figure~\ref{fig:mpc_vel_prediction_vs_actual} and Figure~\ref{fig:mpc_theta_prediction_vs_actual}
        it appears that the data-driven linear model results in acceptable control with an \gls{MPC}.
        However, the actual dynamics do not follow the optimised trajectory of the \gls{MPC} and the \gls{MPC} does not outperform the \gls{LQR} implementation.

        % ?? Add discussion of how it could be improved:
        % Constrain \gls{MPC} to stay within domain
        % Adaptive \gls{MPC} to include larger domain

\FloatBarrier\section{Conclusion}

    \paragraph
    From simulations without wind disturbances, it was shown that the \gls{MPC} and \gls{LQR} architectures deliver similar control performances for a range of different payload parameters.
    Both controllers result in a similar response time and velocity overshoot, and the payload angle is damped well by both controllers.
    Therefore, in the absence of wind, 
    the control performance does not conclusively differentiate between the \gls{LQR} and \gls{MPC} architectures for a simple suspended payload.
    As expected, the \gls{PID} controller does not provide acceptable control of the multirotor with a suspended payload and does not actively damp the payload oscillations.
    
    \paragraph
    Both the \gls{LQR} and the \gls{MPC} architectures handle different payload parameters well.
    Even though the parameter estimation techniques (used with the LQR) did not consider the mass of the multirotor ($m_Q$),
    the \gls{LQR} architecture still provided acceptable swing damping control with small changes in $m_Q$.
    However, it was shown that the \gls{LQR} architecture produces an undesirable control performance for large changes in $m_Q$.
    Therefore, parameter estimation procedure will need to be redesigned to account for such changes in other system parameters.
    However, the data-driven approach does not rely on modelling assumptions, hence the \gls{MPC} still provides good control performance for different values of $m_Q$.
    
    \paragraph
    Both the \gls{LQR} and the \gls{MPC} effectively rejected the unmeasured input disturbance caused by wind, resulting in zero steady-state tracking error for the multirotor velocity.
    However, the \gls{LQR} implementations which are tuned for effective disturbance rejection result in large velocity overshoots due to the integrator build-up.
    The \gls{MPC} implementation applies integral action with a disturbance estimator and achieves zero steady-state error without increasing the velocity overshoot.

    \paragraph
    For simulations with a dynamic payload, both the \gls{LQR} and \gls{MPC} effectively applied swing damping control.
    However, there the trajectories were not as smooth as with the simple suspended payload.
    Even though the simulated dynamics differed significantly from the simple suspended payload model used by the LQR,
    the \gls{LQR} still managed to damp the payload oscillations quickly. 

    \paragraph
    The \gls{MPC} also managed to damp the payload oscillations, but did not conclusively outperform the \gls{LQR}.
    Even though the estimated model used by the \gls{MPC} showed good prediction accuracy with the given testing data set, 
    the actual system response did not follow the predicted trajectory of the \gls{MPC} well.
    It appears that the optimised trajectory of the \gls{MPC} is beyond the domain where the linear model provides an accurate approximation of the non-linear dynamics.
    % Therefore, the proposed \gls{DMDc} implementation does not provide a sufficient linear model approximation for effective \gls{MPC} control of such a dynamic payload.
    Therefore, an improved data-driven system identification model, which provides an accurate approximation  of the dynamics for a larger domain, 
    is required for improved \gls{MPC} control of such a dynamic payload.

    \paragraph
    The advantage of the \gls{LQR} architecture is that it is computationally simple in comparison to an \gls{MPC}.
    However, the \gls{LQR} architecture is designed for a specific system configuration and only accounts for changes in specific system parameters. 
    In contrast, the \gls{MPC} architecture provides a good general solution for different system configurations without considering individual parameters and without a priori modelling.

} % end of tikz set