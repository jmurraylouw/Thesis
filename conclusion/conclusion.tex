\graphicspath{{conclusion/fig/}}

\chapter{Conclusion} \label{chap:conclusion}

    \paragraph
    This thesis considered the design and practical implementation of a stabilising control architecture for a multirotor with an unknown suspended payload.
    A broad scope was considered which includes two different area of research, namely, 
    \begin{itemize}
        \item Data-driven system identification of the unknown payload dynamics 
        \item Optimal swing damping control of the multirotor-payload system
    \end{itemize}
    The content and outcome of this work will be discussed in this chapter.

    \section{Literature study}

        \paragraph
        Existing solutions for stabilised multirotor control with a suspended payload were identified in the literature.
        The literature study showed that research seldomly includes experimental results or algorithm testing on practical hardware, even though this would provide valuable insight.
        It also showed that most studies do not account for uncertainty in the controlled system.
        A thorough study of the literature showed that some solutions account for parameter uncertainty, but very few assume no knowledge of the payload dynamics.
        
        \paragraph
        Furthermore, the few studies that achieved stabilised control despite unknown dynamics, 
        counteracted the payload effect as an unknown disturbance instead of actively controlling the payload state. 
        This places the focus on robustness rather than smooth control of the complete multirotor-payload system. 

        % \paragraph
        % Therefore, this study aimed to design a stabilising control architecture that achieves smooth control of the complete system without prior payload knowledge.
        % Furthermore, it was deemed important to demonstrate the technique using experimental data and practical computational hardware.

        % \paragraph
        % The controller design started with the mathematical derivation of a dynamical model for a multirotor named Honeybee.
        % The dynamical equations of the multirotor-payload system and PX4 controller architecture were implemented in a Simulink\texttrademark~simulation for controller design and testing.
        
        \paragraph
        An \gls{LQR} controller was identified as a popular baseline controller in the literature and was selected as the baseline swing damping controller for this work.
        The specific \gls{LQR} implementation considered in this work is based on a previous study that only considers parameter uncertainty. 
        This \gls{LQR} controller is based on a linearised, predetermined model of the multirotor-payload system.
        The payload mass and cable length are unknown prior to a flight and are estimated with \gls{RLS} and \gls{FFT} estimators respectively.
    
    \section{System identification}  

        \paragraph
        The baseline parameter estimation technique was described and applied to data from \gls{SITL} simulations with Gazebo.
        It was shown that the white-box model which uses the estimated parameters captured the general shape of the system state predictions well.
        The white-box model with parameter estimation technique was also applied to a dynamic payload simulation.
        An elongated payload was suspended from the multirotor and acted as a double pendulum, inducing irregular oscillations in the system.
        For this use case, the resultant white-box model predictions did not represent the general shape of the payload dynamics.

        \paragraph
        \gls{DMDc} and \gls{HAVOKc} were introduced as the data-driven system identification techniques proposed by this work.
        These linear regression techniques each produce a discrete, linear space-space model of the considered dynamics based on input and output data only.
        The conventional \gls{HAVOK} is not designed to be applied to controlled systems.
        However, this algorithm was extended in this work to account for control inputs in a dynamical system and will be referred to as \gls{HAVOKc}.
        The conventional \gls{DMDc} algorithm was altered to include delay-coordinates in a similar way to \gls{HAVOK}.
        Furthermore, the mathematical complexity of these techniques was described in detail.

        \paragraph
        These algorithms were applied to multiple \gls{SITL} simulations for testing.
        Data was generated by tracking a sequence of random velocity step inputs with the standard PID controllers from PX4.
        This data was split into training and testing sets.
        The algorithms could then be trained on the set of training data and could be validated on the unseen testing set.
        The prediction accuracy of each model produced by these techniques was quantified with an \gls{NMAE} error metric.
        This metric is based on multiple model prediction runs from different initial conditions over a specified time horizon.

        \paragraph
        A hyperparameters search showed a Pareto elbow as a function of the number of delay-coordinates, $q$, such that increasing $q$ passed this elbow does not significantly increase the model accuracy, but does increase model complexity.
        Furthermore, the `double-descent' phenomenon was identified when testing with various lengths of training data.
        It was consistently observed in different experiments that increasing the length of training data past a specific point decreases the prediction accuracy.
        This is unintuitive because longer lengths of training data are expected to increase model accuracy by reducing overfitting.
        
        \paragraph
        Both techniques were shown to be robust to measurement noise.
        It was also shown that the techniques consistently produced accurate models with a range of different system parameters.
        The techniques were also tested with the dynamic pendulum and the prediction models accurately captured the irregular oscillations, despite having no prior knowledge of the payload.
        This showed a major improvement compared to the white-box model.
        For the range of different tests the prediction accuracies of \gls{DMDc} and \gls{HAVOKc} models were similar, hence \gls{DMDc} is preferred due to lower computational complexity.

        % \paragraph
        % The uncertainty considered by the baseline method is that the payload mass and cable length are unknown.
        % The control architecture proposed by this work is not based on a predetermined model of the payload dynamics.
        % This method assumes that the dynamical model of the payload is also unknown prior to flight.
    
    \section{Swing damping controllers}  

        \paragraph
        Furthermore, the different controllers were discussed and tested.
        The cascaded PID controller was described and the gains of each control loop were tuned in simulation.
        This simulation environment was verified with practical data from the Honeybee multirotor and was shown to be an accurate representation of the actual system.
        
        \paragraph
        The baseline \gls{LQR} controller was also described and the weights were tuned for the simulated multirotor-payload system.
        The \gls{LQR} was designed based on the white-box model which uses estimated parameters for each different payload.
        The control architecture proposed in this work includes an \gls{MPC} controller which uses a data-driven system identification model for predictive control.
        The \gls{MPC} implementation from the Model Predictive Control Toolbox\texttrademark~in Simulink\texttrademark~was used and the algorithm was described in detail.
        This controller, using data-driven system identification models, was also successfully applied for swing damping control in simulation.

% ?? Velocity step tests

        \paragraph
        Numerous tests were performed to evaluate the combined system identification and control architectures.
        For each test, the system identification stages firstly produced the parameter estimates and identified model for the \gls{LQR} and \gls{MPC} approaches respectively.
        Thereafter, each controller was applied based on those results.
        For a payload with a simple pendulum model, the \gls{LQR} and \gls{MPC} both achieved stabilised control resulting in near swing free motion.
        The controllers showed similar swing damping performances, but the \gls{MPC} produced a faster settling time.
        The control architectures consistently achieved swing damping control with different payload masses and cable lengths, despite parameter uncertainty for the \gls{LQR} approach and no prior payload knowledge for the \gls{MPC} approach.
        Both controllers also showed acceptable disturbance rejection during a constant unknown step input force to the multirotor.

% ?? consistently
% ?? Shown to be adaptive

        \paragraph
        The control approaches were also tested for the dynamic payload case.
        Despite the data-driven model showing a much better prediction accuracy than the white-box model, the \gls{LQR} and \gls{MPC} approached produced similar swing damping performances.
        It was shown that even though the data-driven model is accurate in the domain of the state and input vectors considered in training, the optimised trajectory of the \gls{MPC} goes beyond that domain.
        Hence, the model approximation is inaccurate and the resulting \gls{MPC} control is suboptimal.
        Both controller responses were not as smooth as for the simple pendulum model.
        However, both controllers showed stabilised swing damping control of the system and reduced system oscillations quickly.

        \paragraph
        The controller architectures were also applied to a simulation where a different multirotor mass was used. 
        The \gls{LQR} controller induced undesirable, high-frequency oscillations in the system response due to the model inaccuracy.
        This is because the baseline approach only considers parameter uncertainty in the payload mass and cable length.
        However, the \gls{MPC} approach produced the same swing-free motion shown in previous simulations and clearly outperformed the \gls{LQR} approach.
        The proposed control architecture does not rely on prior knowledge of the system dynamics, therefore the architecture can adapt to unconsidered system changes without redesigning the implementation.

    \section{Practical implementation}  

        \paragraph
        For experimental work, the hardware and software toolchains were described.
        Numerous practical flights were performed with different payload masses, cable lengths, and with a dynamics payload.
        All conclusions made with simulation data for the system identification methods were verified with practical flight data.
        This further validated that the simulation environment is a good representative of the actual system.
        This also showed that \gls{DMDc} produces accurate prediction models with practical levels of wind disturbances and measurement noise.

        \paragraph
        Finally, \gls{HITL} simulations were performed to demonstrate the computationally intensive controller algorithm working with the actual hardware and software toolchain.
        This involved a complex system of different, interconnected software tools.
        The \gls{MPC} algorithm ran as a standalone \gls{ROS} node on the \gls{OBC} and was generated from Simulink\texttrademark~code.
        Tests showed that the OBC was acceptable for the processing requirements of the \gls{MPC} and the algorithm could run at the desired speed.
        It was also shown that the final system produces smooth, swing damping control of the multirotor-payload system as seen in previous simulations.

        \paragraph
        Overall, it was shown that the full data-driven system identification with \gls{MPC} control architecture produces swing damping control of the multirotor-payload system despite having no prior knowledge of the payload dynamics.
        It was demonstrated that this control architecture works for various configurations with different system parameters and even with a dynamic payload.
        Furthermore, it was demonstrated that an accurate data-driven prediction model could be determined from practical flight data with wind disturbances and measurement noise.
        It was also demonstrated that the available hardware fulfils the computational requirements of the proposed algorithms.
        Finally, it is concluded that the proposed control architecture is practically feasible for stabilised control of a multirotor and suspended payload with unknown dynamics.
        
    \section{Recommended future work}
        
        \paragraph
        The current work can be continued to further show the practical feasibility of this approach.
        It can also be extended to improve the control performance or to solve different control problems with the same approach.
        Recommendations for future work include:
        \begin{itemize}
            \item Perform practical flights with the \gls{MPC} to demonstrate the practical performance of the controller.
            \item Perform practical flights with a rigidly attached container with sloshing fluid to demonstrate the adaptability of this approach to unknown dynamics.
            % \item Adding an input shaper to the controller input to further improve the swing damping performance.
            \item Test the current solution using obstacle avoidance trajectories and manoeuvres that require the payload to follow a specific trajectory instead of a simple non-swing reference.
            \item Add a continuous excitation functionality to continue training the system identification model during flight and possibly adapt to time-variant uncertainty. 
            \item Quantify the domain represented by the state and input vectors considered in the training data. 
            This could be used to adjust the input signal during training to cover a larger domain and ultimately train a model that better represents the system dynamics.
            This would hopefully improve the controller performance with unknown dynamical models.
            \item Compare the proposed control architecture to a non-linear \gls{MPC} using a non-linear data-driven model like \gls{SINDy}
        \end{itemize}

