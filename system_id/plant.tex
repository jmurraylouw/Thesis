\section{Plant considered for system identification} \label{sec:plant_considered}
 
    % Gebruik dalk hierdie par erens??
    % \paragraph
    % As discussed in \ref{sec:linear_model}, the payload is attached near the \gls{CoM} of the vehicle 
    % and has a minimal effect on the attitude dynamics.
    % The swing damping controllers are therefore applied in the translational velocity loop
    % and the original attitude controls are used for inner loop controllers.
    % Therefore the attitude state variables of the multirotor are excluded from the system identification model.

    % The model identified by \gls{DMDc} or \gls{HAVOKc} will be used to design a longitudinal velocity controller.
    % As shown in \ref{fig:system_id_plant}, the plant considered for system identification includes the dynamics of the inner loop, attitude controllers.
    % The swing damping controllers which will utilise the identified model act only in the translational velocity loop.
    % Because of the large time-scale separation between the inner and outer loop controllers, 
    % the attitude states have a negligible effect on the plant dynamics seen by the velocity controller.
    % As discussed in Section~\ref{sec:linear_model}, the payload minimally effects the multirotor attitude because it is attached near the \gls{CoM} of the vehicle.
    % Therefore the attitude states are excluded from the system identification model.

    \paragraph
    A specific subsystem of the multirotor-payload system will be considered for system identification.
    The proposed controllers in Chapter~\ref{chap:control} will be applied for North velocity control, therefore the longitudinal dynamics will be considered in this section.
    The input vector of the plant model is given by,
    \begin{equation}
        \bm{u} = \begin{bmatrix}
            A_{N_{sp}}
        \end{bmatrix} ,
    \end{equation}
    where $A_{N_{sp}}$ is the North acceleration setpoint in the inertial frame.
    $A_{N_{sp}}$ is used by the attitude controllers which will be explained in Section~\ref{sec:cascaded_pid}.
    The state vector of the considered plant is,
    \begin{equation}
        \bm{x} = \begin{bmatrix}
            V_N & \theta & \dot{\theta}
        \end{bmatrix}^T ,
    \end{equation}
    where $\theta$ and $\dot{\theta}$ are the payload angle and angular rate about the inertial East axis,
    and $V_N$ is the North velocity of the multirotor in the inertial frame. 
    A schematic of this \gls{2D} plant is shown in Figure \ref{fig:floating_pend}, where $m_Q$ is the multirotor mass, $m_p$ is the payload mass, and $l$ is the cable length.

    \begin{figure}[htb]
        \centering
        \includegraphics[width=0.45\linewidth]{floating_pend.png}            
        \caption{Schematic of a floating pendulum model considered for a North velocity controller}
        \label{fig:floating_pend}
    \end{figure}

    

    % Also note that the inner loop controllers handle the attitude dynamics of the multirotor.
    % During simulations of pure longitudinal velocity setpoints, 
    % it was observed that the multirotor experiences negligible altitude changes due the swinging payload.
    % ?? This will be discussed in Section~\ref{sec:}
    % This is due to the speed of the altitude controllers and the weak coupling between the payload angle and the altitude dynamics, 
    % discussed in Chapter~\ref{chap:modelling}.
    % The plant seen by the system identification process therefore mimics the common pendulum-on-a-cart model.
    
    % The differential equations that describe the motion of this system 
    % were derived with Lagrangian mechanics in Chapter~\ref{chap:modelling}.
    % ?? Linear state-space model 

    % From this derivation it is clear that the angular velocity of the payload, $\dot{\theta}$, is required to described the system dynamics.
    % However, $\dot{\theta}$ is not measured directly on the considered practical multirotor setup.
    % Instead, the payload angle, $\theta$, is measured by a potentiometer attached to a \gls{ADC} on Honeybee as described in Chapter \ref{chap:exp_design}.
    % As expected, this measurement is extremely noisy.
    % \murray{Maybe insert figure to show noise}
    % % Figure \ref{} shows the angle measurement during a practical experiment of the payload while Honeybee is held stationary
    % Numerical differentiation is applied to the noisy $\theta$ signal which results in a very inaccurate estimation of $\dot{\theta}$.
    % Therefore it is desirable to rather use $\theta$ in the system identification process. 
    
    \paragraph
    In subsequent sections, simulations of the full, non-linear multirotor-payload system will be performed
    and different methods will be applied to determine system identification models of this plant.
