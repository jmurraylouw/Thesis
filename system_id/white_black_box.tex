\section{White-box and black-box techniques}

    \paragraph
    Models determined from data-driven system identification methods are generally called black-box models.
    The user is only concerned with the inputs and output of the model and does not need to derive mathematical equations from theoretical deductions.
    In contrast, white-box models are determined from a~priori modelling.

    \subsection{White-box techniques}

        \paragraph 
        The underlying physics of a white-box model is usually derived from first principles.
        This is done by modelling physical processes with techniques such as Lagrangian or Newton mechanics.
        Hence, the mathematical relations between system parameters are predefined in the modelling phase.
        The system identification process is therefore reduced to parameter estimation to determine values for the system parameters.

        \paragraph 
        This approach is used by \cite{Erasmus2020} and \cite{Slabber2020} for system identification in a swing damping control architecture for a multirotor with an unknown suspended payload.
        The same system definition and modelling process are used in this thesis.
        Recall from Chapter~\ref{chap:modelling} that the multirotor-payload system is modelled with the following assumptions:
        \begin{itemize}
            \item The payload is a point-mass.
            \item The link is massless.
            \item The link is rigid.
            \item The link is attached to the \gls{CoM} of the multirotor.
        \end{itemize}
        The only unknown parameters in the multirotor-payload model are the payload mass and the link length.
        These parameters are estimated and inserted into the predefined, linearised model.
        This model is used by a \gls{LQR} controller to damp swing angles while controlling the vehicle.

        \paragraph
        The main advantage of this approach is its simplicity.
        In the case considered by \cite{Erasmus2020} and \cite{Slabber2020}, only two parameters are estimated.
        In contrast, numerous values need to be estimated to reproduce the system dynamics with a black-box model.
        Therefore, white-box system identification methods are often less computationally complex and are easily implemented on low-cost hardware.
        Due to the lower complexity, 
        parameter estimation algorithms often require shorter lengths of training data 
        than data-driven methods to produce accurate models.

        \paragraph 
        Therefore the white-box approach works well for systems with predictable dynamics. 
        However, is not very adaptable to systems that deviate from the predefined dynamics.
        The payload considered by \cite{Erasmus2020} and \cite{Slabber2020} is limited to specific modelling assumptions. 
        However, if a payload or cable is used that violates one of the assumptions, the predefined model will no longer represent the system accurately.
        Many payloads considered for practical drone deliveries do not conform to these assumptions.
        
        \paragraph
        Furthermore, the payload uncertainty is limited to selected parameters.
        A change in a parameter value that is not considered by the parameter estimator may result in large modelling errors.
        Since the controller is dependent on this model, 
        this may result in undesirable controller behaviour.
        Therefore a new model and parameter estimation technique will need to be derived for every use case that deviates significantly from the a~priori model.

    \subsection{Black-box techniques}

        \paragraph
        Data-driven system identification methods produce black-box models.
        These models do not require predefined mathematical relations between system parameters.
        % The user only considers what goes into, and comes out of, the black-box.
        % Something imagery about why it is called black-box
        No prior knowledge of the physics of the system are considered and no modelling assumptions are made.
        Black-box techniques determine the mathematical relationship between the inputs and outputs of a system using information from measurement data only.

        \paragraph
        A disadvantage of the data-driven system identification approach is its computational complexity.
        Data-driven algorithms generally have a much higher computational complexity than parameter estimation techniques.
        This is expected since a lot more values are generated to populate a black-box model than a predefined white-box model.
        In the multirotor use case, more expensive computational hardware is required to implement a data-driven method compared to parameter estimation methods.
        Furthermore, data-driven methods generally require more training data than parameter estimation methods.
        This means that more flight time is wasted on system identification before an updated controller can be activated.
        
        \paragraph
        However, black-box techniques are very adaptable and provide a general system identification solution for a broad range of different dynamics.
        This is a major advantage over white-box techniques, which need to be manually redesigned for different use cases.
        % Add more advantages ??

        \paragraph
        Black-box models can be categorised as non-linear or linear models.
        The dynamics of a multirotor and suspended payload are non-linear and are better represented by non-linear models.
        % Examples of black-box models with multirotors and payloads in literature ???
        However, non-linear models are more complex than linear models. 
        Non-linear models may require control implementations that are too computationally intensive to be implemented with the hardware available on multirotors.
        Controllers based on low-complexity, linear models are attractive because these may be simple enough to execute on low-cost hardware.
        % trade-off between accuracy and complexity.??
        
        \paragraph
        \gls{DMDc} and \gls{HAVOKc} are the data-driven system identification methods investigated in this work. 
        These are linear regression techniques that produce linear models to approximate non-linear dynamics.
        Non-linear data-driven techniques like Neural Networks and \gls{SINDy} \cite{Brunton2016a} 
        may produce models with higher accuracy at the cost of greater computational complexity.
        \gls{DMDc} and \gls{HAVOKc} are preferred due to their lower complexity which may improve their practical feasibility for multirotor systems.