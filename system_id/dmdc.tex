\section{Dynamic mode decomposition with control}
\label{sec:dmdc}
    
    \paragraph        
    \gls{DMD} is a regression technique that can be used to approximate a non-linear dynamical system with a linear model \cite{Tu2014}.
    It uses temporal measurements of system outputs to reconstruct system dynamics without prior modelling assumptions.
    \gls{DMDc} is an adaptation of \gls{DMD} that also accounts for control inputs \cite{Proctor2016c}.
    This section provides an overview of the specific implementation of \gls{DMDc} used in this work.        
    Note that this implementation is an adaptation of \gls{DMDc}, and includes time-delay-embedding of multiple variables. 
    Enriching a \gls{DMD} model with time-delay-embedding is a known technique and is also seen in other \gls{DMD} adaptations \cite{Korda2018b, Arbabi2018}.

    \paragraph
    \gls{DMD} produces a linear, discrete state-space model of system dynamics.
    Discrete measurements, $\bm{x}_k$, of the continuous time variable, $\bm{x}(t)$, are used, 
    where $\bm{x}_k = \bm{x}(k T_s)$, and $T_s$ is the sampling time of the model.    
    Delay-coordinates (i.e. $\bm{x}_{k-1}, \bm{x}_{k-2}$, etc.) are also included in the state-space model to
    account for input delay and state delay in the system.
    Input delay refers to the time delay involved with transporting a control signal to a system, 
    whereas state delay refers to time-separated interactions between system variables \cite{Chen1999}.
    Hence, we define a state delay vector as:
    \begin{equation} \label{eq:delay_vector}
        \bm{d}_{k} = 
        \begin{bmatrix}
            \bm{x}_{k-1} & \bm{x}_{k-2} & \cdots & \bm{x}_{k-q+1}
        \end{bmatrix}^T ,
    \end{equation}
    where 
    $q$ is the number of delay-coordinates 
    (including the current time-step) used in the model, and $\bm{d}_k \in \R^{(n_x)(q-1)}$.
    
    \paragraph
    The discrete state-space model is therefore defined as:
    \begin{equation} \label{eq:dmd_state_space}
        \bm{x}_{k+1} = \bm{A}_{dmd} \bm{x}_k + \bm{A}_d \bm{d}_k + \bm{B}_d \bm{u}_k ,
    \end{equation}
    where
    \( \bm{A}_{dmd} \in \R^{n_x \times n_x} \) is the system matrix, 
    \( \bm{A}_d \in \R^{(q-1) \cdot n_x ~ \times ~ (q-1) \cdot n_x} \) is the state delay system matrix and 
    \( \bm{B}_d \in \R^{n_x \times n_u} \) is the input matrix.
    
    \paragraph
    The training data consists of full-state measurements, $\bm{x}_k$, and corresponding inputs, $\bm{u}_k$, 
    taken at regular intervals of $\Delta t = T_s$, during a simulated flight with Cascaded \gls{PID} control.
    In a practical flight, these time-series measurements need to be saved in memory because it is processed as a single batch by \gls{DMD}.
    Note that \gls{DMD} can be applied in a recursive manner as described in \cite{Noack2016}.
    However, this implementation is not considered because an \gls{OBC} with significant memory size can be used.
    
    \paragraph
    The training data is collected into the following matrices:
    \begin{align} \label{eq:dmd_matrices}
        \bm{X^\prime} = & \phantom{.} \left [
            \begin{array}{*{5}{@{}M{\mycolwd}@{}}}
                    \bm{x}_{q+1} & \bm{x}_{q+2} & \bm{x}_{q+3} & \cdots & \bm{x}_{w+q}
            \end{array}
        \right ] , \nonumber \\
        %
        \bm{X} \phantom{''} = & \phantom{.} \left [
            \begin{array}{*{5}{@{}M{\mycolwd}@{}}}
                \bm{x}_{q} & \bm{x}_{q+1} & \bm{x}_{q+2} & \cdots & \bm{x}_{w+q-1}                      
            \end{array}
        \right ] , \nonumber \\
        % 
        \bm{X}_d \phantom{'} = & \phantom{'} \left [
            \begin{array}{*{5}{@{}M{\mycolwd}@{}}}
                \bm{x}_{q-1} & \bm{x}_{q+0} & \bm{x}_{q+1} & \cdots & \bm{x}_{w+q-2} \\
                \vdots   & \vdots   & \vdots   & \ddots & \vdots \\
                \bm{x}_2 & \bm{x}_3 & \bm{x}_4 & \cdots & \bm{x}_{w+1} \\
                \bm{x}_1 & \bm{x}_2 & \bm{x}_3 & \cdots & \bm{x}_{w} \\                       
            \end{array}
        \right ] , \nonumber \\
        % 
        \bm{\Upsilon} \phantom{'} = & \phantom{.} \left [
            \begin{array}{*{5}{@{}M{\mycolwd}@{}}}
                    \bm{u}_{q} & \bm{u}_{q+1} & \bm{u}_{q+2} & \cdots & \bm{u}_{w+q-1}
            \end{array}
        \right ] ,
    \end{align}
    where $w$ is the number of columns in the matrices, 
    $\bm{X^\prime}$ is the matrix $\bm{X}$ shifted forward by one time-step, 
    $\bm{X}_d$ is the matrix with delay states, 
    and $\bm{\Upsilon}$ is the matrix of inputs.
    Equation~\ref{eq:dmd_state_space} can be combined with the matrices in Equation~\ref{eq:dmd_matrices} to produce:
    \begin{equation}
        \bm{X^\prime} = \bm{A}_{dmd} \bm{X} + \bm{A}_d \bm{X}_d + \bm{B}_d \bm{\Upsilon} .
    \end{equation}
    Note that the primary objective of \gls{DMDc} is to determine the best fit model matrices, $\bm{A}_{dmd}$, $\bm{A}_d$ and $\bm{B}_d$, 
    given the data in $\bm{X^\prime}$, $\bm{X}$, $\bm{X}_d$, and $\bm{\Upsilon}$ \cite{Proctor2016c}.
    In order to group the unknowns into a single matrix, Equation~\ref{eq:dmd_state_space} is manipulated into the form,
    \begin{equation} \label{eq:G_Omega}
        \bm{X^\prime} =   
        \begin{bmatrix} 
            \bm{A}_{dmd} & \bm{A}_d & \bm{B}_d 
        \end{bmatrix}
        \begin{bmatrix} 
            \bm{X} \\ \bm{X}_d \\ \bm{\Upsilon} 
        \end{bmatrix} 
        = \bm{G \Omega} ,
    \end{equation} 
    where $\bm{\Omega}$ contains the state and control data, and $\bm{G}$ represents the system and input matrices.
    
    \paragraph
    A \gls{SVD} is performed on $\bm{\Omega}$ resulting in:
    \(
        \bm{\Omega} = \bm{U} \bm{\Sigma} \bm{V}^T
    \).
    Often, only the first $p$ columns of $\bm{U}$ and $\bm{V}$ are required for a good approximation of the dynamics \cite{Brunton2017a}.
    % Talk about Reduced Order Modelling??
    % \gls{POD} modes??
    In many cases, the truncated form results in better models than the exact form when noisy measurements are used.
    This is because the effect of measurement noise is mostly captured by the truncated columns of $\bm{U}$ and $\bm{V}$.
    By truncating these columns, the influence of noise in the regression problem is reduced. 
    % \murray{explain this better ??}
    Hence the \gls{SVD} is used in the truncated form: 
    \begin{equation} \label{eq:tilde_svd}
        \bm{\Omega} \approx \Tilde{\bm{U}} \Tilde{\bm{\Sigma}} \Tilde{\bm{V}}^T ,
    \end{equation}
    where $\phantom{.} \Tilde{ } \phantom{.}$ represents rank-$p$ truncation.
    For the $\bm{U}$ and $\bm{V}$ matrices, rank-$p$ truncation refers to keeping only the first $p$ number of columns and truncating the rest.
    Rank-$p$ truncation of the $\bm{S}$ matrix refers to keeping the first $p$ number of columns and rows and truncating the rest.
    % \murray{maybe insert colour pictures showing matrices ??}        

    \paragraph
    By combining Equation~\ref{eq:tilde_svd} with the over-constrained equality in Equation~\ref{eq:G_Omega}, 
    the least-squared solution, $\bm{G}$, can be found with:
    \begin{equation}
        \bm{G} \approx \bm{X^\prime} \Tilde{\bm{V}} \Tilde{\bm{\Sigma}}^{-1} \Tilde{\bm{U}} .
    \end{equation}
    By reversing Equation~\ref{eq:G_Omega}, $\bm{G}$ can now be decomposed into:
    \(
        \bm{G} = \begin{bmatrix} \bm{A}_{dmd} & \bm{A}_d & \bm{B}_d \end{bmatrix}
    \)
    according to the required dimensions of each matrix.
    Thereby, the state-space model approximated by \gls{DMDc} is complete.
    
