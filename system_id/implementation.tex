\section{Implementation and results}
%     \subsection{Methodology}
%         \paragraph{Simulation environment}

%         \paragraph{Method overview}
%         \murray{Maybe convert this to a flow diagram}

%         \begin{enumerate}
%             \item Takeoff and hover
%             \item Command a series of velocity step inputs with random step sizes and time intervals
%             \item Measure and save input and output data
%             \item Apply algorithm to data and generate model
%         \end{enumerate}

%         \paragraph{}

%         \paragraph{Steps and intervals}
%         For the training period, different velocity step inputs are commanded with varying time intervals between step commands.
%         A algorithm schedules these velocity step commands, by assigning random step values and time-intervals within a specified range.
%         The velocity range is determined in simulation by iteratively increasing the maximum velocity step 
%         to a safe value where the quadrotor and payload system remain in stable flight.
%         The maximum time-interval is set to a value that allows the payload swing to reach a steady-state condition.
%         This ensures that the identified model includes transient and steady-state dynamics.

%         \paragraph{Why velocity steps?}
%         Velocity step commands are used in the training period because this 
%         Frequency decomposition stimulates the system for a large range of frequencies.

%         \paragraph{Testing data}
%         Cross-validate

%         \paragraph{Error metric}
%         Each state error signal is scaled by the reciprocal of the maximum value of that state variable in the training data.
%         % This is to provide a better representative error when taking the mean of state variable errors.
%         This is to ensure that a scale difference in the variable types create a bias in the error metric.
%         For example, the quadrotor velocity reaches values of \SI{3}{\metre/\second} but the payload swing angle has a maximum of only \SI[]{0.526}{\radian}.
%         The velocity prediction error is therefore inherently larger than the payload angle prediction error
%         and will bias the error metric towards favouring models with good velocity predictions.
%         The proposed scaled error metric ensures that the MAE of each state variable can be compared to each other.
%         It also provides an error metric that is better and unbiased representative of the model prediction performance across all state variables. 

%         Add Information Criteria ??
%         AIC Brunton Kutz??

%     \subsection{Hyperparameters}
%         As discussed in Section~\ref{sec:dmdc} and \ref{sec:havok} 
%         DMDc and HAVOK are dependent on two hyperparameters: the number of delay-coordinates, $q$, and the SVD truncation rank, $p$.

%         Parsimony
%         Pareto front
%         cite Data-Driven book

%         The more terms better chance to overfit, lower generalisation
%         \begin{figure}[htb]
    \centering
    \begin{tikzpicture}
        \begin{axis}[            
            xlabel = {Number of delay-coordinates, $q$},
            ylabel = $\overline{NMAE}$ \phantom{~},
            % x unit = \si{\second},
            y unit = \%,
            xmin = 2,     xmax = 40,
            ymin = 3.2,  ymax = 6.5,
            grid = major,
            legend cell align = left,
            legend pos = north east,
            grid style = dashed,
            legend style = {font = \scriptsize},
            label style = {font = \scriptsize},
            tick label style = {font = \scriptsize},
            width = 0.95\columnwidth,
            height = 0.5\columnwidth,
            % initialize Dark2
            cycle list/Dark2,
            % combine it with 'mark list*':
            cycle multiindex* list = {
                Dark2\nextlist
            }
        ]

        \addplot+[mark = none, style = solid, ultra thick] 
        table[x = q, y expr = {\thisrow{NMAE_mean}*100}, col sep = comma] 
        {system_id/csv/NMAE_vs_q_SITL_x_vel_noise_longer_times_q.csv_dmd_angle_Ttrain_60.csv};
        \addlegendentry{DMD}
        
        \addplot+[mark = none, style = solid, ultra thick] 
        table[x = q, y expr = {\thisrow{NMAE_mean}*100}, col sep = comma] 
        {system_id/csv/NMAE_vs_q_SITL_x_vel_noise_longer_times_q.csv_havok_angle_Ttrain_60.csv};
        \addlegendentry{HAVOK}

        \end{axis}
    \end{tikzpicture} 
    
    \caption{\gls{DMD} and \gls{HAVOK} predictions error for different lengths of noisy training data
    ($m_p =$~\SI{0.2}{\kilo\gram}, $l =$~\SI{0.5}{\meter}, $T_s =$~\SI{0.03}{\second}, $T_{train} =$~\SI{60}{\second}.)}
    \label{fig:NMAE_vs_q}
\end{figure}


%         % \begin{figure}[htb]
    \centering
    \begin{tikzpicture}
        \begin{semilogyaxis}[            
            xlabel = Index of mode,
            ylabel = Singular value,
            % x unit = \si{\second},
            % y unit = \si{\second},
            xmin = 0,     xmax = 60,
            ymin = 1e-2,  ymax = 1e3,
            grid = major,
            legend cell align = left,
            legend pos = north east,
            grid style = dashed,
            legend style = {font = \scriptsize},
            label style = {font = \scriptsize},
            tick label style = {font = \scriptsize},
            width = 0.95\columnwidth,
            height = 0.5\columnwidth,
            % initialize Dark2
            cycle list/Dark2,
            % combine it with 'mark list*':
            cycle multiindex* list = {
                Dark2\nextlist
            }
        ]

        \addplot+[only marks, mark = square, ultra thick] 
        table[x = index, y = S, col sep = comma] 
        {system_id/csv/Singular_values_SITL_x_vel_noise_longer_times_q.csv_havok_angle_Ttrain_60_q29_p13.csv};
        \addlegendentry{Significant modes}

        \addplot+[only marks, mark = square, ultra thick] 
        table[x = index, y = S, col sep = comma] 
        {system_id/csv/Singular_values_SITL_x_vel_noise_longer_times_q.csv_havok_angle_Ttrain_60_q29_p13_trunc.csv};
        \addlegendentry{Truncated modes}

        \end{semilogyaxis}
    \end{tikzpicture} 
    
    \caption{Significant and truncated singular values of a \gls{HAVOKc} model produced from noisy data
    ($m_p =$~\SI{0.2}{\kilo\gram}, $l =$~\SI{0.5}{\meter}, $T_s =$~\SI{0.03}{\second}, $T_{train} =$~\SI{60}{\second}.).}
    \label{fig:singular_values}
\end{figure}


%         Fixed size of data
%         Fixed sample time
%         Fixed pendulum params
%         Talk about the "front"
%         Also about singular values
%         For each of the experiments shown in this chapter, a hyperparameters selected tuned to produced

%     \subsection{Sample time}
%         best hyperparameters.
%         best N_train.
%         \begin{figure}[htb]
    \centering
    \begin{tikzpicture}
        \begin{axis}[            
            xlabel = Ts,
            ylabel = $\overline{NMAE}$ \phantom{~},
            % x unit = \si{\second},
            y unit = \%,
            xmin = 0,     xmax = 0.07,
            ymin = 8.5,  ymax = 12.5,
            grid = major,
            legend cell align = left,
            legend pos = north east,
            grid style = dashed,
            legend style = {font = \scriptsize},
            label style = {font = \scriptsize},
            tick label style = {font = \scriptsize},
            width = 0.95\columnwidth,
            height = 0.5\columnwidth,
            % initialize Dark2
            cycle list/Dark2,
            % combine it with 'mark list*':
            cycle multiindex* list = {
                Dark2\nextlist
            }
        ]

        \addplot+[mark = none, style = solid, ultra thick] 
        table[x = Ts, y expr = {\thisrow{NMAE_mean}*100}, col sep = comma] 
        {system_id/csv/NMAE_vs_Ts_SITL_x_vel_noise_longer_times_Ts.csv_dmd_angle.csv};
        \addlegendentry{DMD}
        
        \addplot+[mark = none, style = solid, ultra thick] 
        table[x = Ts, y expr = {\thisrow{NMAE_mean}*100}, col sep = comma] 
        {system_id/csv/NMAE_vs_Ts_SITL_x_vel_noise_longer_times_Ts.csv_havok_angle.csv};
        \addlegendentry{HAVOK}

        \end{axis}
    \end{tikzpicture} 
    
    \caption{DMD and HAVOK predictions error for different sample times of noisy training data
    ($m =$~\SI{0.2}{\kilo\gram}, $l =$~\SI{0.5}{\meter}, $T_{train} =$~\SI{60}{\second}.)}
    % \label{fig:havok_vs_dmd_noise}
\end{figure}

%         Fixed size of data.
%         Fixed pendulum params.

%     \subsection{Choice of payload variable in the state vector}
%         As discussed in Section~\ref{sec:plant_considered}, 
%         the equations of motion of a floating pendulum in continuous-time are dependent on $\dot{\theta}$ and $V_N$, 
%         but are not dependent on $\theta$.
%         Therefore it is expected that 
%         $
%         \bm{x} = \begin{bmatrix}
%             V_N & \dot{\theta}
%         \end{bmatrix}^T
%         $
%         is used as the state vector for system identification.
%         However, if $\dot{\theta}$ is not included in the state vector of a discrete model, 
%         it can still be represented with numerical differentiation like the backward Euler form,
%         \begin{equation}
%             \dot{\theta}_k = (\frac{1}{T_s}) \cdot \theta_k - (\frac{1}{T_s}) \cdot \theta_{k-1} .
%         \end{equation}
%         Therefore the original state vector can also be replaced by,
%         $
%             \bm{x} = \begin{bmatrix}
%                 V_N & \theta
%             \end{bmatrix}^T
%         $
%         in system identification.

%         \paragraph{}
%         Based on the floating pendulum equations, it is expected that a model derived with $\dot{\theta}$ data 
%         will better approximate the actual dynamics than one using $\theta$.
%         This is because $\dot{\theta}$ contains more direct information about the dynamics compared to $\theta$.
%         A model using $\theta$ needs to "learn" numerical differentiation and the effect of $\dot{\theta}$ on the other variables.
%         A model using $\dot{\theta}$ only needs to consider its relationship with other variables.

%         \begin{figure}[h]
    \centering
    \begin{tikzpicture}
        \begin{axis}[            
            xlabel = Length of training data,
            ylabel = MAE,
            x unit = \si{\second},
            % y unit = \si{\second},
            xmin = 0,     xmax = 120,
            ymin = 0.0005, ymax = 0.08,
            grid = major,
            legend cell align = left,
            legend pos = north east,
            grid style = dashed,
            legend style = {font = \scriptsize},
            label style = {font = \scriptsize},
            tick label style = {font = \scriptsize},
            width = 0.95\columnwidth,
            height = 0.5\columnwidth,
            % initialize Dark2
            cycle list/Dark2,
            % combine it with 'mark list*':
            cycle multiindex* list = {
                Dark2\nextlist
            }
        ]
         
        \addplot+[mark = none, style = solid, ultra thick] 
        table[x = T_train, y = MAE_mean, col sep = comma] 
        {system_id/csv/MAE_vs_Ntrain_Simulink_single_pend_mp0.2_l0.5_PID_vel_steps_tune_scale_0.7longer_times.mat_dmd_angle.csv};
        \addlegendentry{DMD with $\theta$}

        \addplot+[mark = none, style = solid, ultra thick] 
        table[x = T_train, y = MAE_mean, col sep = comma] 
        {system_id/csv/MAE_vs_Ntrain_Simulink_single_pend_mp0.2_l0.5_PID_vel_steps_tune_scale_0.7longer_times.mat_havok_angle.csv};
        \addlegendentry{HAVOK with $\theta$}
        
        \addplot+[mark = none, style = dashed, ultra thick] 
        table[x = T_train, y = MAE_mean, col sep = comma] 
        {system_id/csv/MAE_vs_Ntrain_Simulink_single_pend_mp0.2_l0.5_PID_vel_steps_tune_scale_0.7longer_times.mat_dmd_angular_rate.csv};
        \addlegendentry{DMD with $\dot{\theta}$}

        \addplot+[mark = none, style = dashed, ultra thick] 
        table[x = T_train, y = MAE_mean, col sep = comma] 
        {system_id/csv/MAE_vs_Ntrain_Simulink_single_pend_mp0.2_l0.5_PID_vel_steps_tune_scale_0.7longer_times.mat_havok_angular_rate.csv};
        \addlegendentry{HAVOK with $\dot{\theta}$}

        \end{axis}
    \end{tikzpicture} 
    
    \caption{Prediction MAE for models using angle or angular rate measurements 
    ($m =$~\SI{0.2}{\kilo\gram}, $l =$~\SI{0.5}{\meter}, $T_s =$~\SI{0.03}{\second}).}
    \label{fig:SITL_MAE_vs_train_angular_rate}
\end{figure}


%         \paragraph{}
%         % Figure~\ref{} shows the prediction error of techniques using $\dot{\theta}$ or $\theta$ for different amounts of training data.
%         For each length of training data, the hyperparameter combination producing the lowest prediction error was determined and used.
%         From this plot it is clear that models with $\theta$ produce more accurate predictions than those with $\dot{\theta}$.

%     \subsection{Noise}
%         \paragraph{}
%         Measurement noise is \murray{Find reference for measurement noise definition}
%         This is bad for system identification because the output signals no longer represent the actual process
%         hides the actual dynamics of the system under  
%         The IMU, barometer, magnetometer and GPS sensors on the practical quadrotor are used for state estimation 
%         and all experience measurement noise.
%         The EKF performs sensor fusion and smooths out most of the measurement noise to provide a state estimate that is less noisy than raw sensor values.
        
%         \paragraph{}
%         The potentiometer and ADC which measure the payload angle on the quadrotor alos has quite a lot of measurement noise.
%         However, this signal is not smoothed by an onboard EKF.
%         Figure~\ref{fig:payload_noise} shows the noisy payload angle measurement for a practical pendulum test while the quadrotor is held stationary.
%         For models using $\theta$ in the state vector instead of $\dot{\theta}$, 
%         this noisy signal can be smoothed with \murray{matlab smoother}.
%         Figure~\ref{fig:payload_noise_smoothed} compares the noisy payload angle measurement to the smoothed signal and actual payload angle for a simulated flight.
%         The is applied as band-limited white-noise and the noise power was iteratively adjusted to match that of the practical payload measurements.
        
%         \paragraph{}
%         However, since there is no direct measurement of $\dot{\theta}$, 
%         numerical differentiation is performed on the noisy $\theta$ measurement to estimate $\dot{\theta}$. 
%         This amplifies the noise and results in inaccurate $\dot{\theta}$ signal.
%         Total variation differentiation is implemented to estimate $\dot{\theta}$ from the noisy measurements more accurately. \cite{}
%         Figure~\ref{fig:payload_noise_diff} shows
        
%         % \input{system_id/plots/payload_noise_diff.tex} // With TVDiff

%         Noise also affects model prediction accuracy and the length of training data required for adequate predictions. 
        
%         \begin{figure}[htb]
    \centering
    \begin{tikzpicture}
        \begin{axis}[            
            xlabel = Length of training data,
            ylabel = $\overline{NMAE}$ \phantom{~},
            x unit = \si{\second},
            y unit = \%,
            xmin = 5,     xmax = 120,
            ymin = 3.2,  ymax = 5.7,
            grid = major,
            legend cell align = left,
            legend pos = north east,
            grid style = dashed,
            legend style = {font = \scriptsize},
            label style = {font = \scriptsize},
            tick label style = {font = \scriptsize},
            width = 0.95\columnwidth,
            height = 0.5\columnwidth,
            % initialize Dark2
            cycle list/Dark2,
            % combine it with 'mark list*':
            cycle multiindex* list = {
                Dark2\nextlist
            }
        ]

        \addplot+[mark = none, style = solid, ultra thick] 
        table[x = T_train, y expr = {\thisrow{NMAE_mean}*100}, col sep = comma] 
        {system_id/csv/NMAE_vs_Ntrain_SITL_x_vel_no_noise_longer_times.csv_havok_angle.csv};
        \addlegendentry{Without noise}

        \addplot+[mark = none, style = solid, ultra thick] 
        table[x = T_train, y expr = {\thisrow{NMAE_mean}*100}, col sep = comma] 
        {system_id/csv/NMAE_vs_Ntrain_SITL_x_vel_noise_longer_times.csv_havok_angle.csv};
        \addlegendentry{With noise}

        \end{axis}
    \end{tikzpicture} 
    
    \caption{\gls{HAVOK} prediction errors for different lengths of training data with and without noise 
    ($m_p =$~\SI{0.2}{\kilo\gram}, $l =$~\SI{0.5}{\meter}, $T_s =$~\SI{0.03}{\second}).}
    \label{fig:noise_vs_no_noise}
\end{figure}


%         \begin{figure}[htb]
    \centering
    \begin{tikzpicture}
        \begin{axis}[            
            xlabel = Length of training data,
            ylabel = $\overline{NMAE}$ \phantom{~},
            x unit = \si{\second},
            y unit = \%,
            xmin = 5,     xmax = 120,
            ymin = 3.2,  ymax = 5.7,
            grid = major,
            legend cell align = left,
            legend pos = north east,
            grid style = dashed,
            legend style = {font = \scriptsize},
            label style = {font = \scriptsize},
            tick label style = {font = \scriptsize},
            width = 0.95\columnwidth,
            height = 0.5\columnwidth,
            % initialize Dark2
            cycle list/Dark2,
            % combine it with 'mark list*':
            cycle multiindex* list = {
                Dark2\nextlist
            }
        ]

        \addplot+[mark = none, style = solid, ultra thick] 
        table[x = T_train, y expr = {\thisrow{NMAE_mean}*100}, col sep = comma] 
        {system_id/csv/NMAE_vs_Ntrain_SITL_x_vel_noise_longer_times.csv_dmd_angle.csv};
        \addlegendentry{DMD}

        \addplot+[mark = none, style = solid, ultra thick] 
        table[x = T_train, y expr = {\thisrow{NMAE_mean}*100}, col sep = comma] 
        {system_id/csv/NMAE_vs_Ntrain_SITL_x_vel_noise_longer_times.csv_havok_angle.csv};
        \addlegendentry{HAVOK}

        \end{axis}
    \end{tikzpicture} 
    
    \caption{\gls{DMD} and \gls{HAVOK} prediction errors for different lengths of noisy training data
    ($m_p =$~\SI{0.2}{\kilo\gram}, $l =$~\SI{0.5}{\meter}, $T_s =$~\SI{0.03}{\second}).}
    \label{fig:havok_vs_dmd_noise}
\end{figure}


%         HAVOK performs better than DMD.
%         This slight difference in prediciton performance has a negligible effect on control.
        

%         Input data needs to be adjusted.

%         % \input{plot of SITL acc_sp}
    

%     \subsection{Size of training data}
%         The length of training data used for system identification affects the quality of the model produced.
%         In Figure~\ref{fig:SITL_MAE_vs_train_angular_rate} it is clear that prediction error decreases as the amount of training data increases.
%         As more training data is used in the regression problem, 
%         the determined model better approximates the actual dynamics because a large range of the dynamics is "seen" by the algorithm.

        
%         % \input{low data prediction}
%         \paragraph{}
%         Models produced from data lengths as short as \SI{5}{\second} predict the movement of state variables surprisingly well.
%         % Figure~\ref{} shows prediction.
%         Note how the general shape of the prediction represents the training data, 
%         even though it contains a lot more high frequency oscillations.

%         \begin{figure}[htb]
    \centering
    \begin{tikzpicture}
        \begin{axis}[            
            xlabel = Length of training data,
            ylabel = NMAE of time derivative of predictions,
            x unit = \si{\second},
            % y unit = \si{\second},
            xmin = 0,     xmax = 120,
            ymin = 0.004,  ymax = 0.016,
            grid = major,
            legend cell align = left,
            legend pos = north east,
            grid style = dashed,
            legend style = {font = \scriptsize},
            label style = {font = \scriptsize},
            tick label style = {font = \scriptsize},
            width = 0.95\columnwidth,
            height = 0.5\columnwidth,
            % initialize Dark2
            cycle list/Dark2,
            % combine it with 'mark list*':
            cycle multiindex* list = {
                Dark2\nextlist
            }
        ]

        \addplot+[mark = none, style = solid, ultra thick] 
        table[x = T_train, y expr = {\thisrow{NMAE_mean}*100}, col sep = comma] 
        {system_id/csv/NMAE_vs_Ntrain_SITL_x_vel_noise_longer_times_MAEdiff.csv_dmd_angle.csv};
        \addlegendentry{DMD}
        
        \addplot+[mark = none, style = solid, ultra thick] 
        table[x = T_train, y expr = {\thisrow{NMAE_mean}*100}, col sep = comma] 
        {system_id/csv/NMAE_vs_Ntrain_SITL_x_vel_noise_longer_times_MAEdiff.csv_havok_angle.csv};
        \addlegendentry{HAVOK}

        \end{axis}
    \end{tikzpicture} 
    
    \caption{DMD and HAVOK error of time derivative of predictions for different lengths of noisy training data
    ($m =$~\SI{0.2}{\kilo\gram}, $l =$~\SI{0.5}{\meter}, $T_s =$~\SI{0.03}{\second}).}
    % \label{fig:havok_vs_dmd_noise}
\end{figure}


%         \paragraph{}
%         % Figure~\ref{} shows the MAE of prediction state derivative.
%         Define MAE diff with equation ??.
%         % From Figure~\ref{} it appears that at least \SI{}{} training data is required to produce models that represent the dynamics.

%         \paragraph{}
%         The models produced from HAVOK appear to produce slightly better prediction errors, however this small difference has a negligible effect on control performance.

%         \paragraph{}
%         % In Figure~\ref{fig:MAE_vs_train} it can be seen that after approximately \SI{??}{\second} 
%         the prediction error does not significantly improve with more training data.
%         It practice less training data is desirable because less flight time will be wasted on training a model before the quadrotor can fly with a updated controller.
%         Less training data also corresponds to lower memory usage on quadrotor hardware.
%         Such a slight improvement in prediction error also has a negligible effect on control performance and is therefore not worth the increased data requirement.
%         % Therefore, only \SI{??}{\second} of flight data will be used to train system identification models. 


%     \subsection{System parameters}
%         Best hyperparameters.
%         Fixed size of data.
%         Fixed sample time.

%     \subsection{Dynamic payload}
%         Data-driven vs Parameter estimation

%         plot hyperparameterss MAE. Not how much more delays are required

%     \subsection{Practical flight data}
%     \subsection{HIL}
%         \paragraph{Companion computer}
%         \paragraph{Software}
%         \paragraph{CPU}
%         \paragraph{Memory}

