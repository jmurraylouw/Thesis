\section{Implementation and results}
    \subsection{Methodology}
        \paragraph{Simulation environment}

        \paragraph{Method overview}
        \murray{Maybe convert this to a flow diagram}

        \begin{enumerate}
            \item Takeoff and hover
            \item Command a series of velocity step inputs with random step sizes and time intervals
            \item Measure and save input and output data
            \item Apply algorithm to data and generate model
        \end{enumerate}

        \paragraph{}

        \paragraph{Steps and intervals}
        For the training period, different velocity step inputs are commanded with varying time intervals between step commands.
        A algorithm schedules these velocity step commands, by assigning random step values and time-intervals within a specified range.
        The velocity range is determined in simulation by iteratively increasing the maximum velocity step 
        to a safe value where the quadrotor and payload system remain in stable flight.
        The maximum time-interval is set to a value that allows the payload swing to reach a steady-state condition.
        This ensures that the identified model includes transient and steady-state dynamics.

        \paragraph{Why velocity steps?}
        Velocity step commands are used in the training period because this 
        Frequency decomposition stimulates the system for a large range of frequencies.

        \paragraph{Testing data}
        Cross-validate

        \paragraph{Error metric}
        Each state error signal is scaled by the reciprocal of the maximum value of that state variable in the training data.
        % This is to provide a better representative error when taking the mean of state variable errors.
        This is to ensure that a scale difference in the variable types create a bias in the error metric.
        For example, the quadrotor velocity reaches values of \SI{3}{\metre/\second} but the payload swing angle has a maximum of only \SI[]{0.526}{\radian}.
        The velocity prediction error is therefore inherently larger than the payload angle prediction error
        and will bias the error metric towards favouring models with good velocity predictions.
        The proposed scaled error metric ensures that the MAE of each state variable can be compared to each other.
        It also provides an error metric that is better and unbiased representative of the model prediction performance across all state variables. 

        Add Information Criteria ??
        AIC Brunton Kutz??

    \subsection{Hyperparameters}
        As discussed in Section~\ref{sec:dmdc} and \ref{sec:havok} 
        DMDc and HAVOK are dependent on two hyperparameters: the number of delay-coordinates, $q$, and the SVD truncation rank, $p$.

        Parsimony
        Pareto front
        cite Data-Driven book

        The more terms better chance to overfit, lower generalisation
        \begin{figure}[htb]
    \centering
    \begin{tikzpicture}
        \begin{axis}[            
            xlabel = {Number of delay-coordinates, $q$},
            ylabel = $\overline{NMAE}$ \phantom{~},
            % x unit = \si{\second},
            y unit = \%,
            xmin = 2,     xmax = 40,
            ymin = 3.2,  ymax = 6.5,
            grid = major,
            legend cell align = left,
            legend pos = north east,
            grid style = dashed,
            legend style = {font = \scriptsize},
            label style = {font = \scriptsize},
            tick label style = {font = \scriptsize},
            width = 0.95\columnwidth,
            height = 0.5\columnwidth,
            % initialize Dark2
            cycle list/Dark2,
            % combine it with 'mark list*':
            cycle multiindex* list = {
                Dark2\nextlist
            }
        ]

        \addplot+[mark = none, style = solid, ultra thick] 
        table[x = q, y expr = {\thisrow{NMAE_mean}*100}, col sep = comma] 
        {system_id/csv/NMAE_vs_q_SITL_x_vel_noise_longer_times_q.csv_dmd_angle_Ttrain_60.csv};
        \addlegendentry{DMD}
        
        \addplot+[mark = none, style = solid, ultra thick] 
        table[x = q, y expr = {\thisrow{NMAE_mean}*100}, col sep = comma] 
        {system_id/csv/NMAE_vs_q_SITL_x_vel_noise_longer_times_q.csv_havok_angle_Ttrain_60.csv};
        \addlegendentry{HAVOK}

        \end{axis}
    \end{tikzpicture} 
    
    \caption{\gls{DMD} and \gls{HAVOK} predictions error for different lengths of noisy training data
    ($m_p =$~\SI{0.2}{\kilo\gram}, $l =$~\SI{0.5}{\meter}, $T_s =$~\SI{0.03}{\second}, $T_{train} =$~\SI{60}{\second}.)}
    \label{fig:NMAE_vs_q}
\end{figure}


        % \begin{figure}[htb]
    \centering
    \begin{tikzpicture}
        \begin{semilogyaxis}[            
            xlabel = Index of mode,
            ylabel = Singular value,
            % x unit = \si{\second},
            % y unit = \si{\second},
            xmin = 0,     xmax = 60,
            ymin = 1e-2,  ymax = 1e3,
            grid = major,
            legend cell align = left,
            legend pos = north east,
            grid style = dashed,
            legend style = {font = \scriptsize},
            label style = {font = \scriptsize},
            tick label style = {font = \scriptsize},
            width = 0.95\columnwidth,
            height = 0.5\columnwidth,
            % initialize Dark2
            cycle list/Dark2,
            % combine it with 'mark list*':
            cycle multiindex* list = {
                Dark2\nextlist
            }
        ]

        \addplot+[only marks, mark = square, ultra thick] 
        table[x = index, y = S, col sep = comma] 
        {system_id/csv/Singular_values_SITL_x_vel_noise_longer_times_q.csv_havok_angle_Ttrain_60_q29_p13.csv};
        \addlegendentry{Significant modes}

        \addplot+[only marks, mark = square, ultra thick] 
        table[x = index, y = S, col sep = comma] 
        {system_id/csv/Singular_values_SITL_x_vel_noise_longer_times_q.csv_havok_angle_Ttrain_60_q29_p13_trunc.csv};
        \addlegendentry{Truncated modes}

        \end{semilogyaxis}
    \end{tikzpicture} 
    
    \caption{Significant and truncated singular values of a \gls{HAVOKc} model produced from noisy data
    ($m_p =$~\SI{0.2}{\kilo\gram}, $l =$~\SI{0.5}{\meter}, $T_s =$~\SI{0.03}{\second}, $T_{train} =$~\SI{60}{\second}.).}
    \label{fig:singular_values}
\end{figure}


        Fixed size of data
        Fixed sample time
        Fixed pendulum params
        Talk about the "front"
        Also about singular values
        For each of the experiments shown in this chapter, a hyperparameters selected tuned to produced

    \subsection{Sample time}
        The sample time, $T_s$, used for system identification sets the sample time of the discrete model, 
        which determines the sample time of the MPC.
        Resampling strategies can enable the MPC to run at a different frequency to the discrete model but this adds unnecessary complexity to the control architecture.
        
        \paragraph{}
        The MPC acts in the velocity loop and commands an acceleration setpoint.
        The default PID velocity controller runs at \SI{50}{\hertz} which corresponds to $T_s =~$\SI{0.02}{\second}.
        Due to the computational complexity of an MPC, the optimiser will struggle to run at \SI{50}{\hertz} on a quadrotor companion computer.
        \begin{figure}[htb]
    \centering
    \begin{tikzpicture}
        \begin{axis}[            
            xlabel = Ts,
            ylabel = NMAE\textsubscript{mm},
            % x unit = \si{\second},
            % y unit = \si{\second},
            xmin = 0.02,  xmax = 0.05,
            ymin = 0.03,  ymax = 0.055,
            grid = major,
            legend cell align = left,
            legend pos = north east,
            grid style = dashed,
            legend style = {font = \scriptsize},
            label style = {font = \scriptsize},
            tick label style = {font = \scriptsize},
            width = 0.45\columnwidth,
            height = 0.5\columnwidth,
            % initialize Dark2
            cycle list/Dark2,
            % combine it with 'mark list*':
            cycle multiindex* list = {
                Dark2\nextlist
            }
        ]

        % \addplot+[mark = none, style = solid, ultra thick] 
        % table[x = Ts, y = NMAE_mean, col sep = comma] 
        % {system_id/csv/NMAE_vs_Ts_SITL_x_vel_noise_l0-25_m0-2.csv_dmd_angle.csv};
        % \addlegendentry{$l =$~\SI{0.25}{\metre}}
        
        \addplot+[mark = none, style = solid, ultra thick] 
        table[x = Ts, y = NMAE_mean, col sep = comma] 
        {system_id/csv/NMAE_vs_Ts_SITL_x_vel_noise_l0-5_m0-2.csv_dmd_angle.csv};
        \addlegendentry{$l =$~\SI{0.5}{\metre}}
        
        \addplot+[mark = none, style = solid, ultra thick] 
        table[x = Ts, y = NMAE_mean, col sep = comma] 
        {system_id/csv/NMAE_vs_Ts_SITL_x_vel_noise_l1_m0-2.csv_dmd_angle.csv};
        \addlegendentry{$l =$~\SI{1}{\metre}}
        
        \addplot+[mark = none, style = solid, ultra thick] 
        table[x = Ts, y = NMAE_mean, col sep = comma] 
        {system_id/csv/NMAE_vs_Ts_SITL_x_vel_noise_l2_m0-2.csv_dmd_angle.csv};
        \addlegendentry{$l =$~\SI{2}{\metre}}
        
        \end{axis}
    \end{tikzpicture} 
    
    \caption{DMD prediction error using different cable lengths with a range of different sample times of noisy training data
    ($m =$~\SI{0.2}{\kilo\gram})}
    \label{fig:MAE_vs_Ts_vs_L}
\end{figure}

        This is because the  
        % \begin{figure}[htb]
    \centering
    \begin{tikzpicture}
        \begin{axis}[            
            xlabel = Ts,
            ylabel = $\overline{NMAE}$ \phantom{~},
            % x unit = \si{\second},
            y unit = \%,
            xmin = 0,     xmax = 0.07,
            ymin = 8.5,  ymax = 12.5,
            grid = major,
            legend cell align = left,
            legend pos = north east,
            grid style = dashed,
            legend style = {font = \scriptsize},
            label style = {font = \scriptsize},
            tick label style = {font = \scriptsize},
            width = 0.95\columnwidth,
            height = 0.5\columnwidth,
            % initialize Dark2
            cycle list/Dark2,
            % combine it with 'mark list*':
            cycle multiindex* list = {
                Dark2\nextlist
            }
        ]

        \addplot+[mark = none, style = solid, ultra thick] 
        table[x = Ts, y expr = {\thisrow{NMAE_mean}*100}, col sep = comma] 
        {system_id/csv/NMAE_vs_Ts_SITL_x_vel_noise_longer_times_Ts.csv_dmd_angle.csv};
        \addlegendentry{DMD}
        
        \addplot+[mark = none, style = solid, ultra thick] 
        table[x = Ts, y expr = {\thisrow{NMAE_mean}*100}, col sep = comma] 
        {system_id/csv/NMAE_vs_Ts_SITL_x_vel_noise_longer_times_Ts.csv_havok_angle.csv};
        \addlegendentry{HAVOK}

        \end{axis}
    \end{tikzpicture} 
    
    \caption{DMD and HAVOK predictions error for different sample times of noisy training data
    ($m =$~\SI{0.2}{\kilo\gram}, $l =$~\SI{0.5}{\meter}, $T_{train} =$~\SI{60}{\second}.)}
    % \label{fig:havok_vs_dmd_noise}
\end{figure}


    \subsection{Choice of payload variable in the state vector}
        As discussed in Section~\ref{sec:plant_considered}, 
        the equations of motion of a floating pendulum in continuous-time are dependent on $\dot{\theta}$ and $V_N$, 
        but are not dependent on $\theta$.
        Therefore it is expected that 
        $
        \bm{x} = \begin{bmatrix}
            V_N & \dot{\theta}
        \end{bmatrix}^T
        $
        is used as the state vector for system identification.
        However, if $\dot{\theta}$ is not included in the state vector of a discrete model, 
        it can still be represented with numerical differentiation like the backward Euler form,
        \begin{equation}
            \dot{\theta}_k = (\frac{1}{T_s}) \cdot \theta_k - (\frac{1}{T_s}) \cdot \theta_{k-1} .
        \end{equation}
        Therefore the original state vector can also be replaced by,
        $
            \bm{x} = \begin{bmatrix}
                V_N & \theta
            \end{bmatrix}^T
        $
        in system identification.

        \paragraph{}
        Based on the floating pendulum equations, it is expected that a model derived with $\dot{\theta}$ data 
        will better approximate the actual dynamics than one using $\theta$.
        This is because $\dot{\theta}$ contains more direct information about the dynamics compared to $\theta$.
        A model using $\theta$ needs to "learn" numerical differentiation and the effect of $\dot{\theta}$ on the other variables.
        A model using $\dot{\theta}$ only needs to consider its relationship with other variables.

        \begin{figure}[h]
    \centering
    \begin{tikzpicture}
        \begin{axis}[            
            xlabel = Length of training data,
            ylabel = MAE,
            x unit = \si{\second},
            % y unit = \si{\second},
            xmin = 0,     xmax = 120,
            ymin = 0.0005, ymax = 0.08,
            grid = major,
            legend cell align = left,
            legend pos = north east,
            grid style = dashed,
            legend style = {font = \scriptsize},
            label style = {font = \scriptsize},
            tick label style = {font = \scriptsize},
            width = 0.95\columnwidth,
            height = 0.5\columnwidth,
            % initialize Dark2
            cycle list/Dark2,
            % combine it with 'mark list*':
            cycle multiindex* list = {
                Dark2\nextlist
            }
        ]
         
        \addplot+[mark = none, style = solid, ultra thick] 
        table[x = T_train, y = MAE_mean, col sep = comma] 
        {system_id/csv/MAE_vs_Ntrain_Simulink_single_pend_mp0.2_l0.5_PID_vel_steps_tune_scale_0.7longer_times.mat_dmd_angle.csv};
        \addlegendentry{DMD with $\theta$}

        \addplot+[mark = none, style = solid, ultra thick] 
        table[x = T_train, y = MAE_mean, col sep = comma] 
        {system_id/csv/MAE_vs_Ntrain_Simulink_single_pend_mp0.2_l0.5_PID_vel_steps_tune_scale_0.7longer_times.mat_havok_angle.csv};
        \addlegendentry{HAVOK with $\theta$}
        
        \addplot+[mark = none, style = dashed, ultra thick] 
        table[x = T_train, y = MAE_mean, col sep = comma] 
        {system_id/csv/MAE_vs_Ntrain_Simulink_single_pend_mp0.2_l0.5_PID_vel_steps_tune_scale_0.7longer_times.mat_dmd_angular_rate.csv};
        \addlegendentry{DMD with $\dot{\theta}$}

        \addplot+[mark = none, style = dashed, ultra thick] 
        table[x = T_train, y = MAE_mean, col sep = comma] 
        {system_id/csv/MAE_vs_Ntrain_Simulink_single_pend_mp0.2_l0.5_PID_vel_steps_tune_scale_0.7longer_times.mat_havok_angular_rate.csv};
        \addlegendentry{HAVOK with $\dot{\theta}$}

        \end{axis}
    \end{tikzpicture} 
    
    \caption{Prediction MAE for models using angle or angular rate measurements 
    ($m =$~\SI{0.2}{\kilo\gram}, $l =$~\SI{0.5}{\meter}, $T_s =$~\SI{0.03}{\second}).}
    \label{fig:SITL_MAE_vs_train_angular_rate}
\end{figure}

            \murray{Add plot of using thetaand dtheta??}
            \murray{Redo this plot with SITL noise}
        \paragraph{}
        % Figure~\ref{} shows the prediction error of techniques using $\dot{\theta}$ or $\theta$ for different amounts of training data.
        For each length of training data, the hyperparameter combination producing the lowest prediction error was determined and used.
        From this plot it is clear that models with $\theta$ produce more accurate predictions than those with $\dot{\theta}$.

    \subsection{Noise}
        \paragraph{}
        Measurement noise is \murray{Find reference for measurement noise definition}
        This is bad for system identification because the output signals no longer represent the actual process
        hides the actual dynamics of the system under  
        The IMU, barometer, magnetometer and GPS sensors on the practical quadrotor are used for state estimation 
        and all experience measurement noise.
        The EKF performs sensor fusion and smooths out most of the measurement noise to provide a state estimate that is less noisy than raw sensor values.
        
        \paragraph{}
        The potentiometer and ADC which measure the payload angle on the quadrotor alos has quite a lot of measurement noise.
        However, this signal is not smoothed by an onboard EKF.
        Figure~\ref{fig:payload_noise} shows the noisy payload angle measurement for a practical pendulum test while the quadrotor is held stationary.
        For models using $\theta$ in the state vector instead of $\dot{\theta}$, 
        this noisy signal can be smoothed with \murray{matlab smoother}.
        Figure~\ref{fig:payload_noise_smoothed} compares the noisy payload angle measurement to the smoothed signal and actual payload angle for a simulated flight.
        The is applied as band-limited white-noise and the noise power was iteratively adjusted to match that of the practical payload measurements.
        
        \paragraph{}
        However, since there is no direct measurement of $\dot{\theta}$, 
        numerical differentiation is performed on the noisy $\theta$ measurement to estimate $\dot{\theta}$. 
        This amplifies the noise and results in inaccurate $\dot{\theta}$ signal.
        Total variation differentiation is implemented to estimate $\dot{\theta}$ from the noisy measurements more accurately. \cite{}
        Figure~\ref{fig:payload_noise_diff} shows
        
        % \input{system_id/plots/payload_noise_diff.tex} // With TVDiff

        Noise also affects model prediction accuracy and the length of training data required for adequate predictions. 
        
        \begin{figure}[htb]
    \centering
    \begin{tikzpicture}
        \begin{axis}[            
            xlabel = Length of training data,
            ylabel = $\overline{NMAE}$ \phantom{~},
            x unit = \si{\second},
            y unit = \%,
            xmin = 5,     xmax = 120,
            ymin = 3.2,  ymax = 5.7,
            grid = major,
            legend cell align = left,
            legend pos = north east,
            grid style = dashed,
            legend style = {font = \scriptsize},
            label style = {font = \scriptsize},
            tick label style = {font = \scriptsize},
            width = 0.95\columnwidth,
            height = 0.5\columnwidth,
            % initialize Dark2
            cycle list/Dark2,
            % combine it with 'mark list*':
            cycle multiindex* list = {
                Dark2\nextlist
            }
        ]

        \addplot+[mark = none, style = solid, ultra thick] 
        table[x = T_train, y expr = {\thisrow{NMAE_mean}*100}, col sep = comma] 
        {system_id/csv/NMAE_vs_Ntrain_SITL_x_vel_no_noise_longer_times.csv_havok_angle.csv};
        \addlegendentry{Without noise}

        \addplot+[mark = none, style = solid, ultra thick] 
        table[x = T_train, y expr = {\thisrow{NMAE_mean}*100}, col sep = comma] 
        {system_id/csv/NMAE_vs_Ntrain_SITL_x_vel_noise_longer_times.csv_havok_angle.csv};
        \addlegendentry{With noise}

        \end{axis}
    \end{tikzpicture} 
    
    \caption{\gls{HAVOK} prediction errors for different lengths of training data with and without noise 
    ($m_p =$~\SI{0.2}{\kilo\gram}, $l =$~\SI{0.5}{\meter}, $T_s =$~\SI{0.03}{\second}).}
    \label{fig:noise_vs_no_noise}
\end{figure}


        \begin{figure}[htb]
    \centering
    \begin{tikzpicture}
        \begin{axis}[            
            xlabel = Length of training data,
            ylabel = $\overline{NMAE}$ \phantom{~},
            x unit = \si{\second},
            y unit = \%,
            xmin = 5,     xmax = 120,
            ymin = 3.2,  ymax = 5.7,
            grid = major,
            legend cell align = left,
            legend pos = north east,
            grid style = dashed,
            legend style = {font = \scriptsize},
            label style = {font = \scriptsize},
            tick label style = {font = \scriptsize},
            width = 0.95\columnwidth,
            height = 0.5\columnwidth,
            % initialize Dark2
            cycle list/Dark2,
            % combine it with 'mark list*':
            cycle multiindex* list = {
                Dark2\nextlist
            }
        ]

        \addplot+[mark = none, style = solid, ultra thick] 
        table[x = T_train, y expr = {\thisrow{NMAE_mean}*100}, col sep = comma] 
        {system_id/csv/NMAE_vs_Ntrain_SITL_x_vel_noise_longer_times.csv_dmd_angle.csv};
        \addlegendentry{DMD}

        \addplot+[mark = none, style = solid, ultra thick] 
        table[x = T_train, y expr = {\thisrow{NMAE_mean}*100}, col sep = comma] 
        {system_id/csv/NMAE_vs_Ntrain_SITL_x_vel_noise_longer_times.csv_havok_angle.csv};
        \addlegendentry{HAVOK}

        \end{axis}
    \end{tikzpicture} 
    
    \caption{\gls{DMD} and \gls{HAVOK} prediction errors for different lengths of noisy training data
    ($m_p =$~\SI{0.2}{\kilo\gram}, $l =$~\SI{0.5}{\meter}, $T_s =$~\SI{0.03}{\second}).}
    \label{fig:havok_vs_dmd_noise}
\end{figure}


        HAVOK performs better than DMD.
        This slight difference in prediciton performance has a negligible effect on control.
        

        Input data needs to be adjusted.

        % \input{plot of SITL acc_sp}
    

    \subsection{Size of training data}
        The length of training data used for system identification affects the quality of the model produced.
        In Figure~\ref{fig:SITL_MAE_vs_train_angular_rate} it is clear that prediction error decreases as the amount of training data increases.
        As more training data is used in the regression problem, 
        the determined model better approximates the actual dynamics because a large range of the dynamics is "seen" by the algorithm.
        
        % \input{low data prediction}
        \paragraph{}
        Models produced from data lengths as short as \SI{5}{\second} predict the movement of state variables surprisingly well.
        % Figure~\ref{} shows prediction.
        Note how the general shape of the prediction represents the training data, 
        even though it contains a lot more high frequency oscillations.

        \begin{figure}[htb]
    \centering
    \begin{tikzpicture}
        \begin{axis}[            
            xlabel = Length of training data,
            ylabel = NMAE of time derivative of predictions,
            x unit = \si{\second},
            % y unit = \si{\second},
            xmin = 0,     xmax = 120,
            ymin = 0.004,  ymax = 0.016,
            grid = major,
            legend cell align = left,
            legend pos = north east,
            grid style = dashed,
            legend style = {font = \scriptsize},
            label style = {font = \scriptsize},
            tick label style = {font = \scriptsize},
            width = 0.95\columnwidth,
            height = 0.5\columnwidth,
            % initialize Dark2
            cycle list/Dark2,
            % combine it with 'mark list*':
            cycle multiindex* list = {
                Dark2\nextlist
            }
        ]

        \addplot+[mark = none, style = solid, ultra thick] 
        table[x = T_train, y expr = {\thisrow{NMAE_mean}*100}, col sep = comma] 
        {system_id/csv/NMAE_vs_Ntrain_SITL_x_vel_noise_longer_times_MAEdiff.csv_dmd_angle.csv};
        \addlegendentry{DMD}
        
        \addplot+[mark = none, style = solid, ultra thick] 
        table[x = T_train, y expr = {\thisrow{NMAE_mean}*100}, col sep = comma] 
        {system_id/csv/NMAE_vs_Ntrain_SITL_x_vel_noise_longer_times_MAEdiff.csv_havok_angle.csv};
        \addlegendentry{HAVOK}

        \end{axis}
    \end{tikzpicture} 
    
    \caption{DMD and HAVOK error of time derivative of predictions for different lengths of noisy training data
    ($m =$~\SI{0.2}{\kilo\gram}, $l =$~\SI{0.5}{\meter}, $T_s =$~\SI{0.03}{\second}).}
    % \label{fig:havok_vs_dmd_noise}
\end{figure}


        \paragraph{}
        % Figure~\ref{} shows the MAE of prediction state derivative.
        Define MAE diff with equation ??.
        % From Figure~\ref{} it appears that at least \SI{}{} training data is required to produce models that represent the dynamics.

        \paragraph{}
        The models produced from HAVOK appear to produce slightly better prediction errors, however this small difference has a negligible effect on control performance.

        \paragraph{}
        % In Figure~\ref{fig:MAE_vs_train} it can be seen that after approximately \SI{??}{\second} 
        the prediction error does not significantly improve with more training data.
        It practice less training data is desirable because less flight time will be wasted on training a model before the quadrotor can fly with a updated controller.
        Less training data also corresponds to lower memory usage on quadrotor hardware.
        Such a slight improvement in prediction error also has a negligible effect on control performance and is therefore not worth the increased data requirement.
        % Therefore, only \SI{??}{\second} of flight data will be used to train system identification models. 


    \subsection{System parameters}
            
        Works across a range of parameters.

        \paragraph{}
        The payload acting as a single floating pendulum, as described in Section~\ref{sec:plant_considered},
        has two system parameters, $m_p$ and $l$.
        For the practical quadrotor considered, the payload mass is limited to:
        \begin{equation}
            0.01 \leq m_p \leq \SI{0.4}{\kilo\gram} .
        \end{equation}
        When no external payload is attached, the connection device attached to the end of the cable is 
        $m_p = \SI{0.01}{\kilo\gram}$.
        % It becomes unsafe to Flying without a cable attached to the cable, or with a payload with a very small mass, 
        % may become unsafe since the cable may not always  kept taut by the mass. 
        On the other limit, $m_p = \SI{0.4}{\kilo\gram}$ is determined to be the maximum payload mass the quadrotor can carry safely 
        based on the maximum thrust of the motors.

        \paragraph{}
        The cable length is limited to:
        \begin{equation}
            0.5 \leq l \leq \SI{2}{\metre} .
        \end{equation}
        A cable length shorter than \SI{0.5}{\metre} is quite impractical and may rather be attached as a rigid payload.
        There are very few practical applications that may require a shorter cable length.
        It is also unsafe to fly with a shorter cable length, 
        since the payload may collide with the quadrotor during an uncontrolled swing.
        A longer cable guards against a payload and vehicle collision, 
        because more energy needs to be transferred to the payload to reach the height of the vehicle. 
        The maximum cable length is selected as $l=~\SI{2}{\metre}$ by intuition 
        since a cable much longer than this may not be practically useful for a drone delivery flight with the considered quadrotor.

        \paragraph{}
        Plot MAE vs l
        \begin{figure}[h]
    \centering
    \begin{tikzpicture}
        \begin{axis}[            
            xlabel = Length of training data,
            ylabel = MAE,
            x unit = \si{\second},
            % y unit = \si{\second},
            xmin = 0,     xmax = 120,
            ymin = 0.08,  ymax = 0.12,
            grid = major,
            legend cell align = left,
            legend pos = north east,
            grid style = dashed,
            legend style = {font = \scriptsize},
            label style = {font = \scriptsize},
            tick label style = {font = \scriptsize},
            width = 0.45\columnwidth,
            height = 0.5\columnwidth,
            % initialize Dark2
            cycle list/Dark2,
            % combine it with 'mark list*':
            cycle multiindex* list = {
                Dark2\nextlist
            }
        ]
        
        \addplot+[mark = none, style = solid, ultra thick] 
        table[x = T_train, y = MAE_mean, col sep = comma] 
        {system_id/csv/MAE_vs_Ntrain_SITL_x_vel_noise_l0-5_m0-2.csv_havok_angle.csv};
        \addlegendentry{$l =$~\SI{0.5}{\metre}}
        
        \addplot+[mark = none, style = solid, ultra thick] 
        table[x = T_train, y = MAE_mean, col sep = comma] 
        {system_id/csv/MAE_vs_Ntrain_SITL_x_vel_noise_l1_m0-2.csv_havok_angle.csv};
        \addlegendentry{$l =$~\SI{1}{\metre}}
        
        \addplot+[mark = none, style = solid, ultra thick] 
        table[x = T_train, y = MAE_mean, col sep = comma] 
        {system_id/csv/MAE_vs_Ntrain_SITL_x_vel_noise_l2_m0-2.csv_havok_angle.csv};
        \addlegendentry{$l =$~\SI{2}{\metre}}
        
        \end{axis}
    \end{tikzpicture} 
    
    \caption{HAVOK prediction error using different cable lengths with different lengths of noisy training data
    ($m =$~\SI{0.2}{\kilo\gram}, $T_{train} =$~various??.)}
    \label{fig:MAE_vs_Ntrain_vs_L_havok}
\end{figure}

        DMD shows the same trend revealed in Figure~\ref{fig:MAE_vs_Ntrain_vs_L_havok}.

        \paragraph{}
        Plot MAE vs m
        \begin{figure}[h]
    \centering
    \begin{tikzpicture}
        \begin{axis}[            
            xlabel = Length of training data,
            ylabel = MAE,
            x unit = \si{\second},
            % y unit = \si{\second},
            xmin = 0,     xmax = 120,
            ymin = 0.08,  ymax = 0.12,
            grid = major,
            legend cell align = left,
            legend pos = north east,
            grid style = dashed,
            legend style = {font = \scriptsize},
            label style = {font = \scriptsize},
            tick label style = {font = \scriptsize},
            width = 0.95\columnwidth,
            height = 0.5\columnwidth,
            % initialize Dark2
            cycle list/Dark2,
            % combine it with 'mark list*':
            cycle multiindex* list = {
                Dark2\nextlist
            }
        ]
         
        \addplot+[mark = none, style = solid, ultra thick] 
        table[x = T_train, y = MAE_mean, col sep = comma] 
        {system_id/csv/MAE_vs_Ntrain_SITL_x_vel_noise_l1_m0-1.csv_havok_angle.csv};
        \addlegendentry{$m =$~\SI{0.1}{\kilo\gram}}
        
        \addplot+[mark = none, style = solid, ultra thick] 
        table[x = T_train, y = MAE_mean, col sep = comma] 
        {system_id/csv/MAE_vs_Ntrain_SITL_x_vel_noise_l1_m0-2.csv_havok_angle.csv};
        \addlegendentry{$m =$~\SI{0.2}{\kilo\gram}}
        
        \addplot+[mark = none, style = solid, ultra thick] 
        table[x = T_train, y = MAE_mean, col sep = comma] 
        {system_id/csv/MAE_vs_Ntrain_SITL_x_vel_noise_l1_m0-3.csv_havok_angle.csv};
        \addlegendentry{$m =$~\SI{0.3}{\kilo\gram}} 
        
        \end{axis}
    \end{tikzpicture} 
    
    \caption{DMD prediction error using different cable lengths with a range of different sample times of noisy training data
    ($l =$~\SI{1}{\metre}, $T_{train} =$~various??.)}
    \label{fig:MAE_vs_Ntrain_vs_m_havok}
\end{figure}

        DMD shows the same trend revealed in Figure~\ref{fig:MAE_vs_Ntrain_vs_m_havok}.

        \paragraph{}
        % Not very conclusive
        % \begin{figure}[htb]
    \centering
    \begin{tikzpicture}
        \begin{axis}[            
            xlabel = Ts,
            ylabel = NMAE\textsubscript{mm},
            % x unit = \si{\second},
            % y unit = \si{\second},
            xmin = 0.02,  xmax = 0.05,
            ymin = 0.03,  ymax = 0.055,
            grid = major,
            legend cell align = left,
            legend pos = north east,
            grid style = dashed,
            legend style = {font = \scriptsize},
            label style = {font = \scriptsize},
            tick label style = {font = \scriptsize},
            width = 0.45\columnwidth,
            height = 0.5\columnwidth,
            % initialize Dark2
            cycle list/Dark2,
            % combine it with 'mark list*':
            cycle multiindex* list = {
                Dark2\nextlist
            }
        ]

        % \addplot+[mark = none, style = solid, ultra thick] 
        % table[x = Ts, y = NMAE_mean, col sep = comma] 
        % {system_id/csv/NMAE_vs_Ts_SITL_x_vel_noise_l0-25_m0-2.csv_dmd_angle.csv};
        % \addlegendentry{$l =$~\SI{0.25}{\metre}}
        
        \addplot+[mark = none, style = solid, ultra thick] 
        table[x = Ts, y = NMAE_mean, col sep = comma] 
        {system_id/csv/NMAE_vs_Ts_SITL_x_vel_noise_l0-5_m0-2.csv_dmd_angle.csv};
        \addlegendentry{$l =$~\SI{0.5}{\metre}}
        
        \addplot+[mark = none, style = solid, ultra thick] 
        table[x = Ts, y = NMAE_mean, col sep = comma] 
        {system_id/csv/NMAE_vs_Ts_SITL_x_vel_noise_l1_m0-2.csv_dmd_angle.csv};
        \addlegendentry{$l =$~\SI{1}{\metre}}
        
        \addplot+[mark = none, style = solid, ultra thick] 
        table[x = Ts, y = NMAE_mean, col sep = comma] 
        {system_id/csv/NMAE_vs_Ts_SITL_x_vel_noise_l2_m0-2.csv_dmd_angle.csv};
        \addlegendentry{$l =$~\SI{2}{\metre}}
        
        \end{axis}
    \end{tikzpicture} 
    
    \caption{DMD prediction error using different cable lengths with a range of different sample times of noisy training data
    ($m =$~\SI{0.2}{\kilo\gram})}
    \label{fig:MAE_vs_Ts_vs_L}
\end{figure}

        % See how it affects Ts
        % plot MAE vs Ts with contours of l

        Best hyperparameters.
        Fixed size of data.
        Fixed sample time.

    \subsection{Dynamic payload}
        Some payloads attached to the cable may not satisfy the assumptions made in Section~\ref{sec:plant_considered}.
        For example, if a long payload is attached to the cable, the CoM of the payload will be quite a distance below the attachment of the cable.
        This creates a double pendulum model which has different dynamcis than a single pendulum.

        \murray{insert picture a practical quad with long payload}
        \paragraph{}
        Another payload case that will cause inaccuracies in the parameter estimation technique is if a payload is attached rigidly to the quadrotor while it also has a suspended payload.
        The payload mass estimation is based on the assumption that the quadrotor mass is known.
        However if a mass is rigidly attached to the vehicle, the effective quadrotor mass is changed and the RLS payload mass estimation is no longer accurate.

        These payload produce significantly different dynamics than predicted by the a priori model.
        Since the system identification by parameter estimation depends heavily on the pre-determined mathematical model,
        
        \paragraph{}
        For each of these payload cases, a different parameter estimation based techniques would needs to be designed for effective control.
        This is undesirable for practical drone deliveries, especially when the type of paylaod is not known well in advance or changes regularly.
        A data-driven technique provides a more general solution since it accommodates a larger range of payload types and does not require a prioir modelling information.
        
        plot hyperparameterss MAE. Not how much more delays are required
