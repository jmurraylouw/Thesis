\section{Parameter estimation} \label{sec:param_estimation}

\subsection{Predetermined linear model}
    The motivation for paramater estimation is to determine unknown parameter values required by the predetermined model.
    This model was derived a prioiri in Section~\ref{sec:linear_model}.
    
\subsection{Payload mass estimation}
    RLS
    Cannot work if quadrotor mass also changes, because assume that $m_q$ is known.

\subsection{Cable length estimation}
    The cable length is estimated from the measurement of natural frequency of the swinging payload.
    As described by
    \cite{bisgaard},
    the natural frequency is given by:
    \begin{equation} \label{eq:nat_freq}
        \omega_n = \sqrt{ \frac{g}{l} \cdot \frac{m_q + m_p}{m_q}}
    \end{equation}
    The natural frequency is measured by performing a FFT on the payload swing angle response after a position step by the quadrotor.
    The dominant frequency identified by the FFT during free swing is the natural frequency of the payload.
    
    \ref{fig:pos_step_angle}
    shows the payload swing angle after the system is stimulated by a position step setpoint.
    As shown in 
    \ref{fig:pos_step_angle}
    the first few seconds of the step response are not used in the FFT.
    This is to minimise the effect of the quadrotor controllers on the swing angle frequency 
    by excluding the transient response in the FFT.

    \ref{fig:fft} 
    shows the resulting amplitude spectrum of the payload swing angle response.
    The dominant frequency is clearly identified as ??.
    Since $m_q$ and $g$ is known, and $m_p$ and $\omega_n$ has been estimated, $l$ can now be determined from
    \ref{eq:nat_freq}.
    In this case the estimated length is ??, compared to the actual length of ??.
    
    Frequency resolution ??
    error for different lengths??

