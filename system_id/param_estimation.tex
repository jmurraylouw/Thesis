\section{Parameter estimation} \label{sec:param_estimation}

    \paragraph
    The purpose of parameter estimation is to determine unknown values 
    required by a predetermined, white-box model.
    The identified model is used to design a \gls{LQR} controller and
    therefore needs to be in a linear, continuos-time, state-space form.
    This model was derived and linearised a priori in Section~\ref{sec:linear_model}.
    The unknown parameters in the model include the payload mass, $m_p$ and the cable length, $l$.
    Two separate methods are used to estimate each of these parameters.
    This method was used by \cite{Erasmus2020} and \cite{Slabber2020} 
    to design a \gls{LQR} controller and implement swing damping control 
    of a quadrotor with an unknown suspended payload.
    
\subsection{Payload mass estimation}

    \paragraph
    \gls{RLS} is used by \cite{Erasmus2020} and \cite{Slabber2020} to estimate the payload mass.
    It is assumed that the quadrotor mass is known before flight, 
    therefore the payload mass can be estimated from the additional trust required during hover.
    In both \cite{Erasmus2020} and \cite{Slabber2020} it is demonstrated 
    that \gls{RLS} is very accurate for a system nearly identical to the one considered in this work.
    To compare other aspects of the white-box and black-box techniques with more clarity, 
    it will therefore be assumed that the method estimates $m_p$ with perfect accuracy.
    This isolates any inaccuracies in the white-box model to 
    either the a priori modelling or the cable length estimation.

    \paragraph
    It should be noted however that this method is dependant on the assumption 
    that the vehicle mass is known and remains unchanged.
    The method will clearly be inaccurate if a unknown mass is added to the quadrotor 
    in conjunction with the suspended payload.
    Common practical examples of this include adding a camera to the vehicle or using a different battery for a different flight range.
    In these cases the mass estimation method will have to be redesigned.
    This is an inherent problem of the white-box techniques.
    The parameter estimation methods are designed for specific modelling assumptions 
    and are not adaptable to different types of payload loadings.
    In contrast, data-driven techniques are adaptable to different payload loadings 
    because it does not depend on a priori modelling assumptions.

\subsection{Cable length estimation} \label{sec:length_estimation}

    \paragraph
    The cable length is estimated from the measurement of the natural frequency of the swinging payload.
    As described by \cite{Bisgaard2008}, the natural frequency is given by:
    \begin{equation} \label{eq:nat_freq}
        \omega_n = \sqrt{ \frac{g}{l} \cdot \frac{m_q + m_p}{m_q}} .
    \end{equation}
    The cable length can clearly be calculated from (\ref{eq:nat_freq}) if the other parameters are known.
    The natural frequency is measured by performing a \gls{FFT} on the payload swing angle response 
    after a position step by the quadrotor.
    The dominant frequency identified by the \gls{FFT} during free swing is an approximate measurement
    of the natural frequency of the payload.
    Note that the measured frequency actually corresponds to the damped natural frequency, 
    but it is assumed that damping during free swing is negligible.
    Therefore the damped natural frequency closely approximates the natural frequency.
    % This method is applied successfully by \cite{Erasmus2020} and \cite{Slabber2020} for cable length estimation.
    
    \paragraph
    Figure~\ref{fig:pos_step_single_pend} shows the payload swing angle response 
    to a position step setpoint.
    The first few seconds of the step response are excluded from the \gls{FFT} 
    to minimise the effect of the transient response and the quadrotor controllers 
    on the natural frequency measurement.
    Figure~\ref{fig:FFT_pos_step} shows the resulting single-sided amplitude spectrum of the \gls{FFT} of this data.
    
    \paragraph
    The dominant frequency is clearly identified by the peak at \SI{0.520}{\radian/\second}.
    Since $m_q$, $m_p$ and $g$ are known, and $\omega_n$ has been measured, 
    $l$ can be determined from (\ref{eq:nat_freq}).
    The estimated length for this simulation is \SI{0.953}{\metre}.
    The actual cable length is \SI{1}{\metre}, therefore this estimation has an error of 4.7\%.
    As documented \cite{Erasmus2020} and \cite{Slabber2020}, an error of this magnitude is acceptably small
    and still results in effective control with a \gls{LQR}.
    It is also shown by \cite{Erasmus2020} and \cite{Slabber2020} that this estimation method 
    is effective for a range of different payloads.

    
\begin{figure}
    \captionsetup[subfigure]{justification=centering}
    \centering
    \begin{subfigure}[t]{0.45\columnwidth}
    \centering
    \begin{tikzpicture}
        \begin{axis}[            
            xlabel = Time,
            ylabel = Payload angle,
            x unit = \si{\second},
            y unit = \si{\radian},
            xmin = 0,   xmax = 15,
            ymin = -0.8,  ymax = 0.8,
            grid = major,
            legend cell align = left,
            legend pos = north east,
            grid style = dashed,
            legend style = {font = \scriptsize},
            label style = {font = \scriptsize},
            tick label style = {font = \scriptsize},
            width = 0.95\columnwidth,
            height = 0.95\columnwidth,
            % initialize Dark2
            cycle list/Dark2,
            % combine it with 'mark list*':
            cycle multiindex* list = {
                Dark2\nextlist
            }
        ]
        
        \pgfplotsset{cycle list shift=1}
        \addplot+[mark = none, style = solid, ultra thick] 
        table[x = time, y = theta, col sep = comma] 
        {system_id/csv/pos_step_SITL_single_pos_step_m0.3_l1.csv.csv};

        \end{axis}
    \end{tikzpicture} 
    
    \caption{Position step response of the payload swing angle}
    \label{fig:pos_step_single_pend}
\end{subfigure}
 % subfigure
    \begin{subfigure}[t]{0.45\columnwidth}
    \centering
    \begin{tikzpicture}
        \begin{axis}[            
            xlabel = Frequency,
            ylabel = Amplitude,
            x unit = \si{\radian/\second},
            % y unit = \si{\second},
            xmin = 0.3,     xmax = 1,
            ymin = 0,  ymax = 0.006,
            grid = major,
            legend cell align = left,
            legend pos = north east,
            grid style = dashed,
            legend style = {font = \scriptsize},
            label style = {font = \scriptsize},
            tick label style = {font = \scriptsize},
            width = 0.95\columnwidth,
            height = 0.95\columnwidth,
            % initialize Dark2
            cycle list/Dark2,
            % combine it with 'mark list*':
            cycle multiindex* list = {
                Dark2\nextlist
            }
        ]

        \addplot+[mark = none, style = solid, ultra thick] 
        table[x = f, y = P1, col sep = comma] 
        {system_id/csv/FFT_SITL_single_pos_step_m0.3_l1.csv.csv};

        \end{axis}
    \end{tikzpicture} 
    
    \caption{The single-sided amplitude spectrum of the FFT}
    \label{fig:FFT_pos_step}
\end{subfigure}
 % subfigure
    \caption{ Data from a North position step response used for cable length estimation 
    ($l =$~\SI{1}{\metre}, $m_p =$~\SI{0.3}{\kilo\gram}.).}
    \label{fig:FFT_pos_step_subfigs}  
\end{figure}

    

