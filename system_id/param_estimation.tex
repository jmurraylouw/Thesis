\section{Parameter estimation} \label{sec:param_estimation}

    \paragraph
    The purpose of parameter estimation is to determine unknown values 
    required by a predetermined, white-box model.
    The identified model is used to design a \gls{LQR} controller and
    therefore needs to be in the form of a linear, continuous-time, state-space model.
    This model was derived by \cite{Slabber2020} and will be presented in the section below.
    The unknown parameters in the model include the payload mass, $m_p$ and the cable length, $l$.
    Two separate methods are used to estimate each of these parameters.
    This approach was used successfully by \cite{Erasmus2020} and \cite{Slabber2020} 
    to implement a \gls{LQR} swing damping controller for a multirotor with an unknown suspended payload.
    
\subsection{White-box model}

    \paragraph
    A white-box model was derived a priori by \cite{Slabber2020} to represent the dynamics of the quadrotor-payload system.
    The non-linear dynamical equations derived in Chapter~\ref{chap:modelling} were linearised to produce a linear state-space model.
    The equations were linearised around hover by applying small angle approximations.
    As discussed in Section~\ref{sec:payload_assumptions}, the payload is attached to the \gls{CoM} of the multirotor.
    Therefore the vehicle attitude dynamics are decoupled from the payload dynamics and are not considered in this model.
    Aerodynamic drag forces are assumed to be negligible and are ignored in this model.
    It is also assumed that the attitude controllers have a large enough time-scale separation from the velocity controllers, such that $A_{N_{sp}} \approx A_N$.
    
    % \paragraph
    % In the work \cite{Slabber2020}, it was assumed that aerodynamic drag forces are negligible and can be ignored.
    % In this work however, a simplified linear damping model is used to account for aerodynamic drag forces.
    % The aerodynamic drag force in the North axis is approximated by, 
    % \begin{equation}
    %     F_N^A = - c_{air} V_N
    % \end{equation}
    
    \paragraph
    The resultant linear state-space model of the longitudinal dynamics is given by,
    \begin{equation}
        \dot{\bm{x}}_{long} = \bm{A}_{long} \bm{x}_{long} + \bm{B}_{long} \bm{u}_{long} ,
    \end{equation}
    where
    \begin{align}
        \bm{x}_{long} &= 
            \begin{bmatrix}
                V_N & \theta & \dot{\theta}
            \end{bmatrix}^T, \\
        \bm{u}_{long} &= 
            \begin{bmatrix}
                A_{N_{sp}}
            \end{bmatrix}, \\
        \bm{A}_{long} &= 
            \begin{bmatrix}
                0 & \frac{ m_p \cdot g}{m_Q} & \frac{c_{\theta}}{(l\cdot m_Q)} \\
                0 & 0 & 1 \\
                0 & \frac{ (m_p+m_Q) \cdot g}{(m_Q \cdot l) } & \frac{ -c_{\theta} \cdot (m_p+m_Q)}{(l^2 \cdot m_Q \cdot m_p)}
            \end{bmatrix}, \mbox{ and} \\
        \bm{B}_{long} &= 
            \begin{bmatrix}
                1\\
                0 \\
                - \frac{1}{l} \\
            \end{bmatrix} .
    \end{align}

    % matrix with linear air damping:
    % \bm{A}_{long} &= 
    % \begin{bmatrix}
    %     \frac{ -c_{air} }{m_Q} & \frac{ m_p \cdot g}{m_Q} & \frac{c_{\theta}}{(l\cdot m_Q)} \\
    %     0 & 0 & 1 \\
    %     \frac{ c_{air} }{ (m_Q \cdot l) } & \frac{ (m_p+m_Q) \cdot g}{(m_Q \cdot l) } & \frac{ -c_{\theta} \cdot (m_p+m_Q)}{(l^2 \cdot m_Q \cdot m_p)}
    % \end{bmatrix}, \mbox{ and} \\

    \paragraph
    It is assumed that the coefficient of friction, $c_{\theta}$, does not change for different payload, therefore this parameter can be determine experimentally and will known prior to flight.
    Furthermore, it is assumed that $m_Q$ is known.
    The payload parameters, 
    $m_p$ and $l$, are unknown prior to flight. 
    These parameters will be estimated in the sections below.    

\subsection{Payload mass estimation}

    \paragraph
    \gls{RLS} is used by \cite{Erasmus2020} and \cite{Slabber2020} to estimate the payload mass.
    It is assumed that the multirotor mass is known before a flight, 
    therefore the payload mass can be estimated from the additional trust required during hover.
    In both \cite{Erasmus2020} and \cite{Slabber2020} it is demonstrated 
    that \gls{RLS} is very accurate for a system nearly identical to the one considered in this work.
    To compare other aspects of the white-box and black-box techniques with more clarity, 
    it will be assumed that the method estimates $m_p$ with perfect accuracy.
    This isolates any inaccuracies in the white-box model to 
    either the a priori modelling or the cable length estimation.

    \paragraph
    It should be noted however that this method is dependant on the assumption 
    that the vehicle mass is known and remains unchanged.
    The method will clearly be inaccurate if an unknown mass is added to the multirotor 
    in conjunction with the suspended payload.
    Common practical examples of this include adding a camera to the vehicle or using a different battery for a different flight range.
    In these cases, the mass estimation method will have to be redesigned.
    This is an inherent problem of the white-box techniques.
    The parameter estimation methods are designed for specific modelling assumptions 
    and are not adaptable to different types of payload loadings.
    In contrast, data-driven techniques are adaptable to different payload loadings 
    because it does not depend on a priori modelling assumptions.

\subsection{Cable length estimation} \label{sec:length_estimation}

    \paragraph
    The cable length is estimated from the measurement of the natural frequency of the swinging payload.
    As described by \cite{Bisgaard2008}, the natural frequency is given by:
    \begin{equation} \label{eq:nat_freq}
        \omega_n = \sqrt{ \frac{g}{l} \cdot \frac{m_Q + m_p}{m_Q}} .
    \end{equation}
    The cable length can clearly be calculated from Equation~\ref{eq:nat_freq} if the other parameters are known.
    The natural frequency is measured by performing a \gls{FFT} on the payload swing angle response 
    after a velocity step by the multirotor.
    The dominant frequency identified by the \gls{FFT} during free swing is an approximate measurement
    of the natural frequency of the payload.
    Note that the measured frequency rather corresponds to the damped natural frequency.
    The swing angle damping is caused by 
    the velocity controller, 
    friction at the attachment of the cable to the drone,
    and air drag.
    However, it is assumed that the damping coefficient is small enough for
    the damped natural frequency to closely approximate the theoretical natural frequency.
    % This method is applied successfully by \cite{Erasmus2020} and \cite{Slabber2020} for cable length estimation.
    
    \paragraph
    Figure~\ref{fig:vel_step_single_pend} shows the payload swing angle response 
    to a position step setpoint.
    The first few seconds of the step response are excluded from the \gls{FFT} 
    to minimise the effect of the transient response and the multirotor controllers 
    on the natural frequency measurement.
    Figure~\ref{fig:FFT_vel_step} shows the resulting single-sided amplitude spectrum of the \gls{FFT} of this data.
    
    \paragraph
    The dominant frequency is clearly identified by the peak at \SI{0.520}{\radian/\second}.
    Since $m_Q$, $m_p$ and $g$ are known, and $\omega_n$ has been measured, 
    $l$ can be determined from Equation~\ref{eq:nat_freq}.
    The estimated length for this simulation is \SI{0.953}{\metre}.
    The actual cable length is \SI{1}{\metre}, therefore this estimation has an error of 4.7\%.
    As documented by \cite{Erasmus2020} and \cite{Slabber2020}, an error of this magnitude is acceptably small
    and still results in effective control with a \gls{LQR}.
    It was also shown by \cite{Erasmus2020} and \cite{Slabber2020} that this estimation method 
    is effective for a range of different payloads in simulation.

    
\begin{figure}
    \captionsetup[subfigure]{justification=centering}
    \centering
    \begin{subfigure}[t]{0.45\columnwidth}
    \centering
    \begin{tikzpicture}
        \begin{axis}[            
            xlabel = Time,
            ylabel = Payload angle,
            x unit = \si{\second},
            y unit = \si{\degree},
            xmin = 0,   xmax = 15,
            ymin = -12,  ymax = 6,
            grid = major,
            legend cell align = left,
            legend pos = north east,
            grid style = dashed,
            legend style = {font = \scriptsize},
            label style = {font = \scriptsize},
            tick label style = {font = \scriptsize},
            width = 0.95\columnwidth,
            height = 0.95\columnwidth,
            % initialize Dark2
            cycle list/Dark2,
            % combine it with 'mark list*':
            cycle multiindex* list = {
                Dark2\nextlist
            }
        ]
        
        \pgfplotsset{cycle list shift=1}
        % \addplot+[mark = none, style = solid, ultra thick] 
        % table[x = time, y = theta, col sep = comma] 
        % {system_id/csv/pos_step_SITL_single_pos_step_m0.3_l1.csv.csv};

        \addplot+[mark = none, style = solid, ultra thick] 
        table[x = time, y expr = \thisrow{theta} * 57.2958, col sep = comma] 
        {system_id/csv/single_step_predictions_SITL_x_vel_noise_l1_m0-3.csv.csv};

        \end{axis}
    \end{tikzpicture} 
    
    \caption{Position step response of the payload swing angle}
    \label{fig:vel_step_single_pend}
\end{subfigure}
 % subfigure
    \begin{subfigure}[t]{0.45\columnwidth}
    \centering
    \begin{tikzpicture}
        \begin{axis}[            
            xlabel = Frequency,
            ylabel = Amplitude,
            x unit = \si{\radian/\second},
            % y unit = \si{\second},
            xmin = 0.3,     xmax = 1,
            ymin = 0,  ymax = 0.006,
            grid = major,
            legend cell align = left,
            legend pos = north east,
            grid style = dashed,
            legend style = {font = \scriptsize},
            label style = {font = \scriptsize},
            tick label style = {font = \scriptsize},
            width = 0.95\columnwidth,
            height = 0.95\columnwidth,
            % initialize Dark2
            cycle list/Dark2,
            % combine it with 'mark list*':
            cycle multiindex* list = {
                Dark2\nextlist
            }
        ]

        \addplot+[mark = none, style = solid, ultra thick] 
        table[x = f, y = P1, col sep = comma] 
        {system_id/csv/FFT_SITL_single_pos_step_m0.3_l1.csv.csv};

        \end{axis}
    \end{tikzpicture} 
    
    \caption{The single-sided amplitude spectrum of the FFT}
    \label{fig:FFT_vel_step}
\end{subfigure}
 % subfigure
    \caption{Data from a velocity step response used for cable length estimation 
    ($l =$~\SI{1}{\metre}, $m_p =$~\SI{0.3}{\kilo\gram}.).}
    \label{fig:FFT_vel_step_subfigs}  
\end{figure}

    

