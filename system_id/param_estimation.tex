\section{Parameter estimation} \label{sec:param_estimation}

    \paragraph{}
    The purpose of parameter estimation is to determine unknown parameter values required by the predetermined model.
    This model was derived a priori in Section~\ref{sec:linear_model}.
    The unknown parameters in the model include the payload mass, $m_p$ and the cable length, $l$.
    Two seperate methods used to estimate each of these paramaters.
    These methods were used by \cite{Erasmus2020} and \cite{Slabber2020} 
    with a LQR controller to apply swing damping control of a quadrotor with an unknown suspended payload.
    
\subsection{Payload mass estimation}

    \paragraph{}
    RLS is used by \cite{Erasmus2020} and \cite{Slabber2020} to estimate the payload mass.
    Because it is assumed that the quadrotor mass is known, 
    the payload mass can be estimated from the additional trust required during hover.
    In both \cite{Erasmus2020} and \cite{Slabber2020} it is demonstrated 
    that RLS is very accurate for the payload mass estimation of a system nearly identical to the one considered in this work.
    To compare the white-box and black-box techniques with more clarity, 
    it will therefore be assumed that the method estimates $m_p$ with perfect accuracy.
    This isolates any inaccuracies in the white-box model to 
    either the a priori modelling or the cable length estimation.

    \paragraph{}
    It should be noted however that this method is dependant on the assumption 
    that the vehicle mass is known and remains unchanged.
    The method will clearly be inaccurate if a unknown mass is added to the quadrotor 
    in conjunction with the suspended payload.
    In this case the mass estimation method will have to be completely redesigned.
    This is an inherent problem of the white-box techniques.
    The parameter estimation methods are designed for specific modelling assumptions 
    and are not adaptable to different types of payload loadings.
    In contrast, data-driven techniques are adaptable to different payload loadings 
    because it does not depend on a priori modelling assumptions.

\subsection{Cable length estimation} \label{sec:length_estimation}

    \paragraph{}
    The cable length is estimated from the measurement of the natural frequency of the swinging payload.
    As described by
    \cite{bisgaard},
    the natural frequency is given by:
    \begin{equation} \label{eq:nat_freq}
        \omega_n = \sqrt{ \frac{g}{l} \cdot \frac{m_q + m_p}{m_q}}
    \end{equation}
    The natural frequency is measured by performing a FFT on the payload swing angle response 
    after a position step by the quadrotor.
    The dominant frequency identified by the FFT during free swing is the natural frequency of the payload.
    This method is applied successfully by \cite{Erasmus2020} and \cite{Slabber2020} for cable length estimation.
    
    \paragraph{}
    \ref{fig:pos_step_angle}
    shows the payload swing angle after the system is stimulated by a position step setpoint.
    As shown in 
    \ref{fig:pos_step_angle}
    the first few seconds of the step response are not used in the FFT.
    This is to minimise the effect of the quadrotor controllers on the swing angle frequency 
    by excluding the transient response in the FFT.

    \paragraph{}
    \ref{fig:fft} 
    shows the resulting amplitude spectrum of the payload swing angle response.
    The dominant frequency is clearly identified as ??.
    Since $m_q$ and $g$ is known, and $m_p$ and $\omega_n$ has been estimated, $l$ can now be determined from
    \ref{eq:nat_freq}.
    In this case the estimated length is ??, compared to the actual length of ??.
    
    Frequency resolution ??
    error for different lengths??

