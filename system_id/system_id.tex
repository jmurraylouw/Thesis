\graphicspath{{system_id/fig/}}

\chapter{System identification}
\label{chap:system_id}

    \par
    System identification is the process of creating mathematical models of a dynamical system by using input and output measurements of the system.
    Two major approaches are used to represent the dynamics of such a system, resulting in white-box or black-box models.

 
\section{Parameter estimation} 

    \subsection{White-box models}

        In white-box models the physics of a model are understood by the user.    
        These models are therefore determined from first principles.
        This can be done by modelling physical processes with techniques like Lagrangian mechanics or Newton equations.
        In this case, the mathematical relations between physical properties are predefined in the modelling phase.
        System identification is then reduced to parameter estimation to determine values for the parameters used in die model.
        
        \paragraph{} 
        This is the approach used by \cite{erasmus} and \cite{slabber} for swing damping control of a quadrotor with an unknown suspended payload.
        The system was modelled as two rigid bodies connected by a link and the following assumptions were made regarding the suspended payload:
        \begin{itemize}
            \item The payload is a point mass.
            \item The link is massless.
            \item The link is rigid.
            \item The link is attached to the CoM of the quadrotor.
        \end{itemize}
        The only unknown parameters in the quadrotor and payload model is the payload mass and link length.
        These parameters are first estimated and then inserted into the predefined, linearised model.
        This model is used by a LQR controller to damp swing angles while also controlling the vehicle.

        \paragraph{} 
        The approach works well for systems with predictable dynamics, but it is not very adaptable.
        The payload considered by \cite{erasmus} and \cite{slabber} is limited to a small rigid mass suspended from the quadrotor by a non-stretching cable. 
        In this use case it was shown that a LQR controller successfully controls a quadrotor while minimising payload swing angles.
        However, if a payload or cable is used that violates one of the modelling assumptions, the predefined model no longer accurately represent the system.
        Since the controller is dependent on this model, the mismatch between the model and actual dynamics may result in undesirable controller behaviour.


    Parameter estimation

    \subsection{Payload mass estimation}
        RLS

    \subsection{Cable length estimation}
        The cable length is estimated from the measurement of natural frequency of the swinging payload.
        As described by
        \cite{bisgaard},
        the natural frequency is given by:
        \begin{equation} \label{eq:nat_freq}
            \omega_n = \sqrt{ \frac{g}{l} \cdot \frac{m_q + m_p}{m_q}}
        \end{equation}
        The natural frequency is measured by performing a FFT on the payload swing angle response after a position step by the quadrotor.
        The dominant frequency identified by the FFT during free swing is the natural frequency of the payload.
        
        \ref{fig:pos_step_angle}
        shows the payload swing angle after the system is stimulated by a position step setpoint.
        As shown in 
        \ref{fig:pos_step_angle}
        the first few seconds of the step response are not used in the FFT.
        This is to minimise the effect of the quadrotor controllers on the swing angle frequency 
        by excluding the transient response in the FFT.

        \ref{fig:pos_step_angle} 
        shows the resulting amplitude spectrum of the payload swing angle response.
        
        Since $m_q$ and $g$ is known, and $m_p$ and $\omega_n$ has been estimated, $l$ can now be determined from
        \ref{eq:nat_freq}
        

\section{Data-driven system identification}
    \subsection{Black-box models}
        In contrast to white-box models, black-box models do not require predefined mathematical relations between system parameters.
        % The user only considers what goes into, and comes out of, the black-box.
        % Something imagery about why it is called black box
        No prior knowledge of the physics of the system are considered and no modelling assumptions are made.
        Black-box techniques determine the mathematical relationship between inputs and outputs of a system based only on measurement data.
        This is referred to as data-driven system identification.

        \paragraph{}
        Black-box models can be categorised as either non-linear or linear models.
        Non-linear models are often more accurate than linear models because complex, real-world dynamics are better approximated by non-linear systems.
        The dynamics of a quadrotor and suspended payload are also non-linear.
        Examples of black box models with quadrotors and payloads in literature ???

        \paragraph{}
        However, non-linear models are inherently more complex than linear models. 
        Controllers that use non-linear models are usually more computationally complex than those with linear models.
        Control archetictures for quadrotors used in practical applications are mostly implemented on onboard hardware.
        Therefore there is value in low-complexity, linear models since these may be simple enough to execute on low cost hardware.
        trade-off between accuracy and complexity.
        Non-linear models may require control implementations that are too computationally expensive and may not be practically realisable on the available hardware on a quadrotor.

    \subsection{DMD}
    \subsection{HAVOK}


