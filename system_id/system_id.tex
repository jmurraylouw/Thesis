\graphicspath{{system_id/fig/}, {system_id/plots}}

\chapter{System identification}
\label{chap:system_id}

    \paragraph
    System identification is the process of creating a mathematical model of a dynamical system by using input and output measurements of that system.
    Two major approaches are used to represent the dynamics of such a system:
    \begin{enumerate}
        \item A priori mathematical modelling with parameter estimation
        \item Data-driven system identification
    \end{enumerate}

    \paragraph
    This chapter discusses these system identification approaches and describes the differences between them.
    For each approach, specific estimation techniques are explained and applied to the quadrotor and payload system.
    The results of these techniques are then compared to each other.

    \section{White-box and black-box techniques}

    \paragraph
    Models determined from data-driven system identification methods are generally called black-box models.
    The user is only concerned with the inputs and output of the model
    and does not need to derive mathematical relationships from theoretical deductions.
    In contrast, white-box models are determined from a~priori modelling and
    the user defines the physics of white-box models.

    \subsection{White-box techniques}

        \paragraph 
        The underlying physics of a white-box model is usually determined from first principles.
        This is done by modelling physical processes with techniques such as Lagrangian or Newton mechanics.
        Hence, the mathematical relations between system parameters in the model are predefined in the modelling phase.
        The system identification process is therefore reduced to parameter estimation 
        which determines the best fit values of the system parameters.

        \paragraph 
        This approach is used by \cite{Erasmus2020} and \cite{Slabber2020} for system identification 
        for swing damping control of a multirotor with an unknown suspended payload.
        Recall from Chapter~\ref{chap:modelling} that the system was modelled as two connected rigid bodies with the following assumptions:
        \begin{itemize}
            \item The payload is a point-mass.
            \item The link is massless.
            \item The link is rigid.
            \item The link is attached to the \gls{CoM} of the multirotor.
        \end{itemize}
        The only unknown parameters in the multirotor and payload model is the payload mass and the link length.
        These parameters are first estimated and then inserted into the predefined, linearised model.
        This model is used by a \gls{LQR} controller to damp swing angles while also controlling the vehicle.

        \paragraph
        The main advantage of this approach is its simplicity.
        In the case considered by \cite{Erasmus2020} and \cite{Slabber2020}, only two parameters are estimated.
        In contrast, numerous values need to be estimated to reproduce the system dynamics with a black-box model.
        Therefore, white-box system identification methods are often less computationally complex
        and can easily be applied on low cost hardware.
        Due to the lower complexity, 
        parameter estimation algorithms often require shorter lengths of training data 
        than data-driven methods to produce accurate models.

        \paragraph 
        Therefore the white-box approach works well for systems with predictable physics, 
        however is not very adaptable to systems that deviate from the predefined dynamics.
        The payload considered by \cite{Erasmus2020} and \cite{Slabber2020} is limited to a small rigid mass 
        suspended from the multirotor by a non-stretching cable. 
        In this configuration it was shown that a \gls{LQR} controller 
        successfully controls a multirotor and minimises the payload swing angles.
        However, if a payload or cable is used that violates one of the modelling assumptions, 
        the predefined model no longer accurately represents the system.
        Many payloads considered for practical drone deliveries do not conform to these assumptions.
        Since the controller is dependent on this model, 
        the mismatch between the model and actual dynamics may result in undesirable controller behaviour.
        Therefore a new model and parameter estimation technique will need to be derived 
        for every use case that deviates significantly from the a~priori model.

    \subsection{Black-box techniques}

        \paragraph
        Data-driven system identification methods produce black-box models.
        These models do not require predefined mathematical relations between system parameters.
        % The user only considers what goes into, and comes out of, the black-box.
        % Something imagery about why it is called black-box
        No prior knowledge of the physics of the system are considered and no modelling assumptions are made.
        Black-box techniques determine the mathematical relationship between inputs and outputs of a system 
        using information from measurement data only.

        \paragraph
        A disadvantage of the data-driven system identification approach is its computational complexity.
        Data-driven algorithms generally have a much higher computational complexity than parameter estimation techniques.
        This is expected since a lot more model parameter values are generated 
        to populate a black-box model than a predefined white-box model.
        In the multirotor use case, this may mean that more expensive computational hardware is required 
        to implement a data-driven method compared to parameter estimation methods.
        Also, most data-driven methods have hyperparameters that affect the performance of the method 
        and need to be tuned for a specific use case.
        This can be done automatically, but this process increases the computational demand on the hardware.
        Furthermore, data-driven methods generally require more training data than parameter estimation methods.
        This means that more flight time is wasted on system identification before an updated controller can be activated.
        
        \paragraph
        However, black-box techniques are very adaptable 
        and provide a general system identification solution for a broad range of different dynamics.
        This is a major advantage over white-box system identification techniques, 
        which need to be manually redesigned for different use cases.
        % Add more advantages ??

        \paragraph
        Dynamical models can be categorised as either non-linear or linear models.
        Non-linear models are often more accurate than linear models 
        because real-world system mostly contain non-linear dynamics.
        The dynamics of a multirotor and suspended payload are also non-linear.
        % Examples of black-box models with multirotors and payloads in literature ???

        \paragraph
        However, non-linear models are inherently more complex than linear models. 
        Controllers based on non-linear models are usually more computationally complex than those with linear models.
        The control architectures used for multirotors in practical applications are mostly implemented on onboard hardware.
        Therefore there is value in low-complexity, linear models because these may be simple enough to execute on low cost hardware.
        % trade-off between accuracy and complexity.??
        Non-linear models may require control implementations that are too computationally expensive and may not be practically realisable on the available hardware on a multirotor.
        
        \paragraph
        \gls{DMDc} and \gls{HAVOKc} are the two data-driven system identification methods investigated in this work. 
        These are linear regression techniques that produce linear models that approximate non-linear dynamics.
        Non-linear data-driven techniques like Neural Networks and SINDy \cite{Brunton2016a} 
        may produce models that are more accurate than linear techniques.
        However the gain in accuracy will be at the cost of a greater computational complexity.
        % \murray{Name more techniques ??}
        \gls{DMDc} and \gls{HAVOKc} are less computationally complex 
        and their models are suitable for linear \gls{MPC}, 
        which is significantly faster than non-linear \gls{MPC}.
        This is desirable for a practical multirotor implementation, where onboard computational power is limited.

    % \subsection{Comparison of the techniques} ??
    \section{Plant considered for system identification} \label{sec:plant_considered}
    
    \begin{figure}[htb]
        \centering
        \includegraphics[width=0.45\linewidth]{floating_pend.png}            
        \caption{Floating pendulum model considered for system identification for a North velocity controller}
        \label{fig:floating_pend}
    \end{figure}

    \paragraph
    The considered LQR and MPC needs to control the North velocity of the quadrotor 
    and therefore has the input vector,
    \begin{equation}
        \bm{u} = \begin{bmatrix}
            A_{N,sp}
        \end{bmatrix} .
    \end{equation}
    The controllers require state feedback from the quadrotor and payload,
    hence the state vector of the considered plant is defined as,
    \begin{equation}
        \bm{x} = \begin{bmatrix}
            V_N & \theta & \dot{\theta}
        \end{bmatrix}^T .
    \end{equation}
    % This is the state vector of the considered plant, 
    % but the controllers apply an augmented state vector that will be discussed in Chapter~\ref{chap:control_systems}.
    Note that position is not included in the state vector because it does not affect the velocity controller.
    Also note that the inner loop controllers handle the attitude dynamics of the quadrotor.
    During pure longitudinal velocity setpoints the quadrotor experiences negligible altitude changes
    because of sufficient speed in the altitude controllers.
    The plant seen by the system identification process therefore mimics the common pendulum-on-a-cart model.
    A schematic of this 2D plant considered for system identification is shown in Figure \ref{fig:floating_pend}.
    In the following sections, simulations of the full quadrotor and payload system will be performed
    and different methods will be applied to identify models of this plant.
    % The differential equations that describe the motion of this system 
    % were derived with Lagrangian mechanics in Chapter~\ref{chap:modelling}. 

    % From this derivation it is clear that the angular velocity of the payload, $\dot{\theta}$, is required to described the system dynamics.
    % However, $\dot{\theta}$ is not measured directly on the considered practical quadrotor setup.
    % Instead, the payload angle, $\theta$, is measured by a potentiometer attached to a ADC on Honeybee as described in Chapter \ref{chap:system_overview}.
    % As expected, this measurement is extremely noisy.
    % \murray{Maybe insert figure to show noise}
    % % Figure \ref{} shows the angle measurement during a practical experiment of the payload while Honeybee is held stationary
    % Numerical differentiation is applied to the noisy $\theta$ signal which results in a very inaccurate estimation of $\dot{\theta}$.
    % Therefore it is desirable to rather use $\theta$ in the system identification process. 


    \section{Parameter estimation} \label{sec:param_estimation}

    \paragraph
    The purpose of parameter estimation is to determine unknown values 
    required by a predetermined, white-box model.
    The identified model is used to design a \gls{LQR} controller and
    therefore needs to be in a linear, continuous-time, state-space form.
    This model was derived and linearised a priori in Section~\ref{sec:linear_model}.
    The unknown parameters in the model include the payload mass, $m_p$ and the cable length, $l$.
    Two separate methods are used to estimate each of these parameters.
    This method was used by \cite{Erasmus2020} and \cite{Slabber2020} 
    to design a \gls{LQR} controller and implement swing damping control 
    of a multirotor with an unknown suspended payload.
    
\subsection{Payload mass estimation}

    \paragraph
    \gls{RLS} is used by \cite{Erasmus2020} and \cite{Slabber2020} to estimate the payload mass.
    It is assumed that the multirotor mass is known before a flight, 
    therefore the payload mass can be estimated from the additional trust required during hover.
    In both \cite{Erasmus2020} and \cite{Slabber2020} it is demonstrated 
    that \gls{RLS} is very accurate for a system nearly identical to the one considered in this work.
    To compare other aspects of the white-box and black-box techniques with more clarity, 
    it will be assumed that the method estimates $m_p$ with perfect accuracy.
    This isolates any inaccuracies in the white-box model to 
    either the a priori modelling or the cable length estimation.

    \paragraph
    It should be noted however that this method is dependant on the assumption 
    that the vehicle mass is known and remains unchanged.
    The method will clearly be inaccurate if an unknown mass is added to the multirotor 
    in conjunction with the suspended payload.
    Common practical examples of this include adding a camera to the vehicle or using a different battery for a different flight range.
    In these cases, the mass estimation method will have to be redesigned.
    This is an inherent problem of the white-box techniques.
    The parameter estimation methods are designed for specific modelling assumptions 
    and are not adaptable to different types of payload loadings.
    In contrast, data-driven techniques are adaptable to different payload loadings 
    because it does not depend on a priori modelling assumptions.

\subsection{Cable length estimation} \label{sec:length_estimation}

    \paragraph
    The cable length is estimated from the measurement of the natural frequency of the swinging payload.
    As described by \cite{Bisgaard2008}, the natural frequency is given by:
    \begin{equation} \label{eq:nat_freq}
        \omega_n = \sqrt{ \frac{g}{l} \cdot \frac{m_q + m_p}{m_q}} .
    \end{equation}
    The cable length can clearly be calculated from Equation~\ref{eq:nat_freq} if the other parameters are known.
    The natural frequency is measured by performing a \gls{FFT} on the payload swing angle response 
    after a velocity step by the multirotor.
    The dominant frequency identified by the \gls{FFT} during free swing is an approximate measurement
    of the natural frequency of the payload.
    Note that the measured frequency rather corresponds to the damped natural frequency.
    The swing angle damping is caused by 
    the velocity controller, 
    friction at the attachment of the cable to the drone,
    and air drag.
    However, it is assumed that the damping coefficient is small enough for
    the damped natural frequency to closely approximate the theoretical natural frequency.
    % This method is applied successfully by \cite{Erasmus2020} and \cite{Slabber2020} for cable length estimation.
    
    \paragraph
    Figure~\ref{fig:vel_step_single_pend} shows the payload swing angle response 
    to a position step setpoint.
    The first few seconds of the step response are excluded from the \gls{FFT} 
    to minimise the effect of the transient response and the multirotor controllers 
    on the natural frequency measurement.
    Figure~\ref{fig:FFT_vel_step} shows the resulting single-sided amplitude spectrum of the \gls{FFT} of this data.
    
    \paragraph
    The dominant frequency is clearly identified by the peak at \SI{0.520}{\radian/\second}.
    Since $m_q$, $m_p$ and $g$ are known, and $\omega_n$ has been measured, 
    $l$ can be determined from Equation~\ref{eq:nat_freq}.
    The estimated length for this simulation is \SI{0.953}{\metre}.
    The actual cable length is \SI{1}{\metre}, therefore this estimation has an error of 4.7\%.
    As documented by \cite{Erasmus2020} and \cite{Slabber2020}, an error of this magnitude is acceptably small
    and still results in effective control with a \gls{LQR}.
    It was also shown by \cite{Erasmus2020} and \cite{Slabber2020} that this estimation method 
    is effective for a range of different payloads in simulation.

    
\begin{figure}
    \captionsetup[subfigure]{justification=centering}
    \centering
    \begin{subfigure}[t]{0.45\columnwidth}
    \centering
    \begin{tikzpicture}
        \begin{axis}[            
            xlabel = Time,
            ylabel = Payload angle,
            x unit = \si{\second},
            y unit = \si{\radian},
            xmin = 0,   xmax = 15,
            ymin = -0.2,  ymax = 0.1,
            grid = major,
            legend cell align = left,
            legend pos = north east,
            grid style = dashed,
            legend style = {font = \scriptsize},
            label style = {font = \scriptsize},
            tick label style = {font = \scriptsize},
            width = 0.95\columnwidth,
            height = 0.95\columnwidth,
            % initialize Dark2
            cycle list/Dark2,
            % combine it with 'mark list*':
            cycle multiindex* list = {
                Dark2\nextlist
            }
        ]
        
        \pgfplotsset{cycle list shift=1}
        % \addplot+[mark = none, style = solid, ultra thick] 
        % table[x = time, y = theta, col sep = comma] 
        % {system_id/csv/pos_step_SITL_single_pos_step_m0.3_l1.csv.csv};

        \addplot+[mark = none, style = solid, ultra thick] 
        table[x = time, y = theta, col sep = comma] 
        {system_id/csv/single_step_predictions_SITL_x_vel_noise_l1_m0-3.csv.csv};

        \end{axis}
    \end{tikzpicture} 
    
    \caption{Position step response of the payload swing angle}
    \label{fig:vel_step_single_pend}
\end{subfigure}
 % subfigure
    \begin{subfigure}[t]{0.45\columnwidth}
    \centering
    \begin{tikzpicture}
        \begin{axis}[            
            xlabel = Frequency,
            ylabel = Amplitude,
            x unit = \si{\radian/\second},
            % y unit = \si{\second},
            xmin = 0.3,     xmax = 1,
            ymin = 0,  ymax = 0.006,
            grid = major,
            legend cell align = left,
            legend pos = north east,
            grid style = dashed,
            legend style = {font = \scriptsize},
            label style = {font = \scriptsize},
            tick label style = {font = \scriptsize},
            width = 0.95\columnwidth,
            height = 0.95\columnwidth,
            % initialize Dark2
            cycle list/Dark2,
            % combine it with 'mark list*':
            cycle multiindex* list = {
                Dark2\nextlist
            }
        ]

        \addplot+[mark = none, style = solid, ultra thick] 
        table[x = f, y = P1, col sep = comma] 
        {system_id/csv/FFT_SITL_single_pos_step_m0.3_l1.csv.csv};

        \end{axis}
    \end{tikzpicture} 
    
    \caption{The single-sided amplitude spectrum of the FFT}
    \label{fig:FFT_vel_step}
\end{subfigure}
 % subfigure
    \caption{Data from a velocity step response used for cable length estimation 
    ($l =$~\SI{1}{\metre}, $m_p =$~\SI{0.3}{\kilo\gram}.).}
    \label{fig:FFT_vel_step_subfigs}  
\end{figure}

    


    \section{Dynamic mode decomposition with control}
\label{sec:dmdc}
    
    \paragraph        
    DMD is a regression technique that can be used to approximate a non-linear dynamical system with a linear model \cite{Tu2014}.
    It uses temporal measurements of system outputs to reconstruct system dynamics without prior modelling assumptions.
    DMDc is an adaptation of DMD that also accounts for control inputs \cite{Proctor2016c}.
    This section provides an overview of the specific implementation of DMDc used in this work.        
    Note that this implementation is an adaptation of DMDc, and includes time-delay-embedding of multiple observables. 
    Enriching a DMD model with time-delay-embedding is a known technique and is also seen in other DMD adaptations \cite{Korda2018b, Arbabi2018}.

    \paragraph
    DMD produces a linear, discrete state-space model of system dynamics.
    Discrete measurements, $\bm{x}_k$, of the continuous time observable, $\bm{x}(t)$, are used, 
    where $\bm{x}_k = \bm{x}(k T_s)$, and $T_s$ is the sampling time of the model.    
    Delay-coordinates (i.e. $\bm{x}_{k-1}, \bm{x}_{k-2}$, etc.) are also included in the state-space model to
    account for input delay and state delay in the system.
    Input delay refers to the time delay involved with transporting a control signal to a system, 
    whereas state delay refers to time-separated interactions between system variables \cite{Chen1999}.
    Hence, we define an state delay vector as:
    \begin{equation}
        \bm{d}_{k} = 
        \begin{bmatrix}
            \bm{x}_{k-1} & \bm{x}_{k-2} & \cdots & \bm{x}_{k-q+1}
        \end{bmatrix}^T ,
    \end{equation}
    $\bm{d}_k \in \R^{(n_x)(q-1)}$ and where $q$ is the number of delay-coordinates 
    (including the current time-step) used in the model.
    
    \paragraph
    The discrete state-space model is therefore defined as:
    \begin{equation} \label{eq:dmd_state_space}
        \bm{x}_{k+1} = \bm{A} \bm{x}_k + \bm{A}_d \bm{d}_k + \bm{B} \bm{u}_k ,
    \end{equation}
    \( \bm{A} \in \R^{n_x \times n_x} \) is the system matrix, 
    \( \bm{A}_1 \in \R^{(q-1) \cdot n_x ~ \times ~ (q-1) \cdot n_x} \) is the state delay system matrix and 
    \( \bm{B} \in \R^{n_x \times n_u} \) is the input matrix.
    
    \paragraph
    The training data consists of full-state measurements, $\bm{x}_k$, and corresponding inputs, $\bm{u}_k$, 
    taken at regular intervals of $\Delta t = T_s$, during a simulated flight with Cascaded PID control.
    In a practical flight, these time-series measurements need to be saved in memory because it is usd as a single batch by DMD.
    Note that DMD can be applied in a recursive manner as described in \cite{Noack2016}, 
    However this implementation is not considered because memory size will not be a limitation since a companion computer will be used.
    
    \paragraph
    The training data is collected into the following matrices:
    \begin{align} \label{eq:dmd_matrices}
        \bm{X^\prime} = & \phantom{.} \left [
            \begin{array}{*{5}{@{}M{\mycolwd}@{}}}
                    \bm{x}_{q+1} & \bm{x}_{q+2} & \bm{x}_{q+3} & \cdots & \bm{x}_{w+q}
            \end{array}
        \right ] , \nonumber \\
        %
        \bm{X} \phantom{''} = & \phantom{.} \left [
            \begin{array}{*{5}{@{}M{\mycolwd}@{}}}
                \bm{x}_{q} & \bm{x}_{q+1} & \bm{x}_{q+2} & \cdots & \bm{x}_{w+q-1}                      
            \end{array}
        \right ] , \nonumber \\
        % 
        \bm{X}_d \phantom{'} = & \phantom{'} \left [
            \begin{array}{*{5}{@{}M{\mycolwd}@{}}}
                \bm{x}_{q-1} & \bm{x}_{q+0} & \bm{x}_{q+1} & \cdots & \bm{x}_{w+q-2} \\
                \vdots   & \vdots   & \vdots   & \ddots & \vdots \\
                \bm{x}_2 & \bm{x}_3 & \bm{x}_4 & \cdots & \bm{x}_{w+1} \\
                \bm{x}_1 & \bm{x}_2 & \bm{x}_3 & \cdots & \bm{x}_{w} \\                       
            \end{array}
        \right ] , \nonumber \\
        % 
        \bm{\Upsilon} \phantom{'} = & \phantom{.} \left [
            \begin{array}{*{5}{@{}M{\mycolwd}@{}}}
                    \bm{u}_{q} & \bm{u}_{q+1} & \bm{u}_{q+2} & \cdots & \bm{u}_{w+q-1}
            \end{array}
        \right ] ,
    \end{align}
    where $w$ is the number of columns in the matrices, 
    $\bm{X^\prime}$ is the matrix $\bm{X}$ shifted forward by one time-step, 
    $\bm{X}_d$ is the matrix with delay states, 
    and $\bm{\Upsilon}$ is the matrix of inputs.
    Equation (\ref{eq:dmd_state_space}) can be combined with the matrices in Equation (\ref{eq:dmd_matrices}) to produce:
    \begin{equation}
        \bm{X^\prime} = \bm{A} \bm{X} + \bm{A}_d \bm{X}_d + \bm{B} \bm{\Upsilon} .
    \end{equation}
    Note that the primary objective of DMDc is to determine the best fit model matrices, $\bm{A}$, $\bm{A}_d$ and $\bm{B}$, 
    given the data in $\bm{X^\prime}$, $\bm{X}$, $\bm{X}_d$, and $\bm{\Upsilon}$ \cite{Proctor2016c}.
    In order to group the unknowns into a single matrix, (\ref{eq:dmd_state_space}) is manipulated into the form,
    \begin{equation} \label{eq:G_Omega}
        \bm{X^\prime} =   
        \begin{bmatrix} 
            \bm{A} & \bm{A}_d & \bm{B} 
        \end{bmatrix}
        \begin{bmatrix} 
            \bm{X} \\ \bm{X}_d \\ \bm{\Upsilon} 
        \end{bmatrix} 
        = \bm{G \Omega} ,
    \end{equation} 
    where $\bm{\Omega}$ contains the state and control data, and $\bm{G}$ represents the system and input matrices.
    
    \paragraph
    A SVD is performed on $\bm{\Omega}$ resulting in:
    \(
        \bm{\Omega} = \bm{U} \bm{\Sigma} \bm{V}^T
    \).
    Often, only the first $p$ columns of $\bm{U}$ and $\bm{V}$ are required for a good approximation of the dynamics \cite{Brunton2017a}.
    % Talk about Reduced Order Modelling??
    % POD modes??
    In many cases, the truncated form results in better models than the exact form when noisy measurements are used.
    This is because the effect of measurement noise is mostly captured by the truncated columns of $\bm{U}$ and $\bm{V}$.
    By truncating these columns, the influence of noise in the regression problem is reduced. 
    % \murray{explain this better ??}
    Hence the SVD is used in the truncated form: 
    \begin{equation} \label{eq:tilde_svd}
        \bm{\Omega} \approx \Tilde{\bm{U}} \Tilde{\bm{\Sigma}} \Tilde{\bm{V}}^T ,
    \end{equation}
    where $\phantom{.} \Tilde{ } \phantom{.}$ represents rank-$p$ truncation.
    % \murray{maybe insert colour pictures showing matrices ??}        

    \paragraph
    By combining (\ref{eq:tilde_svd}) with the over-constrained equality in (\ref{eq:G_Omega}), 
    the least-squared solution, $\bm{G}$, can be found with:
    \begin{equation}
        \bm{G} \approx \bm{X^\prime} \Tilde{\bm{V}} \Tilde{\bm{\Sigma}}^{-1} \Tilde{\bm{U}} .
    \end{equation}
    By reversing \ref{eq:G_Omega}, $\bm{G}$ can now be separated into:
    \(
        \bm{G} = \begin{bmatrix} \bm{A} & \bm{A}_d & \bm{B} \end{bmatrix}.
    \)
    according to the required dimensions of each matrix.
    Thereby, the state-space model approximated by DMDc is complete.
    

    \section{Hankel alternative view of Koopman} 
\label{sec:havok}
    
    \murray{q = number of delays, from here up}
    
    \paragraph{}
    HAVOK is a data-driven, regression technique that provides a connection between DMD and Koopman operator theory \cite{Brunton2017a, Champion2019}. 
    We have adapted the standard HAVOK algorithm slightly to account for the effect of control and to extract a discrete, linear model that approximates the behaviour of a controlled dynamical system.
    In this section, a brief overview is provided for this implementation and expansion of \mbox{HAVOK}.
    % 
    \paragraph{}
    The extracted discrete state-space model is defined as:
    \begin{equation} \label{eq:havoc_state_space}
        \bm{a}_{k+1} = \Tilde{\bm{A}} \bm{a}_k + \Tilde{\bm{B}} \bm{u}_k ,
    \end{equation}
    where $\bm{a}_k$ is the state vector previously defined in Section \ref{sec:dmdc}, 
    \( \Tilde{\bm{A}} \in \R^{(q \cdot n_x) \times (q \cdot n_x)} \) is the system matrix, 
    and \( \Tilde{\bm{B}} \in \R^{(q \cdot n_x) \times n_u} \) is the input matrix. 
    Here, $\Tilde{\phantom{a}}$ is used to differentiate these matrices from $\bm{A}$ and $\bm{B}$ used in DMDc.
    % 
    \paragraph{}
    The original HAVOK algorithm, developed by \cite{Brunton2017}, constructs a Hankel matrix from output variables only. 
    In order to incorporate the effect of control, an extended Hankel matrix, $\bm{\Pi}$, is created by appending a matrix of inputs to a Hankel matrix of measurements:
    \begin{equation} \label{eq:pi_hankel}
        \bm{\Pi} = \phantom{.} \left [
            \begin{array}{*{5}{@{}M{\mycolwd}@{}}}
                    \bm{a}_{q} & \bm{a}_{q+1} & \bm{a}_{q+2} & \cdots & \bm{a}_{w+q-1} \\
                    \bm{u}_{q} & \bm{u}_{q+1} & \bm{u}_{q+2} & \cdots & \bm{u}_{w+q-1}
            \end{array}
        \right ] ,
    \end{equation}
    where $w$ is the number of columns in $\bm{\Pi}$.
    A truncated SVD of this Hankel matrix results in following approximation:
    \begin{equation} \label{eq:havok_svd_tilde}
        \bm{\Pi} \approx \Tilde{\bm{U}} \Tilde{\bm{\Sigma}} \Tilde{\bm{V}}^T ,
    \end{equation}
    where $\Tilde{\phantom{a}}$ represents rank-$p$ truncation.
    It is important to note that the model extracted by HAVOK depends on the choice of hyperparameters, $p$ and $q$.
    The number of samples in the training data, $N_{train} = w + q -1$, also influences the accuracy of the model.
    % 
    \paragraph{}
    The columns of $\Tilde{\bm{V}}$ are the most significant principal components of the system dynamics \cite{Kamb2020}.
    This matrix, $\Tilde{\bm{V}}$, can be considered to contain a time-series of the pseudo-state, $\bm{v}$, such that
    \(
        \Tilde{\bm{V}}^T = \begin{bmatrix} 
            \bm{v}_q & \bm{v}_{q+1} & \cdots & \bm{v}_w 
        \end{bmatrix} ,
    \)
    characterises the evolution of the actual dynamics in an eigen-time-delay coordinate system \cite{Brunton2017}.
    % 
    Consider the following discrete, state-space formulation:
    \begin{equation} \label{eq:v_ss}
        \bm{v}_{k+1} = \bm{\Lambda} \bm{v}_k .
    \end{equation}
    Recall that DMDc finds a best fit linear operator that directly maps $\bm{a}_{k}$ to $\bm{a}_{k+1}$.
    Similarly, HAVOK determines the best fit linear operator $\bm{\Lambda}$ that maps the pseudo-state $\bm{v}_k$ to $\bm{v}_{k+1}$.
    So, in order to setup an over-determined equality for (\ref{eq:v_ss}), $\Tilde{\bm{V}}^T$ is divided into two matrices:
    \begin{align} \label{eq:v1v2}
        \bm{V}_1 &= \left [
            \begin{array}{*{5}{@{}M{\mycolwd}@{}}} 
                \bm{v}_{q \phantom{-1}}     & \bm{v}_{q+1} & ... & \bm{v}_{w-1} \\
            \end{array} 
        \right ] , \nonumber \\ 
        \bm{V}_2 &= \left [
            \begin{array}{*{5}{@{}M{\mycolwd}@{}}} 
                \bm{v}_{q+1}     & \bm{v}_{q+2} & ... & \bm{v}_{w \phantom{-1}} \\
            \end{array} 
        \right ] ,
    \end{align} 
    where $\bm{V}_2$ is $\bm{V}_1$ advanced a single step forward in time.
    The matrices from Equation (\ref{eq:v1v2}) are now combined with Equation (\ref{eq:v_ss}) and the best fit $\bm{\Lambda}$ is determined with the Moore-Penrose pseudoinverse:
    \begin{equation} \label{eq:v_dmd}
        \bm{V}_2 = \bm{\Lambda} \bm{V}_1 \phantom{---} \Rightarrow \phantom{---} \bm{\Lambda} \approx \bm{V}_1 \bm{V}_1^{\dagger}
    \end{equation}
    % 
    It can be shown from Equation (\ref{eq:havok_svd_tilde}) that Equation (\ref{eq:v_ss}) is transformed from the eigen-time-delay coordinate system to the original coordinate system as the following:
    \begin{equation} \label{eq:v_ss_a} 
        \begin{bmatrix}
            \bm{a}_{k+1}  \\  \bm{u}_{k+1} 
        \end{bmatrix}
    \phantom{.} = \phantom{.} (\Tilde{\bm{U}} \Tilde{\bm{\Sigma}}) \bm{\Lambda} (\Tilde{\bm{U}}  \Tilde{\bm{\Sigma}})^{\dagger} \phantom{.}
        \begin{bmatrix}
            \bm{a}_{k}  \\  \bm{u}_{k} 
        \end{bmatrix} .
    \end{equation}    
    % 
    This form is used to extract $\Tilde{\bm{A}}$ and $\Tilde{\bm{B}}$ from the matrix,
    \( 
        (\Tilde{\bm{U}} \Tilde{\bm{\Sigma}}) \bm{\Lambda} (\Tilde{\bm{U}}  \Tilde{\bm{\Sigma}})^{\dagger}
    \), in the following way:
    \begin{equation} \label{matrix_decomp}
        \begin{bmatrix}
            \bm{a}_{k+1}  \\  \bm{u}_{k+1} 
        \end{bmatrix}
        \phantom{.} = \phantom{.} 
        \begin{bmatrix}
            \Tilde{\bm{A}} \phantom{.....} \Tilde{\bm{B}} \\
            \textit{(discarded)}
        \end{bmatrix}
        \phantom{.}
        \begin{bmatrix}
            \bm{a}_{k}  \\  \bm{u}_{k} 
        \end{bmatrix}.
    \end{equation}    
    % 
    % This decomposition is illustrated in Fig.~\ref{fig:lambda_decomp}, where blocks represent different groups of entries in the matrix.
    % \begin{figure}[h]
    %     \includegraphics[scale = 0.45]{Lambda_decomp.png}
    %     \centering
    %     \caption{Illustration of the extraction of $\Tilde{\bm{A}}$ and $\Tilde{\bm{B}}$ from (\ref{eq:v_ss_a})}
    %     \label{fig:lambda_decomp}
    % \end{figure}
    % 
    Note that the matrix entries in (\ref{matrix_decomp}) that map $\bm{u}_k$ to $\bm{u}_{k+1}$ are meaningless for our purposes and are discarded.
    Similarly to DMDc, some matrix entries in $\Tilde{\bm{A}}$ and $\Tilde{\bm{B}}$ are known a priori due to the relative positions of delay coordinates. These are forced to 1 or 0 to improve the prediction performance of the model.
    
    \murray{merge these paragraphs}
    Since the state vector, $\bm{a}$, includes delay-coordinates, some matrix entries are known a priori and are independent of the dynamics. 
    For example, the values of $\bm{x}_{k}$ should be mapped from their position in $\bm{a}_k$ to specific indices in $\bm{a}_{k+1}$. 
    Due to the least-squares fitting and coordinate transformation, DMDc will not produce these exact values in $\bm{A}$ and $\bm{B}$. 
    By forcing each of these matrix entries to 1 or 0, the state-prediction performance of the model is improved.





    \section{Implementation and results}
    \subsection{Methodology}
        \paragraph{Simulation environment}

        \paragraph{Method overview}
        \murray{Maybe convert this to a flow diagram}

        \begin{enumerate}
            \item Takeoff and hover
            \item Command a series of velocity step inputs with random step sizes and time intervals
            \item Measure and save input and output data
            \item Apply algorithm to data and generate model
        \end{enumerate}

        \paragraph{}

        \paragraph{Steps and intervals}
        For the training period, different velocity step inputs are commanded with varying time intervals between step commands.
        A algorithm schedules these velocity step commands, by assigning random step values and time-intervals within a specified range.
        The velocity range is determined in simulation by iteratively increasing the maximum velocity step 
        to a safe value where the quadrotor and payload system remain in stable flight.
        The maximum time-interval is set to a value that allows the payload swing to reach a steady-state condition.
        This ensures that the identified model includes transient and steady-state dynamics.

        \paragraph{Why velocity steps?}
        Velocity step commands are used in the training period because this 
        Frequency decomposition stimulates the system for a large range of frequencies.

        \paragraph{Testing data}
        Cross-validate

        \paragraph{Error metric}
        Each state error signal is scaled by the reciprocal of the maximum value of that state variable in the training data.
        % This is to provide a better representative error when taking the mean of state variable errors.
        This is to ensure that a scale difference in the variable types create a bias in the error metric.
        For example, the quadrotor velocity reaches values of \SI{3}{\metre/\second} but the payload swing angle has a maximum of only \SI[]{0.526}{\radian}.
        The velocity prediction error is therefore inherently larger than the payload angle prediction error
        and will bias the error metric towards favouring models with good velocity predictions.
        The proposed scaled error metric ensures that the MAE of each state variable can be compared to each other.
        It also provides an error metric that is better and unbiased representative of the model prediction performance across all state variables. 

        Add Information Criteria ??
        AIC Brunton Kutz??

    \subsection{Hyperparameters}
        As discussed in Section~\ref{sec:dmdc} and \ref{sec:havok} 
        DMDc and HAVOK are dependent on two hyperparameters: the number of delay-coordinates, $q$, and the SVD truncation rank, $p$.

        Parsimony
        Pareto front
        cite Data-Driven book

        The more terms better chance to overfit, lower generalisation
        \begin{figure}[htb]
    \centering
    \begin{tikzpicture}
        \begin{axis}[            
            xlabel = {Number of delay-coordinates, $q$},
            ylabel = $\overline{NMAE}$ \phantom{~},
            % x unit = \si{\second},
            y unit = \%,
            xmin = 2,     xmax = 40,
            ymin = 3.2,  ymax = 6.5,
            grid = major,
            legend cell align = left,
            legend pos = north east,
            grid style = dashed,
            legend style = {font = \scriptsize},
            label style = {font = \scriptsize},
            tick label style = {font = \scriptsize},
            width = 0.95\columnwidth,
            height = 0.5\columnwidth,
            % initialize Dark2
            cycle list/Dark2,
            % combine it with 'mark list*':
            cycle multiindex* list = {
                Dark2\nextlist
            }
        ]

        \addplot+[mark = none, style = solid, ultra thick] 
        table[x = q, y expr = {\thisrow{NMAE_mean}*100}, col sep = comma] 
        {system_id/csv/NMAE_vs_q_SITL_x_vel_noise_longer_times_q.csv_dmd_angle_Ttrain_60.csv};
        \addlegendentry{DMDc}
        
        \addplot+[mark = none, style = solid, ultra thick] 
        table[x = q, y expr = {\thisrow{NMAE_mean}*100}, col sep = comma] 
        {system_id/csv/NMAE_vs_q_SITL_x_vel_noise_longer_times_q.csv_havok_angle_Ttrain_60.csv};
        \addlegendentry{HAVOK}

        \end{axis}
    \end{tikzpicture} 
    
    \caption{\gls{DMDc} and \gls{HAVOKc} predictions error for different lengths of noisy training data
    ($m_p =$~\SI{0.2}{\kilo\gram}, $l =$~\SI{0.5}{\meter}, $T_s =$~\SI{0.03}{\second}, $T_{train} =$~\SI{60}{\second}.).}
    \label{fig:NMAE_vs_q}
\end{figure}


        % \begin{figure}[htb]
    \centering
    \begin{tikzpicture}
        \begin{semilogyaxis}[            
            xlabel = Index of mode,
            ylabel = Singular value,
            % x unit = \si{\second},
            % y unit = \si{\second},
            xmin = 0,     xmax = 60,
            ymin = 1e-2,  ymax = 1e3,
            grid = major,
            legend cell align = left,
            legend pos = north east,
            grid style = dashed,
            legend style = {font = \scriptsize},
            label style = {font = \scriptsize},
            tick label style = {font = \scriptsize},
            width = 0.95\columnwidth,
            height = 0.5\columnwidth,
            % initialize Dark2
            cycle list/Dark2,
            % combine it with 'mark list*':
            cycle multiindex* list = {
                Dark2\nextlist
            }
        ]

        \addplot+[only marks, mark = square, ultra thick] 
        table[x = index, y = S, col sep = comma] 
        {system_id/csv/Singular_values_SITL_x_vel_noise_longer_times_q.csv_havok_angle_Ttrain_60_q29_p13.csv};
        \addlegendentry{Significant modes}

        \addplot+[only marks, mark = square, ultra thick] 
        table[x = index, y = S, col sep = comma] 
        {system_id/csv/Singular_values_SITL_x_vel_noise_longer_times_q.csv_havok_angle_Ttrain_60_q29_p13_trunc.csv};
        \addlegendentry{Truncated modes}

        \end{semilogyaxis}
    \end{tikzpicture} 
    
    \caption{Significant and truncated singular values of a \gls{HAVOK} model produced from noisy data
    ($m_p =$~\SI{0.2}{\kilo\gram}, $l =$~\SI{0.5}{\meter}, $T_s =$~\SI{0.03}{\second}, $T_{train} =$~\SI{60}{\second}.)}
    \label{fig:singular_values}
\end{figure}


        Fixed size of data
        Fixed sample time
        Fixed pendulum params
        Talk about the "front"
        Also about singular values
        For each of the experiments shown in this chapter, a hyperparameters selected tuned to produced

    \subsection{Sample time}
        The sample time, $T_s$, used for system identification sets the sample time of the discrete model, 
        which determines the sample time of the MPC.
        Resampling strategies can enable the MPC to run at a different frequency to the discrete model but this adds unnecessary complexity to the control architecture.
        
        \paragraph{}
        The MPC acts in the velocity loop and commands an acceleration setpoint.
        The default PID velocity controller runs at \SI{50}{\hertz} which corresponds to $T_s =~$\SI{0.02}{\second}.
        Due to the computational complexity of an MPC, the optimiser will struggle to run at \SI{50}{\hertz} on a quadrotor companion computer.
        \begin{figure}[h]
    \centering
    \begin{tikzpicture}
        \begin{axis}[            
            xlabel = Ts,
            ylabel = NMAE\textsubscript{mm},
            % x unit = \si{\second},
            % y unit = \si{\second},
            xmin = 0.02,  xmax = 0.05,
            ymin = 0.03,  ymax = 0.055,
            grid = major,
            legend cell align = left,
            legend pos = north east,
            grid style = dashed,
            legend style = {font = \scriptsize},
            label style = {font = \scriptsize},
            tick label style = {font = \scriptsize},
            width = 0.45\columnwidth,
            height = 0.5\columnwidth,
            % initialize Dark2
            cycle list/Dark2,
            % combine it with 'mark list*':
            cycle multiindex* list = {
                Dark2\nextlist
            }
        ]

        % \addplot+[mark = none, style = solid, ultra thick] 
        % table[x = Ts, y = NMAE_mean, col sep = comma] 
        % {system_id/csv/NMAE_vs_Ts_SITL_x_vel_noise_l0-25_m0-2.csv_dmd_angle.csv};
        % \addlegendentry{$l =$~\SI{0.25}{\metre}}
        
        \addplot+[mark = none, style = solid, ultra thick] 
        table[x = Ts, y = NMAE_mean, col sep = comma] 
        {system_id/csv/NMAE_vs_Ts_SITL_x_vel_noise_l0-5_m0-2.csv_dmd_angle.csv};
        \addlegendentry{$l =$~\SI{0.5}{\metre}}
        
        \addplot+[mark = none, style = solid, ultra thick] 
        table[x = Ts, y = NMAE_mean, col sep = comma] 
        {system_id/csv/NMAE_vs_Ts_SITL_x_vel_noise_l1_m0-2.csv_dmd_angle.csv};
        \addlegendentry{$l =$~\SI{1}{\metre}}
        
        \addplot+[mark = none, style = solid, ultra thick] 
        table[x = Ts, y = NMAE_mean, col sep = comma] 
        {system_id/csv/NMAE_vs_Ts_SITL_x_vel_noise_l2_m0-2.csv_dmd_angle.csv};
        \addlegendentry{$l =$~\SI{2}{\metre}}
        
        \end{axis}
    \end{tikzpicture} 
    
    \caption{DMD prediction error using different cable lengths with a range of different sample times of noisy training data
    ($m =$~\SI{0.2}{\kilo\gram})}
    \label{fig:MAE_vs_Ts_vs_L}
\end{figure}

        This is because the  
        % \begin{figure}[htb]
    \centering
    \begin{tikzpicture}
        \begin{axis}[            
            xlabel = Ts,
            ylabel = $\overline{NMAE}$ \phantom{~},
            % x unit = \si{\second},
            y unit = \%,
            xmin = 0,     xmax = 0.07,
            ymin = 8.5,  ymax = 12.5,
            grid = major,
            legend cell align = left,
            legend pos = north east,
            grid style = dashed,
            legend style = {font = \scriptsize},
            label style = {font = \scriptsize},
            tick label style = {font = \scriptsize},
            width = 0.95\columnwidth,
            height = 0.5\columnwidth,
            % initialize Dark2
            cycle list/Dark2,
            % combine it with 'mark list*':
            cycle multiindex* list = {
                Dark2\nextlist
            }
        ]

        \addplot+[mark = none, style = solid, ultra thick] 
        table[x = Ts, y expr = {\thisrow{NMAE_mean}*100}, col sep = comma] 
        {system_id/csv/NMAE_vs_Ts_SITL_x_vel_noise_longer_times_Ts.csv_dmd_angle.csv};
        \addlegendentry{DMDc}
        
        \addplot+[mark = none, style = solid, ultra thick] 
        table[x = Ts, y expr = {\thisrow{NMAE_mean}*100}, col sep = comma] 
        {system_id/csv/NMAE_vs_Ts_SITL_x_vel_noise_longer_times_Ts.csv_havok_angle.csv};
        \addlegendentry{HAVOK}

        \end{axis}
    \end{tikzpicture} 
    
    \caption{\gls{DMDc} and \gls{HAVOK} predictions error for different sample times of noisy training data
    ($m_p =$~\SI{0.2}{\kilo\gram}, $l =$~\SI{0.5}{\meter}, $T_{train} =$~\SI{60}{\second}.)}
    % \label{fig:havok_vs_dmd_noise}
\end{figure}


    \subsection{Choice of payload variable in the state vector}
        As discussed in Section~\ref{sec:plant_considered}, 
        the equations of motion of a floating pendulum in continuous-time are dependent on $\dot{\theta}$ and $V_N$, 
        but are not dependent on $\theta$.
        Therefore it is expected that 
        $
        \bm{x} = \begin{bmatrix}
            V_N & \dot{\theta}
        \end{bmatrix}^T
        $
        is used as the state vector for system identification.
        However, if $\dot{\theta}$ is not included in the state vector of a discrete model, 
        it can still be represented with numerical differentiation like the backward Euler form,
        \begin{equation}
            \dot{\theta}_k = (\frac{1}{T_s}) \cdot \theta_k - (\frac{1}{T_s}) \cdot \theta_{k-1} .
        \end{equation}
        Therefore the original state vector can also be replaced by,
        $
            \bm{x} = \begin{bmatrix}
                V_N & \theta
            \end{bmatrix}^T
        $
        in system identification.

        \paragraph{}
        Based on the floating pendulum equations, it is expected that a model derived with $\dot{\theta}$ data 
        will better approximate the actual dynamics than one using $\theta$.
        This is because $\dot{\theta}$ contains more direct information about the dynamics compared to $\theta$.
        A model using $\theta$ needs to "learn" numerical differentiation and the effect of $\dot{\theta}$ on the other variables.
        A model using $\dot{\theta}$ only needs to consider its relationship with other variables.

        \input{system_id/plots/MAE_vs_train_Simulink_angular_rate.tex}
            \murray{Add plot of using thetaand dtheta??}
            \murray{Redo this plot with SITL noise}
        \paragraph{}
        % Figure~\ref{} shows the prediction error of techniques using $\dot{\theta}$ or $\theta$ for different amounts of training data.
        For each length of training data, the hyperparameter combination producing the lowest prediction error was determined and used.
        From this plot it is clear that models with $\theta$ produce more accurate predictions than those with $\dot{\theta}$.

    \subsection{Noise}
        \paragraph{}
        Measurement noise is \murray{Find reference for measurement noise definition}
        This is bad for system identification because the output signals no longer represent the actual process
        hides the actual dynamics of the system under  
        The IMU, barometer, magnetometer and GPS sensors on the practical quadrotor are used for state estimation 
        and all experience measurement noise.
        The EKF performs sensor fusion and smooths out most of the measurement noise to provide a state estimate that is less noisy than raw sensor values.
        
        \paragraph{}
        The potentiometer and ADC which measure the payload angle on the quadrotor alos has quite a lot of measurement noise.
        However, this signal is not smoothed by an onboard EKF.
        Figure~\ref{fig:payload_noise} shows the noisy payload angle measurement for a practical pendulum test while the quadrotor is held stationary.
        For models using $\theta$ in the state vector instead of $\dot{\theta}$, 
        this noisy signal can be smoothed with \murray{matlab smoother}.
        Figure~\ref{fig:payload_noise_smoothed} compares the noisy payload angle measurement to the smoothed signal and actual payload angle for a simulated flight.
        The is applied as band-limited white-noise and the noise power was iteratively adjusted to match that of the practical payload measurements.
        
        \paragraph{}
        However, since there is no direct measurement of $\dot{\theta}$, 
        numerical differentiation is performed on the noisy $\theta$ measurement to estimate $\dot{\theta}$. 
        This amplifies the noise and results in inaccurate $\dot{\theta}$ signal.
        Total variation differentiation is implemented to estimate $\dot{\theta}$ from the noisy measurements more accurately. \cite{}
        Figure~\ref{fig:payload_noise_diff} shows
        
        % \input{system_id/plots/payload_noise_diff.tex} // With TVDiff

        Noise also affects model prediction accuracy and the length of training data required for adequate predictions. 
        
        \begin{figure}[h]
    \centering
    \begin{tikzpicture}
        \begin{axis}[            
            xlabel = Length of training data,
            ylabel = MAE,
            x unit = \si{\second},
            % y unit = \si{\second},
            xmin = 0,     xmax = 120,
            ymin = 0.085,  ymax = 0.12,
            grid = major,
            legend cell align = left,
            legend pos = north east,
            grid style = dashed,
            legend style = {font = \scriptsize},
            label style = {font = \scriptsize},
            tick label style = {font = \scriptsize},
            width = 0.95\columnwidth,
            height = 0.5\columnwidth,
            % initialize Dark2
            cycle list/Dark2,
            % combine it with 'mark list*':
            cycle multiindex* list = {
                Dark2\nextlist
            }
        ]

        \addplot+[mark = none, style = solid, ultra thick] 
        table[x = T_train, y = MAE_mean, col sep = comma] 
        {system_id/csv/MAE_vs_Ntrain_SITL_x_vel_no_noise_longer_times.csv_havok_angle.csv};
        \addlegendentry{Without noise}

        \addplot+[mark = none, style = solid, ultra thick] 
        table[x = T_train, y = MAE_mean, col sep = comma] 
        {system_id/csv/MAE_vs_Ntrain_SITL_x_vel_noise_longer_times.csv_havok_angle.csv};
        \addlegendentry{With noise}

        \end{axis}
    \end{tikzpicture} 
    
    \caption{HAVOK prediction error for different lengths of training data with and without noise 
    ($m =$~\SI{0.2}{\kilo\gram}, $l =$~\SI{0.5}{\meter}, $T_s =$~\SI{0.03}{\second}).}
    \label{fig:noise_vs_no_noise}
\end{figure}


        \begin{figure}[h]
    \centering
    \begin{tikzpicture}
        \begin{axis}[            
            xlabel = Length of training data,
            ylabel = MAE,
            x unit = \si{\second},
            % y unit = \si{\second},
            xmin = 0,     xmax = 120,
            ymin = 0.085,  ymax = 0.12,
            grid = major,
            legend cell align = left,
            legend pos = north east,
            grid style = dashed,
            legend style = {font = \scriptsize},
            label style = {font = \scriptsize},
            tick label style = {font = \scriptsize},
            width = 0.95\columnwidth,
            height = 0.5\columnwidth,
            % initialize Dark2
            cycle list/Dark2,
            % combine it with 'mark list*':
            cycle multiindex* list = {
                Dark2\nextlist
            }
        ]

        \addplot+[mark = none, style = solid, ultra thick] 
        table[x = T_train, y = MAE_mean, col sep = comma] 
        {system_id/csv/NMAE_vs_Ntrain_SITL_x_vel_noise_longer_times.csv_dmd_angle.csv};
        \addlegendentry{DMD}

        \addplot+[mark = none, style = solid, ultra thick] 
        table[x = T_train, y = MAE_mean, col sep = comma] 
        {system_id/csv/NMAE_vs_Ntrain_SITL_x_vel_noise_longer_times.csv_havok_angle.csv};
        \addlegendentry{HAVOK}

        \end{axis}
    \end{tikzpicture} 
    
    \caption{DMD and HAVOK prediction error for different lengths of noisy training data
    ($m =$~\SI{0.2}{\kilo\gram}, $l =$~\SI{0.5}{\meter}, $T_s =$~\SI{0.03}{\second}).}
    \label{fig:havok_vs_dmd_noise}
\end{figure}


        HAVOK performs better than DMD.
        This slight difference in prediciton performance has a negligible effect on control.
        

        Input data needs to be adjusted.

        % \input{plot of SITL acc_sp}
    

    \subsection{Size of training data}
        The length of training data used for system identification affects the quality of the model produced.
        In Figure~\ref{fig:SITL_MAE_vs_train_angular_rate} it is clear that prediction error decreases as the amount of training data increases.
        As more training data is used in the regression problem, 
        the determined model better approximates the actual dynamics because a large range of the dynamics is "seen" by the algorithm.
        
        % \input{low data prediction}
        \paragraph{}
        Models produced from data lengths as short as \SI{5}{\second} predict the movement of state variables surprisingly well.
        % Figure~\ref{} shows prediction.
        Note how the general shape of the prediction represents the training data, 
        even though it contains a lot more high frequency oscillations.

        \begin{figure}[htb]
    \centering
    \begin{tikzpicture}
        \begin{axis}[            
            xlabel = Length of training data,
            ylabel = NMAE of time derivative of predictions,
            x unit = \si{\second},
            % y unit = \si{\second},
            xmin = 0,     xmax = 120,
            ymin = 0.004,  ymax = 0.016,
            grid = major,
            legend cell align = left,
            legend pos = north east,
            grid style = dashed,
            legend style = {font = \scriptsize},
            label style = {font = \scriptsize},
            tick label style = {font = \scriptsize},
            width = 0.95\columnwidth,
            height = 0.5\columnwidth,
            % initialize Dark2
            cycle list/Dark2,
            % combine it with 'mark list*':
            cycle multiindex* list = {
                Dark2\nextlist
            }
        ]

        \addplot+[mark = none, style = solid, ultra thick] 
        table[x = T_train, y expr = {\thisrow{NMAE_mean}*100}, col sep = comma] 
        {system_id/csv/NMAE_vs_Ntrain_SITL_x_vel_noise_longer_times_MAEdiff.csv_dmd_angle.csv};
        \addlegendentry{DMDc}
        
        \addplot+[mark = none, style = solid, ultra thick] 
        table[x = T_train, y expr = {\thisrow{NMAE_mean}*100}, col sep = comma] 
        {system_id/csv/NMAE_vs_Ntrain_SITL_x_vel_noise_longer_times_MAEdiff.csv_havok_angle.csv};
        \addlegendentry{HAVOK}

        \end{axis}
    \end{tikzpicture} 
    
    \caption{\gls{DMDc} and \gls{HAVOKc} error of time derivative of predictions for different lengths of noisy training data
    ($m_p =$~\SI{0.2}{\kilo\gram}, $l =$~\SI{0.5}{\meter}, $T_s =$~\SI{0.03}{\second}).}
    % \label{fig:havok_vs_dmd_noise}
\end{figure}


        \paragraph{}
        % Figure~\ref{} shows the MAE of prediction state derivative.
        Define MAE diff with equation ??.
        % From Figure~\ref{} it appears that at least \SI{}{} training data is required to produce models that represent the dynamics.

        \paragraph{}
        The models produced from HAVOK appear to produce slightly better prediction errors, however this small difference has a negligible effect on control performance.

        \paragraph{}
        % In Figure~\ref{fig:MAE_vs_train} it can be seen that after approximately \SI{??}{\second} 
        the prediction error does not significantly improve with more training data.
        It practice less training data is desirable because less flight time will be wasted on training a model before the quadrotor can fly with a updated controller.
        Less training data also corresponds to lower memory usage on quadrotor hardware.
        Such a slight improvement in prediction error also has a negligible effect on control performance and is therefore not worth the increased data requirement.
        % Therefore, only \SI{??}{\second} of flight data will be used to train system identification models. 


    \subsection{System parameters}
            
        Works across a range of parameters.

        \paragraph{}
        The payload acting as a single floating pendulum, as described in Section~\ref{sec:plant_considered},
        has two system parameters, $m_p$ and $l$.
        For the practical quadrotor considered, the payload mass is limited to:
        \begin{equation}
            0.01 \leq m_p \leq \SI{0.4}{\kilo\gram} .
        \end{equation}
        When no external payload is attached, the connection device attached to the end of the cable is 
        $m_p = \SI{0.01}{\kilo\gram}$.
        % It becomes unsafe to Flying without a cable attached to the cable, or with a payload with a very small mass, 
        % may become unsafe since the cable may not always  kept taut by the mass. 
        On the other limit, $m_p = \SI{0.4}{\kilo\gram}$ is determined to be the maximum payload mass the quadrotor can carry safely 
        based on the maximum thrust of the motors.

        \paragraph{}
        The cable length is limited to:
        \begin{equation}
            0.5 \leq l \leq \SI{2}{\metre} .
        \end{equation}
        A cable length shorter than \SI{0.5}{\metre} is quite impractical and may rather be attached as a rigid payload.
        There are very few practical applications that may require a shorter cable length.
        It is also unsafe to fly with a shorter cable length, 
        since the payload may collide with the quadrotor during an uncontrolled swing.
        A longer cable guards against a payload and vehicle collision, 
        because more energy needs to be transferred to the payload to reach the height of the vehicle. 
        The maximum cable length is selected as $l=~\SI{2}{\metre}$ by intuition 
        since a cable much longer than this may not be practically useful for a drone delivery flight with the considered quadrotor.

        \paragraph{}
        Plot MAE vs l
        \begin{figure}[h]
    \centering
    \begin{tikzpicture}
        \begin{axis}[            
            xlabel = Length of training data,
            ylabel = MAE,
            x unit = \si{\second},
            % y unit = \si{\second},
            xmin = 0,     xmax = 120,
            ymin = 0.08,  ymax = 0.12,
            grid = major,
            legend cell align = left,
            legend pos = north east,
            grid style = dashed,
            legend style = {font = \scriptsize},
            label style = {font = \scriptsize},
            tick label style = {font = \scriptsize},
            width = 0.45\columnwidth,
            height = 0.5\columnwidth,
            % initialize Dark2
            cycle list/Dark2,
            % combine it with 'mark list*':
            cycle multiindex* list = {
                Dark2\nextlist
            }
        ]
        
        \addplot+[mark = none, style = solid, ultra thick] 
        table[x = T_train, y = MAE_mean, col sep = comma] 
        {system_id/csv/MAE_vs_Ntrain_SITL_x_vel_noise_l0-5_m0-2.csv_havok_angle.csv};
        \addlegendentry{$l =$~\SI{0.5}{\metre}}
        
        \addplot+[mark = none, style = solid, ultra thick] 
        table[x = T_train, y = MAE_mean, col sep = comma] 
        {system_id/csv/MAE_vs_Ntrain_SITL_x_vel_noise_l1_m0-2.csv_havok_angle.csv};
        \addlegendentry{$l =$~\SI{1}{\metre}}
        
        \addplot+[mark = none, style = solid, ultra thick] 
        table[x = T_train, y = MAE_mean, col sep = comma] 
        {system_id/csv/MAE_vs_Ntrain_SITL_x_vel_noise_l2_m0-2.csv_havok_angle.csv};
        \addlegendentry{$l =$~\SI{2}{\metre}}
        
        \end{axis}
    \end{tikzpicture} 
    
    \caption{HAVOK prediction error using different cable lengths with a range of different sample times of noisy training data
    ($m =$~\SI{0.2}{\kilo\gram}, $T_{train} =$~various??.)}
    \label{fig:MAE_vs_Ntrain_vs_L_havok}
\end{figure}

        DMD shows the same trend revealed in Figure~\ref{fig:MAE_vs_Ntrain_vs_L_havok}.

        \paragraph{}
        Plot MAE vs m
        \input{system_id/plots/MAE_vs_Ntrain_vs_m_havok.tex}
        DMD shows the same trend revealed in Figure~\ref{fig:MAE_vs_Ntrain_vs_m_havok}.

        \paragraph{}
        % Not very conclusive
        % \begin{figure}[h]
    \centering
    \begin{tikzpicture}
        \begin{axis}[            
            xlabel = Ts,
            ylabel = NMAE\textsubscript{mm},
            % x unit = \si{\second},
            % y unit = \si{\second},
            xmin = 0.02,  xmax = 0.05,
            ymin = 0.03,  ymax = 0.055,
            grid = major,
            legend cell align = left,
            legend pos = north east,
            grid style = dashed,
            legend style = {font = \scriptsize},
            label style = {font = \scriptsize},
            tick label style = {font = \scriptsize},
            width = 0.45\columnwidth,
            height = 0.5\columnwidth,
            % initialize Dark2
            cycle list/Dark2,
            % combine it with 'mark list*':
            cycle multiindex* list = {
                Dark2\nextlist
            }
        ]

        % \addplot+[mark = none, style = solid, ultra thick] 
        % table[x = Ts, y = NMAE_mean, col sep = comma] 
        % {system_id/csv/NMAE_vs_Ts_SITL_x_vel_noise_l0-25_m0-2.csv_dmd_angle.csv};
        % \addlegendentry{$l =$~\SI{0.25}{\metre}}
        
        \addplot+[mark = none, style = solid, ultra thick] 
        table[x = Ts, y = NMAE_mean, col sep = comma] 
        {system_id/csv/NMAE_vs_Ts_SITL_x_vel_noise_l0-5_m0-2.csv_dmd_angle.csv};
        \addlegendentry{$l =$~\SI{0.5}{\metre}}
        
        \addplot+[mark = none, style = solid, ultra thick] 
        table[x = Ts, y = NMAE_mean, col sep = comma] 
        {system_id/csv/NMAE_vs_Ts_SITL_x_vel_noise_l1_m0-2.csv_dmd_angle.csv};
        \addlegendentry{$l =$~\SI{1}{\metre}}
        
        \addplot+[mark = none, style = solid, ultra thick] 
        table[x = Ts, y = NMAE_mean, col sep = comma] 
        {system_id/csv/NMAE_vs_Ts_SITL_x_vel_noise_l2_m0-2.csv_dmd_angle.csv};
        \addlegendentry{$l =$~\SI{2}{\metre}}
        
        \end{axis}
    \end{tikzpicture} 
    
    \caption{DMD prediction error using different cable lengths with a range of different sample times of noisy training data
    ($m =$~\SI{0.2}{\kilo\gram})}
    \label{fig:MAE_vs_Ts_vs_L}
\end{figure}

        % See how it affects Ts
        % plot MAE vs Ts with contours of l

        Best hyperparameters.
        Fixed size of data.
        Fixed sample time.

    \subsection{Dynamic payload}
        Some payloads attached to the cable may not satisfy the assumptions made in Section~\ref{sec:plant_considered}.
        For example, if a long payload is attached to the cable, the CoM of the payload will be quite a distance below the attachment of the cable.
        This creates a double pendulum model which has different dynamcis than a single pendulum.

        \murray{insert picture a practical quad with long payload}
        \paragraph{}
        Another payload case that will cause inaccuracies in the parameter estimation technique is if a payload is attached rigidly to the quadrotor while it also has a suspended payload.
        The payload mass estimation is based on the assumption that the quadrotor mass is known.
        However if a mass is rigidly attached to the vehicle, the effective quadrotor mass is changed and the RLS payload mass estimation is no longer accurate.

        These payload produce significantly different dynamics than predicted by the a priori model.
        Since the system identification by parameter estimation depends heavily on the pre-determined mathematical model,
        
        \paragraph{}
        For each of these payload cases, a different parameter estimation based techniques would needs to be designed for effective control.
        This is undesirable for practical drone deliveries, especially when the type of paylaod is not known well in advance or changes regularly.
        A data-driven technique provides a more general solution since it accommodates a larger range of payload types and does not require a prioir modelling information.
        
        plot hyperparameterss MAE. Not how much more delays are required


    \section{Conclusion}

        \paragraph
        \gls{DMDc} and \gls{HAVOK} produce very similar prediction errors for a range of different simulation conditions.
        \gls{HAVOK} generally has slightly lower errors, but this may have a negligible effect on control.
        \gls{DMDc} will be used in the remainder of this work 
        because the algorithm has been studied more, is less computationally complex and provides similar performance to \gls{HAVOK}.
        These data-driven methods were shown to produce accurate predictions of the payload dynamics 
        with a practical length of training data, sample time and noise level.
        This proves to be promising for practical implementation with a \gls{MPC} on a quadrotor companion computer.
        The data-driven methods also proved to provide accurate prediction with a range of different payload parameters
        and even with payloads that act as a double pendulum.
        In contrast, the considered white-box system identification technique 
        failed to identify a relevant model for the double pendulum dynamics.
        Therefore the data-driven approach provides a more general system identification method 
        that can be used for a range of different suspended payloads without being redesigned for specific dynamics.

        % \gls{DMD} and \gls{HAVOK} work very similarly with single pend.
        % \gls{HAVOK} has slightly better prediction accuracy, but this small difference has neglible effect on control.
        % It is difficult to compare the white-box to the black-box models because the real effect will only be seen during control.
        % However it is clear that the accuracy of the white-box model degrades significantly 
        % when a payload that causes double pendulum dynamics.

        % The major advantage of the data-driven approach which was demonstrated in this section 
        % is that the method was applied to different dynamical systems   

        % \paragraph
        % For each of these payload cases, a different parameter estimation based techniques would needs to be designed for effective control.
        % This is undesirable for practical drone deliveries, especially when the type of paylaod is not known well in advance or changes regularly.
        % A data-driven technique provides a more general solution since it accommodates a larger range of payload types and does not require a prioir modelling information.
            
        % \paragraph
        % Since the a priori white-box model is based on a single pendulum model, 
        % the dynamics described by the model are significantly different from the actual dynamics.
        % This will have a detrimental effect on the control performance of a controller based on such a model,
        % since the controller will be designed for different plant than what it is controlling actually controlling.
       