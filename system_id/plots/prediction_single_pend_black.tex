\begin{figure}[htb]
    \centering
    \begin{tikzpicture}
        \begin{axis}[            
            xlabel = Time,
            ylabel = Payload angle,
            x unit = \si{\second},
            y unit = \si{\radian},
            xmin = 0,   xmax = 20,
            ymin = -0.2,  ymax = 0.1,
            grid = major,
            legend cell align = left,
            legend pos = north east,
            grid style = dashed,
            legend style = {font = \scriptsize},
            label style = {font = \scriptsize},
            tick label style = {font = \scriptsize},
            width = 0.95\columnwidth,
            height = 0.5\columnwidth,
            % initialize Dark2
            cycle list/Dark2,
            % combine it with 'mark list*':
            cycle multiindex* list = {
                Dark2\nextlist
            }
        ]
        
        \addplot+[mark = none, style = solid, ultra thick] 
        table[x = time, y = theta, col sep = comma] 
        {system_id/csv/single_step_predictions_SITL_x_vel_noise_l1_m0-3.csv.csv};
        \addlegendentry{Actual}

        \addplot+[mark = none, style = solid, ultra thick] 
        table[x = time, y = theta_dmd, col sep = comma] 
        {system_id/csv/single_step_predictions_SITL_x_vel_noise_l1_m0-3.csv.csv};
        \addlegendentry{DMD}

        \addplot+[mark = none, style = dashed, ultra thick] 
        table[x = time, y = theta_havok, col sep = comma] 
        {system_id/csv/single_step_predictions_SITL_x_vel_noise_l1_m0-3.csv.csv};
        \addlegendentry{HAVOK}

        % \addplot+[mark = none, style = dashed, ultra thick] 
        % table[x = time, y = theta_white, col sep = comma] 
        % {system_id/csv/single_step_predictions_SITL_x_vel_noise_l1_m0-3.csv.csv};
        % \addlegendentry{White-box}

        \end{axis}
    \end{tikzpicture} 
    
    \caption{Data-driven model predictions of a single pendulum for a North velocity step input
        ($m =$~\SI{0.3}{\kilo\gram}, $l =$~\SI{1}{\meter}).}
    \label{fig:prediction_single_pend_black}
\end{figure}
