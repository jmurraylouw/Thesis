\graphicspath{{modelling/fig/}}

\chapter{Modelling}
\label{chap:modelling}

This chapter discusses the mathematical modelling of a quadrotor with a suspended
payload which is based on a practical quadrotor UAV named Honeybee.
The model is first derived as a 2D model.
The system identification and control system techniques in later chapters will then be explained based on the 2D model to avoid unnecessary complexity.
Finally, it will be described how this model and the techniques in later chapters are extended to the 3D case.
This 3D mathematical model will be used in a nonlinear simulation of a quadrotor and suspended payload.
1
\section{Coordinate frames}
\section{States}
\section{Forces and moments}
\section{Lagrangian mechanics}
\section{Linearised model}
    \label{sec:linear_model}

\section{Discretised model}
\section{Model verification}
\section{Dynamic payloads}
\begin{itemize}
    \item Most control in literature model payload as rigid Mass. Give references.
    \item Some add spring stiffness of cable (give references e.g. QuadLoad ElasticCable Prasanth Kotaru)
    \item but not practical because can design which cable you used. Cannot design which payload needs to be transported
    \item water looks like double payload
\end{itemize}
% https://hybrid-robotics.berkeley.edu/publications/ACC2017_QuadLoad_ElasticCable.pdf), 



