\graphicspath{{modelling/fig/}}

\chapter{Modelling}
\label{chap:modelling}

This chapter discusses the mathematical modelling of a multirotor with a suspended
payload which is based on a practical multirotor UAV named Honeybee.
The model is first derived as a \gls{2D} model.
The system identification and control system techniques in later chapters will then be explained based on the \gls{2D} model to avoid unnecessary complexity.
Finally, it will be described how this model and the techniques in later chapters are extended to the 3D case.
This 3D mathematical model will be used in a nonlinear simulation of a multirotor and suspended payload.
1
\section{Coordinate frames}
\section{States}
\section{Forces and moments}
\section{Lagrangian mechanics}
    \begin{enumerate}
        \item Discuss weak link between altitude dynamics and swing angle
    \end{enumerate}
\section{Linearised model}
    \label{sec:linear_model}

\section{Discretised model}
\FloatBarrier\section{Model verification}

    \paragraph
    Intro

    \begin{figure}[htb]
    \centering
    \begin{tikzpicture}
        \begin{axis}[            
            xlabel = Time,
            ylabel = North velocity,
            x unit = \si{\second},
            y unit = \si{\metre/\second},
            xmin = 0,   xmax = 16,
            ymin = -0.1,  ymax = 2.5,
            grid = major,
            legend cell align = left,
            legend pos = south east,
            grid style = dashed,
            legend style = {font = \scriptsize},
            label style = {font = \scriptsize},
            tick label style = {font = \scriptsize},
            width = 0.95\columnwidth,
            height = 0.5\columnwidth,
            % initialize Dark2
            cycle list/Dark2,
            % combine it with 'mark list*':
            cycle multiindex* list = {
                Dark2\nextlist
            }
        ]
                
        \addplot+[mark = none, style = solid, ultra thick] 
        table[x = time, y = vel_sp, col sep = comma] 
        {modelling/csv/prac_vs_sim_vel_step_Simulink_2021-08-20_04_no_load_velocity_steps_wind_0.5.csv.csv};
        \addlegendentry{$V_{N_{sp}}$}

        \addplot+[mark = none, style = solid, ultra thick] 
        table[x = time, y = vel.prac, col sep = comma] 
        {modelling/csv/prac_vs_sim_vel_step_Simulink_2021-08-20_04_no_load_velocity_steps_wind_0.5.csv.csv};
        \addlegendentry{$V_N$ (Practical)}

        \addplot+[mark = none, style = dashed, ultra thick] 
        table[x = time, y = vel.sim, col sep = comma] 
        {modelling/csv/prac_vs_sim_vel_step_Simulink_2021-08-20_04_no_load_velocity_steps_wind_0.5.csv.csv};
        \addlegendentry{$V_N$ (Simulated)}

        \end{axis}
    \end{tikzpicture} 
    \caption{Comparison of simulated and practical data from Honeybee.}
    \label{fig:prac_vs_sim_vel_step_no_payload}
\end{figure}


    \paragraph
    No load

    \begin{figure}[htb]
    \centering
    \begin{tikzpicture}
        \begin{axis}[            
            xlabel = Time,
            ylabel = North velocity,
            x unit = \si{\second},
            y unit = \si{\metre/\second},
            xmin = 0,   xmax = 16,
            ymin = -0.1,  ymax = 2.8,
            grid = major,
            legend cell align = left,
            legend pos = south east,
            grid style = dashed,
            legend style = {font = \scriptsize},
            label style = {font = \scriptsize},
            tick label style = {font = \scriptsize},
            width = 0.95\columnwidth,
            height = 0.5\columnwidth,
            % initialize Dark2
            cycle list/Dark2,
            % combine it with 'mark list*':
            cycle multiindex* list = {
                Dark2\nextlist
            }
        ]
                
        \addplot+[mark = none, style = solid, ultra thick] 
        table[x = time, y = vel_sp, col sep = comma] 
        {modelling/csv/prac_vs_sim_vel_step_Simulink_2021-08-20_02_l-2_mp-0.3-wind-0.5.csv.csv};
        \addlegendentry{$V_{N_{sp}}$}

        \addplot+[mark = none, style = solid, ultra thick] 
        table[x = time, y = vel.prac, col sep = comma] 
        {modelling/csv/prac_vs_sim_vel_step_Simulink_2021-08-20_02_l-2_mp-0.3-wind-0.5.csv.csv};
        \addlegendentry{$V_N$ (Practical)}

        \addplot+[mark = none, style = dashed, ultra thick] 
        table[x = time, y = vel.sim, col sep = comma] 
        {modelling/csv/prac_vs_sim_vel_step_Simulink_2021-08-20_02_l-2_mp-0.3-wind-0.5.csv.csv};
        \addlegendentry{$V_N$ (Simulated)}

        \end{axis}
    \end{tikzpicture} 
    \caption{Velocity step comparison of simulated and practical data for Honeybee with a suspended payload.}
    \label{fig:prac_vs_sim_vel_step_with_payload}
\end{figure}


    \paragraph
    With payload vel

    \begin{figure}[htb]
    \centering
    \begin{tikzpicture}
        \begin{axis}[            
            xlabel = Time,
            ylabel = Payload angle,
            x unit = \si{\second},
            y unit = \si{\degree},
            xmin = 0,   xmax = 16,
            ymin = -20,  ymax = 20,
            grid = major,
            legend cell align = left,
            legend pos = south east,
            grid style = dashed,
            legend style = {font = \scriptsize},
            label style = {font = \scriptsize},
            tick label style = {font = \scriptsize},
            width = 0.95\columnwidth,
            height = 0.5\columnwidth,
            % initialize Dark2
            cycle list/Dark2,
            % combine it with 'mark list*':
            cycle multiindex* list = {
                Dark2\nextlist
            }
        ]
                
        \addplot+[mark = none, style = solid, ultra thick] 
        table[x = time, y = theta.prac, col sep = comma] 
        {modelling/csv/prac_vs_sim_vel_step_Simulink_2021-08-20_02_l-2_mp-0.3-wind-0.5.csv.csv};
        \addlegendentry{Practical}

        \addplot+[mark = none, style = dashed, ultra thick] 
        table[x = time, y = theta.sim, col sep = comma] 
        {modelling/csv/prac_vs_sim_vel_step_Simulink_2021-08-20_02_l-2_mp-0.3-wind-0.5.csv.csv};
        \addlegendentry{Simulated}

        \end{axis}
    \end{tikzpicture} 
    \caption{Payload angle comparison of simulated and practical data for Honeybee with a suspended payload}
    \label{fig:prac_vs_sim_theta_with_payload}
\end{figure}


    \paragraph
    With payload theta



    \begin{enumerate}
        \item Plot velocity step
        \item Plot position step
    \end{enumerate}


\FloatBarrier\section{Dynamic payloads}

    In this work, a dynamic payload refers to a payload with dynamics that differ significantly from a rigid mass.
    When considering suspended payloads, a dynamic payload induces dynamics which differ from that of a simple pendulum.

    \begin{itemize}
        \item Most control in literature model payload as rigid Mass. Give references.
        \item Some add spring stiffness of cable (give references e.g. QuadLoad ElasticCable Prasanth Kotaru)
        \item but not practical because can design which cable you used. Cannot design which payload needs to be transported
        \item water looks like double payload
    \end{itemize}
% https://hybrid-robotics.berkeley.edu/publications/ACC2017_QuadLoad_ElasticCable.pdf), 



