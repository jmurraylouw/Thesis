\graphicspath{{modelling/fig/}}
{
\tikzset{external/figure name/.add={modelling/}{}}

\chapter{Modelling}
\label{chap:modelling}

\paragraph
In this chapter, mathematical models of a multirotor with a suspended payload will be derived.
These models will be used to simulate the multirotor system in subsequent chapters.
The models will be based on the real multirotor named \emph{Honeybee}, which was built by \cite{Grobler2020} and will be described further in Chapter~\ref{chap:exp_design}.

\paragraph
% The system identification and control system techniques in later chapters will then be explained based on the \gls{2D} model to avoid unnecessary complexity.
Finally, it will be described how this model and the techniques in later chapters are extended to the 3D case.
This 3D mathematical model will be used in a non-linear simulation of a multirotor and suspended payload.

% ?? For good modelling of multirotor, survey: https://arxiv.org/pdf/2011.11104.pdf

\FloatBarrier\section{Coordinate frames}

    \paragraph
    A quadrotor with a standard X-configuration is considered in this work.
    Figure~\ref{fig:coord_frames} shows a quadrotor schematic and two coordinate frames that will be used to describe this system.
    
    \begin{figure}[htb]
        \centering
        \includegraphics[width=0.818\linewidth]{modelling/fig/coord_frames}
        \caption{Coordinate frames of a quadrotor \cite{Erasmus2020}}
        \label{fig:coord_frames}
    \end{figure}

    \paragraph
    The inertial frame is denote by $ \mathcal{I} = \{ \bm{\bar{x}}_\mathcal{I}, \bm{\bar{y}}_\mathcal{I}, \bm{\bar{z}}_\mathcal{I} \} $ and describes a \gls{NED} axis system.
    The $x$, $y$, and $z$ axis, align with the North, East, and Down inertial directions respectively.
    The inertial frame assumes a flat, non-rotating earth, since the quadrotor will travel small distances in comparison to the radius of the earth.
    The origin of this frame is fixed at the takeoff location of the multirotor.

    \paragraph
    The body frame is denote by $ \mathcal{B} = \{ \bm{\bar{x}}_\mathcal{B}, \bm{\bar{y}}_\mathcal{B}, \bm{\bar{z}}_\mathcal{B} \} $ and is fixed to the quadrotor body.
    The origin of this frame is at the \gls{CoM} of the vehicle and the $x$, $y$, and $z$ axis, align with the forwards, rightwards, and downwards directions of the quadrotor body respectively.
    The body frame is described by a translation and rotation relative to the inertial frame.

\FloatBarrier\section{Multirotor with a suspended payload}


\FloatBarrier\section{Forces and moments}
\FloatBarrier\section{Lagrangian mechanics}
    \begin{enumerate}
        \item Discuss weak link between altitude dynamics and swing angle
    \end{enumerate}
\FloatBarrier\section{Linearised model}
    \label{sec:linear_model}

\FloatBarrier\section{Discretised model}
\FloatBarrier\section{Model verification} \label{sec:model_verification}

    \paragraph
    Intro

    \begin{figure}[htb]
    \centering
    \begin{tikzpicture}
        \begin{axis}[            
            xlabel = Time,
            ylabel = North velocity,
            x unit = \si{\second},
            y unit = \si{\metre/\second},
            xmin = 0,   xmax = 16,
            ymin = -0.1,  ymax = 2.5,
            grid = major,
            legend cell align = left,
            legend pos = south east,
            grid style = dashed,
            legend style = {font = \scriptsize},
            label style = {font = \scriptsize},
            tick label style = {font = \scriptsize},
            width = 0.95\columnwidth,
            height = 0.5\columnwidth,
            % initialize Dark2
            cycle list/Dark2,
            % combine it with 'mark list*':
            cycle multiindex* list = {
                Dark2\nextlist
            }
        ]
                
        \addplot+[mark = none, style = solid, ultra thick] 
        table[x = time, y = vel_sp, col sep = comma] 
        {modelling/csv/prac_vs_sim_vel_step_Simulink_2021-08-20_04_no_load_velocity_steps_wind_0.5.csv.csv};
        \addlegendentry{$V_{N_{sp}}$}

        \addplot+[mark = none, style = solid, ultra thick] 
        table[x = time, y = vel.prac, col sep = comma] 
        {modelling/csv/prac_vs_sim_vel_step_Simulink_2021-08-20_04_no_load_velocity_steps_wind_0.5.csv.csv};
        \addlegendentry{$V_N$ (Practical)}

        \addplot+[mark = none, style = dashed, ultra thick] 
        table[x = time, y = vel.sim, col sep = comma] 
        {modelling/csv/prac_vs_sim_vel_step_Simulink_2021-08-20_04_no_load_velocity_steps_wind_0.5.csv.csv};
        \addlegendentry{$V_N$ (Simulated)}

        \end{axis}
    \end{tikzpicture} 
    \caption{Comparison of simulated and practical data from Honeybee.}
    \label{fig:prac_vs_sim_vel_step_no_payload}
\end{figure}


    \paragraph
    No load

    \begin{figure}[htb]
    \centering
    \begin{tikzpicture}
        \begin{axis}[            
            xlabel = Time,
            ylabel = Velocity,
            x unit = \si{\second},
            y unit = \si{\metre/\second},
            xmin = 0,   xmax = 16,
            ymin = -0.1,  ymax = 2.8,
            grid = major,
            legend cell align = left,
            legend pos = south east,
            grid style = dashed,
            legend style = {font = \scriptsize},
            label style = {font = \scriptsize},
            tick label style = {font = \scriptsize},
            width = 0.95\columnwidth,
            height = 0.5\columnwidth,
            % initialize Dark2
            cycle list/Dark2,
            % combine it with 'mark list*':
            cycle multiindex* list = {
                Dark2\nextlist
            }
        ]
                
        \addplot+[mark = none, style = solid, ultra thick] 
        table[x = time, y = vel_sp, col sep = comma] 
        {modelling/csv/prac_vs_sim_vel_step_Simulink_2021-08-20_02_l-2_mp-0.3-wind-0.5.csv.csv};
        \addlegendentry{$V_{N,sp}$}

        \addplot+[mark = none, style = solid, ultra thick] 
        table[x = time, y = vel.prac, col sep = comma] 
        {modelling/csv/prac_vs_sim_vel_step_Simulink_2021-08-20_02_l-2_mp-0.3-wind-0.5.csv.csv};
        \addlegendentry{$V_N$ (Practical)}

        \addplot+[mark = none, style = dashed, ultra thick] 
        table[x = time, y = vel.sim, col sep = comma] 
        {modelling/csv/prac_vs_sim_vel_step_Simulink_2021-08-20_02_l-2_mp-0.3-wind-0.5.csv.csv};
        \addlegendentry{$V_N$ (Simulated)}

        \end{axis}
    \end{tikzpicture} 
    \caption{Velocity step comparison of simulated and practical data for Honeybee with a suspended payload}
    \label{fig:prac_vs_sim_vel_step_with_payload}
\end{figure}


    \paragraph
    With payload vel

    \begin{figure}[htb]
    \centering
    \begin{tikzpicture}
        \begin{axis}[            
            xlabel = Time,
            ylabel = Payload angle,
            x unit = \si{\second},
            y unit = \si{\degree},
            xmin = 0,   xmax = 16,
            ymin = -20,  ymax = 20,
            grid = major,
            legend cell align = left,
            legend pos = south east,
            grid style = dashed,
            legend style = {font = \scriptsize},
            label style = {font = \scriptsize},
            tick label style = {font = \scriptsize},
            width = 0.95\columnwidth,
            height = 0.5\columnwidth,
            % initialize Dark2
            cycle list/Dark2,
            % combine it with 'mark list*':
            cycle multiindex* list = {
                Dark2\nextlist
            }
        ]
                
        \addplot+[mark = none, style = solid, ultra thick] 
        table[x = time, y = theta.prac, col sep = comma] 
        {modelling/csv/prac_vs_sim_vel_step_Simulink_2021-08-20_02_l-2_mp-0.3-wind-0.5.csv.csv};
        \addlegendentry{Practical}

        \addplot+[mark = none, style = dashed, ultra thick] 
        table[x = time, y = theta.sim, col sep = comma] 
        {modelling/csv/prac_vs_sim_vel_step_Simulink_2021-08-20_02_l-2_mp-0.3-wind-0.5.csv.csv};
        \addlegendentry{Simulated}

        \end{axis}
    \end{tikzpicture} 
    \caption{Payload angle comparison of simulated and practical data for Honeybee with a suspended payload.}
    \label{fig:prac_vs_sim_theta_with_payload}
\end{figure}


    \paragraph
    With payload theta



    \begin{enumerate}
        \item Plot velocity step
        \item Plot position step
    \end{enumerate}


\FloatBarrier\section{Dynamic payloads}

    In this work, a dynamic payload refers to a payload with dynamics that differ significantly from a rigid mass.
    When considering suspended payloads, a dynamic payload induces dynamics which differ from that of a simple pendulum.

    \begin{itemize}
        \item Most control in literature model payload as rigid Mass. Give references.
        \item Some add spring stiffness of cable (give references e.g. QuadLoad ElasticCable Prasanth Kotaru)
        \item but not practical because can design which cable you used. Cannot design which payload needs to be transported
        \item water looks like double payload
    \end{itemize}
% https://hybrid-robotics.berkeley.edu/publications/ACC2017_QuadLoad_ElasticCable.pdf), 

}

