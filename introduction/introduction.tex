\graphicspath{{introduction/fig/}}

\chapter{Introduction}
\label{chap:introduction}

    \section{Background}


    \section{Project definition and objectives}

        \paragraph
        The aim of this project is to design and implement a controller architecture for stabilised control of a multirotor with an unknown suspended payload.
        The architecture has no prior knowledge of the suspended payload dynamics and applies a data-driven approach to determine a model of the unknown dynamics.
        The pendulum motion of the payload significantly affects the flight dynamics by inducing oscillations into the system.
        A controller uses the identified model for stabilised control of the multirotor by actively damping the payload swing angles.

        \paragraph
        Therefore, the research objectives are summarised as:
        \begin{enumerate}
            \item 
        \end{enumerate}

        % ?? Use words from evaluation sheet. Like large scope...


    \section{Thesis outline}

        \paragraph
        Chapter~\ref{chap:introduction} provides the background for the research, the project definition and objectives, and the outline of the thesis.
        
        \paragraph
        Chapter~\ref{chap:lit_study} presents a study of the literature regarding multirotors payload transportation, with a focus on suspended payloads and uncertain payload dynamics.
        
        \paragraph
        Chapter~\ref{chap:modelling} contains a derivation of a mathematical model for the multirotor and suspended payload dynamics, which is used in subsequent chapters.
        
        \paragraph
        Chapter~\ref{chap:system_id} describes the baseline and the proposed system identification methods methods considered in this thesis. 
        Furthermore the performance of these methods are compared based on tests with simulation data.
        
        \paragraph
        Chapter~\ref{chap:control} describes the different controllers and the corresponding controller design process used in this project.
        Using the system identification models from the previous chapter, the controllers are also applied to the multirotor-payload system in simulation and the results are compared.
        
        \paragraph
        Chapter~\ref{chap:exp_design} provides an overview of the practical multirotor setup used for experimental work with the proposed algorithms.
        Thereby, the hardware components, software toolchain, and \gls{HITL} simulations are discussed.
        
        \paragraph
        Chapter~\ref{chap:results} presents and discusses the experimental results from implementing the system identification methods to practical flight data.
        \gls{HITL} results are also presented to test the controller algorithms with the practical hardware and software systems.
        
        \paragraph
        Chapter~\ref{chap:conclusion} provides a summary of the work in this thesis. The major conclusions of this work is also presented and future recommendations are discussed.


