\graphicspath{{introduction/fig/}}

\chapter{Introduction}
\label{chap:introduction}

    \section{Background}

        \paragraph
        Recent years have seen a rise in the popularity of payload transportation with \glspl{UAV}.
        Examples like Erasmus2020
        Especially multirotors are popular vehicles for payload transportation tasks due to their hover and \gls{VTOL} abilities.

        \paragraph
        Sensors vs freight
        Suspended vs rigid
        
        \paragraph
        With sensor payloads, the payload dynamics are often well known and constant, hence a control system can be designed based on a predetermined model.
        However, in package delivery application, the payload parameters is often unknown prior to flight.
        Some payloads, such as elongated payloads or fluids in contains, induce interesting dynamics which are also unknown prior to flight.
        This significantly alters the flight dynamics of a multirotor.
        Therefore, the multirotor controller architecture needs to account for the payload dynamics for stabilised control.

        \paragraph
        In summary, multirotor 


    \section{Project definition and objectives}

        \paragraph
        The aim of this project is to design and implement a controller architecture for stabilised control of a multirotor with an unknown suspended payload.
        The pendulum motion of the payload significantly affects the multirotor dynamics by inducing oscillations into the system.
        The proposed controller should be compared to previous work involving a swing damping controller for a suspended payload with an unknown mass and cable length. 
        
        \paragraph
        In contrast to an architecture based on a predetermined model with only two unknown parameters,
        the proposed architecture should assume no prior knowledge of the suspended payload dynamics.
        A data-driven approach should be applied to estimate a dynamical model of the unknown dynamics.
        Based on the estimated model, a controller should stabilise the multirotor by actively damping the payload swing angles.

        \paragraph
        The research objectives can therefore be stated as the following:
        \begin{enumerate}
            \item Investigate the literature regarding multirotor payload transportation systems and specifically consider solutions or unknown suspended payload dynamics.
            \item Derive a dynamical model to describe a multirotor with a suspended payload.
            \item Identify and implement a baseline system identification and control method for this system in simulation.
            \item Design a data-driven system identification method for this system and implement it in simulation.
            \item Design a controller based on the proposed system identification model and implement it in simulation.
            \item Identify a hardware platform and software toolchain to implement the proposed control architecture.
            \item Implement and verify the data-driven system identification method with experimental data from the practical system.
            \item Implement, simulate and verify the controller algorithms on the practical hardware for effective swing damping control of the unknown suspended payload system.
        \end{enumerate}

        % ?? Use words from evaluation sheet. Like large scope...


    \section{Thesis outline}

        \paragraph
        Chapter~\ref{chap:introduction} provides the background for the research, the project definition and objectives, and the outline of the thesis.
        
        \paragraph
        Chapter~\ref{chap:lit_study} presents a study of the literature regarding multirotors payload transportation, with a focus on suspended payloads and uncertain payload dynamics.
        
        \paragraph
        Chapter~\ref{chap:modelling} contains a derivation of a mathematical model for the multirotor and suspended payload dynamics, which is used in subsequent chapters.
        
        \paragraph
        Chapter~\ref{chap:system_id} describes the baseline and the proposed system identification methods methods considered in this thesis. 
        Furthermore the performance of these methods are compared based on tests with simulation data.
        
        \paragraph
        Chapter~\ref{chap:control} describes the different controllers and the corresponding controller design process used in this project.
        Using the system identification models from the previous chapter, the controllers are also applied to the multirotor-payload system in simulation and the results are compared.
        
        \paragraph
        Chapter~\ref{chap:exp_design} provides an overview of the practical multirotor setup used for experimental work with the proposed algorithms.
        Thereby, the hardware components, software toolchain, and \gls{HITL} simulations are discussed.
        
        \paragraph
        Chapter~\ref{chap:results} presents and discusses the experimental results from implementing the system identification methods to practical flight data.
        \gls{HITL} results are also presented to test the controller algorithms with the practical hardware and software systems.
        
        \paragraph
        Chapter~\ref{chap:conclusion} provides a summary of the work in this thesis. The major conclusions of this work is also presented and future recommendations are discussed.


