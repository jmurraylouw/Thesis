\graphicspath{{introduction/fig/}}

\chapter{Introduction}
\label{chap:introduction}

    \section{Background}

        \paragraph
        Recent years have seen a rise in the popularity of payload transportation with \glspl{UAV} \cite{Nakamura2019}.
        \gls{UAV} payloads are usually categorised as either a sensor, or freight \cite{Vergouw2016}.
        Sensors like cameras or meteorological instruments can be carried by \glspl{UAV} for aerial photography or surveying.
        Payloads carried as freight can be pesticides sprayed over agricultural land, medical parcels carried to hard to reach areas or mail parcels for consumer deliveries.
    
        \paragraph
        Commercial package deliveries with \glspl{UAV} have become especially popular.
        In 2015, the first \gls{FFA} approved drone delivery was successfully completed by Flirtey in the United States \cite{Vanian2015}.
        Domino's pizza has also been delivered by Flirtey multirotors in New Zealand \cite{Hidalgo2021}.
        Another commercial example includes Wing food deliveries in Australia with multirotors \cite{Wing2021}.

        \paragraph
        Multirotor \glspl{UAV} are popular vehicles for payload transportation tasks due to their hover and \gls{VTOL} abilities.
        In some applications, payloads are rigidly attached to a \gls{UAV}.
        However, the flying characteristics of multirotors allow them to transport suspended payloads.
        In this configuration, the payload is attached below the vehicle with a suspended cable and the payload is free to swing during flight.
        This oscillatory motion affects the flight dynamics of the multirotor and makes stabilised control a challenging task.
        
        \paragraph
        Control becomes even more difficult with increased uncertainty of the system dynamics.
        With sensor payloads, the payload dynamics are often well known and constant.
        However, package delivery applications often have payload parameter uncertainty before a flight.
        Specific payloads such as elongated payloads or fluid containers add more uncertainty to the system by inducing interesting dynamics which are also unknown before a flight.
        This significantly alters the flight dynamics of a multirotor and the controller may need to account for this uncertainty for effective control.

        \paragraph
        In summary, multirotor-payload transportation is becoming increasingly popular.
        The suspended payload configuration offers strategic benefits but increases the difficulty of the control task.
        Furthermore, the uncertainty in payload dynamics makes the control task more challenging.
        In this study, a control architecture will be designed to address this problem.

    \section{Project definition and objectives}

        \paragraph
        This project aims to design and implement a control architecture for stabilised control of a multirotor with an unknown suspended payload.
        This system includes payload uncertainty and the oscillatory motion of the payload significantly affects the multirotor dynamics.
        The proposed controller should be compared to previous work involving a swing damping controller for a suspended payload with an unknown mass and cable length. 
        
        \paragraph
        In contrast to the architecture based on a predetermined model with only two unknown parameters,
        the proposed architecture should assume no prior knowledge of the suspended payload dynamics.
        A data-driven approach should be applied to estimate a dynamical model of the unknown dynamics.
        Based on the estimated model, a controller should stabilise the multirotor by actively damping the payload swing angles.

        \paragraph
        Therefore, the research objectives are stated as:
        \begin{enumerate}
            \item Investigate the literature regarding multirotor-payload transportation systems and specifically consider solutions or unknown suspended payload dynamics.
            \item Derive a dynamical model to describe a multirotor with a suspended payload.
            \item Identify and implement a baseline system identification and control method for this system in simulation.
            \item Design a data-driven system identification method for this system and implement it in simulation.
            \item Design a controller based on the proposed system identification model and implement it in simulation.
            \item Identify a hardware platform and software toolchain to implement the proposed control architecture.
            \item Implement and verify the data-driven system identification method with experimental data from the practical system.
            \item Implement, simulate and verify the controller algorithms on the practical hardware for effective swing damping control of the unknown suspended payload system.
        \end{enumerate}

    \section{Thesis outline}

        \paragraph
        Chapter~\ref{chap:introduction} provides the background for the research, the project definition and objectives, and the outline of the thesis.
        
        \paragraph
        Chapter~\ref{chap:lit_study} presents a study of the literature regarding multirotors payload transportation, with a focus on suspended payloads and uncertain payload dynamics.
        
        \paragraph
        Chapter~\ref{chap:modelling} contains a derivation of a mathematical model for the multirotor and suspended payload dynamics, which is used in subsequent chapters.
        
        \paragraph
        Chapter~\ref{chap:system_id} describes the baseline and the proposed system identification methods considered in this thesis. 
        Furthermore, the performances of these methods are evaluated based on tests with simulation data.
        
        \paragraph
        Chapter~\ref{chap:control} describes the different controllers and the corresponding controller design process used in this project.
        Using the system identification models from the previous chapter, the controllers are also applied to the multirotor-payload system in simulation and the results are compared.
        
        \paragraph
        Chapter~\ref{chap:exp_design} provides an overview of the practical multirotor setup used for experimental work with the proposed algorithms.
        Thereby, the hardware components, software toolchain, and \gls{HITL} simulations are discussed.
        
        \paragraph
        Chapter~\ref{chap:results} presents and discusses the experimental results from implementing the system identification methods to practical flight data.
        \gls{HITL} results are also presented to test the controller algorithms with the practical hardware and software systems.
        
        \paragraph
        Chapter~\ref{chap:conclusion} provides a summary of the work in this thesis. The major conclusions of this work is also presented and future recommendations are discussed.


