\chapter*{Abstract}
\addcontentsline{toc}{chapter}{Abstract}
\makeatletter\@mkboth{}{Abstract}\makeatother

\paragraph
This thesis considers the problem of a stabilising a multirotor with an unknown suspended payload.
The controller has no prior knowledge of the payload dynamics.
The swinging payload negatively affects the multirotor flight dynamics by inducing oscillations in the system.
In this work, an adaptive control architecture is proposed to damp these oscillations and produce stable flight.
A data-driven system identification method is demonstrated in simulation and with practical flights.
The final control architecture is also verified on practical hardware with \gls{HITL} simulations.  

\paragraph
An architecture with a parameter estimator and \gls{LQR} is used as a baseline method.
For the controller design, this uses a predetermined model of the payload dynamics and only estimates the payload mass and cable length.
The proposed architecture uses \gls{DMDc} to estimate a state-space model from a short length of flight data to approximate the unknown multirotor-payload dynamics.
Thereafter, an \gls{MPC} uses this model to control the multirotor and damp the payload oscillations.

\paragraph
A Simulink\texttrademark~simulation environment is designed and verified with practical data.
Within simulation, both the baseline and proposed architectures produce near swing-free payload motion for multirotor velocity control with different payload masses, cable lengths, external disturbances, and sensor noise.
Both architectures also produce effective disturbance rejection.
With a dynamic payload producing irregular oscillations, both methods resulted in stabilised control.
Despite the baseline using an accurate predetermined model, the proposed method produces an equal swing damping performance without an prior knowledge of the payload dynamics.
The baseline control performance degrades significantly when the multirotor mass is changed, because this parameter is not considered unknown.
However, the proposed method produced a consistently good performance. 

\paragraph
The accuracy of the \gls{DMDc} prediction models was successfully verified with training and testing data from practical flights.
The proposed control architecture was also tested in \gls{HITL} simulations.
The hardware successfully ran the \gls{MPC} at the desired speed, producing near swing-free control with a Gazebo simulator.

% ?? make past tense

\glsresetall