\chapter*{Uittreksel}
% \addcontentsline{toc}{chapter}{Uittreksel}
\makeatletter\@mkboth{}{Uittreksel}\makeatother

\vspace{-5mm}

\paragraph
Hierdie tesis hanteer die probleem van gestabiliseerde beheer vir 'n multirotor hommeltuig met 'n onbekende hangende loonvrag.
Die beheerder veronderstel geen voorafkennis van die loonvragdinamika nie.
Die swaaiende loonvrag be\"invloed die multirotor-vlugdinamika negatief deur ossillasies in die stelsel te veroorsaak.
'n Aanpasbare beheerargitektuur word voorgestel om hierdie ossillasies te demp en \'n stabiele vlug met verskillende onbekende loonvragte te verskaf.
’n Datagedrewe stelsel-identifikasiemetode vorm deel van die argitektuur en word in simulasie en met praktiese vlugdata gedemonstreer.
\gls{MPC} word toegepas vir swaaidempingsbeheer en word geverifieer met \gls{HITL}-simulasies.

\paragraph
'n Parameterberamer en \gls{LQR} word in die basislynargitektuur gebruik.
Die \gls{LQR} gebruik 'n voorafbepaalde model van die loonvrag, wat voltooi word met skattings van die loonvragmassa en kabellengte.
Die argitektuur wat in hierdie werk voorgestel word, gebruik \gls{DMDc} om 'n line\^ere toestand-ruimte model te skat en die onbekende dinamika te benader.
'n Kort lengte van vlugdata word vir opleiding gebruik.
'n \gls{MPC} gebruik die data-gedrewe model om die multirotor te beheer en die loonvrag ossillasies te demp.

% ?? Last paragraph from english to afrikaans
% ?? Afrikaans acronymns

\paragraph 
A Simulink\texttrademark~simulator was designed and verified with practical data.
Within simulations, both the baseline and proposed architectures produced near swing-free multirotor control with different payload masses and cable lengths.
Even with a dynamic payload producing irregular oscillations, both methods resulted in stabilised control.
Both architectures also showed effective disturbance rejection.
Despite the baseline method using an accurate predetermined model, the proposed method produced equal performances without prior knowledge of the dynamics.
The baseline performance degraded significantly for a different multirotor mass because this parameter was not considered as an unknown.
In contrast, the proposed method consistently produced good performances. 

\paragraph
Die akkuraatheid van die \gls{DMDc} modelle is geverifieer met praktiese vlugdata.
Die voorgestelde beheerargitektuur is ook in \gls{HITL}-simulasies gedemonstreer.
Die hardeware het die \gls{MPC} teen die verlangde spoed uitgevoer, wat byna swaaivrye beheer in 'n Gazebo-simulator gelewer het.
Oor die algemeen is aangetoon dat die voorgestelde beheerargitektuur prakties uitvoerbaar is.
Sonder kennis van die loonvragdinamika word 'n data-gedrewe model met \gls{MPC} gebruik vir effektiewe swaaidempingsbeheer met 'n multirotor.

\glsresetall