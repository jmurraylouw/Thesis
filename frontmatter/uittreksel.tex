\chapter*{Uittreksel}
% \addcontentsline{toc}{chapter}{Uittreksel}
\makeatletter\@mkboth{}{Uittreksel}\makeatother

\vspace{-5mm}

\paragraph
Hierdie tesis hanteer die probleem van gestabiliseerde beheer vir 'n multirotor hommeltuig met 'n onbekende hangende loonvrag.
Die swaaiende loonvrag be\"invloed die vlugdinamika deur ossillasies in die stelsel te veroorsaak.
'n Aanpasbare beheerargitektuur word voorgestel om hierdie ossillasies te demp vir stabiele vlugte met verskillende onbekende loonvragte.
Die argitektuur maak gebruik van 'n datagedrewe stelsel-identifikasiemetode wat geen voorafkennis van die loonvragdinamika gebruik nie.
Hierdie metode word in simulasies en met praktiese vlugdata gedemonstreer.
Model Voorspellende Beheer (MVB) word toegepas vir swaaidempingsbeheer en word geverifieer met Hardeware-in-die-Lus (HIDL) simulasies.

\paragraph
'n Parameter-afskatter en Line\^ere Kwadratiese Gaussiese (LKG) word in die basislyn beheerargitektuur gebruik.
Die LKG gebruik 'n voorafbepaalde model van die sisteem wat voltooi word met afskattings van die loonvragmassa en kabellengte.
Die nuwe voorgestelde argitektuur gebruik Dinamiese Modus Ontbinding met beheer (DMOb) om 'n line\^ere toestand-ruimte model te bereken en die dinamika af te skat sonder 'n voorafbepaalde model.
Die argitektuur is ook getoets met 'n Hankel Alternatiewe Siening van Koopman (HASK)-algoritme wat in hierdie werk uitgebrei is om beheer in te sluit.
'n MVB gebruik die data-gedrewe model om die multirotor te beheer en die loonvrag se ossillasies te demp.

\paragraph 
'n Simulink\texttrademark-simululeerder is ontwerp en geverifieer met praktiese data.
In simulasies het beide die basislyn en voorgestelde argitekture byna-swaaivrye beheer met verskillende loonvragmassas en kabellengtes geproduseer.
Selfs met 'n dinamiese loonvrag wat onre\"elmatige ossillasies voortbring, het beide metodes gestabiliseerde beheer tot gevolg gehad.
Beide argitekture het ook effektiewe versteuringsverwerping getoon.
Al gebruik die basislynmetode 'n akkurate voorafbepaalde model, het die voorgestelde metode gelyke prestasies gelewer sonder voorafkennis van die dinamika.
Die basislyn prestasie het aansienlik afgeneem vir 'n aangepaste multirotormassa omdat hierdie parameter nie as 'n onbekende beskou is nie.
Daarteenoor het die voorgestelde metode deurgaans goeie prestasies gelewer. 

\paragraph
Die akkuraatheid van die DMOb modelle is geverifieer met praktiese vlugdata.
Die voorgestelde beheerargitektuur is ook in HIDL-simulasies gedemonstreer.
MVB is teen die verlangde frekwensie uitgevoer en het byna-swaaivrye beheer in 'n Gazebo-simululeerder gelewer.
In die geheel is dit gewys dat die voorgestelde beheerargitektuur prakties uitvoerbaar is.
Sonder kennis van die loonvragdinamika kan 'n data-gedrewe model met MVB gebruik word vir effektiewe swaaidempingsbeheer met 'n multirotor.

\glsresetall