\chapter*{Uittreksel}
% \addcontentsline{toc}{chapter}{Uittreksel}
\makeatletter\@mkboth{}{Uittreksel}\makeatother

\vspace{-5mm}

\paragraph
Hierdie tesis hanteer die probleem van gestabiliseerde beheer vir 'n multirotor hommeltuig met 'n onbekende hangende loonvrag.
Die beheerder aanvaar geen voorafkennis van die loonvragdinamika nie.
Die swaaiende loonvrag het 'n negatiewe invloed op die multirotor-vlugdinamika deur ossillasies in die stelsel te veroorsaak.
'n Aanpasbare beheerargitektuur word voorgestel om hierdie ossillasies te demp vir stabiele vlugte met verskillende onbekende loonvragte.
'n Datagedrewe stelsel-identifikasiemetode vorm deel van die argitektuur en word in simulasie en met praktiese vlugdata gedemonstreer.
Model Voorspellende Beheer (MVB) word toegepas vir swaaidempingsbeheer en word geverifieer met Hardeware-in-die-Lus (HIDL) simulasies.

\paragraph
'n Parameter-afskatter en Line\^ere Kwadratiese Gaussiese (LKG) word in die basislynargitektuur gebruik.
Die LKG gebruik 'n voorafbepaalde model van die loonvrag, wat voltooi word met afskattings van die loonvragmassa en kabellengte.
Die nuwe argitektuur, wat in hierdie werk voorgestel word, gebruik Dinamiese Modus Ontbinding met Beheer (DMOB) om 'n line\^ere toestand-ruimte model te bereken en die onbekende dinamika te benader.
'n Kort lengte van vlugdata word vir opleiding gebruik.
'n MVB gebruik die data-gedrewe model om die multirotor te beheer en die loonvrag se ossillasies te demp.

\paragraph 
'n Simulink\texttrademark~simulator is ontwerp en geverifieer met praktiese data.
In simulasies het beide die basislyn en voorgestelde argitekture byna swaaivrye multirotorbeheer met verskillende loonvragmassas en kabellengtes geproduseer.
Selfs met 'n dinamiese loonvrag wat onreëlmatige ossillasies voortbring, het beide metodes gestabiliseerde beheer tot gevolg gehad.
Beide argitekture het ook effektiewe versteuringsverwerping getoon.
Ten spyte daarvan dat die basislynmetode 'n akkurate voorafbepaalde model gebruik het, het die voorgestelde metode gelyke prestasies gelewer sonder voorafkennis van die dinamika.
Die basislyn prestasie het aansienlik afgeneem vir 'n ander multirotor massa omdat hierdie parameter nie as 'n onbekende beskou is nie.
Daarteenoor het die voorgestelde metode deurgaans goeie prestasies gelewer. 

\paragraph
Die akkuraatheid van die DMOB modelle is geverifieer met praktiese vlugdata.
Die voorgestelde beheerargitektuur is ook in HIDL-simulasies gedemonstreer.
Die hardeware het die MVB teen die verlangde frekwensie uitgevoer, wat byna swaaivrye beheer in 'n Gazebo-simululeerder gelewer het.
In die geheel is dit gewys dat die voorgestelde beheerargitektuur prakties uitvoerbaar is.
Sonder kennis van die loonvragdinamika kan 'n data-gedrewe model met MVB gebruik word vir effektiewe swaaidempingsbeheer met 'n multirotor.

\glsresetall